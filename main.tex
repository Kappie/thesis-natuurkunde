\documentclass[9pt, ebook, openany, oneside]{memoir}

% Packages for baskervillef font.
\usepackage[T1]{fontenc}
\usepackage{baskervillef}
\usepackage[varqu,varl,var0]{inconsolata}
\usepackage[scale=.95,type1]{cabin}
\usepackage[baskerville,vvarbb]{newtxmath}
\usepackage[cal=boondoxo]{mathalfa}

% Bulk of packages:
\usepackage[english]{babel}
\usepackage{standalone}
\usepackage[utf8]{inputenc}
\usepackage{graphicx}
\usepackage{todonotes}
\usepackage{amsmath}
\usepackage{amssymb}
\usepackage{braket}
\usepackage{bm}
\usepackage{tabularx}
\usepackage{subcaption}
\usepackage{mathtools}
\usepackage{booktabs}
\usepackage{tablefootnote}

% Input packages for tikz pictures.
\usepackage{tikz}
\usepackage{pgfplots}
\usepackage{tikzscale}

\usepgfplotslibrary{colorbrewer}
 \definecolor{scientific1}{rgb}{0.028, 0.5376, 0.5936}
 \definecolor{scientific2}{rgb}{0.75, 0.315, 0.}
 \definecolor{scientific3}{rgb}{0.531753, 0.331477, 0.920616}
 \definecolor{scientific4}{rgb}{0.627887, 0.708654, 0.}
 \definecolor{scientific5}{rgb}{0.237882, 0.510711, 0.979357}
 \definecolor{scientific6}{rgb}{0.975692, 0.628459, 0.0656322}
 \definecolor{scientific7}{rgb}{0.629898, 0.28, 0.684772}
 \definecolor{scientific8}{rgb}{0.198854, 0.7, 0.446913}
 \definecolor{scientific9}{rgb}{0.910038, 0.300188, 0.226913}
 \definecolor{scientific10}{rgb}{0.440765, 0.385541, 1.}
 \definecolor{scientific11}{rgb}{0.797526, 0.718694, 0.00486471}
 \definecolor{scientific12}{rgb}{0.792386, 0.272363, 0.429254}
 \definecolor{scientific13}{rgb}{0.289887, 0.478466, 0.985608}
 \definecolor{scientific14}{rgb}{0.96081, 0.554048, 0.104292}
 \definecolor{scientific15}{rgb}{0.633426, 0.29116, 0.697739}
\pgfplotscreateplotcyclelist{my mark list}{%
    {scientific1, mark=o},
    {scientific2, mark=square},
    {scientific3, mark=triangle},
    {scientific4, mark=diamond},
    {scientific5, mark=pentagon},
    {scientific6, mark=x},
    {scientific7, mark=plus},
    {scientific8, mark=asterisk},
    {scientific9, mark=star},
}

\pgfplotsset{
  compat=1.5,
  table/search path={plots/data},
  cycle list name = my mark list,
  every tick/.style={
    semithick,
  },
  every line/.style={
    semithick,
  }
  % cycle list/Dark2-8,
  % cycle multiindex* list={
  %   my mark list\nextlist
  %   % Dark2-8\nextlist
  % },
}

\tikzset{
  every mark/.append style={solid, thick},
  mark size = 2pt,
}

\pgfplotsset{mediumMarker/.style={%
  mark size = 2.0pt,
  }}

\pgfplotsset{smallMarker/.style={%
  mark size = 1.3pt,
  }}

\pgfplotsset{miniMarker/.style={%
  mark size = 0.7pt,
  }}

\newlength{\plotWidthTwoColumn}
\setlength{\plotWidthTwoColumn}{0.5\textwidth}

\usetikzlibrary{decorations.markings}
\usetikzlibrary{decorations.pathreplacing}
\usetikzlibrary{shapes.geometric}
\usetikzlibrary{positioning}
\usetikzlibrary{plotmarks}
\usetikzlibrary{arrows.meta}
\usepgfplotslibrary{patchplots}
\usepgfplotslibrary{groupplots}
\usepackage{grffile}



\pgfdeclarelayer{edgelayer}
\pgfdeclarelayer{nodelayer}
\pgfsetlayers{edgelayer,nodelayer,main}


\def\tensorsize{1.3em}

\definecolor{princess-orange}{rgb}{0.988,0.824,0.745}
\definecolor{princess-green}{rgb}{0.929,0.929,0.792}
\definecolor{princess-blue}{rgb}{0.627,0.808,0.808}
\definecolor{princess-grey}{rgb}{0.6549,0.612, 0.655}

\tikzstyle{none}=[inner sep=0pt]
\tikzstyle{tensor}=[circle, draw=black, thick, minimum size = \tensorsize]
\tikzstyle{generic} = [tensor, fill=princess-orange]
\tikzstyle{generic2} = [tensor, very thick, fill=princess-green, minimum size =
1.15*\tensorsize]
\tikzstyle{a-tensor}=[tensor,fill=princess-blue]
\tikzstyle{p-tensor-up} = [tensor, minimum size = 0.5*\tensorsize, shape=semicircle,
fill=princess-green]
\tikzstyle{p-tensor-right} = [p-tensor-up, shape border rotate = 90]
\tikzstyle{p-tensor-down} = [p-tensor-up, shape border rotate = 180]
\tikzstyle{p-tensor-left} = [p-tensor-up, shape border rotate = 270]
\tikzstyle{q-tensor} = [tensor, fill=princess-green]
\tikzstyle{delta-tensor}=[tensor, shape=diamond, minimum size = 0.75*\tensorsize, fill=princess-orange]
\tikzstyle{white no border}=[tensor,minimum size=1.75*\tensorsize,fill=white,draw=white]
\tikzstyle{white no border small}=[minimum size = .01*\tensorsize]
\tikzstyle{largest_eigenvector}=[tensor, shape=rectangle, fill=princess-green, minimum
height = 4cm, minimum width = 0.75*\tensorsize]
\tikzstyle{isometry}=[tensor, shape=isosceles triangle, isosceles triangle stretches,
fill=princess-orange, minimum width = 2*\tensorsize]
\tikzstyle{square}=[tensor, shape=rectangle, minimum width = 0.75*\tensorsize, minimum
height = 2*\tensorsize, fill=princess-orange]
\tikzstyle{singular values matrix} = [tensor, shape=diamond, minimum size =
0.5*\tensorsize, fill=white]
\tikzstyle{ctm} = [generic2, fill=princess-grey]

\tikzstyle{simple}=[-,draw=black,thick]
\tikzstyle{thick leg}=[simple, very thick]
\tikzstyle{arrow}=[-,draw=black,postaction={decorate},decoration={markings,mark=at position .5 with {\arrow{>}}},line width=2.000]
\tikzstyle{tick}=[-,draw=black,postaction={decorate},decoration={markings,mark=at position .5 with {\draw (0,-0.1) -- (0,0.1);}},line width=2.000]


% Just to make sure: import these packages last.
\usepackage[backend=biber, style=phys, hyperref]{biblatex}
\usepackage[]{hyperref}

\newcommand\myshade{80}
\hypersetup{
  linktocpage,
  backref,
  pdfpagemode=FullScreen,
  linkcolor  = scientific7,
  citecolor  = scientific8,
  urlcolor   = scientific9,
  colorlinks = true,
}
\addto\extrasenglish{%
  \def\subsectionautorefname{section}%
  \def\subsubsectionautorefname{section}%
  \def\equationautorefname{Eq.}%
  \def\figureautorefname{Fig.}%
  \def\chapterautorefname{Chap.}%
}

\definecolor{orange}{RGB}{252,210,190}



\addbibresource{references.bib}
\graphicspath{{images/}{plots/}}



\maxtocdepth{subsubsection}
\setsecnumdepth{subsubsection}


% Commands below define chapter title style.
\makeatletter
 \renewcommand{\chapterheadstart}{\vspace*{\beforechapskip}}
 \renewcommand{\printchaptername}{}
 \renewcommand{\chapternamenum}{\space}
 \renewcommand{\printchapternum}{\centering\chapnumfont\thechapter}
 \renewcommand{\afterchapternum}{\par\nobreak\vskip \midchapskip}
 \renewcommand{\printchapternonum}{}
 % \renewcommand{\printchaptertitle}[1]{\centering\chaptitlefont\MakeLowercase{ \oldstylenums #1}}
 \renewcommand{\printchaptertitle}[1]{\centering\chaptitlefont #1}
 % \renewcommand{\printchaptertitle}[1]{\centering\chaptitlefont}
 \renewcommand{\afterchaptertitle}{%
 \vskip2em
 \hrule height 0.6pt
 \vskip2em
 }
 \renewcommand{\chapnamefont}{\normalfont\huge\bfseries}
 \renewcommand{\chapnumfont}{\normalfont\Huge\bfseries}
 \renewcommand{\chaptitlefont}{\normalfont\huge\bfseries}
 \setlength{\beforechapskip}{0.5em}
 \setlength{\midchapskip}{0.5em}
 \setlength{\afterchapskip}{4em}
\makeatother

% To make chapter abstracts regular font
\renewcommand*{\precisfont}{\normalfont}

\DeclareMathOperator{\tr}{Tr}


\begin{document}

\pagestyle{simple}



\frontmatter

\title{Mijn Titel}
\author{Geert Kapteijns}
\date{\today}

\newlength\drop
\newcommand*{\titleMS}{
\thispagestyle{empty}
\begingroup% MS Thesis
\drop=0.1\textheight
\vspace*{\drop}
\centering
{\LARGE University of Amsterdam}\\[2\baselineskip]
{\LARGE Finite bond dimension scaling with the corner transfer matrix renormalization group method}\par
\vspace{\drop}
{\large supervisor: Dr. P. Corboz\\
second examiner: Prof. Dr. B. Nienhuis}\par
\vfill
{\large\bfseries Geert Kapteijns}\par
% \vspace*{\drop}
\clearpage
\endgroup}

\titleMS


\begingroup
\renewcommand{\afterchaptertitle}{\vskip1.5em}

\tableofcontents*
\endgroup

% \chapter{Abstract}
% \noindent This thesis investigates scaling in the number of basis states
kept (the \emph{bond dimension} $m$) in approximating the partition function
of two-dimensional classical models with the corner transfer matrix
renormalization group (CTMRG) method.

For the Ising model, it is shown that exponents and the transition temperature may be approximated with a scaling
analysis in the correlation length defined in terms of the row-to-row transfer matrix at the (pseudo)critical point,
as was suggested by Nishino et al.
However, the calculated quantities show inherent deviations from the basic scaling laws,
due to the spectrum of the underlying corner transfer matrix (CTM).
These deviations are mitigated to some extent when we define the correlation length in terms of the classical analogue
of the entanglement entropy.
Scaling directly in the bond dimension $m$ is also possible, but less accurate since the law for the correlation length
$\xi \propto m^{\kappa}$ holds only in the limit $m \to \infty$ and does not take into account the spectrum of the CTM
that is obtained.

It is found that finite-$m$ scaling and finite-size scaling yield comparable accuracy for critical exponents and the
transition temperature.
With finite-$m$ scaling larger effective system sizes are obtainable,
but finite-size approximations do not suffer from the deviations due to the CTM spectrum and are
consequently of higher quality. Therefore it is plausible that finite-size results will improve significantly if
corrections to scaling are included in the fits.

We also present a numerical analysis of the clock model with $q = \{5,
6\}$ states, concluding that the Kosterlitz-Thouless picture is plausible.
We find values of the transition temperatures that are in agreement with values found by other authors.
Results for the exponent $\eta$ indicate that the critical temperatures found in both this study and previous work might
be too close together.
It is conceivable that, after considering larger systems and taking into account finite-size corrections,
both critical temperatures and the values of $\eta$ will be adjusted outwards towards their true values,
thereby completely reconciling the results.

Overall, we conclude that finite-$m$ scaling is a valuable alternative to finite-size scaling within CTMRG,
since larger system sizes are accessible.
The CTMRG analysis is itself a valuable addition to other approximation methods such as Monte Carlo,
yielding comparable results, while obtaining estimates from completely different principles.
Furthermore it reveals information, such as the the spectrum of the transfer matrices and the central charge of the
massless phase, that is not accessible otherwise.

%
% \chapter{Acknowledgements}
% I want to thank Philippe Corboz for all his help during the past year.
Thanks for being always enthusiastic and ready to answer questions, for including me in the group from the start,
and for showing me the messy reality of research.

Thanks to Piotr Czarnik, Sangwoo Chung, Schelto Crone, Karel Temmink and Ido Niesen for many helpful discussions.
Especially Ido, who took more time for me than could reasonably be expected for someone with such a busy schedule.

Thanks to Bernard Nienhuis, for very helpful discussions towards the end of my thesis.

Thanks to Tobias Bouma, for good times and for sharing your arcane knowledge of LaTeX.

Thanks to Boris Ponsioen for the hours and hours we spent drinking coffee,
discussing the intricacies of this-or-that algorithm or our next career move.
Those significantly broadened my perspective.

Thanks to Daan Mulder for our discussions about physics, politics and music.
If I remembered one line of a song, you could always declaim the whole verse,
at least in the case of Cohen (may he rest in peace).

Thanks to my family, whom I didn't see much the past year. And finally, thanks to Marianne.


\mainmatter

% \chapter{Introduction}
% This thesis investigates a numerical approximation method put forth by
Baxter in 1978 \cite{baxter1978variational, baxter1982exactly_ctm,
tsang1979square} based on the corner transfer matrix formulation of the
partition function for two-dimensional classical lattice models with
a round-a-face interaction. 

The method rose to prominence in 1996, under the name \emph{corner
transfer matrix renormalization group} (CTMRG), when Nishino showed
\cite{nishino1996corner} that in the thermodynamic limit, it is equivalent
to the hugely successful density matrix renormalization group (DMRG)
method for one-dimensional quantum systems \cite{white1992density},
discovered a few years earlier by White.

The error in the CTMRG method comes from the fact that the corner transfer
matrices, whose dimension diverges exponentially in the lattice size, have
to be truncated at a maximum size $m$ in order to make numerical
manipulation possible. This finite \emph{bond dimension} $m$ introduces
finite-size effects, comparable to those observed for systems that are
finite in one or more spatial dimensions. This was already realized by
Nishino \cite{nishino1996numerical}.

The main objective of this thesis is to study how \emph{finite bond
dimension scaling} may be performed with the CTMRG method.

Before the structure of this thesis is laid out, I will first make some
general remarks on statistical mechanics and phase transitions, and on how
the CTMRG method relates to the class of methods for simulating many-body
systems that grew out of White's breakthrough, known as \emph{tensor
network algorithms}.

\section{Phase transitions}
Statistical mechanics is concerned with describing the average properties
of systems consisting of many particles. Examples of such systems are the
atoms making up a bar magnet, the water molecules in a glass of water, or
virtually any other instance of matter around us.

Matter can arrange itself in various structures with fundamentally
different properties. We call these distinct states of matter
\emph{phases}. When matter changes from one phase to another, we say it
undergoes a phase transition.

Physics has made great strides in understanding these transitions. The
complete history of the field is beyond the scope of this introduction and this
thesis, but the reader may wish to consult \cite{kadanoff2009more,
domb1996critical} to get an idea.

Only as late as 1936, the occurrence of a phase transition
within the framework of statistical physics was established by R. Peierls
\cite{peierls1936on_ising}. He showed that the two-dimensional Ising model
has a non-zero magnetization for sufficiently low temperatures. Since for
high enough temperatures the Ising model loses its magnetization, it
follows that there must be phase transition in between.

The effort to understand the Ising model culminated with Onsager's exact
solution in 1944 \cite{onsager1944two_dimensional}, which rigorously
established a sharp transition point in the thermodynamic limit.

One may question the relevance of studying very simple models such as the
Ising model. As it turns out, systems that are at first sight vastly
different may show qualitatively similar behaviour near a phase
transition. For example, exponents that characterize the divergence of
quantities near a transition are conjectured to be independent on
microscopic details of the interactions between particles, but instead
fall into distinct \emph{universality classes}
\cite{griffiths1970dependence, fisher1966quantum}. Thus, studying the very
simplest model may yield universal results.

\section{Baxter's method as a precursor to tensor network methods}

Baxter showed that the optimal truncation of corner transfer matrices
corresponds, in the thermodynamic limit, to a variational optimization of
the row-to-row transfer matrix within a certain subspace, now known as the
subspace of \emph{matrix product states} (MPSs) \cite{baxter1968dimers,
baxter1982exactly_ctm}.

After the success of White's DMRG, which, as Nishino showed, is equivalent
to Baxter's method, the underlying matrix-product structure was
rediscovered in the context of one-dimensional quantum systems by Östlund
and Rommer \cite{ostlund1995thermodynamic, rommer1997class}.

It is historically significant but little known that Nightingale, in
a footnote of a 1986 paper \cite{nightingale1986gap}, already made the
remark that \enquote{The generalization [of Baxter's method] to quantum
mechanical systems is straightforward.}

After Östlund and Rommer, it was realized that reformulating White's
algorithm directly in terms of matrix product states provided the
explanation of the algorithm's shortcomings around phase transitions. An
MPS-ansatz fundamentally limits the entropy of the ground state
approximation and since the entropy diverges at a conformally invariant
critical point \cite{calabrese2004entanglement}, DMRG gives inaccurate
results.

This gave rise to other ansätze, formulated in the language of tensor
networks \cite{orus2014practical}, specifically designed to represent
states with a certain amount of entropy. Examples are multi-scale
entanglement renormalization ansatz (MERA) for critical one-dimensional
quantum systems \cite{vidal2007entanglement} and projected entangled-pair
states (PEPS) \cite{verstraete2004renormalization}
for two-dimensional quantum systems.

Other tensor network algorithms, such as infinite time-evolving block
decimation \cite{vidal2007classical} in one dimension and iPEPS (infinite
PEPS) \cite{jordan2008classical} in two dimensions, made it possible to
directly approximate quantum systems in the thermodynamic limit.



\section{Structure of this thesis}

It is in the mostly quantum-oriented context sketched above that the
research for this thesis was carried out. Therefore, I have chosen to
begin by introducing White's algorithm in its original description
(chapter two), before making the connection to two-dimensional classical
lattices and properly introducing the corner transfer matrix formulation
(chapter three).

Chapter four introduces the concepts of critical behaviour and finite-size
scaling and chapter five 



% We investigate finite-size scaling behaviour in two-dimensional classical systems using
% the corner transfer matrix renormalization group (CTMRG) method. Instead of scaling in the
% system size $N$, we perform a scaling analysis in the bond dimension -- or numbers of
% basis states kept -- in approximating the corner transfer matrix of the system. This
% dimension is denoted $m$, though in other works it may appear as $\chi$ or $D$.
%
% This thesis is laid out as follows. Chapter two introduces the density matrix
% renormalization group (DMRG) method in the context of one-dimensional quantum systems.
% Since most of the literature and research focuses on quantum lattice models, it is helpful
% to keep this picture in mind.
%
% Chapter three explains how the ideas of DMRG can be applied to two-dimensional classical
% lattices. After making this connection clear, the corner transfer matrix
% renormalization group (CTMRG) method, a unification of DMRG and earlier ideas from Baxter, is introduced.
%
% Chapter four introduces the concepts of critical behaviour and finite-size scaling,
% while chapter five makes the connection to finite-size effects that appear as a consequence of a finite bond dimension
% $m$ within the CTMRG method.
%
% Chapter six describes implementation details and convergence behaviour of the CTMRG algorithm.
% Chapter seven presents numerical results of finite-$m$ scaling for the Ising model.
%
% Chapter eight introduces the clock model and the basic concepts of the Kosterlitz-Thouless transition,
% presents results of finite-$m$ scaling and compares with works using other numerical methods.

%
% \chapter{Density matrix renormalization group method}
% \section{Abstract}

The density matrix renormalization group, proposed in 1992 by White
\cite{white1992density}, is introduced in its historical context. To highlight the ideas
that led to this method, we explain the real-space renormalization group, proposed by
Wilson \cite{wilson1975renormalization} in 1975. We then explain how the shortcomings of
Wilson's method led to the density matrix renormalization group.

\section{Introduction}
Consider the problem of numerically finding the ground state $\ket{\Psi_0}$ of the
$N$-site 1-dimensional transverse-field Ising-model, given by

\begin{equation}
  H = -J \sum_{i = 1}^{N} \sigma_{i}^{z}\sigma_{i+1}^{z}
  - h \sum_{i=1}^{N} \sigma_{i}^{x}.
\end{equation}

The underlying Hilbert space of the system is a tensor product of the
local Hilbert spaces $\mathcal{H}_{\text{site}}$, which are spanned by the
states $\{\ket{\uparrow}, \ket{\downarrow}\}$. Thus, a general state of the system is a unit vector in
a $2^N$-dimensional space
\begin{equation}
  \ket{\Psi} = \sum_{\sigma_1, \sigma_2, \ldots \in \{\ket{\uparrow}, \ket{\downarrow}\}}
  c_{\sigma_1, \sigma_2, \ldots, \sigma_N} \ket{\sigma_1} \otimes \ket{\sigma_2} \otimes
  \ldots \otimes \ket{\sigma_N}.
\end{equation}

For a system with 1000 particles, the dimensionality of the Hilbert
space comes in at about $10^{301}$, some 220 orders of magnitude larger than the number of
atoms in the observable universe. How can we possibly hope to approximate states in this
space?

As it turns out, nature is very well described by Hamiltonians that are local -- that do
not contain interactions between an arbitrary number of bodies. And for these
Hamiltonians, only an exponentially small subset of states can be explored in the lifetime
of the universe \cite{poulin2011quantum}. That is, only exponentially few states are
physical. The low-energy states, especially, have special properties that allow them to be
very well approximated by a polynomial number of parameters. This explains the existence
of algorithms, of which the density matrix normalization group is the most widely
celebrated one, that can approximate certain quantum systems to machine precision.

\todo[inline]{Refer to where this will be made more precise.}

\section{Density matrix renormalization group}

The density matrix renormalization group (DMRG), introduced in 1992 by
White \cite{white1992density}, aims to find the best approximation of
a many-body quantum state, given that only a fixed amount of basis vectors
is kept. This amounts to finding the best truncation

\begin{equation}
  \mathcal{H}_N \rightarrow \mathcal{H}_{\text{eff}}
\end{equation}
from the full $N$-particle Hilbert space to an effective lower dimensional
one. This corresponds to renormalizing the Hamiltonian $H$.

Before DMRG, several methods for achieving this truncation were proposed, most notably
Wilson's real-space renormalization group \cite{wilson1975renormalization}. We will
discuss this method first, and highlight its shortcomings, which eventually led to the
invention of the density-matrix renormalization group method by White.

\subsection{Real-space renormalization group}

Consider again the problem of finding the ground state of a many-body
Hamiltonian $H$. A natural way of renormalizing $H$ in real-space is by
partitioning the lattice in blocks, and writing $H$ as

\begin{equation}
  H = H_A \otimes \ldots \otimes H_A
\end{equation}

where $H_A$ is the Hamiltonian of a block. \todo[inline]{Make figures.} The real-space
renormalization procedure now entails finding an effective Hamiltonian $H_{A}'$ of the
two-block Hamiltonian $H_{AA} = H_A \otimes H_A$. In the method introduced
by Wilson, $H_{A}'$ is formed by keeping the $m$ lowest lying eigenstates
$\ket{\epsilon_{i}}$ of $H_{AA}$:
\begin{equation}
  H_{A}' = \sum_{i = 1}^{m} \epsilon_{i} \ket{\epsilon_{i}}\bra{\epsilon_{i}}.
 \end{equation}

This is equivalent to writing
\begin{equation}
  H_{A}' = O H_{AA} O^{\dagger},
\end{equation}
with $O$ an $m \times 2^L$ matrix, with rows being the $m$ lowest-lying
eigenvectors of $H_{AA}$, and $L$ the number of lattice sites of a block. At
the fixed point of this iteration procedure, $H_A$ represents the
Hamiltonian of an infinite chain. \todo[inline]{Explain renormalization
group idea somewhere.} In choosing this truncation, it is assumed that the
low-lying eigenstates of the system in the thermodynamic limit are
composed of low-lying eigenstates of smaller blocks.

It turns out that this method gives poor results for many lattice systems. Following an
example put forth by White and Noack \cite{white1992real}, we establish an intuition
why.

\subsection{Single particle in a box}

Consider the Hamiltonian
\begin{equation}
  H = 2 \sum_i \ket{i}\bra{i} - \sum_{\langle i, j \rangle} \ket{i}\bra{j},
\end{equation}
where the second summation is over nearest neighbors $\langle i,
j \rangle$. $H$ represents the discretized version of the
particle-in-a-box Hamiltonian, so we expect its ground state to be
approximately a standing wave with wavelength double the box size.
However, the blocking procedure just described tries to build the ground
state iteratively from ground states of smaller blocks. No matter the
amount of states kept, the final result will always incur large errors.

For this simple model, White and Noack solved the problem by diagonalizing
the Hamiltonian of a block with different boundary conditions, and
combining the lowest eigenstates of each.

Additionally, they noted that
diagonalizing $p > 2$ blocks, and projecting out $p - 2$ blocks to arrive
at $H_{AA}$ also gives accurate results, and that this is a generalization
of applying multiple boundary conditions. \todo[inline]{Figure.}

In the limit $p \to \infty$
this method becomes exact, since we then find exactly the correct
contribution of $H_{AA}$ to the final ground state. It is a slightly
changed version of this last method that is now known as DMRG.

\subsection{Density matrix method}

The fundamental idea of the density matrix renormalization group method
rests on the fact that if we know the state of the final lattice, we can find the $m$
most important states for $H_{AA}$ by diagonalizing the reduced density
matrix $\rho_{AA}$ of the two blocks.

To see this, suppose, for simplicity, that the entire lattice is in a pure
state\footnote{For a proof for a mixed state, see \cite{noack1999workshop}} $\ket{\Psi} = \sum c_{b, e} \ket{b} \ket{e}$, with $b = 1, \ldots, l$ the
states of $H_{AA}$ and $e = 1, \ldots, N_{\text{env}}$ the environment states. The
reduced density matrix is given by
\begin{equation}\label{eq:density_matrix_superblock}
  \rho_{AA} = \sum_{e} \ket{\Psi} \bra{\Psi} = \sum_{b, b'} c_{b, e} c_{b', e} \ket{b} \bra{b'}
\end{equation}
We now wish to find a set of orthonormal states $\ket{\lambda} \in \mathcal{H}_{AA}$,
$\lambda = 1, \ldots, m$ with $m < l$, such that the quadratic norm
\begin{equation}\label{eq:quadratic_norm}
  \Vert \ket{\Psi} - \ket{\widetilde{\Psi}} \Vert = 1 - 2 \sum_{\lambda, b, e} a_{\lambda, e} c_{b, e} u_{\lambda, b} + \sum_{\lambda, e} a_{\lambda, e}^2
\end{equation}
is minimized. Here,
\begin{equation}
  \ket{\widetilde{\Psi}} = \sum_{\lambda = 1}^{m} \sum_{e = 1}^{N_{\text{env}}} a_{\lambda, e} \ket{\lambda} \ket{e}
\end{equation}
is the representation of $\ket{\Psi}$ given the constraint that we can only use
$m$ states from $\mathcal{H}_{AA}$. The $u_{\lambda, b}$ are given by
\begin{equation}
  \lambda = \sum_{b} u_{\lambda, b} \ket{b}.
\end{equation}

We need to minimize \eqref{eq:quadratic_norm} with respect to $a_{\lambda, e}$
and $u_{\lambda, b}$. Setting the derivative with respect to $a_{\lambda, e}$ equal to 0 yields
\begin{equation}
  -2 \sum_{\lambda, b, e} c_{b, e} u_{\lambda, b} + 2 \sum_{\lambda, e} a_{\lambda, e} = 0
\end{equation}
So we see that $a_{\lambda, e} = \sum_{b} c_{b, e} u_{\lambda, b}$, and we are left to minimize
\begin{equation}
  1 - \sum_{\lambda, b, b'} u_{\lambda, b} (\rho_{AA})_{b, b'} u_{\lambda, b'}
\end{equation}
with respect to $u_{\lambda, b}$. But this is equal to
\begin{equation}
  1 - \sum_{\lambda = 1}^{m} \bra{\lambda} \rho_{AA} \ket{\lambda}
\end{equation}
and because the eigenvalues of $\rho_{AA}$ represent probabilities and are thus
non-negative, this is clearly minimal when $\ket{\lambda}$ are the $m$
eigenvectors of $\rho_{AA}$ corresponding to the largest eigenvalues. This minimal value is
\begin{equation}\label{eq:truncation_error}
  1 - \sum_{\lambda = 1}^{m} w_{\lambda}
\end{equation}
with $w_{\lambda}$ the eigenvalues of the reduced density matrix.

\eqref{eq:truncation_error} is called the truncation error or residual
probability, and quantifies the incurred error when taking a number $m < l$ states to
represent $\mathcal{H}_{AA}$.

\todo[inline]{Look ahead to SVD}

We have proven that the optimal (in the sense that $\Vert \ket{\Psi}
- \ket{\widetilde{\Psi}} \Vert$ is minimized\footnote{There are several other
arguments for why these states are optimal, for example, they minimize the
error in expectation values $\langle A \rangle$ of operators. For an overview,
see \cite{schollwock2005density}.}) states to keep for a subsystem are the
states given by the reduced density matrix, obtained by tracing out the entire
lattice in the ground state (or some other target state).

The problem, of
course, is that we do not know the state of the entire lattice, since that is
exactly what we're trying to approximate.

Instead then, we should try to calculate the reduced density matrix of the
system embedded in \textit{some} larger environment, as closely as possible
resembling the one in which it should be embedded.  The combination of the
system block and this environment block is usually called \textit{superblock}.

Analogous to how White and Noack solved the particle in a box problem, we could
calculate the ground state of $p > 2$ blocks,
and trace out all but 2, doubling our block size each iteration. In
practice, this doesn't work well for interacting Hamiltonians, since this
would involve finding the largest eigenvalue of a $N_{\text{block}}^p
\times N_{\text{block}}^p$ matrix (compare this with the particle in a box
Hamiltonian, which only grows linearly in the amount of lattice sites).

The widely adopted algorithm proposed by White \cite{white1993density} for
finding the ground state of a system in the thermodynamic limit proceeds
as follows.

\subsection{Infinite-system method}

\todo[inline]{Add figures.} \todo[inline]{Mention boundary conditions somewere}
Instead of using an exponential blocking procedure (doubling or tripling the
amount of effective sites in a block at each iteration), the infinite-system
method in the DMRG formulation adds a single site before truncating the Hilbert
space to have at most $m$ basis states.

\begin{enumerate}
  \item \label{step1} Consider a block $A$ of size $l$, with $l$ small. Suppose, for
  simplicity, that the number of basis states of the block is already
  $m$. States of this block can be written as
  \begin{equation}
    \ket{\Psi_A} = \sum_{b = 1}^{m} c_{i} \ket{b}.
  \end{equation}

  The Hamiltonian is written as (similarly for other operators):
  \begin{equation}
    \hat{H}_{A} = \sum_{b, b'}^{m} H_{b b'} \ket{b} \bra{b}.
  \end{equation}

  \item Construct an enlarged block with one additional site, denoted by $A
  \cdot$. States are now written
  \begin{equation}
    \ket{\Psi_{A\cdot}} = \sum_{b, \sigma} c_{b, \sigma} \ket{b} \otimes \ket{\sigma}.
  \end{equation}
  Here, $\sigma$ runs over the $d$ local basis states of $\mathcal{H}_{\text{site}}$.

  \item Construct a superblock, consisting of the enlarged system block $A
  \cdot$ and a reflected environment block $\cdot A$, together denoted by $A \cdot
  \cdot A$. Find the ground state $\ket{\Psi_0}$ of $A \cdot \cdot A$, for example
  with the Lanczos method \cite{lehoucq1996deflation}.

  \item Obtain the reduced density matrix of the enlarged block by tracing out
  the environment, and write it in diagonal form.
  \begin{equation}
  \begin{aligned}
    \rho_{A \cdot} & = \sum_{e, \sigma} (\bra{\sigma} \otimes \bra{e}) \ket{\Psi_0}
    \bra{\Psi_0} (\ket{\sigma} \otimes \ket{e}), \\
    & = \sum_{i = 1}^{d m} w_{i} \ket{\lambda_i} \bra{\lambda_i}.
  \end{aligned}
  \end{equation}

  Here, we have chosen $w_0 >= w_1 \ldots >= w_{d m}$. In this basis, the Hamiltonian is
  written as
  \begin{equation}
    \hat{H}_{A \cdot} = \sum_{i, j}^{d m} H_{ij} \ket{\lambda_i}\bra{\lambda_j}.
  \end{equation}

  \item Truncate the Hilbert space by keeping only the $m$ eigenstates of
  $\rho_{A \cdot}$ with largest eigenvalues. Operators truncate as follows:
  \begin{align}
    \widetilde{\rho}_{A \cdot} & = \sum_{i = 1}^{m} w_i \ket{\lambda_i}\bra{\lambda_j}, \\
    \widetilde{H}_{A \cdot} & = \sum_{i, j}^{m} H_{ij} \ket{\lambda_i}\bra{\lambda_j}.
  \end{align}

  \item Set $H_{A} \leftarrow \widetilde{H}_{A \cdot}$ and return to \ref{step1}.

\end{enumerate}

\todo[inline]{Expand. Present or link to some results. Finite-system algorithm. Maybe in
other chapter: rephrase in MPS, validity of approximation: primer on entropy and
eigenvalue spectrum of density matrix.}

This methods finds ground state energies with astounding accuracy, and has been
the reference point in all 1D quantum lattice simulation since its invention.

%
% \chapter{DRMG applied to two-dimensional classical lattice models}
% \section{Partition functions of classical lattices}
The central quantity in equilibrium statistical mechanics is the partition
function $Z$, which, for a discrete system such as a lattice, is defined as
\begin{equation}
  Z = \sum_{s} \exp{(-\beta H(s))}
\end{equation}
where the sum is over all microstates $s$, $H$ is the energy function, and
$\beta = T^{-1}$ the inverse temperature.

\section{Transfer matrices of lattice models}

\subsection{1D Ising model}

\todo[inline]{refer to Ising, talk a bit about model (magnetism etc).}

Consider the 1D zero-field ferromagnetic Ising model, defined by the energy function
\begin{equation}\label{ising_energy_function}
  H(\sigma) = -J \sum_{\langle i j \rangle} \sigma_i \sigma_j
\end{equation}
Here, we sum over nearest neighbors $\langle i j \rangle$ and the spins
$\sigma_i$ take the values $\pm 1$. $J > 0$.

Assume, for the moment, that the chain
consists of $N$ spins, and apply periodic boundary conditions.
The partition function of this system is given by
\begin{equation}
  Z_{N} = \sum_{\sigma_1, \dotsc, \sigma_N \in \{-1, 1\}} \exp (-\beta H(\sigma))
\end{equation}
Exploiting the local nature of the interaction between spins, we can write
\begin{equation}
  Z_{N} = \sum_{\sigma_1, \cdots, \sigma_N \in \{-1, 1\}} \prod_{\langle i, j \rangle} e^{K\sigma_i \sigma_j}
\end{equation}
where we defined $K \equiv \beta J$.

Now, we can define
the $2 \times 2$ matrix
\begin{equation}\label{eq:transfer_matrix_1d_ising}
  T_{\sigma \sigma'} = \exp(K \sigma \sigma')
\end{equation}
to obtain
\begin{equation}
  Z_N = \sum_{\sigma_1, \cdots, \sigma_N} T_{\sigma_1 \sigma_2} \dotsm T_{\sigma_N \sigma_1} = \tr T^N
\end{equation}

$T$ is called the transfer matrix. Since $T$ is, in fact, diagonalizable, $T^N = P D^N
P^{-1}$, where $P$ consists of the eigenvectors of $T$. By the cyclic property of the
trace, we have
\begin{equation}
  Z_N = \lambda_{1}^{N} + \lambda_{2}^{N}
\end{equation}

Thus, we have reduced the problem of finding the partition function to an
eigenvalue problem, which is quite easy in this case.

Note that in the thermodynamic limit $N \to \infty$
\begin{equation}
  Z = \lim_{N \to \infty} \lambda_{1}^{N}
\end{equation}
where $\lambda_1$ is the non-degenerate largest eigenvalue (in absolute value) of $T$.

\subsection{2D Ising model}
\todo[inline]{Talk about exact solution (Onsager). Why is it important? Maybe
star-triangle relation (Baxter). Not all IRF models solvable.}

Next, we treat the two-dimensional, square-lattice Ising model. In two
dimensions, the energy function is still written as in
\eqref{ising_energy_function}, but now every lattice site has four neighbors.

Let $N$ be the number of columns and $l$ be the number of rows of the lattice, and assume
$l \gg N$. In the vertical direction, we apply periodic boundary conditions, as in the
one-dimensional case. In the horizontal direction, we keep an open boundary. We refer to
$N$ as the system size.

Similarly to the 1D case, the partition function can be written as
\begin{equation}
  Z_N = \sum_{\bm{\sigma}} \prod_{\langle i, j, k, l \rangle} W(\sigma_i, \sigma_j, \sigma_k, \sigma_l)
\end{equation}
where the product runs over all groups of four spins sharing the same face. The Boltzmann weight of such a face is given by
\begin{equation}\label{eq:boltzmann_weight_face_ising_model}
  W(\sigma_i, \sigma_j, \sigma_k, \sigma_l) = \exp \left\{ \frac{K}{2} (\sigma_i \sigma_j + \sigma_j \sigma_k + \sigma_k \sigma_l + \sigma_l \sigma_i) \right\}
\end{equation}

We can express the Boltzmann weight of a configuration of the whole lattice as
a product of the Boltzmann weights of the rows
\begin{equation}
  Z_N = \sum_{\bm{\sigma}} \prod_{r = 1}^{l} W(\sigma_{1}^{r}, \sigma_{2}^{r}, \sigma_{1}^{r+1}, \sigma_{2}^{r+1}) \dots W(\sigma_{N-1}^{r}, \sigma_{N}^{r}, \sigma_{N-1}^{r+1}, \sigma_{N}^{r+1})
\end{equation}
where $\sigma_{i}^{r}$ denotes the value of the $i$th spin of row $r$.

Now, we can generalize the definition of the transfer matrix to two dimensions, by
defining it as the Boltzmann weight of an entire row
\begin{equation}\label{eq:row_to_row_transfer_matrix}
  T_{N}(\bm{\sigma}, \bm{\sigma'}) = W(\sigma_1, \sigma_2, \sigma_1', \sigma_2') \dots W(\sigma_{N-1}, \sigma_N, \sigma_{N-1}', \sigma_{N}')
\end{equation}
The dimensions of $T_N$ are $2^N \times 2^N$.

Similarly as in the one-dimensional case, the partition function now becomes
\begin{equation}\label{z_n_times_infty}
  Z_N = \sum_{\bm{\sigma}} \prod_{r = 1}^{l} T_{N}(\bm{\sigma}^r, \bm{\sigma}^{r+1}) = \tr T_{N}^l
\end{equation}

In the limit of an $N \times \infty$ cylinder, the partition function is once again
determined by the largest eigenvalue\footnote{As in the 1D case, $T$ is symmetric, so it
orthogonally diagonalizable.}.
\begin{equation}\label{largest_eigenvalue_transfer_matrix}
  Z_N = \lim_{l \to \infty} T_{N}^{l} = \lim_{l \to \infty} (\lambda_0)_{N}^{l}
\end{equation}

The partition function in the thermodynamic limit is given by
\begin{equation}
  Z = \lim_{N \to \infty} Z_N
\end{equation}

\section{Partition function of the 2D Ising model as a tensor network}
In calculating the partition function of 1D and 2D lattices, matrices of Boltzmann weights
like $W$ and $T$ play a crucial role. We have formulated them in a way that is valid for
any interaction-round-a-face (IRF) model, defined by
\begin{equation}
  H \thicksim \sum_{\langle i, j, k, l \rangle} W(\sigma_i, \sigma_j, \sigma_k,
  \sigma_l)
\end{equation}
where the summation is over all spins sharing a face. $W$ can contain 4-spin,
3-spin, 2-spin and 1-spin interaction terms. The Ising model is a special case of the IRF
model, with $W$ given by \autoref{eq:boltzmann_weight_face_ising_model}.

We will now express the partition function of the 2D Ising model as a tensor network. The
transfer matrix $T$ is redefined in the process. This allows us to visualize the equations
in a way that is consistent with the many other tensor network algorithms under research
today.

\todo[inline]{Help! I don't know what the formal connection between the
  transfer matrix and the tensor network representation is! Yes you do! For any
  $l$, the trace over the transfer matrix is the same in both definitions of
  $T$. It is symmetric, so they should be related by a basis transformation. Is
  it, perhaps, exactly the basis in which the CTM is diagonal? }

\subsection{A system of four spins}

We define
\begin{equation}
  Q(\sigma_i, \sigma_j) = \exp(K \sigma_i \sigma_j)
\end{equation}
as the Boltzmann weight of the bond between $\sigma_i$ and $\sigma_j$. It is the
same as the 1D transfer matrix in \autoref{eq:transfer_matrix_1d_ising}.

The Boltzmann weight of a face $W$ decomposes into a product of Boltzmann weights of
bonds
\begin{equation}
  W(\sigma_i, \sigma_j, \sigma_k, \sigma_l) =
  Q(\sigma_i, \sigma_j)Q(\sigma_j, \sigma_l)Q(\sigma_l, \sigma_k)Q(\sigma_k, \sigma_i)
\end{equation}

It is now easy to see that the partition function is equal to the contracted tensor
network in \autoref{fig:tensor_network_4_sites}
\begin{equation}
  \begin{split}
    Z_{2 \times 2} & =
    \sum_{\sigma_1, \sigma_2, \sigma_3, \sigma_4} \sum_{a, b, c, d}
    \delta_{\sigma_1, a} Q(a, b) \delta_{\sigma_2, b} Q(b, c)
    \delta_{\sigma_3, c} Q(c, d) \delta_{\sigma_4, d} Q(d, a) \\
    & =
    \sum_{\sigma_1, \sigma_2, \sigma_3, \sigma_4} W(\sigma_1, \sigma_2, \sigma_3, \sigma_4)
  \end{split}
\end{equation}
where the Kronecker delta is defined as usual:
\begin{equation}
  \delta_{i j} =
  \begin{cases}
    1 & \text{if } i = j \\
    0 & \text{if } i \neq j
  \end{cases}
\end{equation}

\begin{figure}
  \caption{Hallo hier staat nog niks}
  \label{fig:tensor_network_4_sites}
\end{figure}


\subsection{Thermodynamic limit}

We define the matrix $P$ by
\begin{equation}
  P^2 = Q
\end{equation}
and can now write the partition function of an arbitrary square lattice as a tensor
network of a single recurrent tensor $a_{i j k l}$, given by
\begin{equation}
  a_{i j k l} = \sum_{a, b, c, d} \delta_{a b c d} P_{i a} P_{j b} P_{k c} P_{l d}
\end{equation}
where the generalization of the Kronecker delta is defined as
\begin{equation}
  \delta_{i_1 \dots i_n} =
  \begin{cases}
    1 & \text{if } i_1 = \ldots = i_n \\
    0 & \text{otherwise}
  \end{cases}
\end{equation}

See \autoref{fig:2d_classical_partition_function_as_tensor_network}. At the edges and
corners, we have suitable 2 and 3 legged tensors, which we will also denote by $a$.
\begin{align*}
  a_{i j k} &= \sum_{a b c} \delta_{a b c} P_{i a} P_{j b} P_{k c} \\
  a_{i j} &= \sum_{a b} \delta_{a b} P_{i a} P_{j b}
\end{align*}

The challenge is to approximate this tensor network in the thermodynamic limit.


\subsection{The transfer matrix as a tensor network}
\todo[inline]{Say something about reshaping legs.}

We can now redefine the row-to-row transfer matrix from
\autoref{eq:row_to_row_transfer_matrix} as the tensor network expressed in
\autoref{fig:transfer_matrix_as_tensor_network}. For all $l$, it is still true that
\begin{equation}
  Z_{N \times l} = \tr T_{N}^{l} = \sum_{i = 1}^{2^N} \lambda_{i}^{l}
\end{equation}
so the eigenvalues must be the same. That means that the new definition of the transfer
matrix is related to the old one by a basis transformation
\begin{equation}
  T_{\text{new}} = P T_{\text{old}} P^{-1}
\end{equation}



\begin{figure}
  \caption{lalala nog niks}
  \label{fig:transfer_matrix_as_tensor_network}
\end{figure}



\section{Transfer matrix renormalization group}
\todo[inline]{More about connection between 1D quantum and 2D classical}
\todo[inline]{Pictures}

Nishino \cite{nishino1995density} was the first one to apply DMRG in its modern
form to approximate the transfer matrix of a two-dimensional classical lattice system.

In the limit of an $N \times \infty$ lattice, $T_{N}^{l}$ becomes proportional
to the projector onto the eigenspace of the largest eigenvalue

\begin{equation}
  \lim_{l \to \infty} T_{N}^{l} = \lim_{l \to \infty} (\lambda_0)_{N}^{l} \ket{\lambda_0}_{N}\bra{\lambda_0}_{N} \thicksim \ket{\lambda_0}_{N}\bra{\lambda_0}_N
\end{equation}

Now, the connection to ground state DMRG is clear. Recall the full-system
density matrix of an $N$-site 1D quantum lattice system (the superblock) in the
ground state (cf. \eqref{eq:density_matrix_superblock})

\begin{equation}
  \rho_N = \ket{\Psi_0}_{N}\bra{\Psi_0}_{N}
\end{equation}


\todo[inline]{Write down equations that are analogous to reduced density matrix. Pictures}



\section{Corner transfer matrices}

The concept of corner transfer matrices for 2D lattices was first introduced by
Baxter \cite{baxter1968dimers, baxter1978variational, baxter1982exactly}.
Whereas the row-to-row transfer matrix \eqref{eq:row_to_row_transfer_matrix}
corresponds to adding a row to the lattice, the corner transfer matrix adds
a quadrant of spins. It is defined as

\begin{equation}
  A_{\bm{\sigma}, \bm{\sigma'}} =
  \begin{cases}
    \sum \prod_{\langle i, j, k, l \rangle} W(\sigma_i, \sigma_j, \sigma_k, \sigma_l) & \text{if } \sigma_{1} = \sigma_{1}' \\
    0 & \text{if } \sigma_{1} \neq \sigma_{1}'
  \end{cases}
\end{equation}

\todo[inline]{Picture.}
Here, the product runs over groups of four spins that share the same face, and
the sum is over all spins in the interior of the quadrant.
In a symmetric and isotropic model, we have

\begin{equation}
  W(a, b, c, d) = W(b, a, d, c) = W(c, a, d, b) = W(d, c, b, a)
\end{equation}

and the partition of an $N \times N$ lattice is expressed as

\begin{equation}\label{z_n_times_n}
  Z_{N \times N} = \tr A^4
\end{equation}

In the thermodynamic limit, \eqref{z_n_times_n} is equal to \eqref{z_n_times_infty}.



\subsection{Corner transfer matrix renormalization group}

\todo[inline]{More about Baxter's variational approach.}
Nishino and Okunishi combined ideas from Baxter and White to formulate the
corner transfer matrix renormalization group \cite{nishino1996corner}.

%
% \chapter{Critical behaviour and finite-size scaling}
% \section{Abstract}

In this chapter, we introduce the central concepts in critical phenomena
and finite-size scaling.

We follow the excellent review by Barber \cite{barber1983finite} and
chapter five of Cardy's book \cite{cardy1996scaling}.

\section{Introduction}

\section{Phase transitions}
When matter exhibits a sudden change in behaviour, characterized by a sudden change in one or more thermodynamic
quantities, we say it undergoes a \emph{phase transition}.
A quantity that signifies this change is called an \emph{order parameter},
that can take vastly different forms across systems and transitions.
For example, for the transition of a ferromagnet, the order parameter is the net magnetization of the system,
while for a percolation transition, it is the size of the largest connected graph.

For a historical account of the classification of phase transitions,
see \cite{jaeger1998ehrenfest}.
At the present time, we distinguish between two different types \cite{kadanoff2009more}.

When some thermodynamic quantity changes discontinuously, i.e.
shows a jump, we call the transition \emph{first order}.
In contrast, during a \emph{continuous} phase transition a variable undergoes change continuously.
The point at which a continuous phase transition occurs, is called the critical point.

The two-dimensional Ising model (\emph{ref here}) in a magnetic field shows both types of transition.
At zero magnetic field and $T = T_c = 1 / (\log(1 + \sqrt{2}))$, the magnetization changes from zero for $T > T_c$ to
a finite value for $T < T_c$ in a continuous manner.

Below the critical temperature $T_c$, when the magnetic field $h$ tends to zero from $h > 0$,
the magnetization tends to a positive value.
Conversely, when the magnetic field tends to zero from $h < 0$,
the magnetization tends to a negative value.
Thus, across the region $h = 0, T < T_c$ the system undergoes a first-order phase transition.

\subsection{Finite systems}

We will now argue that a phase transition cannot occur in a finite system,
but only happens when the number of particles tends to infinity.

Because thermodynamic quantities are averages over all possible
microstates of a system, those quantities are completely defined in terms
of the system's partition function, or equivalently its free energy.

Since in a finite system, the partition function is a finite sum of
exponentials, it is analytic (infinitely differentiable). Hence, thermodynamic
quantities cannot show true discontinuities and the phase transitions
described in the above section do not occur.

\todo[inline]{pictures?}

\section{Critical behaviour}
We will now focus our attention on continuous phase transitions, more specifically the one that occurs in the
two-dimensional Ising model.
Before we discuss the behaviour of the free energy around the critical point,
we briefly summarize how the thermodynamic limit is approached far away from it.
Here, we largely follow \cite{barber1983finite}.

We assume that the free energy per site in the thermodynamic limit
\begin{equation}
  f_{\infty}(T) = \lim_{N \to \infty} \frac{F(T, N)}{N}
\end{equation}
exists, and is not dependent on boundary conditions.
By definition, it is not analytic in a region around the critical point.

Outside that region, however, we can write
\begin{equation}\label{eq:free_energy_finite_system_outside_critical_region}
  F(T, N) = N f_{\infty}(T) + o(N),
\end{equation}
where corrections $f(N)$ of $o(N)$ (little-o of $N$) mean that
\begin{equation}
  \lim_{N \to \infty} \frac{f(N)}{N} = 0.
\end{equation}
These corrections, of course, do depend on boundary conditions.

\autoref{eq:free_energy_finite_system_outside_critical_region} is valid only outside the critical region precisely
because $F(T, N)$ is analytic \emph{everywhere}, and $f_{\infty}(T)$ is only analytic away from the critical point.

The behaviour of $F(T, N)$ (and hence, all thermodynamic quantities) at criticality is approached is described by
\emph{finite-size scaling}.

\subsection{Finite-size scaling}

\autoref{fig:order_parameter_finite_N_exact} shows the behaviour of the order parameter obtained by exact
diagonalization of the partition function of small lattices.
It is clear that far from the critical point, the order parameter is essentially not dependent on system size,
while in critical region there are significant deviations from the thermodynamic behaviour.

One can now define two characteristic temperatures \cite{fisher1967interfacial,
barber1983finite}.
The first being the cross-over temperature $T_X$ at which finite-size effects become important,
which is predicted to scale as
\begin{equation}
  |T_X - T_c| \propto N^{-\theta}.
\end{equation}
$\theta$ is called the cross-over or rounding exponent.

The second characteristic temperature is the pseudocritical temperature, denoted by $T^{\star}$.
It can be defined in several ways, one being the point where the order parameter becomes almost zero (in which case it is the same as the cross-over temperature), or the point where the heat capacity
\begin{equation}
  C = T^2 \frac{\partial^2 F}{\partial T^2}
\end{equation}
reaches its maximum.
$T^{\star}$ can be regarded as the point where the finite system in some sense comes closest to undergoing a transition.
Generally $T^{\star}$ will not equal $T_X$, if only because $T^{\star}$ depends on boundary conditions:
periodic or fixed boundary conditions will nudge the system into an ordered state,
therefore $T^{\star} > T_c$.
Free boundary conditions will cause the system to favor disorder and the pseudocritical temperature to be lowered.

In any case, it is predicted that
\begin{equation}
  |T^{\star} - T_c| \propto N^{-\lambda}.
\end{equation}

It is generally accepted that \cite{barber1983finite}
\begin{equation}
  \lambda = \theta.
\end{equation}

Furthermore, if one assumes that finite-size effects become important once the correlation length of the system becomes of order of the system size, i.e. \cite{fisher1967interfacial}
\begin{equation}
  \xi(T_X(N)) \propto N,
\end{equation}
then the correlation length exponent $\nu$, given by
\begin{equation}
  \xi(T) \propto |T - T_c|^{-\nu}
\end{equation}
can be related to $\theta$ as
\begin{equation}
  \theta = \frac{1}{\nu}.
\end{equation}


\begin{figure}
  \includegraphics[]{order_parameter_finite_N_exact.tikz}
  \caption{The magnetization of the central spin for small lattices with boundary spins fixed to
  $+1$. The black line is the exact solution in the thermodynamic limit.}\label{fig:order_parameter_finite_N_exact}
\end{figure}




\section{Effective length scale related to the bond dimension $m$}

To cite

\begin{itemize}
  \item \cite{fisher1972scaling} original physical derivation of finite-size scaling
  \item \cite{fisher1967interfacial} rounding and displacement exponents
  \item \cite{kadanoff2009more} Basic (philosophical) introduction to phase transitions
  \item \cite{jaeger1998ehrenfest} Historical account of classification of phase transitions
\end{itemize}

Todo
\begin{itemize}
  \item finite-size scaling ansatz, basic scaling laws, data collapse, etc.
  \item renormalization group derivation? (in appendix?)
\end{itemize}

%
% \chapter{Finite-$m$ scaling in the CTMRG algorithm}
% \chapterprecishere{abstract bla bla}

Up until now, we have developed our scaling analysis in terms of a finite system size $N$.
But the approximation of the infinite-system partition function with the CTMRG algorithm depends on two parameters;
the system size $N$ and the bond dimension $m$.

A finite bond dimension $m$ carries a characteristic length scale.
Baxter \cite{baxter1978variational}, and later Östlund and Rommer \cite{ostlund1995thermodynamic} (in the context of
one-dimensional quantum systems) showed that in the thermodynamic limit,
CTMRG and DMRG are variational optimizations in the space of matrix product states.
\todo[inline]{Can extend this idea a bit.}

It is known that an MPS-ansatz with finite bond dimension inherently limits the
correlation length of the system to a finite value \cite{wolf2006quantum, rommer1997class}. Hence,
thermodynamic quantities obtained from the CTMRG algorithm with finite $m$, in the limit
$N \to \infty$, cannot diverge and must show finite-size effects similar to those of some
effective finite system of size $N_{\text{eff}}(m)$ depending on the bond dimension $m$.

\autoref{fig:order_parameter_vs_T} shows the behaviour of the order parameter of the
two-dimensional Ising model for systems of finite-size,
where the result is converged in $m$, and for systems of finite $m$, where
the result is converged in the system size $N$. The results look very similar and support
the claim that there are two relevant length scales in the critical region, namely the system size $N$ and
the length scale associated to the finite bond dimension $m$.

\begin{figure}
\includegraphics[]{order_parameter_vs_T.tikz}
\caption{Upper panel: expectation value of the central spin $\langle \sigma_0 \rangle$
  after $n$ CTMRG steps. $m$ is chosen such that the truncation error is smaller than
  $10^{-6}$. Lower panel: $\langle \sigma_0 \rangle$ for systems with bond dimension $m$.}\label{fig:order_parameter_vs_T}
\end{figure}

\section{Definition of the effective length scale in terms of the correlation length at $T_c$}\label{sec:definition_effective_length_scale_in_terms_of_xi}

The first direct comparison of finite-size scaling in the system size $N$ with scaling in
the bond dimension of the CTMRG method $m$ was done
in \cite{nishino1996numerical}.

In the thermodynamic limit (corresponding to infinite $m$ and $N$), we have the following
expression for the correlation length of a classical system
\cite{baxter1982exactly_correlation_length}
\begin{equation}\label{eq:correlation_length_row_to_row_transfer_matrix}
  \xi(T) = \frac{1}{\log\left(\frac{T_0}{T_1}\right)}.
\end{equation}
Here, $T_0$ and $T_1$ are the largest and second-largest eigenvalues of the row-to-row
transfer matrix $T$, respectively. With $N$ tending towards infinity and finite $m$, near
the critical point $\xi(T)$ should obey a scaling law of the form
\begin{equation}
  \xi(T, m) = N_{\text{eff}}(m) \mathcal{F}(N_{\text{eff}}(m) / \xi(T))
\end{equation}
with
\begin{equation}
  \mathcal{F}(x) = \begin{cases}
      \text{const} & \text{if } x \to 0, \\
      x^{-1} & \text{if } x \to \infty.
    \end{cases}
\end{equation}

Hence, the effective length scale corresponding to the finite bond dimension $m$ is
proportional to the correlation length of the system at the critical point $t = 0$.
\begin{equation}
  N_{\text{eff}}(m) \propto \xi(T = T_c, m).
\end{equation}

\todo[inline]{Look ahead to replicating this in results section?}

\section{Relation to finite-entropy scaling and the exponent $\kappa$.}


The first numerical evidence of a law for the correlation length at the critical point of the form
\begin{equation}\label{eq:xi_propto_m_kappa}
  \xi(m) \propto m^{\kappa}
\end{equation}
was given by the authors of \cite{andersson1999density}, who found
\begin{equation}
  \kappa \approx 1.3
\end{equation}
for a gapless system of free fermions, using DMRG calculations. Later, using the iTEBD algorithm
\cite{vidal2007classical}, the authors of \cite{tagliacozzo2008scaling} presented numerical evidence for such a relation
for the Ising model with transverse field and the Heisenberg model, with
\begin{align}
  \kappa_{\text{Ising}} & \approx 2, \\
  \kappa_{\text{Heisenberg}} & \approx 1.37.
\end{align}

\subsection{Quantitative theory for $\kappa$}
A quantitative theory of the existence of an exponent $\kappa$ was given in \cite{pollmann2009theory}.
We reproduce the argument, which is presented in the language of one-dimensional quantum systems, below.

We start by noting that in the critical region, the entanglement of a half-infinite subsystem $A$ diverges as
\begin{equation}\label{eq:entropy_scaling_near_criticality}
  S_A \propto \mathcal{A}(c/6)\log(\xi),
\end{equation}
where $\mathcal{A}$ is the number of boundary points of $A$ and $c$ is the central charge of the conformal field theory
at the critical point \cite{calabrese2004entanglement, vidal2003entanglement, ercolessi2010exact}.

Recalling the definition of the entanglement entropy
\begin{equation}
  S_A = - \tr(\rho_A \log \rho_A) = - \sum_{\alpha} \omega_{\alpha} \log \omega_{\alpha},
\end{equation}
it is trivially seen that the entropy of a state given by the DMRG (or any other MPS), which only
retains $m$ basis states of $\rho_A$, is limited by
\begin{equation}
  S^{\text{max}}_A(m) = \log m
\end{equation}
by putting $\omega_{\alpha} = 1/m$ for $\alpha = 1, \dots, m$.

This is, incidentally, another way to see that DMRG or CTMRG, or any other algorithm which produces ground states with a
matrix-product structure have an inherently finite correlation length.

The leading energy correction to the free energy per site of a one-dimensional quantum system at a conformally invariant
critical point at finite temperature $T$ in the thermodynamic limit is \cite{affleck1986universal}
\begin{equation}\label{eq:correction_free_energy_critical_point_finite_temperature}
  f(T) = f_0 + aT^2 + \mathcal{O}(T^3).
\end{equation}

Due to the quantum-classical correspondence, this is equivalent to a two-dimensional classical $N \times \infty$ lattice
with strip width $N = 1/T$.
This implies also that the correlation length of a critical one-dimensional quantum system at finite temperature cannot
diverge and goes as $\xi \propto 1/T$.
In terms of this finite correlation length, \autoref{eq:correction_free_energy_critical_point_finite_temperature} is
written as
\begin{equation}\label{eq:correction_free_energy_critical_point_finite_correlation_length}
  f(\xi) = f_{\infty} + \frac{A}{\xi^2} + \mathcal{O(\frac{1}{\xi^3})}.
\end{equation}

Empirically, optimized ground states with a matrix-product structure at criticality do not simply maximize their
entropy, as they should if we take \autoref{eq:correction_free_energy_critical_point_finite_correlation_length} to be
true for ground states with a matrix-product structure.

We will now show that \autoref{eq:correction_free_energy_critical_point_finite_correlation_length} needs,
in fact, an additional term due to the matrix-product structure with finite bond dimension $m$.

The ground state with finite correlation length and energy density as in
\autoref{eq:correction_free_energy_critical_point_finite_correlation_length} has a Schmidt decomposition that in
principle can have infinitely many terms
\begin{equation}\label{eq:ground_state_infinite_schmidt_decomposition}
  \ket{\psi_0} = \sum_{n = 1}^{\infty} \lambda_n \ket{\psi_{n}^{L}}\ket{\psi_{n}^{R}},
\end{equation}
where $\ket{\psi_{n}^{L}}$ and $\ket{\psi_{n}^{R}}$ are states of the left and right infinite half-chains. Normalization
requires
\begin{equation}
  \sum_{n}^{\infty} \lambda_{n}^2 = 1.
\end{equation}

The ground state with a matrix-product structure with finite bond dimension $m$ has an additional constraint:
its Schmidt decomposition carries only the $m$ $\ket{\psi_n}$ with largest $\lambda_n$.
It is written as
\begin{equation}
  \ket{\psi_{0}^{\text{MPS}}} = \frac{\sum_{n = 1}^{m} \lambda_n
  \ket{\psi_{n}^{L}}\ket{\psi_{n}^{R}}}{\sqrt{\sum_{n=1}^{m} \lambda_{n}^2}}.
\end{equation}

To find the extra energy cost of only keeping the first $m$ terms in the Schmidt decomposition,
note that in the limit of $m$ large, $\ket{\psi_{0}^{\text{MPS}}}$ almost entirely overlaps with $\ket{\psi_0}$,
hence can be written as
\begin{equation}
  \ket{\psi_{0}^{\text{MPS}}} = \sqrt{1 - \epsilon^2} \ket{\psi_0} + \epsilon \ket{\psi_{\text{ex}}},
\end{equation}
where $\ket{\psi_{\text{ex}}}$ is some excited state and $\epsilon \ll 1$. This leads to an energy of
\begin{equation}
  E_{0}^{\text{MPS}} = \braket{\psi_{0}^{\text{MPS}} | \hat{H} | \psi_{0}^{\text{MPS}}} = E_0 + \epsilon^2 (E_{\text{ex}} - E_0),
\end{equation}
with
\begin{equation}
  \epsilon^2 = \left(1 - \braket{\psi_0 | \psi_{0}^{\text{MPS}}}^2 \right) = 1 - \sum_{n = 1}^{m} \lambda_{n}^2 \equiv
  P_{\text{res}}(m).
\end{equation}
Here, we have defined the residual probability $P_{\text{res}}$, also known as the truncation error,
as the part of the spectrum that is thrown away.

If we now assume that $E_0 - E_{\text{ex}}$ is proportional to the energy gap $\Delta$, which scales as \cite{lieb1961two, mata1989energy, pfeuty1970one}
\begin{equation}
  \Delta \propto \frac{1}{\xi},
\end{equation}
we arrive at
\begin{equation}\label{eq:correction_energy_mps_ground_state}
  E_{0}^{\text{MPS}} = E_{\infty} + \frac{A}{\xi^2} + \frac{B P_{\text{res}}(m)}{\xi}.
\end{equation}

It is clear that when the correlation length is very large, by \autoref{eq:entropy_scaling_near_criticality} the entropy
and $P_{\text{res}}(m)$ must be too.
So, the third term in \autoref{eq:correction_energy_mps_ground_state} dominates.

If the correlation length is small, the second term dominates.
The correlation length that belongs to the MPS ground state with fixed $m$ is the optimum that minimizes this
expression.

The details of the calculation, which can be found in the supplementary material of \cite{pollmann2009theory},
depend on the asymptotic form of $P_{\text{res}}$, found in \cite{calabrese2008entanglement}. In the limit $m \to \infty$, the correlation is indeed of the form in \autoref{eq:xi_propto_m_kappa} with
\begin{equation}\label{eq:exact_value_kappa}
  \kappa = \frac{6}{c \left( \sqrt{12/c} + 1 \right) },
\end{equation}
which is in good agreement with the values found in \cite{tagliacozzo2008scaling}.

\todo[inline]{Refer back to chapter on spectrum of CTM}

\section{Locating the critical point with the entanglement spectrum}\label{sec:locating_critical_point_entanglement}
Since phase transitions of quantum systems can be located by studying their entanglement spectrum
\cite{huang2017holographic, osborne2002entanglement}, classical systems may be investigated in the same way through the
correspondence in \autoref{eq:correspondence_density_matrix_ctm}.
This is an alternative to the usual approach of studying an order parameter or derivatives of thermodynamical
observables.

Examples of studies using the spectrum of the corner transfer matrix to analyze two-dimensional classical systems are
\cite{krvcmar2015reentrant, PhysRevE.94.022134, krvcmar2016phase}.

At the critical point, the entropy must diverge (cf.
\autoref{eq:entropy_scaling_near_criticality}).
For finite systems the entropy will remain finite, but the pseudocritical temperature $T^{\star}$ is defined as the
point of maximum entropy.
The critical point is then located by fitting the scaling law in \autoref{eq:scaling_law_T_star}.

% %
% \chapter{Methods}
% \section{Abstract}

We describe the technical details of the algorithms used to compute quantities of interest.
We report the convergence behaviour of the algorithms and discuss validity and sources of error.

\section{Technical details}
For the models treated in this thesis, the corner transfer matrix $A$ and the row-to-row transfer matrix $T$ are
symmetric. But due to the accumulation of machine-precision sized errors in the matrix multiplication and singular value
decomposition, this will, after many algorithm steps, no longer be the case. In order for results to remain valid, we
manually enforce symmetricity after each step.

The tensor network contractions at each algorithm step will cause the elements of $A$ and $T$ to tend to infinity, which
means that they will at some point exceed the maximum value of a floating point number as it can be stored in memory.
But because the elements of $A$ and $T$ represent Boltzmann weights, they can be scaled by a constant factor, which
allows us to prevent this overflow if we use a suitable scaling. For example by requiring that
\begin{equation}
  \tr A^4 = 1,
\end{equation}
so that the interpretation of $A^4$ as a reduced density matrix of an effective one-dimensional quantum is valid.

\section{Convergence criteria}

\subsection{Simulations with finite bond dimension}
The convergence of the CTMRG algorithm with fixed bond dimension $m$ (the infinite system algorithm) can be defined
in multiple ways (\emph{cite}). In this thesis, the convergence after step $i$ of the algorithm is defined as
\begin{equation}
  c_i = \sum_{\alpha = 1}^{m} | s_{\alpha}^{(i)} - s_{\alpha}^{(i - 1)} |,
\end{equation}
where $s_{\alpha}$ are the singular values of the corner transfer matrix $A$. If the convergence falls below some
threshold $\epsilon$, the algorithm terminates.

The assumption is that once the singular values stop changing to some precision, the optimal projection is sufficiently
close to its fixed point and the transfer matrices $A$ and $T$ represent an environment only limited by the length scale
given by $m$, i.e.
\begin{equation}
  \xi(m) \ll N
\end{equation}
is satisfied.

\begin{figure}
  \includegraphics[]{convergence_finite_chi.tikz}
  \caption{hallootjes}\label{fig:convergence_finite_chi}
\end{figure}

\subsubsection{Convergence of quantities at the critical point}
The convergence of the various quantities at the critical point of the Ising model is shown in
\autoref{fig:convergence_finite_chi}.

\todo[inline]{Cross check with correlation length, report on boundary conditions}



\subsection{Simulations with finite system size}
In the finite-system algorithm, we require
\begin{equation}
  N \ll \xi(m),
\end{equation}
so the question becomes at which $m$ the results are sufficiently converged in $m$. Throughout this work, we have used
the residual probability (also called truncation error)
\begin{equation}
  P(m)^{(i)} = \frac{\sum_{\alpha = m + 1}^{dm} (s_{\alpha}^{(i)})^2 }{ \sum_{\alpha = 1}^{dm} (s_{\alpha}^{(i)})^2 },
\end{equation}
which quantifies the fraction of the spectrum of the corner transfer matrix that is thrown away, as a measure of how
accurate the transfer matrices represent the finite system of size $N$. Here, $d$ is the
dimension of the local tensors ($d = 2$ for the Ising model).

If, for a given $m$, we have
\begin{equation}
  P(m) < P_{\text{max}}
\end{equation}
we deem the result accurate enough.

In the limit $m \to d^n$, with
\begin{equation}
  n = \frac{N - 1}{2}
\end{equation}
the number of algorithm steps, we obtain the exact result for the transfer matrices and hence the partition function,
i.e. $P(m) \to 0$.

To justify that for small enough $P_{\max}$, we obtain good results for a wide range of $N$, figure bla bla.

% % %
% \chapter{Numerical results for the Ising model}
% \chapterprecishere{abstract}

\section{At the critical point}

\subsection{Existence of two length scales}

First, we reproduce the results presented in \cite{nishino1996numerical} to validate the assumption that at the critical
point, the only relevant length scales are the system size $N$ and the length scale associated to a finite dimension $m$
of the corner transfer matrix $\xi(m)$.
Here, we assume that $\xi(m)$ is given by the correlation length at the critical point,
see \autoref{sec:definition_effective_length_scale_in_terms_of_xi}.

The order parameter should obey the following scaling relation at
the critical temperature
\begin{equation}\label{eq:order_param_scaling_relation_finite_m}
  M(T = T_c, m) \propto \xi(T = T_c, m)^{-\beta/\nu}.
\end{equation}
The left panel of \autoref{fig:order_parameter_power_law_fit} shows that this scaling
relation holds. The fit yields $\frac{\beta}{\nu} \approx 0.125(5)$, close to the true
value of $\frac{1}{8}$.

The right panel shows the conventional finite-size scaling relation
\begin{equation}\label{eq:order_param_scaling_relation_finite_N}
  M(T = T_c, N) \propto N^{-\beta/\nu},
\end{equation}
yielding $\beta/\nu \approx 0.1249(1)$.

The correlation length $\xi(m)$ shows characteristic half-moon patterns on a
log-log scale, stemming from the degeneracies in the corner transfer matrix spectrum. This
makes the data harder to interpret, since the effect of increasing $m$ depends on how much
of the spectrum is currently retained.

\todo[inline]{Talk about how to alleviate this partially by using entropy $S$ as
length scale.}

\begin{figure}
  % \includegraphics[width=\textwidth, axisratio=1]{order_parameter_power_law_fit.tikz}
  \includegraphics[]{order_parameter_power_law_fit.tikz}
  \caption{Left panel: fit to the relation in
  \autoref{eq:order_param_scaling_relation_finite_m}, yielding $\frac{\beta}{\nu} \approx
  0.125(5)$. The data points are obtained from simulations with $m = 2, 4, \dots, 64$. The
  smallest 10 values of $m$ have not been used for fitting, to diminish correction terms
  to the basic scaling law. Right panel: fit to conventional finite-size scaling law
  given in \autoref{eq:order_param_scaling_relation_finite_N}.
  }
  \label{fig:order_parameter_power_law_fit}
\end{figure}

To further test the hypothesis that $N$ and $\xi(m)$ are the only relevant length scales,
the authors of \cite{nishino1996numerical} propose a scaling relation for the order
parameter $M$ at the critical temperature of the form
\begin{equation}\label{eq:order_param_scaling_relation}
  M(N, m) = N^{-\beta/\nu} \mathcal{G}(\xi(m) / N)
\end{equation}
with
\begin{equation}
  \mathcal{G}(x) =
  \begin{cases}
    \text{const} & \text{if } x \to \infty, \\
    x^{-\beta/\nu} & \text{if } x \to 0,
  \end{cases}
\end{equation}
meaning that \autoref{eq:order_param_scaling_relation} reduces to
\autoref{eq:order_param_scaling_relation_finite_N} in the limit $\xi(m) \gg N$ and to
\autoref{eq:order_param_scaling_relation_finite_m} in the limit $N \gg \xi(m)$.
\autoref{fig:data_collapse_nishino} shows that the scaling relation of \autoref{eq:order_param_scaling_relation}
is justified.

\autoref{fig:crossover_nishino} shows the cross-over behaviour from the $N$-limiting regime, where
$M(N, m) \propto N^{-\beta/\nu}$ to the $\xi(m)$-limiting regime, where $M(N, m)$ does not depend on $N$.

\begin{figure}
  \includegraphics[]{data_collapse_nishino.tikz}
  \caption{Scaling function $\mathcal{G}(\xi(m)/N)$ given in
  \autoref{eq:order_param_scaling_relation}.}\label{fig:data_collapse_nishino}
\end{figure}

\begin{figure}
  \includegraphics[]{crossover_nishino.tikz}
  \caption{Behaviour of the order parameter at fixed $m$ as function of
  the number of renormalization steps $n$. For small $n$, all curves coincide, since the system size is the only
  limiting length scale. For large enough $n$, the order parameter is only limited by the length scale
  $\xi(m)$. In between, there is a cross-over described by $\mathcal{G}(\xi(m)/N)$, given in
  \autoref{eq:order_param_scaling_relation}.}\label{fig:crossover_nishino}
\end{figure}

\subsection{Central charge}
We may directly verify the value of the central charge $c$ associated with the conformal field theory at the critical
point by fitting to
\begin{equation}\label{eq:entropy_vs_correlation_length}
  S_{\text{classical}} \propto \frac{c}{6} \log \xi(m),
\end{equation}
which yields $c = 0.501$, shown in the left panel of \autoref{fig:entropy_vs_correlation_length}.

The right panel of \autoref{fig:entropy_vs_correlation_length} shows the fit to the scaling relation in $N$ (or,
equivalently the number of CTMRG steps $n$)
\begin{equation}\label{eq:entropy_vs_N}
  S_{\text{classical}} \propto \frac{c}{6} \log N,
\end{equation}
which yields $c = 0.498$.

\begin{figure}
  \includegraphics[]{entropy_vs_correlation_length.tikz}
  \caption{Left panel: numerical fit to \autoref{eq:entropy_vs_correlation_length}, yielding $c = 0.501$. Right panel:
  numerical fit to \autoref{eq:entropy_vs_N}, yielding $c = 0.498$. }\label{fig:entropy_vs_correlation_length}
\end{figure}

\subsection{Using the entropy to define the correlation length}\label{sec:entropy_to_define_correlation_length}
Via \autoref{eq:entropy_scaling_near_criticality}, the correlation length is expressed as
\begin{equation}\label{eq:correlation_length_as_function_of_entropy}
  \xi \propto \exp(\frac{6}{c}S).
\end{equation}

\autoref{fig:order_parameter_power_law_fit_entropy} shows the results of fitting the relation in
\autoref{eq:order_param_scaling_relation_finite_m} with this definition of the correlation length. The fit is an order
of magnitude better in the least-squares sense, and the half-moon shapes have almost disappeared,
yielding a much more robust exponent of $\beta/\nu = 0.12498$.

The entropy uses all eigenvalues of the corner transfer matrix, making it apparently less prone to structure in the
spectrum than the correlation length as defined in \autoref{eq:correlation_length_row_to_row_transfer_matrix},
which uses only two eigenvalues of the row-to-row transfer matrix.
Furthermore, the corner transfer matrix $A$ is kept diagonal in the CTMRG algorithm,
so $S$ is much cheaper to compute than $\xi$.

\begin{figure}
  \includegraphics[]{order_parameter_power_law_fit_entropy.tikz}
  \caption{}\label{fig:order_parameter_power_law_fit_entropy}
\end{figure}

\subsection{Exponent $\kappa$}

We now check the validity of the relation
\begin{equation}\label{eq:xi_propto_kappa_2}
  \xi(m) \propto m^{\kappa}
\end{equation}
in the context of the CTMRG method for two-dimensional
classical systems. Similar checks were done for one-dimensional quantum systems in \cite{tagliacozzo2008scaling}.

Let us first state that boundary conditions are relevant.
From (\emph{cite here!!}) we expect that for fixed boundary conditions,
the entropy and therefore the correlation length is lower for a given bond dimension $m$.

There are various ways of extracting the exponent $\kappa$.
\autoref{fig:support_for_kappa} shows the results for fixed boundary conditions and \autoref{fig:support_for_kappa_free}
for free boundary conditions.

Directly checking \autoref{eq:xi_propto_kappa_2} yields $\kappa = 1.93$ for a fixed boundary
and $\kappa = 1.96$ for a free boundary.

Under the assumption of \autoref{eq:xi_propto_kappa_2}, we have the following scaling laws at the critical point
\begin{align}\label{eq:scaling_laws_order_param_free_energy_kappa}
  M(m) & \propto m^{-\beta \kappa / \nu} \\
  f(m) - f_{\text{exact}} & \propto m^{(2-\alpha)\kappa / \nu}
\end{align}
for the order parameter and the singular part of the free energy, respectively.
With a fixed boundary, a fit to $M(m)$ yields $\kappa = 1.93$.
For a free boundary we cannot extract any exponent, since $M = 0$ for every temperature.
A fit to $f(m) - f_{\text{exact}}$ yields $\kappa = 1.90$ for a fixed boundary and $\kappa = 1.93$ for a free boundary.
\autoref{fig:support_for_kappa}. Here, we have used $\beta = 1/8$, $\nu = 1$ and $\alpha = 0$ for the Ising model.
\todo[inline]{Tell that the $\kappa$ law is indeed valid, since it is a good fit.}

We may use \autoref{eq:entropy_scaling_near_criticality} and \autoref{eq:classical_entropy} to check the
relation
\begin{equation}\label{eq:scaling_law_entropy_kappa}
  S_{\text{classical}} \propto \frac{c\kappa}{6}\log m,
\end{equation}
which yields $\kappa = 1.93$ for a fixed boundary and $\kappa = 1.96$ for a free boundary,
with $c = 1/2$ for the Ising model.

\begin{figure}
  \includegraphics[]{support_for_kappa.tikz}
  \caption{Numerical evidence for \autoref{eq:xi_propto_kappa_2}, \autoref{eq:scaling_laws_order_param_free_energy_kappa},
  \autoref{eq:scaling_law_entropy_kappa} with fixed boundary, yielding, from left to right and top to bottom, $\kappa = \{ 1.93, 1.93, 1.90,
  1.93 \}$.}\label{fig:support_for_kappa}
\end{figure}

\begin{figure}
  \includegraphics[]{support_for_kappa_free.tikz}
  \caption{Numerical evidence for \autoref{eq:xi_propto_kappa_2} with free boundary,
  yielding from left to right and then bottom $\kappa = \{ 1.96,
  1.93, 1.96 \}$.}\label{fig:support_for_kappa_free} \end{figure}

\subsubsection{Comparison with exact result in asymptotic limit}

The predicted value for $\kappa$ \cite{pollmann2009theory} is $2.034\dots$ (see also \autoref{eq:exact_value_kappa}).
With the CTMRG method, we extract the slightly lower value of $1.96$ (corresponding to free boundary conditions).
But, the structure in the quantities as function of $m$ makes it hard to get an accurate fit to $\kappa$.

It is interesting to note that for fixed boundary conditions, the relation in \autoref{eq:xi_propto_kappa_2} holds,
but with a lower exponent $\kappa$. This is to be expected, since half the spectrum is missing.
\todo[inline]{phrase this better. Maybe subsubsection?}
\todo[inline]{MUST SAY THAT I LEFT AWAY VALUES OF $m$ in symmetric case!!}

\section{Locating the critical point}\label{sec:locating_the_critical_point}

In general, the critical point is not known, but it may be located using the spectrum of the corner transfer matrix as
described in \autoref{sec:locating_critical_point_entanglement}. For a given value of $m$ (or $N$, in a finite-size approximation), the pseudocritical temperature is defined as the point of maximum entropy.

\autoref{fig:entropy_vs_T} shows the classical analogue to the entanglement entropy as a function of temperature for
different values of $m$.

The critical point is located by fitting the scaling law in \autoref{eq:scaling_law_T_star}.

\subsection{Finite $m$}
For approximations with finite bond dimension $m$, it is not clear what length scale should be used to fit the scaling
behaviour of $T^{\star}(m)$.
\autoref{fig:T_pseudocrit_chi_power_law_fit} shows the results for different choices of this length scale.
To obtain $T^{\star}$, a convergence threshold of $10^{-8}$ and a temperature tolerance of $10^{-6}$ are used.
The boundaries are fixed to $+1$.

\todo[inline]{Maybe I've used a too low TolX??}
\todo[inline]{Explain results here}

We denote the estimated value of the critical temperature as $\widetilde{T_c}$. Recall that the exact value is
\begin{equation}
  T_c = 2.2691853\dots
\end{equation}
and $\nu = 1$.

When using $\xi(T_c, m)$, the correlation length at the exact critical point,
the result shows a lot of structure, yielding $\widetilde{T_c} = 2.269172$ and $\nu = 1.057$.

If, instead, the correlation length at the estimated pseudocritical temperature $\xi(T^{\star}(m))$ is used,
the data shows less structure and we obtain the much more precise results $\widetilde{T_c} = 2.269183$ and $\nu =
1.002$.

Another option is to use the entropy to define the correlation length,
via \autoref{eq:correlation_length_as_function_of_entropy}, which gave more accurate results than using the transfer
matrix definition in \autoref{sec:entropy_to_define_correlation_length}.
In this case, the results are slightly worse than the transfer matrix definition:
$T_c = 2.269183$ and $\nu = 1.02$.

Finally, we may directly fit the law
\begin{equation}
  |T_c - T^{\star}(m)| \propto m^{-\kappa/\nu},
\end{equation}
yielding $T_c = 2.269181$ and $\kappa/\nu = 1.91$.
Incidentally, this is another way to confirm $\kappa \approx 1.9$ for systems with a fixed boundary.

\todo[inline]{Why do length scales defined at $T^{\star}$ work better??
It is fortunate that we don't need the length scales at $T_c$, since we don't know it.}

\subsection{Finite $N$}
As a cross check, we can instead use systems of finite size to extract $T_c$ and $\nu$.
This yields \emph{values}. See the \emph{figure}.

\begin{figure}
  \includegraphics[]{entropy_vs_T.tikz}
  \caption{Classical analogue to the entanglement entropy, as in \autoref{eq:classical_entropy},
  near the critical point (shown as dashed line).}\label{fig:entropy_vs_T}
\end{figure}

\begin{figure}
  \includegraphics[]{T_pseudocrit_chi_power_law_fit.tikz}
  \caption{haloalaoalalalalaalkk}\label{fig:T_pseudocrit_chi_power_law_fit}
\end{figure}

\section{Away from the critical point}

We may also verify the validity of the different length scales by asserting that the data for different values of $m$
should collapse on a single curve
\begin{equation}
  \mathcal{G}(t N_{\text{eff}}(m)^{1/\nu}) = M(T, m) N_{\text{eff}}(m)^{\beta/\nu}.
\end{equation}

All data points were calculated with a convergence threshold of $10^{-7}$.
The values of the pseudocritical temperatures are taken from the results in \autoref{sec:locating_the_critical_point}.
No temperatures beyond $T_c$ is considered because the order parameter drops off sharply,
causing the curve $\mathcal{G}(x)$ to tend to zero almost vertically, making the fitness $P$ unreliable.

\autoref{fig:data_collapse_chi} shows that for all length scales, the results more or less fall on one curve.
\autoref{table:fitness_data_collapse_different_length_scales} shows the fitness of the data collapse
\cite{bhattacharjee2001measure} (given by \autoref{eq:fitness_data_collapse}) for all length scales used.

\todo[inline]{Say which length scales apparently don't work so well}

Using $m^{\kappa}$ as a length scale for optimized fitness $P(\kappa)$ yields $\kappa \approx 1.98$,
substantially higher than found previously for fixed boundary conditions.

As a cross-check, the bottom-right panel of \autoref{fig:data_collapse_chi} shows data points for finite-$N$
simulations. Here, the bond dimension is chosen such that the truncation error is smaller than $10^{-6}$.

\begin{figure}
  \includegraphics[]{data_collapse_chi.tikz}
  \caption{Data collapses using different length scales.
  For the bottom-right plot, approximations with finite $N$ instead of finite $m$ have been used,
  with $n = \{160, 480, 1000, 1500 \}$ ($n = \frac{N - 1}{2}$ is the number of algorithm
  steps).}\label{fig:data_collapse_chi}
\end{figure}

\begin{table}[]
\centering
\begin{tabular}{@{}ll@{}} \toprule
$N_{\text{eff}}$                  & fitness $P$ \\ \midrule
$\xi(T_c, m)$                     & 0.0075      \\
$\xi(T^{\star}(m))$               & 0.066       \\
$\exp((6/c)S(T_c, m))$            & 0.057       \\
$\exp((6/c)S(T^{\star}(m))$       & 0.087       \\
$m^{\kappa}$                      & 0.0080      \\
$N$                               & 0.0075      \\ \bottomrule
\end{tabular}
  \caption{Fitness of data collapse (\autoref{eq:fitness_data_collapse}) for different length scales.
  $\kappa \approx 1.98$ was found to be optimal for the length scale $m^{\kappa}$.}
  \label{table:fitness_data_collapse_different_length_scales}
\end{table}


% \chapter{Numerical results for the clock model}
% \chapterprecishere{We present results of scaling in bond dimension and system size with the CTMRG algorithm for the
five- and six-state clock model.}

\todo[inline]{Not finished yet.}

\section{Introduction}
In the field of phase transitions and critical phenomena, the two-dimensional topological phase transition discovered by
Kosterlitz and Thouless \cite{kosterlitz1973ordering, kosterlitz1974critical} receives much attention. This phase
transition is characterized not by an order parameter which indicates a breaking of symmetry, but by the proliferation
of topological defects.

In the low-temperature phase, the two-point correlation functions decay with a power-law with varying
exponent $\eta(T)$. At the transition, the correlation length diverges as
\begin{equation}
  \xi \propto \exp(A |T - T_c|^{-1/2}),
\end{equation}
with $A$ a non-universal constant. Above the transition, the two-point correlators decay exponentially.

The XY model consists of planar rotors on the square lattice. It exhibits the Kosterlitz-Thouless (KT) phase transition
and by the Mermin-Wagner-Hohenberg theorem the symmetry of the ground state is broken for all temperatures, due to
the $O(2)$ (planar rotational) symmetry of the potential \cite{mermin1966absence, hohenberg1967existence}.

The $q$-state clock model possesses the discrete $\mathbb{Z}_q$ symmetry and is an interpolation between the Ising
model, which corresponds to $q = 2$, and the XY model, which corresponds to $q \to \infty$. Its energy function is given
by
\begin{equation}\label{eq:hamiltonian_clock_model}
  H_q = -\sum_{\langle i j \rangle} \cos(\theta_i - \theta_j),
\end{equation}
with the spins taking the values
\begin{equation}
  \theta = \frac{2 \pi n}{q} \qquad n \in \{ 0, \dots, q-1 \}.
\end{equation}

It has been proven that for high enough $q$, this model indeed exhibits a Kosterlitz-Thouless transition
\cite{frohlich1981kosterlitz}. Furthermore, it has been proven that for $q \geq 5$, a general model with $\mathbb{Z}_q$
symmetry (of which \autoref{eq:hamiltonian_clock_model} is a special case) has three phases: a symmetry broken phase for
$T < T_1$, an intermediate phase with power law decay of the correlation function, and a high-temperature phase with
exponential decay of the correlation function for $T > T_2$ \cite{cardy1980general}.

In the Villain formulation of the potential \cite{villain1975theory}, it has been proven that the transition at $T_2$ is
a BK-transition \cite{jose1977renormalization}, and numerical results suggest that for a broad range of temperatures,
the thermodynamic behaviour becomes identical to the XY model for high enough $q$ \cite{lapilli2006universality}.

Furthermore, in the Villain formulation it is known that \cite{elitzur1979phase, nienhuis1984critical}
\begin{equation}\label{eq:eta_villain}
  \eta(T_1) =\frac{4}{q^2}, \qquad \eta(T_2) = \frac{1}{4},
\end{equation}
where $\frac{\eta}{2} = \frac{\beta}{\nu}$, the magnetization exponent in the finite-size regime.

For the cosine model in \autoref{eq:hamiltonian_clock_model}, the value $q_c$ for which it first exhibits a
BK-transition and critical indices are not precisely known, though it is generally accepted that $q = 6$ exhibits two
BK-transitions. The $q = 5$ case is contested (see previous numerical results below).

In our simulations we will focus on the cases $q = 5, 6$, to (i) study the nature of the phase transition from a new
perspective in the case of $q = 5$ and (ii) compare the accuracy of finite-$m$ and finite-$N$ scaling within the CTMRG
method to other established methods for both cases.

We briefly summarise previous numerical results, then present results from the CTMRG algorithm.

\section{Previous numerical results}
\subsection{The $q = 5$ clock model}

The general consensus is that the two transitions of the $q = 5$ clock model with cosine potential are of the KT-type,
though there are no rigorous results.
It is also assumed that the critical indices are the same as those in the Villain formulation.

The disagreement about the nature of the phase transitions
stems from numerical results for the helicity modulus
\cite{fisher1973helicity}.

Most notably, Baek and Minnhagen \cite{baek2010non} claim that since the helicity modulus does
not vanish in the high-temperature phase, the upper transition is not of the KT-type.

It was shown by Kumano et al.
in \cite{kumano2013response}, however, that the definition used by Baek and Minnhagen is not suitable for systems with a
discrete symmetry.
The correct discrete definition yields the expected result, namely that the helicity modulus does vanish and the
three-phase KT-picture holds.

The conclusion of Kumano et al, which was obtained by a Monte Carlo study,
was verified by Chatelain \cite{chatelain2014dmrg} using the TMRG algorithm \cite{nishino1995density} (see also
\autoref{sec:tmrg}).
Chatelain also found that the critical indices match those of the Villain model (\autoref{eq:eta_villain}),
implying the cosine model is in the same universality class as the Villain model.

After the rebuttal by Kumano et al., Baek et al.
published another work \cite{baek2013residual} in which they again use the (in the eyes of Kumano et al.) wrong
definition of the helicity modulus, yet calculated in a different way.
Again they conclude the transition is not of the KT-type.

Meanwhile, Borisenko et al.
\cite{borisenko2011numerical} carried out a very detailed Monte Carlo study confirming the KT-picture,
using Binder-cumulants to find the critical points and the magnetization and susceptability to find the critical
indices.

Brito et al.
\cite{brito2010twodimensional} conclude from a Monte Carlo study that while the transition is of KT-type,
the resolution of their numerical method is not high enough to distinguish between $T_1$ and $T_2$.

\begin{table}[]
\centering
\begin{tabular}{@{}lll@{}}
\toprule
 & $T_1$ & $T_2$ \\ \midrule
Brito et al. (2010) & 0.91(2) & 0.90(2) \\
Borisenko et al. (2011) &  &  \\
Kumano et al. (2013) &  &  \\
Chatelain (2014) &  &  \\ \bottomrule
\end{tabular}
\caption{My caption}
\label{my-label}
\end{table}

\todo[inline]{List critical points found by other authors}

\subsection{The $q = 6$ clock model}


For $q = 5$:
\begin{itemize}
  \item \cite{chatelain2014dmrg} 2014 claims BK-transition for $q = 5$ by discrete helicity modulus using TMRG.
  \item \cite{kumano2013response} 2013 claims BK-transition for $q = 5$ by discrete helicity modelus using Monte Carlo.
  Shows infinitesimal helicity modulus does not vanish for finite $q$.
  \item \cite{borisenko2011numerical} 2011 most comprehensive study? Claims BK-transition for $q = 5$ by calculating
  Binder cumulants + stuff I don't understand using Monte Carlo.
  \item \cite{borisenko2012phase} 2012 don't understand. Some numerical and theoretical validations for certain
  assumptions for high $q$.
  \item \cite{brito2010twodimensional} 2010 claims BK-transition for $q = 5$ based on Monte Carlo simulation and mapping
  to solid-on-solid growth model.
  \item \cite{baek2010non} 2010 claims that $q = 5$ exhibits non-KT-transition, based on the fact that their (wrong)
  definition of the helicity modulus does not vanish.
  \item \cite{baek2013residual} 2013 claims that $q = 5$ exhibits weaker cousin of KT-transition, based on things I do
  not fully understand, but among others on the correct, discrete definition of the helicity modulus (true?).
  \item \cite{kim2017partition} 2017 uses partition function zeros (don't understand that concept) to show that the
  behaviour of zeros of the $q = 5$ model significantly departs from the behaviour of $q \geq 6$.
\end{itemize}

For $q = 6$:
\begin{itemize}
  \item \cite{krvcmar2016phase} 2016 KT-transition from entropy.
  \item \cite{baek2013residual}
  \item \cite{kumano2013response}
\end{itemize}

\section{Results}

What are the results that really should be in there? Focus on $q = 5, 6$.

\begin{itemize}
  \item show that scaling relation holds for magnetization and correlation length and calculate exponent. Should vary as
  $\eta \propto T$ in the massless phase and probably (?) $\eta(T_2) = 1/4$ and $\eta(T_1) \propto 1/q^2$ or something.
  \item show that $c = 1$ massless phase exists.
  \item show that $T^{\star} - T_c$ is compatible with correlation length at essential singularity (but what about
  logarithmic corrections...?)
  \item data collapse in magnetization around Kosterlitz-transition
\end{itemize}


\appendix
% \chapter{Correspondence of quantum and classical lattice
% systems}\label{chapter:correspondence_quantum_classical}
% The partition of a discrete quantum mechanical system is given by
\begin{equation}\label{eq:quantum_partition_function}
  Z_{q} = \tr \exp(-\beta H_{q}) =
  \sum_{\sigma} \bra{\sigma} \exp(-\beta H_{q}) \ket{\sigma}
\end{equation}
Imagine splitting the imaginary time interval $\beta$ into $N$ smaller steps:
\begin{align}
  \beta &= N \delta \tau, \\
  \exp(-\beta H_q) &= \exp(-\delta \tau H_q)^N.
\end{align}
Recall that for any orthonormal basis, the identity can be expressed as a sum over
projectors onto the basis states
\begin{equation}
    \mathbb{1} = \sum_{\sigma} \ket{\sigma}\bra{\sigma}.
\end{equation}
If we insert $N - 1$ resolutions of indentity into
\autoref{eq:quantum_partition_function}, we obtain
\begin{equation}
  Z_q = \sum_{\sigma} \sum_{\sigma_1, \dots, \sigma_{N-1}}
  \braket{\sigma | \exp(-\beta \delta \tau) | \sigma_1} \dots
  \braket{\sigma_{N-1} | \exp(-\beta \delta \tau) | \sigma}.
\end{equation}

This is the imaginary time path integral formulation of quantum mechanics. Similar to the
real-time path integral, an evolution in the imaginary time direction is expressed as a
sum over all paths connecting the initial and final state, which are the same here, since
we are taking the trace.

We turn to the partition function of a classical system, written as a product of
its transfer matrix, as in \autoref{eq:partition_function_transfer_matrix_1d}:
\begin{equation}
  Z_{\text{cl}} = \tr T^N.
\end{equation}
There is a striking similarity between a quantum mechanical partition function in $d$
dimensions and a classical partition function in $d + 1$ dimensions.
Adding a row to the classical lattice by applying the transfer matrix corresponds to time
evolution of a quantum system:
\begin{equation}
  T \longleftrightarrow \exp(-\delta \tau H_q).
\end{equation}
The classical temperature corresponds to the coupling constants in the Hamiltonian
$H_{\text{q}}$.

Letting $\beta \to \infty$ (or equivalently $T \to 0$) amounts to taking $N \to \infty$,
while keeping $\delta \tau$ fixed. In this limit, analogously to the transfer matrix for
the classical system (cf. \autoref{eq:largest_eigenvector}), the operator $\exp(-\beta
H_{\text{q}})$ becomes a projector onto the ground state.

Thus, a quantum lattice in the ground state corresponds to a
classical lattice that is infinite in its additional dimension.

\todo[inline]{What is the role $\delta \tau$? Why should it be small?
If we want to make the correspondence $T \equiv H_q$, we need $\delta \tau \to 0$.
But what does this imply for the lattice? How does the 'scaling limit' come into play?
Also: explain the correspondence between the energy scale of the quantum system and the
correlation length of the classical system. Maybe do an example of 0D quantum to 1D
classical and 1D quantum to 2D classical (Ising model?).}

%
% \chapter{Introduction to tensor networks}\label{chapter:introduction_to_tensor_networks}
% \section{Tensors, or multidimensional arrays}
In the field of tensor networks, a tensor is a multidimensional table with numbers -- a
convenient way to organize information. It is the generalization of a vector
\begin{equation}
  v_i =
  \begin{bmatrix}
    v_1 \\
    \vdots \\
    v_n
  \end{bmatrix},
\end{equation}
which has one index, and a matrix
\begin{equation}
  M_{i j} =
  \begin{bmatrix}
  M_{1 1} & \dots & M_{1 n} \\
  \vdots  & & \vdots \\
  M_{m 1} & \dots & M_{m n}
  \end{bmatrix},
\end{equation}
which has two.
A tensor of rank $N$ has $N$ indices:\footnote{The definition of rank in this
context is not to be confused with the rank of a matrix, which is the number of
linearly independent columns. Synonyms of tensor rank are tensor degree and
tensor order.}
\begin{equation}
  T_{i_1 \dots i_N}.
\end{equation}

A tensor of rank zero is just a scalar.

\section{Tensor contraction}

Tensor contraction is the higher-dimensional generalization of the dot product
\begin{equation}
  \bm{a} \cdot \bm{b} = \sum_i a_i b_i,
\end{equation}
where a lower-dimensional tensor (in this case, a scalar, which is a
zero-dimensional tensor) is obtained by summing over all values of a repeated
index.

Examples are matrix-vector multiplication
\begin{equation}
  (M \bm{a})_{i} = \sum_j M_{i j} a_j,
\end{equation}
and matrix-matrix multiplication
\begin{equation}
  (A B)_{i j} = \sum_k A_{i k} B_{k j},
\end{equation}
but a more elaborate tensor multiplication could look like
\begin{equation}
  w_{a b c} = \sum_{d, e, f} T_{a b c d e f} v_{d e f}.
\end{equation}

As with the dot product between vectors, matrix-vector multiplication and
matrix-matrix multiplication, a contraction between tensors is only defined if
the dimensions of the indices match.

\section{Tensor networks}

A tensor network is specified by a set of tensors, together with a set of contractions to be performed. For example:
\begin{equation}
  M_{a b} = \sum_{i, j, k} A_{a i} B_{i j} C_{j k} D_{k b},
\end{equation}
which corresponds to the matrix product $A B C D$.

\subsection{Graphical notation}
It is highly convenient to introduce a graphical notation that is common in the
tensor network community. It greatly simplifies expressions and makes certain
properties manifest.

Each tensor is represented by a shape. Open-ended lines, called legs, represent unsummed
indices. See \autoref{fig:tensors_graphical_notation}. If it clear from the context, index
labels may be omitted from the open legs.

Each contracted index is represented by a connected line. See
\autoref{fig:contracted_tensors}.

Many tensor equations, while burdensome when written out, are readily
understood in this graphical way. As an example, consider the matrix trace in
\autoref{fig:contracted_tensors}, where its cyclic property is manifest.

\begin{figure}
  \begin{tikzpicture}
	\begin{pgfonlayer}{nodelayer}
		\node [style=generic, label={$c$}] (0) at (-3, 0) {};
		\node [style=generic, label={$v_i$}] (1) at (-1, 0) {};
		\node [style=generic, label={$M_{i j}$}] (2) at (1, 0) {};
		\node [style=generic, label={$T_{i j k}$}] (3) at (3, 0) {};
		\node [style=white no border] (4) at (-2, 0) {};
		\node [style=white no border] (5) at (0, 0) {};
		\node [style=white no border] (6) at (2, 0) {};
		\node [style=white no border] (7) at (4, 0) {};
		\node [style=white no border] (8) at (3, -1) {};
	\end{pgfonlayer}
	\begin{pgfonlayer}{edgelayer}
		\draw [style=simple] (1) to node[below]{$i$} (5);
		\draw [style=simple] (2) to node[below]{$i$} (5);
		\draw [style=simple, in=180, out=0, looseness=1.00] (2) to node[below]{$j$} (6);
		\draw [style=simple, in=0, out=180, looseness=1.00] (3) to node[below]{$i$} (6);
		\draw [style=simple] (3) to node[below]{$j$} (7);
		\draw [style=simple] (3) to node[left]{$k$} (8);
	\end{pgfonlayer}
\end{tikzpicture}

  \caption{Open-ended lines, called legs, represent unsummed indices. A tensor
  with no open legs is a scalar.}
  \label{fig:tensors_graphical_notation}
\end{figure}

\begin{figure}
  \begin{tikzpicture}
	\begin{pgfonlayer}{nodelayer}
		\node [style=generic, label={$\bm{a}$}] (0) at (-3, 0) {};
		\node [style=generic, label={$\bm{b}$}] (1) at (-2, 0) {};
		\node [style=generic, label={$M$}] (2) at (0, 0) {};
		\node [style=generic, label={$\bm{v}$}] (3) at (1, 0) {};
		\node [style=white no border] (4) at (-1, 0) {};
		\node [style=generic, label={$A$}] (5) at (-3, -2) {};
		\node [style=generic, label={$B$}] (6) at (-2, -2) {};
		\node [style=white no border] (7) at (-1, -2) {};
		\node [style=white no border] (8) at (-4, -2) {};
		\node [style=generic] (9) at (0, -1) {};
		\node [style=generic] (10) at (0, -2) {};
		\node [style=generic] (11) at (1, -2) {};
		\node [style=generic] (12) at (1, -1) {};
	\end{pgfonlayer}
	\begin{pgfonlayer}{edgelayer}
		\draw [style=simple] (0) to (1);
		\draw [style=simple] (4) to (2);
		\draw [style=simple] (2) to (3);
		\draw [style=simple] (8) to (5);
		\draw [style=simple] (5) to (6);
		\draw [style=simple] (6) to (7);
		\draw [style=simple] (10) to (11);
		\draw [style=simple] (11) to (12);
		\draw [style=simple] (12) to (9);
		\draw [style=simple] (9) to (10);
	\end{pgfonlayer}
\end{tikzpicture}

  \caption{Connected legs represent contracted indices. The networks in the
  figure represent $\sum_i a_i b_i$ (dot product),
  $\sum_j M_{i j} a_j$ (matrix-vector product), $\sum_{k} A_{i k} B_{k
  j}$ (matrix-matrix product) and $\tr A B C D$, respectively.}
  \label{fig:contracted_tensors}
\end{figure}

\subsection{Reshaping tensors}
Several indices can be taken together to form a single, joint index, that runs over all
combinations of the indices that fused into it. For example, an $m \times n$ matrix can be
reshaped into an $m n$ vector.
\begin{equation}
  M_{i j} = \bm{v_a} \qquad a \in \{ 1, \dots, m n \}.
\end{equation}

Some convention has to be chosen to map the joint index $i, j$ onto the single index $a$,
for example
\begin{equation}
  a = (n - 1)i + j,
\end{equation}
which orders the indices of the matrix $M$ row by row
\begin{equation}
  1 1, 1 2, \dots, 1 n, 2 1, 2 2, \dots, m n - 1, m n.
\end{equation}

Graphically, an index contraction is represented as a bundling of the open legs of a
tensor network. See \autoref{fig:reshaped_tensor} for an example.

\begin{figure}
  \begin{tikzpicture}
	\begin{pgfonlayer}{nodelayer}
		\node [style=generic] (0) at (-2, 0) {};
		\node [style=generic] (1) at (-1, 0) {};
		\node [style=generic] (2) at (0, 0) {};
		\node [style=white no border] (3) at (-2, 1) {};
		\node [style=white no border] (4) at (-1, 1) {};
		\node [style=white no border] (5) at (0, 1) {};
		\node [style=white no border] (6) at (0, -1) {};
		\node [style=white no border] (7) at (-1, -1) {};
		\node [style=white no border] (8) at (-2, -1) {};
		\node [style=white no border] (9) at (1, 0) {$=$};
		\node [style=generic2] (10) at (2, 0) {};
		\node [style=white no border] (11) at (2, 1) {};
		\node [style=white no border] (12) at (2, -1) {};
	\end{pgfonlayer}
	\begin{pgfonlayer}{edgelayer}
		\draw [style=simple] (0) to (1);
		\draw [style=simple] (1) to (2);
		\draw [style=simple] (2) to node[right]{$c$} (5);
		\draw [style=simple] (1) to node[right]{$b$} (4);
		\draw [style=simple] (0) to node[right]{$a$} (3);
		\draw [style=simple] (0) to node[right]{$d$} (8);
		\draw [style=simple, in=90, out=-90, looseness=1.00] (1) to node[right]{$e$} (7);
		\draw [style=simple] (2) to node[right]{$f$} (6);
		\draw [style=simple, ultra thick] (10) to node[right]{$i$} (11);
		\draw [style=simple, ultra thick] (10) to node[right]{$j$} (12);
	\end{pgfonlayer}

		\draw [style=simple, decoration={brace, amplitude=0.5em}, decorate] (3.center) to (5.center);
    \draw [style=simple, decoration={brace, amplitude=0.5em}, decorate, rotate=90] (6.center) to
    (8.center);
\end{tikzpicture}

  \caption{Reshaping a tensor $T_{a b c d e f}$ to $T_{i j}$}
  \label{fig:reshaped_tensor}
\end{figure}

\subsection{Computational complexity of contraction}
\todo[inline]{Computational complexity.}


% \chapter{Singular value decomposition}
% \input{chapters/singular_value_decomposition}


\backmatter
\printbibliography

\end{document}
