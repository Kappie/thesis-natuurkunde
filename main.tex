\documentclass[9pt, ebook, openany, oneside]{memoir}

% Packages for baskervillef font.
\usepackage[T1]{fontenc}
\usepackage{baskervillef}
\usepackage[varqu,varl,var0]{inconsolata}
\usepackage[scale=.95,type1]{cabin}
\usepackage[baskerville,vvarbb]{newtxmath}
\usepackage[cal=boondoxo]{mathalfa}

% Bulk of packages:
\usepackage[english]{babel}
\usepackage{standalone}
\usepackage[utf8]{inputenc}
\usepackage{graphicx}
\usepackage{todonotes}
\usepackage{amsmath}
\usepackage{amssymb}
\usepackage{braket}
\usepackage{bm}
\usepackage{tabularx}
\usepackage{subcaption}
\usepackage{mathtools}
\usepackage{booktabs}

% Input packages for tikz pictures.
\usepackage{tikz}
\usetikzlibrary{decorations.markings}
\usetikzlibrary{decorations.pathreplacing}
\usetikzlibrary{shapes.geometric}
\usetikzlibrary{positioning}

\pgfdeclarelayer{edgelayer}
\pgfdeclarelayer{nodelayer}
\pgfsetlayers{edgelayer,nodelayer,main}

\def\tensorsize{1.3em}

\definecolor{princess-orange}{rgb}{0.988,0.824,0.745}
\definecolor{princess-green}{rgb}{0.929,0.929,0.792}
\definecolor{princess-blue}{rgb}{0.627,0.808,0.808}

\tikzstyle{none}=[inner sep=0pt]
\tikzstyle{tensor}=[circle, draw=black, thick, minimum size = \tensorsize]
\tikzstyle{generic} = [tensor, fill=princess-orange]
\tikzstyle{generic2} = [tensor, very thick, fill=princess-green, minimum size =
1.15*\tensorsize]
\tikzstyle{a-tensor}=[tensor,fill=princess-blue]
\tikzstyle{p-tensor-up} = [tensor, minimum size = 0.5*\tensorsize, shape=semicircle,
fill=princess-green]
\tikzstyle{p-tensor-right} = [p-tensor-up, shape border rotate = 90]
\tikzstyle{p-tensor-down} = [p-tensor-up, shape border rotate = 180]
\tikzstyle{p-tensor-left} = [p-tensor-up, shape border rotate = 270]
\tikzstyle{q-tensor} = [tensor, fill=princess-green]
\tikzstyle{delta-tensor}=[tensor, shape=diamond, minimum size = 0.75*\tensorsize, fill=princess-orange]
\tikzstyle{white no border}=[tensor,minimum size=1.75*\tensorsize,fill=white,draw=white]
\tikzstyle{white no border small}=[minimum size = .01*\tensorsize]
\tikzstyle{largest_eigenvector}=[tensor, shape=rectangle, fill=princess-green, minimum
height = 4cm, minimum width = 0.75*\tensorsize]
\tikzstyle{isometry}=[tensor, shape=isosceles triangle, isosceles triangle stretches,
fill=princess-orange, minimum width = 2*\tensorsize]
\tikzstyle{square}=[tensor, shape=rectangle, minimum width = 0.75*\tensorsize, minimum
height = 2*\tensorsize, fill=princess-orange]
\tikzstyle{singular values matrix} = [tensor, shape=diamond, minimum size =
0.5*\tensorsize, fill=white]

\tikzstyle{simple}=[-,draw=black,thick]
\tikzstyle{thick leg}=[simple, very thick]
\tikzstyle{arrow}=[-,draw=black,postaction={decorate},decoration={markings,mark=at position .5 with {\arrow{>}}},line width=2.000]
\tikzstyle{tick}=[-,draw=black,postaction={decorate},decoration={markings,mark=at position .5 with {\draw (0,-0.1) -- (0,0.1);}},line width=2.000]


% Just to make sure: import these packages last.
% \usepackage[backend=biber, citestyle=numeric-comp, bibstyle=phys, hyperref]{biblatex}
\usepackage[backend=biber, style=phys, hyperref]{biblatex}
\usepackage[]{hyperref}

\newcommand\myshade{80}
\hypersetup{
  backref,
  pdfpagemode=FullScreen,
  linkcolor  = scientific7,
  citecolor  = scientific8,
  urlcolor   = scientific9,
  colorlinks = true,
}

\definecolor{orange}{RGB}{252,210,190}



\addbibresource{references.bib}
\graphicspath{{images/}{plots/}}



\maxtocdepth{subsubsection}
\setsecnumdepth{subsubsection}


% Commands below define chapter title style.
\makeatletter
 \renewcommand{\chapterheadstart}{\vspace*{\beforechapskip}}
 \renewcommand{\printchaptername}{}
 \renewcommand{\chapternamenum}{\space}
 \renewcommand{\printchapternum}{\centering\chapnumfont\thechapter}
 \renewcommand{\afterchapternum}{\par\nobreak\vskip \midchapskip}
 \renewcommand{\printchapternonum}{}
 % \renewcommand{\printchaptertitle}[1]{\centering\chaptitlefont\MakeLowercase{ \oldstylenums #1}}
 \renewcommand{\printchaptertitle}[1]{\centering\chaptitlefont #1}
 % \renewcommand{\printchaptertitle}[1]{\centering\chaptitlefont}
 \renewcommand{\afterchaptertitle}{%
 \vskip2em
 \hrule height 0.6pt
 \vskip2em
 }
 \renewcommand{\chapnamefont}{\normalfont\huge\bfseries}
 \renewcommand{\chapnumfont}{\normalfont\Huge\bfseries}
 \renewcommand{\chaptitlefont}{\normalfont\huge\bfseries}
 \setlength{\beforechapskip}{0.5em}
 \setlength{\midchapskip}{0.5em}
 \setlength{\afterchapskip}{4em}
\makeatother

% To make chapter abstracts regular font
\renewcommand*{\precisfont}{\normalfont}

\DeclareMathOperator{\tr}{Tr}


\begin{document}

\pagestyle{simple}



\frontmatter

\title{Mijn Titel}
\author{Geert Kapteijns}
\date{\today}

\begin{titlingpage}
  \maketitle
\end{titlingpage}

\begingroup
\renewcommand{\afterchaptertitle}{\vskip1.5em}

\tableofcontents*
\endgroup

\chapter{Abstract}
\noindent This thesis investigates scaling in the number of basis states
kept (the \emph{bond dimension} $m$) in approximating the partition function
of two-dimensional classical models with the corner transfer matrix
renormalization group (CTMRG) method.

For the Ising model, it is shown that exponents and the transition temperature may be approximated with a scaling
analysis in the correlation length defined in terms of the row-to-row transfer matrix at the (pseudo)critical point,
as was suggested by Nishino et al.
However, the calculated quantities show inherent deviations from the basic scaling laws,
due to the spectrum of the underlying corner transfer matrix (CTM).
These deviations are mitigated to some extent when we define the correlation length in terms of the classical analogue
of the entanglement entropy.
Scaling directly in the bond dimension $m$ is also possible, but less accurate since the law for the correlation length
$\xi \propto m^{\kappa}$ holds only in the limit $m \to \infty$ and does not take into account the spectrum of the CTM
that is obtained.

It is found that finite-$m$ scaling and finite-size scaling yield comparable accuracy for critical exponents and the
transition temperature.
With finite-$m$ scaling larger effective system sizes are obtainable,
but finite-size approximations do not suffer from the deviations due to the CTM spectrum and are
consequently of higher quality. Therefore it is plausible that finite-size results will improve significantly if
corrections to scaling are included in the fits.

We also present a numerical analysis of the clock model with $q = \{5,
6\}$ states, concluding that the Kosterlitz-Thouless picture is plausible.
We find values of the transition temperatures that are in agreement with values found by other authors.
Results for the exponent $\eta$ indicate that the critical temperatures found in both this study and previous work might
be too close together.
It is conceivable that, after considering larger systems and taking into account finite-size corrections,
both critical temperatures and the values of $\eta$ will be adjusted outwards towards their true values,
thereby completely reconciling the results.

Overall, we conclude that finite-$m$ scaling is a valuable alternative to finite-size scaling within CTMRG,
since larger system sizes are accessible.
The CTMRG analysis is itself a valuable addition to other approximation methods such as Monte Carlo,
yielding comparable results, while obtaining estimates from completely different principles.
Furthermore it reveals information, such as the the spectrum of the transfer matrices and the central charge of the
massless phase, that is not accessible otherwise.


\chapter{Acknowledgements}
I want to thank Philippe Corboz for all his help during the past year.
Thanks for being always enthusiastic and ready to answer questions, for including me in the group from the start,
and for showing me the messy reality of research.

Thanks to Piotr Czarnik, Sangwoo Chung, Schelto Crone, Karel Temmink and Ido Niesen for many helpful discussions.
Especially Ido, who took more time for me than could reasonably be expected for someone with such a busy schedule.

Thanks to Bernard Nienhuis, for very helpful discussions towards the end of my thesis.

Thanks to Tobias Bouma, for good times and for sharing your arcane knowledge of LaTeX.

Thanks to Boris Ponsioen for the hours and hours we spent drinking coffee,
discussing the intricacies of this-or-that algorithm or our next career move.
Those significantly broadened my perspective.

Thanks to Daan Mulder for our discussions about physics, politics and music.
If I remembered one line of a song, you could always declaim the whole verse,
at least in the case of Cohen (may he rest in peace).

Thanks to my family, whom I didn't see much the past year. And finally, thanks to Marianne.


\mainmatter

\chapter{Introduction}
This thesis investigates a numerical approximation method put forth by
Baxter in 1978 \cite{baxter1978variational, baxter1982exactly_ctm,
tsang1979square} based on the corner transfer matrix formulation of the
partition function for two-dimensional classical lattice models.

The method rose to prominence in 1996, under the name \emph{corner
transfer matrix renormalization group} (CTMRG), when Nishino showed
\cite{nishino1996corner} that in the thermodynamic limit, it is equivalent
to the hugely successful density matrix renormalization group (DMRG)
method for one-dimensional quantum systems \cite{white1992density},
discovered a few years earlier by White.

The error in the CTMRG method comes from the fact that the corner transfer
matrices, whose dimension diverges exponentially in the lattice size, have
to be truncated at a maximum dimension $m$ in order to make numerical
manipulation possible. This finite \emph{bond dimension} $m$ (also denoted
by $\chi$ or $D$ in the literature) introduces
finite-size effects, comparable to those observed for systems that are
finite in one or more spatial dimensions. This was already realized by
Nishino \cite{nishino1996numerical}.

The main objective of this thesis is to study how \emph{finite bond
dimension scaling} may be performed with the CTMRG method.

Before the structure of this thesis is laid out, I will first make some
general remarks on statistical mechanics and phase transitions, and on how
the CTMRG method relates to the class of newer methods for simulating
many-body systems that grew out of White's breakthrough, known as
\emph{tensor network algorithms}.

\section{Statistical mechanics and phase transitions}

Statistical mechanics is concerned with describing the average properties
of systems consisting of many particles. Examples of such systems are the
atoms making up a bar magnet, the water molecules in a glass of water, or
virtually any other instance of matter around us.

Matter can arrange itself in various structures with fundamentally
different properties. We call these distinct states of matter
\emph{phases}. When matter changes from one phase to another, we say it
undergoes a phase transition.

Physics has made great strides in understanding these transitions. The
complete history of the field is beyond the scope of this introduction and this
thesis, but the reader may wish to consult \cite{kadanoff2009more,
domb1996critical} to get an idea.

Only as late as 1936, the occurrence of a phase transition
within the framework of statistical physics was established by R. Peierls
\cite{peierls1936on_ising}. He showed that the two-dimensional Ising model
has a non-zero magnetization for sufficiently low temperatures. Since for
high enough temperatures the Ising model loses its magnetization, it
follows that there must be phase transition in between.

The effort to understand the Ising model culminated with Onsager's exact
solution in 1944 \cite{onsager1944two_dimensional}, which rigorously
established a sharp transition point in the thermodynamic limit.

One may question the relevance of studying very simple models such as the
Ising model. As it turns out, systems that are at first sight vastly
different may show qualitatively similar behaviour near a phase
transition. For example, exponents that characterize the divergence of
quantities near a transition are conjectured to be independent on
microscopic details of the interactions between particles, but instead
fall into distinct \emph{universality classes}
\cite{griffiths1970dependence, fisher1966quantum}. Thus, studying the very
simplest model may yield universal results.

\section{Baxter's method as a precursor to tensor network methods}

Baxter showed that the optimal truncation of corner transfer matrices
corresponds, in the thermodynamic limit, to a variational optimization of
the row-to-row transfer matrix within a certain subspace, now known as the
subspace of \emph{matrix product states} (MPSs) \cite{baxter1968dimers,
baxter1982exactly_ctm}.

After the success of White's DMRG, which, as Nishino pointed out, is equivalent
to Baxter's method, the underlying matrix-product structure was rediscovered in
the context of one-dimensional quantum systems by Östlund and Rommer
\cite{ostlund1995thermodynamic, rommer1997class}.

It is historically significant but little known that Nightingale, in
a footnote of a 1986 paper \cite{nightingale1986gap}, already made the
remark that \enquote{The generalization [of Baxter's method] to quantum
mechanical systems is straightforward.}

After Östlund and Rommer, it was realized that reformulating White's
algorithm directly in terms of matrix product states provided the
explanation of the algorithm's shortcomings around phase transitions. An
MPS-ansatz fundamentally limits the entropy of the ground state
approximation and since the entropy diverges at a conformally invariant
critical point \cite{calabrese2004entanglement}, DMRG gives inaccurate
results.

This gave rise to other ansätze, formulated in the language of tensor
networks \cite{orus2014practical}, specifically designed to represent
states with a certain amount of entropy. Examples are multi-scale
entanglement renormalization ansatz (MERA) for critical one-dimensional
quantum systems \cite{vidal2007entanglement} and projected entangled-pair
states (PEPS) \cite{verstraete2004renormalization}
for two-dimensional quantum systems.

Other tensor network algorithms, such as infinite time-evolving block
decimation \cite{vidal2007classical} in one dimension and iPEPS (infinite
PEPS) \cite{jordan2008classical} in two dimensions made it possible to
directly approximate quantum systems in the thermodynamic limit.

iTEBD was used to study finite bond dimension scaling (under the slightly
different name of \emph{finite-entropy scaling})
\cite{tagliacozzo2008scaling}. Some theoretical predictions were later
made in \cite{pollmann2009theory}.

The goal of this thesis is twofold: (i) investigate how finite bond
dimension scaling works in the CTMRG algorithm for classical systems,
where we can directly compare it with finite-size scaling, and (ii)
investigate how it compares to different numerical approaches, such as
iTEBD or Monte Carlo.

For (i), I have studied the Ising model, for which all results may be checked
against the exact solution. For (ii), I have studied the clock model with $q
= \{5,
6 \}$ states, which is regarded as difficult numerically and subject to some
controverse.

\section{Structure of this thesis}

It is in the mostly quantum-oriented research field sketched above that the
work for this thesis was done. Therefore, I have chosen to
begin by introducing White's algorithm in its original description
(chapter two), before making the connection to two-dimensional classical
lattices and properly introducing the corner transfer matrix formulation
(chapter three).

In chapter four, the concepts of critical behaviour and finite-size
scaling are introduced and in chapter five these concepts are connected to
the work already done on finite bond dimension (or finite-entropy)
scaling.

Technical details and convergence behaviour of the CTMRG algorithm are reported in chapter six.
It is found that the values of observables may be accurately extrapolated in the chosen convergence criterion of the
algoritm.

Results for the Ising model are presented and analyzed in chapter seven.
With finite-$m$ simulations, it is much easier to reach large system sizes,
but thermodynamic quantities do not grow smoothly as a function of the bond dimension,
as a result of the underlying spectrum of the corner transfer matrix.

Quantities calculated with finite-size simulations do not suffer this unsmooth behaviour.
Results for both methods are comparable, but it is plausible that finite-size data turns out to be more accurate when
corrections to scaling are included.

A numerical analysis of the clock model with $q = \{5, 6\}$ states is given in chapter eight.
The model has a low-temperature ordered phase, a massless phase and a high-temperature disordered phase.
The transition temperatures $T_1$ and $T_2$ are located by extrapolating the positions of pseudocritical temperatures,
assuming the transitions are of the Kosterlitz-Thouless type.
I find slightly contradictory results, based on exact results in a related formulation of the model,
but argue it is plausible that this is due to finite-size effects.




\chapter{Density matrix renormalization group method}

\chapterprecishere{The density matrix renormalization group, proposed in 1992 by White
\cite{white1992density}, is introduced in its historical context. To highlight the ideas
that led to this method, we explain the real-space renormalization group, proposed by
Wilson \cite{wilson1975renormalization} in 1975. We then explain how the shortcomings of
Wilson's method led to the density matrix renormalization group.}

\section{Introduction}
Consider the problem of numerically finding the ground state $\ket{\Psi_0}$ of an
$N$-site one-dimensional spin-$\frac{1}{2}$ system.

The underlying Hilbert space of the system is a tensor product of the
local Hilbert spaces $\mathcal{H}_{\text{site}}$, which are spanned by the
states $\{\ket{\uparrow}, \ket{\downarrow}\}$. Thus, a general state of the system is a unit vector in
a $2^N$-dimensional space
\begin{equation}
  \ket{\Psi} = \sum_{\sigma_1, \sigma_2, \ldots \in \{\ket{\uparrow}, \ket{\downarrow}\}}
  c_{\sigma_1, \sigma_2, \ldots, \sigma_N} \ket{\sigma_1} \otimes \ket{\sigma_2} \otimes
  \ldots \otimes \ket{\sigma_N}.
\end{equation}

For a system with 1000 particles, the dimensionality of the Hilbert
space comes in at about $10^{301}$, some 220 orders of magnitude larger than the number of
atoms in the observable universe. How can we possibly hope to approximate states in this
space?

As it turns out, nature is very well described by Hamiltonians that are local -- that do
not contain interactions between an arbitrary number of bodies. And for these
Hamiltonians, only an exponentially small subset of states can be explored in the lifetime
of the universe \cite{poulin2011quantum}. That is, only exponentially few states are
physical. The low-energy states, especially, have special properties that allow them to be
very well approximated by a polynomial number of parameters. This explains the existence
of algorithms, of which the density matrix normalization group is the most widely
celebrated one, that can approximate certain quantum systems to machine precision.

\todo[inline]{Refer to where this will be made more precise.}

\section{Density matrix renormalization group}

The density matrix renormalization group (DMRG), introduced in 1992 by
White \cite{white1992density}, aims to find the best approximation of
a many-body quantum state, given that only a fixed amount of basis vectors
is kept. This amounts to finding the best truncation

\begin{equation}
  \mathcal{H}_N \rightarrow \mathcal{H}_{\text{eff}}
\end{equation}
from the full $N$-particle Hilbert space to an effective lower dimensional
one. This corresponds to renormalizing the Hamiltonian $H$.

Before DMRG, several methods for achieving this truncation were proposed, most notably
Wilson's real-space renormalization group \cite{wilson1975renormalization}. We will
discuss this method first, and highlight its shortcomings, which eventually led to the
invention of the density-matrix renormalization group method by White.

\subsection{Real-space renormalization group}

Consider again the problem of finding the ground state of a many-body
Hamiltonian $H$. A natural way of renormalizing $H$ in real-space is by
partitioning the lattice in blocks, and writing $H$ as

\begin{equation}
  H = H_A \otimes \ldots \otimes H_A
\end{equation}

where $H_A$ is the Hamiltonian of a block. \todo[inline]{Make figures.} The real-space
renormalization procedure now entails finding an effective Hamiltonian $H_{A}'$ of the
two-block Hamiltonian $H_{AA} = H_A \otimes H_A$. In the method introduced
by Wilson, $H_{A}'$ is formed by keeping the $m$ lowest lying eigenstates
$\ket{\epsilon_{i}}$ of $H_{AA}$:
\begin{equation}
  H_{A}' = \sum_{i = 1}^{m} \epsilon_{i} \ket{\epsilon_{i}}\bra{\epsilon_{i}}.
 \end{equation}

This is equivalent to writing
\begin{equation}
  H_{A}' = O H_{AA} O^{\dagger},
\end{equation}
with $O$ an $m \times 2^L$ matrix, with rows being the $m$ lowest-lying
eigenvectors of $H_{AA}$, and $L$ the number of lattice sites of a block. At
the fixed point of this iteration procedure, $H_A$ represents the
Hamiltonian of an infinite chain.
In choosing this truncation, it is assumed that the
low-lying eigenstates of the system in the thermodynamic limit are
composed of low-lying eigenstates of smaller blocks.

It turns out that this method gives poor results for many lattice systems. Following an
example put forth by White and Noack \cite{white1992real}, we establish an intuition
why.

\subsection{Single particle in a box}

Consider the Hamiltonian
\begin{equation}
  H = 2 \sum_i \ket{i}\bra{i} - \sum_{\langle i, j \rangle} \ket{i}\bra{j},
\end{equation}
where the second summation is over nearest neighbors $\langle i,
j \rangle$. $H$ represents the discretized version of the
particle-in-a-box Hamiltonian, so we expect its ground state to be
approximately a standing wave with wavelength double the box size.
However, the blocking procedure just described tries to build the ground
state iteratively from ground states of smaller blocks. No matter the
amount of states kept, the final result will always incur large errors.

For this simple model, White and Noack solved the problem by diagonalizing
the Hamiltonian of a block with different boundary conditions, and
combining the lowest eigenstates of each.

Additionally, they noted that
diagonalizing $p > 2$ blocks, and projecting out $p - 2$ blocks to arrive
at $H_{AA}$ also gives accurate results, and that this is a generalization
of applying multiple boundary conditions. \todo[inline]{Figure.}

In the limit $p \to \infty$
this method becomes exact, since we then find exactly the correct
contribution of $H_{AA}$ to the final ground state. It is a slightly
changed version of this last method that is now known as DMRG.

\subsection{Density matrix method}

The fundamental idea of the density matrix renormalization group method
rests on the fact that if we know the state of the final lattice, we can find the $m$
most important states for $H_{AA}$ by diagonalizing the reduced density
matrix $\rho_{AA}$ of the two blocks.

To see this, suppose, for simplicity, that the entire lattice is in a pure
state\footnote{For a proof for a mixed state, see \cite{noack1999workshop}} $\ket{\Psi} = \sum c_{b, e} \ket{b} \ket{e}$, with $b = 1, \ldots, l$ the
states of $H_{AA}$ and $e = 1, \ldots, N_{\text{env}}$ the environment states. The
reduced density matrix is given by
\begin{equation}\label{eq:density_matrix_superblock}
  \rho_{AA} = \sum_{e} \ket{\Psi} \bra{\Psi} = \sum_{b, b'} c_{b, e} c_{b', e} \ket{b} \bra{b'}
\end{equation}
We now wish to find a set of orthonormal states $\ket{\lambda} \in \mathcal{H}_{AA}$,
$\lambda = 1, \ldots, m$ with $m < l$, such that the quadratic norm
\begin{equation}\label{eq:quadratic_norm}
  \Vert \ket{\Psi} - \ket{\widetilde{\Psi}} \Vert = 1 - 2 \sum_{\lambda, b, e} a_{\lambda, e} c_{b, e} u_{\lambda, b} + \sum_{\lambda, e} a_{\lambda, e}^2
\end{equation}
is minimized. Here,
\begin{equation}
  \ket{\widetilde{\Psi}} = \sum_{\lambda = 1}^{m} \sum_{e = 1}^{N_{\text{env}}} a_{\lambda, e} \ket{\lambda} \ket{e}
\end{equation}
is the representation of $\ket{\Psi}$ given the constraint that we can only use
$m$ states from $\mathcal{H}_{AA}$. The $u_{\lambda, b}$ are given by
\begin{equation}
  \lambda = \sum_{b} u_{\lambda, b} \ket{b}.
\end{equation}

We need to minimize \eqref{eq:quadratic_norm} with respect to $a_{\lambda, e}$
and $u_{\lambda, b}$. Setting the derivative with respect to $a_{\lambda, e}$ equal to 0 yields
\begin{equation}
  -2 \sum_{\lambda, b, e} c_{b, e} u_{\lambda, b} + 2 \sum_{\lambda, e} a_{\lambda, e} = 0
\end{equation}
So we see that $a_{\lambda, e} = \sum_{b} c_{b, e} u_{\lambda, b}$, and we are left to minimize
\begin{equation}
  1 - \sum_{\lambda, b, b'} u_{\lambda, b} (\rho_{AA})_{b, b'} u_{\lambda, b'}
\end{equation}
with respect to $u_{\lambda, b}$. But this is equal to
\begin{equation}
  1 - \sum_{\lambda = 1}^{m} \bra{\lambda} \rho_{AA} \ket{\lambda}
\end{equation}
and because the eigenvalues of $\rho_{AA}$ represent probabilities and are thus
non-negative, this is clearly minimal when $\ket{\lambda}$ are the $m$
eigenvectors of $\rho_{AA}$ corresponding to the largest eigenvalues. This minimal value is
\begin{equation}\label{eq:truncation_error}
  1 - \sum_{\lambda = 1}^{m} w_{\lambda}
\end{equation}
with $w_{\lambda}$ the eigenvalues of the reduced density matrix.

\eqref{eq:truncation_error} is called the truncation error or residual
probability, and quantifies the incurred error when taking a number $m < l$ states to
represent $\mathcal{H}_{AA}$.

We have proven that the optimal (in the sense that $\Vert \ket{\Psi}
- \ket{\widetilde{\Psi}} \Vert$ is minimized\footnote{There are several other
arguments for why these states are optimal, for example, they minimize the
error in expectation values $\langle A \rangle$ of operators. For an overview,
see \cite{schollwock2005density}.}) states to keep for a subsystem are the
states given by the reduced density matrix, obtained by tracing out the entire
lattice in the ground state (or some other target state).

The problem, of
course, is that we do not know the state of the entire lattice, since that is
exactly what we're trying to approximate.

Instead then, we should try to calculate the reduced density matrix of the
system embedded in \textit{some} larger environment, as closely as possible
resembling the one in which it should be embedded.  The combination of the
system block and this environment block is usually called \textit{superblock}.

Analogous to how White and Noack solved the particle in a box problem, we could
calculate the ground state of $p > 2$ blocks,
and trace out all but 2, doubling our block size each iteration. In
practice, this doesn't work well for interacting Hamiltonians, since this
would involve finding the largest eigenvalue of a $N_{\text{block}}^p
\times N_{\text{block}}^p$ matrix (compare this with the particle in a box
Hamiltonian, which only grows linearly in the amount of lattice sites).

The widely adopted algorithm proposed by White \cite{white1993density} for
finding the ground state of a system in the thermodynamic limit proceeds
as follows.

\subsection{Infinite-system method}

\todo[inline]{Add figures.} \todo[inline]{Mention boundary conditions somewere}
Instead of using an exponential blocking procedure (doubling or tripling the
amount of effective sites in a block at each iteration), the infinite-system
method in the DMRG formulation adds a single site before truncating the Hilbert
space to have at most $m$ basis states.

\begin{enumerate}
  \item \label{step1} Consider a block $A$ of size $l$, with $l$ small. Suppose, for
  simplicity, that the number of basis states of the block is already
  $m$. States of this block can be written as
  \begin{equation}
    \ket{\Psi_A} = \sum_{b = 1}^{m} c_{i} \ket{b}.
  \end{equation}

  The Hamiltonian is written as (similarly for other operators):
  \begin{equation}
    \hat{H}_{A} = \sum_{b, b'}^{m} H_{b b'} \ket{b} \bra{b}.
  \end{equation}

  \item Construct an enlarged block with one additional site, denoted by $A
  \cdot$. States are now written
  \begin{equation}
    \ket{\Psi_{A\cdot}} = \sum_{b, \sigma} c_{b, \sigma} \ket{b} \otimes \ket{\sigma}.
  \end{equation}
  Here, $\sigma$ runs over the $d$ local basis states of $\mathcal{H}_{\text{site}}$.

  \item Construct a superblock, consisting of the enlarged system block $A
  \cdot$ and a reflected environment block $\cdot A$, together denoted by $A \cdot
  \cdot A$. Find the ground state $\ket{\Psi_0}$ of $A \cdot \cdot A$, for example
  with the Lanczos method \cite{lehoucq1996deflation}.

  \item Obtain the reduced density matrix of the enlarged block by tracing out
  the environment, and write it in diagonal form.
  \begin{equation}
  \begin{aligned}
    \rho_{A \cdot} & = \sum_{e, \sigma} (\bra{\sigma} \otimes \bra{e}) \ket{\Psi_0}
    \bra{\Psi_0} (\ket{\sigma} \otimes \ket{e}), \\
    & = \sum_{i = 1}^{d m} w_{i} \ket{\lambda_i} \bra{\lambda_i}.
  \end{aligned}
  \end{equation}

  Here, we have chosen $w_0 >= w_1 \ldots >= w_{d m}$. In this basis, the Hamiltonian is
  written as
  \begin{equation}
    \hat{H}_{A \cdot} = \sum_{i, j}^{d m} H_{ij} \ket{\lambda_i}\bra{\lambda_j}.
  \end{equation}

  \item Truncate the Hilbert space by keeping only the $m$ eigenstates of
  $\rho_{A \cdot}$ with largest eigenvalues. Operators truncate as follows:
  \begin{align}
    \widetilde{\rho}_{A \cdot} & = \sum_{i = 1}^{m} w_i \ket{\lambda_i}\bra{\lambda_j}, \\
    \widetilde{H}_{A \cdot} & = \sum_{i, j}^{m} H_{ij} \ket{\lambda_i}\bra{\lambda_j}.
  \end{align}

  \item Set $H_{A} \leftarrow \widetilde{H}_{A \cdot}$ and return to \ref{step1}.

\end{enumerate}

\todo[inline]{Expand. Present or link to some results. Finite-system algorithm. Maybe in
other chapter: rephrase in MPS, validity of approximation: primer on entropy and
eigenvalue spectrum of density matrix.}

This methods finds ground state energies with astounding accuracy, and has been
the reference point in all 1D quantum lattice simulation since its invention.


\chapter{DRMG applied to two-dimensional classical lattice models}
\begin{abstract}
\noindent This chapter explains how to apply the ideas of the density matrix renormalization group to
two-dimensional classical lattices.

First, we explain the transfer-matrix formulation for
classical partition functions.
Then, we show how to renormalize the transfer matrix using DMRG.
This was first done by Nishino \cite{nishino1995density}.
To make notation easier and up-to-date with current approaches, we redefine the transfer matrix in terms of a tensor
network.
Then, we explain the
corner transfer matrix renormalization group (CTMRG) method.
This method, first introduced by Nishino and
Okunishi \cite{nishino1996corner}, unifies ideas from Baxter \cite{baxter1968dimers,
baxter1978variational, baxter1982exactly_ctm} and White \cite{white1992density} to
significantly speed up the renormalization of the transfer matrix.

Finally, we show how to compute various quantities with the CTMRG method and make some remarks about the spectrum of the
corner transfer matrix.
\end{abstract}

\section{Statistical mechanics on classical lattices}
For a general introduction to statistical mechanics, we refer to \cite{tong2011lectures}.

The central quantity in equilibrium statistical mechanics is the partition
function $Z$, which, for a discrete system such as a lattice, is defined as
\begin{equation}
  Z = \sum_{s} \exp{(-\beta H(s))},
\end{equation}
where the sum is over all microstates $s$, $H$ is the energy function, and
$\beta = T^{-1}$ the inverse temperature.

The probability that the system is in a particular microstate
\begin{equation}
  p(s) = \frac{\exp (-\beta H(s)) }{Z}
\end{equation}
is also called the \emph{Boltzmann weight}.

At first glance, the partition function is a simple normalization factor.
But its importance stems from the fact that since it contains all statistical information about the system,
all thermodynamic quantities can be expressed as a function of $Z$.

The energy of the system is expressed as
\begin{equation}
  \langle E \rangle = \frac{\sum_{s} H(s) \exp{(-\beta H(s))}}{Z} = -\frac{\partial}{\partial \beta} \log Z,
\end{equation}
the entropy as
\begin{equation}
  S = - \sum_{s} p(s) \log p(s) = \frac{\partial}{\partial T}(T \log Z),
\end{equation}
and the free energy as
\begin{equation}
  F = \langle E \rangle - TS = T^2 \frac{\partial}{\partial T} \log Z - T \frac{\partial}{\partial T}(T \log Z) = -T \log Z.
\end{equation}





\section{Transfer matrices of lattice models}

Transfer matrices are used to re-express the partition function of classical
lattice systems, allowing them to be solved exactly or approximated.

We will introduce the transfer matrix in the context of the 1D classical
Ising model, first introduced and solved using the transfer matrix method by Ising
\cite{ising1925beitrag} in his PhD thesis.

\subsection{1D Ising model}
Consider the 1D ferromagnetic Ising model \cite{ising1925beitrag}, defined by the energy function
\begin{equation}\label{ising_energy_function}
  H(\sigma) = -J \sum_{\langle i j \rangle} \sigma_i \sigma_j - h \sum_{i} \sigma_i.
\end{equation}
Here, we sum over nearest neighbors $\langle i j \rangle$ and the spins
$\sigma_i$ take the values $\pm 1$. $J > 0$ is the spin coupling and $h > 0$ an external magnetic field.

Assume, for the moment, that the chain
consists of $N$ spins, and apply periodic boundary conditions.
The partition function of this system is given by
\begin{equation}
  Z_{N} = \sum_{\sigma_1, \dotsc, \sigma_N \in \{-1, 1\}} \exp (-\beta H(\sigma))
\end{equation}
Exploiting the local nature of the interaction between spins, we can write
\begin{equation}
  Z_{N} = \sum_{\sigma_1, \cdots, \sigma_N \in \{-1, 1\}} \prod_{\langle i, j \rangle} e^{K\sigma_i \sigma_j + \frac{H}{2}(\sigma_i + \sigma_j)}
\end{equation}
where we defined $K \equiv \beta J$ and $H \equiv \beta h$.

Now, we define the $2 \times 2$ matrix
\begin{equation}\label{eq:transfer_matrix_1d_ising}
  T_{\sigma \sigma'} = \exp(K \sigma \sigma' + \frac{H}{2}(\sigma + \sigma')).
\end{equation}
for which a possible choice of basis is
\begin{equation}\label{eq:basis_transfer_matrix_1d}
  \bigl( \ket{\uparrow} = 1, \ket{\downarrow} = -1 \bigr) =
  \bigl(
  \begin{bmatrix}
    1 \\
    0
  \end{bmatrix},
  \begin{bmatrix}
    0 \\
    1
  \end{bmatrix}
  \bigr).
\end{equation}

In terms of this matrix, $Z_N$ is written as
\begin{equation}\label{eq:partition_function_transfer_matrix_1d}
  Z_N = \sum_{\sigma_1, \cdots, \sigma_N} T_{\sigma_1 \sigma_2} \dotsm T_{\sigma_N \sigma_1} = \tr T^N.
\end{equation}
$T$ is called the transfer matrix. In the basis of \autoref{eq:basis_transfer_matrix_1d},
it is written as
\begin{equation}
  T = \begin{bmatrix}
    e^{K+H} & e^{-K} \\
    e^{-K} & e^{K-H}
  \end{bmatrix}.
\end{equation}

$T$ is, in fact, diagonalizable. So, we can write $T^N = P D^N
P^{-1}$, where $P$ consists of the eigenvectors of $T$, and $D$ has the corresponding eigenvalues on the diagonal. By the cyclic property of the
trace, we have
\begin{equation}
  Z_N = \lambda_{+}^{N} + \lambda_{-}^{N},
\end{equation}
where
\begin{equation}
  \lambda_{\pm} = e^{K} \left[ \cosh(H) \pm \sqrt{\sinh^2(H) + e^{-4K}} \right]
\end{equation}
Thus, we have reduced the problem of finding the partition function to an
eigenvalue problem.

In the thermodynamic limit $N \to \infty$
\begin{equation}
  Z = \lim_{N \to \infty} \lambda_{+}^{N}
\end{equation}
where $\lambda_+$ is the non-degenerate largest eigenvalue (in absolute value) of $T$.
Thermodynamic quantities like the free energy per site
\begin{equation}
  \frac{F}{N} = -T \log \lambda_{+}
\end{equation}
and the magnetization per site
\begin{equation}
   M = \frac{\sum_{i}^{N} \langle \sigma_i \rangle}{N} = - \frac{1}{N}\frac{\partial F}{\partial h}
\end{equation}
can now readily be calculated.

\subsubsection{Fixed boundary conditions}\label{sec:fixed_boundary_conditions}
We may also apply fixed boundary conditions. The partition function is then written as
\begin{equation}
  Z_N = \bra{\sigma'}T^N\ket{\sigma},
\end{equation}
where $\ket{\sigma}$ and $\ket{\sigma'}$ are the right and left boundary spins.

In the large-$N$ limit, $T^N$ tends towards the projector onto the eigenspace spanned by
the eigenvector belonging to the largest eigenvalue
\begin{equation}\label{eq:largest_eigenvector}
  \ket{\lambda_+} = \lim_{N \to \infty} \frac{T^N \ket{\sigma}}{\left\lVert T^N \ket{\sigma} \right\rVert}.
\end{equation}

\autoref{eq:largest_eigenvector} is true for any $\ket{\sigma}$ that is not orthogonal to $\ket{\lambda_+}$.

The physical significance of the normalized lowest-lying eigenvector $\ket{\lambda_1}$ is
that
$\braket{\lambda_1 | \uparrow}$ and
$\braket{\lambda_1 | \downarrow}$ represent the Boltzmann weight of $\ket{\uparrow}$ and
$\ket{\downarrow}$ at the boundary of a half-infinite chain.

\subsection{2D Ising model}
Next, we treat the two-dimensional, square-lattice Ising model, which was solved in 1944 by Onsager
\cite{onsager1944two_dimensional} in a groundbreaking effort\footnote{Onsager's solution rigorously showed,
for the first time, that phase transitions could appear in simple statistical models and remained for a long time the
only exactly solved model exhibiting critical behaviour.
For a historical overview, see \cite{bhattacharjee1995fifty}.}. The energy function is still written as in
\autoref{ising_energy_function}, but now every lattice site has four neighbors.

Let $N$ be the number of columns and $l$ be the number of rows of the lattice, and assume
$l \gg N$. In the vertical direction, we apply periodic boundary conditions, as in the
one-dimensional case. In the horizontal direction, we keep an open boundary. We refer to
$N$ as the system size.

Similarly as in the 1D case, the partition function can be written as
\begin{equation}
  Z_N = \sum_{\bm{\sigma}} \prod_{\langle i, j, k, l \rangle} W(\sigma_i, \sigma_j, \sigma_k, \sigma_l)
\end{equation}
where the product runs over all groups of four spins sharing the same face. The Boltzmann weight of such a face is given by
\begin{equation}\label{eq:boltzmann_weight_face_ising_model}
  W(\sigma_i, \sigma_j, \sigma_k, \sigma_l) = \exp \left\{ \frac{K}{2} (\sigma_i \sigma_j + \sigma_j \sigma_k + \sigma_k \sigma_l + \sigma_l \sigma_i) \right\}
\end{equation}

We can express the Boltzmann weight of a configuration of the whole lattice as
a product of the Boltzmann weights of the rows
\begin{equation}
  Z_N = \sum_{\bm{\sigma}} \prod_{r = 1}^{l} W(\sigma_{1}^{r}, \sigma_{2}^{r}, \sigma_{1}^{r+1}, \sigma_{2}^{r+1}) \dots W(\sigma_{N-1}^{r}, \sigma_{N}^{r}, \sigma_{N-1}^{r+1}, \sigma_{N}^{r+1})
\end{equation}
where $\sigma_{i}^{r}$ denotes the value of the $i$th spin of row $r$.

Now, we can generalize the definition of the transfer matrix to two dimensions, by
defining it as the Boltzmann weight of an entire row
\begin{equation}\label{eq:row_to_row_transfer_matrix}
  T_{N}(\bm{\sigma}, \bm{\sigma'}) = W(\sigma_1, \sigma_2, \sigma_1', \sigma_2') \dots W(\sigma_{N-1}, \sigma_N, \sigma_{N-1}', \sigma_{N}')
\end{equation}
If we take the spin configurations of an entire row as basis vectors, $T_N$ can be written
as a matrix of dimensions $2^N \times 2^N$.

Similarly as in the one-dimensional case, the partition function now becomes
\begin{equation}\label{eq:z_n_times_infty}
  Z_N = \sum_{\bm{\sigma}} \prod_{r = 1}^{l} T_{N}(\bm{\sigma}^r, \bm{\sigma}^{r+1}) = \tr T_{N}^l
\end{equation}

In the limit of an $N \times \infty$ cylinder, the partition function is once again
determined by the largest eigenvalue\footnote{As in the 1D case, $T$ is symmetric, so it
is orthogonally diagonalizable.}.
\begin{equation}\label{largest_eigenvalue_transfer_matrix}
  Z_N = \lim_{l \to \infty} T_{N}^{l} = \lim_{l \to \infty} (\lambda_0)_{N}^{l}
\end{equation}

The partition function in the thermodynamic limit is given by
\begin{equation}
  Z = \lim_{N \to \infty} Z_N
\end{equation}

\section{Partition function of the 2D Ising model as a tensor network}
In calculating the partition function of 1D and 2D lattices, matrices of Boltzmann weights
like $W$ and $T$ play a crucial role. We have formulated them in a way that is valid for
any interaction-round-a-face (IRF) model, defined by
\begin{equation}
  H \propto \sum_{\langle i, j, k, l \rangle} W(\sigma_i, \sigma_j, \sigma_k,
  \sigma_l),
\end{equation}
where the summation is over all spins sharing a face. $W$ can contain 4-spin,
3-spin, 2-spin and 1-spin interaction terms. The Ising model is a special case of the IRF
model, with $W$ given by \autoref{eq:boltzmann_weight_face_ising_model}.

We will now express the partition function of the 2D Ising model as a tensor network. The
transfer matrix $T$ is redefined in the process. This allows us to visualize the equations
in a way that is consistent with the many other tensor network algorithms under research
today. For an introduction to tensor network notation, see
\autoref{chapter:introduction_to_tensor_networks}.

\subsection{Tensor network of the partition function of a system of four spins}

We define
\begin{equation}
  Q(\sigma_i, \sigma_j) = \exp(K \sigma_i \sigma_j)
\end{equation}
as the Boltzmann weight of the bond between $\sigma_i$ and $\sigma_j$. It is the
same as the 1D transfer matrix in \autoref{eq:transfer_matrix_1d_ising}.

The Boltzmann weight of a face $W$ decomposes into a product of Boltzmann weights of
bonds
\begin{equation}
  W(\sigma_i, \sigma_j, \sigma_k, \sigma_l) =
  Q(\sigma_i, \sigma_j)Q(\sigma_j, \sigma_l)Q(\sigma_l, \sigma_k)Q(\sigma_k, \sigma_i).
\end{equation}

It is now easy to see that the partition function is equal to the contracted tensor
network in \autoref{fig:tensor_network_4_sites}:
\begin{equation}\label{eq:tensor_network_4_sites}
  \begin{split}
    Z_{2 \times 2} & =
    \sum_{\sigma_1, \sigma_2, \sigma_3, \sigma_4} \sum_{a, b, c, d}
    \delta_{\sigma_1, a} Q(a, b) \delta_{\sigma_2, b} Q(b, c)
    \delta_{\sigma_3, c} Q(c, d) \delta_{\sigma_4, d} Q(d, a) \\
    & =
    \sum_{\sigma_1, \sigma_2, \sigma_3, \sigma_4} W(\sigma_1, \sigma_2, \sigma_3, \sigma_4).
  \end{split}
\end{equation}
where the Kronecker delta is defined as usual:
\begin{equation}
  \delta_{i j} =
  \begin{cases}
    1 & \text{if } i = j \\
    0 & \text{if } i \neq j.
  \end{cases}
\end{equation}

\begin{figure}
  \centering
  \begin{tikzpicture}
	\begin{pgfonlayer}{nodelayer}
		\node [style=delta-tensor] (0) at (-1, 0) {};
		\node [style=delta-tensor] (1) at (1, 0) {};
		\node [style=delta-tensor] (2) at (1, 2) {};
		\node [style=delta-tensor, label={$\delta$}] (3) at (-1, 2) {};
		\node [style=q-tensor] (4) at (1, 1) {};
		\node [style=q-tensor, label={$Q$}] (5) at (0, 2) {};
		\node [style=q-tensor] (6) at (-1, 1) {};
		\node [style=q-tensor] (7) at (0, 0) {};
		\node [style=white no border] (8) at (-2.5, 1) {$Z_{2 \times 2} = $};
	\end{pgfonlayer}
	\begin{pgfonlayer}{edgelayer}
		\draw [style=simple] (3) to (6);
		\draw [style=simple] (6) to (0);
		\draw [style=simple] (0) to (7);
		\draw [style=simple] (7) to (1);
		\draw [style=simple] (1) to (4);
		\draw [style=simple] (4) to (2);
		\draw [style=simple] (2) to (5);
		\draw [style=simple] (5) to (3);
	\end{pgfonlayer}
\end{tikzpicture}
  \caption{A tensor network representation of the partition function of the Ising model on
  a $2 \times 2$ lattice. See \autoref{eq:tensor_network_4_sites}.}
  \label{fig:tensor_network_4_sites}
\end{figure}


\subsection{Thermodynamic limit}
\begin{figure}
  \centering
  \begin{tikzpicture}
	\begin{pgfonlayer}{nodelayer}
		\node [style=q-tensor, label={$Q$}] (0) at (0, 0) {};
		\node [style=white no border] (1) at (-1, 0) {};
		\node [style=white no border] (2) at (1, 0) {$=$};
		\node [style=p-tensor-right, label={$P$}] (3) at (2, 0) {};
		\node [style=p-tensor-left, label={$P$}] (4) at (2.75, 0) {};
		\node [style=white no border] (5) at (3.75, 0) {};
	\end{pgfonlayer}
	\begin{pgfonlayer}{edgelayer}
		\draw [style=simple] (0) to (1);
		\draw [style=simple] (0) to (2);
		\draw [style=simple] (3) to (4);
		\draw [style=simple] (4) to (5);
		\draw [style=simple] (3) to (2);
	\end{pgfonlayer}
\end{tikzpicture}

  \caption{Graphical form of \autoref{eq:q_to_p}.}
  \label{fig:q_to_p}
\end{figure}

\begin{figure}
  \centering
  \begin{tikzpicture}
	\begin{pgfonlayer}{nodelayer}
		\node [style=delta-tensor] (0) at (-2, 0) {};
		\node [style=p-tensor-up] (1) at (-2, -0.75) {};
		\node [style=p-tensor-left] (2) at (-2.75, 0) {};
		\node [style=p-tensor-down] (3) at (-2, 0.75) {};
		\node [style=p-tensor-right] (4) at (-1.25, 0) {};
		\node [style=white no border] (5) at (-0.5, 0) {$=$};
		\node [style=white no border] (6) at (-2, 1.5) {};
		\node [style=white no border] (7) at (-3.5, 0) {};
		\node [style=white no border] (8) at (-2, -1.5) {};
		\node [style=a-tensor] (9) at (0.5, 0) {$a$};
		\node [style=white no border] (10) at (0.5, 1) {};
		\node [style=white no border] (11) at (1.5, 0) {};
		\node [style=white no border] (12) at (0.5, -1) {};
	\end{pgfonlayer}
	\begin{pgfonlayer}{edgelayer}
		\draw [style=simple] (9) to (11);
		\draw [style=simple] (9) to (10);
		\draw [style=simple] (9) to (12);
		\draw [style=simple] (9) to (5);
		\draw [style=simple] (5) to (4);
		\draw [style=simple] (4) to (0);
		\draw [style=simple] (0) to (3);
		\draw [style=simple] (0) to (1);
		\draw [style=simple] (0) to (2);
		\draw [style=simple] (2) to (7);
		\draw [style=simple] (1) to (8);
		\draw [style=simple] (3) to (6);
	\end{pgfonlayer}
\end{tikzpicture}

  \caption{Graphical form of \autoref{eq:a_tensor}.}
  \label{fig:a_tensor}
\end{figure}

We define the matrix $P$ by
\begin{equation}\label{eq:q_to_p}
  P^2 = Q.
\end{equation}
as in \autoref{fig:q_to_p}. This allows us to write the partition function of an arbitrary
$N \times l$ square lattice as a tensor network of a single recurrent tensor $a_{i j k
l}$, given by
\begin{equation}\label{eq:a_tensor}
  a_{i j k l} = \sum_{a, b, c, d} \delta_{a b c d} P_{i a} P_{j b} P_{k c} P_{l d},
\end{equation}
where the generalization of the Kronecker delta is defined as
\begin{equation}
  \delta_{i_1 \dots i_n} =
  \begin{cases}
    1 & \text{if } i_1 = \ldots = i_n \\
    0 & \text{otherwise.}
  \end{cases}
\end{equation}

See \autoref{fig:a_tensor} and \autoref{fig:2d_ising_as_tensor_network}. At the edges and
corners, we define suitable tensors of rank 3 and 2, which we will also denote by $a$:
\begin{align*}
  a_{i j k} &= \sum_{a b c} \delta_{a b c} P_{i a} P_{j b} P_{k c}, \\
  a_{i j} &= \sum_{a b} \delta_{a b} P_{i a} P_{j b}.
\end{align*}

The challenge is to approximate this tensor network in the thermodynamic limit.

\begin{figure}
  \centering
  \begin{tikzpicture}
	\begin{pgfonlayer}{nodelayer}
		\node [style=delta-tensor] (0) at (2, -1) {};
		\node [style=white no border, rotate=90] (1) at (2, -2.5) {$\dots$};
		\node [style=p-tensor-down] (2) at (2, -0.25) {};
		\node [style=p-tensor-up] (3) at (2, -1.75) {};
		\node [style=p-tensor-left] (4) at (1.25, -1) {};
		\node [style=p-tensor-right] (5) at (2.75, -1) {};
		\node [style=white no border] (6) at (0.5, -1) {$\dots$};
		\node [style=p-tensor-left] (7) at (1.25, 1) {};
		\node [style=p-tensor-right] (8) at (2.75, 1) {};
		\node [style=p-tensor-down] (9) at (2, 1.75) {};
		\node [style=delta-tensor] (10) at (2, 1) {};
		\node [style=p-tensor-up] (11) at (2, 0.25) {};
		\node [style=p-tensor-left] (12) at (3.25, -1) {};
		\node [style=p-tensor-right] (13) at (4.75, -1) {};
		\node [style=p-tensor-down] (14) at (4, -0.25) {};
		\node [style=delta-tensor] (15) at (4, -1) {};
		\node [style=p-tensor-up] (16) at (4, -1.75) {};
		\node [style=p-tensor-left] (17) at (3.25, 1) {};
		\node [style=p-tensor-right] (18) at (4.75, 1) {};
		\node [style=p-tensor-down] (19) at (4, 1.75) {};
		\node [style=delta-tensor] (20) at (4, 1) {};
		\node [style=p-tensor-up] (21) at (4, 0.25) {};
		\node [style=white no border, rotate=90] (22) at (4, -2.5) {$\dots$};
		\node [style=white no border] (23) at (5.5, -1) {$\dots$};
		\node [style=white no border] (24) at (5.5, 1) {$\dots$};
		\node [style=white no border, rotate=90] (25) at (4, 2.5) {$\dots$};
		\node [style=white no border, rotate=90] (26) at (2, 2.5) {$\dots$};
		\node [style=white no border] (27) at (0.5, 1) {$\dots$};
		\node [style=delta-tensor] (28) at (-1.5, -1) {};
		\node [style=delta-tensor] (29) at (-1.5, 1) {};
		\node [style=delta-tensor] (30) at (-3.5, -1) {};
		\node [style=delta-tensor] (31) at (-3.5, 1) {};
		\node [style=q-tensor] (32) at (-2.5, -1) {};
		\node [style=q-tensor] (33) at (-1.5, 0) {};
		\node [style=q-tensor] (34) at (-2.5, 1) {};
		\node [style=q-tensor] (35) at (-3.5, 0) {};
		\node [style=white no border] (36) at (-0.5, 1) {$\dots$};
		\node [style=white no border] (37) at (-0.5, -1) {$\dots$};
		\node [style=white no border, rotate=90] (38) at (-1.5, -2) {$\dots$};
		\node [style=white no border, rotate=90] (39) at (-3.5, -2) {$\dots$};
		\node [style=white no border, rotate=90] (40) at (-3.5, 2) {$\dots$};
		\node [style=white no border, rotate=90] (41) at (-1.5, 2) {$\dots$};
		\node [style=white no border] (42) at (-4.5, 1) {$\dots$};
		\node [style=white no border] (43) at (-4.5, -1) {$\dots$};
		\node [style=white no border] (44) at (0, 0) {$=$};
		\node [style=a-tensor] (45) at (-3, -4.25) {$a$};
		\node [style=a-tensor] (46) at (-2, -4.25) {$a$};
		\node [style=a-tensor] (47) at (-3, -5.25) {$a$};
		\node [style=a-tensor] (48) at (-2, -5.25) {$a$};
		\node [style=white no border] (49) at (-1, -4.25) {$\dots$};
		\node [style=white no border, rotate=90] (50) at (-2, -3.25) {$\dots$};
		\node [style=white no border, rotate=90] (51) at (-3, -3.25) {$\dots$};
		\node [style=white no border] (52) at (-4, -4.25) {$\dots$};
		\node [style=white no border] (53) at (-4, -5.25) {$\dots$};
		\node [style=white no border, rotate=90] (54) at (-3, -6.25) {$\dots$};
		\node [style=white no border, rotate=90] (55) at (-2, -6.25) {$\dots$};
		\node [style=white no border] (56) at (-1, -5.25) {$\dots$};
		\node [style=white no border] (57) at (-4.5, -4.75) {$=$};
	\end{pgfonlayer}
	\begin{pgfonlayer}{edgelayer}
		\draw [style=simple] (5) to (0);
		\draw [style=simple] (0) to (2);
		\draw [style=simple] (0) to (3);
		\draw [style=simple] (0) to (4);
		\draw [style=simple] (4) to (6);
		\draw [style=simple] (3) to (1);
		\draw [style=simple] (8) to (10);
		\draw [style=simple] (10) to (9);
		\draw [style=simple] (10) to (11);
		\draw [style=simple] (10) to (7);
		\draw [style=simple] (13) to (15);
		\draw [style=simple] (15) to (14);
		\draw [style=simple] (15) to (16);
		\draw [style=simple] (15) to (12);
		\draw [style=simple] (18) to (20);
		\draw [style=simple] (20) to (19);
		\draw [style=simple] (20) to (21);
		\draw [style=simple] (20) to (17);
		\draw [style=simple] (9) to (26);
		\draw [style=simple] (19) to (25);
		\draw [style=simple] (18) to (24);
		\draw [style=simple] (7) to (27);
		\draw [style=simple] (13) to (23);
		\draw [style=simple] (16) to (22);
		\draw [style=simple] (11) to (2);
		\draw [style=simple] (21) to (14);
		\draw [style=simple] (5) to (12);
		\draw [style=simple] (8) to (17);
		\draw [style=simple] (30) to (32);
		\draw [style=simple] (32) to (28);
		\draw [style=simple] (28) to (33);
		\draw [style=simple] (33) to (29);
		\draw [style=simple] (29) to (34);
		\draw [style=simple] (34) to (31);
		\draw [style=simple] (31) to (35);
		\draw [style=simple] (35) to (30);
		\draw [style=simple] (28) to (37);
		\draw [style=simple] (28) to (38);
		\draw [style=simple] (30) to (39);
		\draw [style=simple] (30) to (43);
		\draw [style=simple] (31) to (42);
		\draw [style=simple] (31) to (40);
		\draw [style=simple] (29) to (41);
		\draw [style=simple] (29) to (36);
		\draw [style=simple] (45) to (46);
		\draw [style=simple] (45) to (47);
		\draw [style=simple] (47) to (48);
		\draw [style=simple] (48) to (46);
		\draw [style=simple] (46) to (49);
		\draw [style=simple] (48) to (56);
		\draw [style=simple] (48) to (55);
		\draw [style=simple] (47) to (54);
		\draw [style=simple] (47) to (53);
		\draw [style=simple] (45) to (52);
		\draw [style=simple] (45) to (51);
		\draw [style=simple] (46) to (50);
	\end{pgfonlayer}
\end{tikzpicture}
  \caption{$Z_{N \times l}$ can be written as a contracted tensor network of $N \times l$
  copies of the tensor $a$.}
  \label{fig:2d_ising_as_tensor_network}
\end{figure}

\subsection{The transfer matrix as a tensor network}
With our newfound representation of the partition function as a tensor network, we can
redefine the row-to-row transfer matrix from
\autoref{eq:row_to_row_transfer_matrix} as the tensor network expressed in
\autoref{fig:transfer_matrix_as_tensor_network}. For all $l$, it is still true that
\begin{equation}
  Z_{N \times l} = \tr T_{N}^{l} = \sum_{i = 1}^{2^N} \lambda_{i}^{l},
\end{equation}
so the eigenvalues must be the same. That means that the new definition of the transfer
matrix is related to the old one by a basis transformation
\begin{equation}
  T_{\text{new}} = P T_{\text{old}} P^{T}.
\end{equation}

\begin{figure}
  \centering
  \begin{tikzpicture}
	\begin{pgfonlayer}{nodelayer}
		\node [style={a-tensor}] (0) at (-1, -0) {$a$};
		\node [style={a-tensor}] (1) at (0, -0) {$a$};
		\node [style={white no border}] (2) at (1, -0) {$\dots$};
		\node [style={a-tensor}] (3) at (2, -0) {$a$};
		\node [style={a-tensor}] (4) at (3, -0) {$a$};
		\node [style={white no border}] (5) at (3, 1) {};
		\node [style={white no border}] (6) at (2, 1) {};
		\node [style={white no border}] (7) at (0, 1) {};
		\node [style={white no border}] (8) at (-1, 1) {};
		\node [style={white no border}] (9) at (-1, -1) {};
		\node [style={white no border}] (10) at (0, -1) {};
		\node [style={white no border}] (11) at (2, -1) {};
		\node [style={white no border}] (12) at (3, -1) {};
	\end{pgfonlayer}
	\begin{pgfonlayer}{edgelayer}
		\draw [style=simple] (0) to (1);
		\draw [style=simple] (1) to (2);
		\draw [style=simple] (2) to (3);
		\draw [style=simple] (3) to (4);
		\draw [style=simple] (4) to (5);
		\draw [style=simple] (3) to (6);
		\draw [style=simple] (1) to (7);
		\draw [style=simple] (0) to (8);
		\draw [style=simple] (0) to (9);
		\draw [style=simple] (1) to (10);
		\draw [style=simple] (3) to (11);
		\draw [style=simple] (4) to (12);
	\end{pgfonlayer}
\end{tikzpicture}

  \caption{The definition of $T_N$ as a network of $N$ copies of the tensor $a$.}
  \label{fig:transfer_matrix_as_tensor_network}
\end{figure}

\section{Transfer matrix renormalization group}\label{sec:tmrg}
There is a deep connection between quantum mechanical lattice systems in $d$ dimensions
and classical lattice systems in $d + 1$ dimensions. Via the imaginary time path integral
formulation, the partition function of a one-dimensional quantum system can be written as
the partition function of an effective two-dimensional classical system. The ground state
of the quantum system corresponds to the largest eigenvector of the transfer matrix of this corresponding
classical system.

For more on the quantum-classical correspondence, see
\autoref{chapter:correspondence_quantum_classical}.

\subsection{The infinite system algorithm for the transfer matrix}
Nishino \cite{nishino1995density, nishino1996corner} was the first to apply density matrix
renormalization group methods in the context of two-dimensional classical
lattices.

Analogous to the infinite system DMRG algorithm for approximating the Hamiltonian of
quantum spin chains, our goal is to
approximate the transfer matrix in the thermodynamic limit as well as possible within a
restriced number of basis states $m$. We will do this by adding a single site at a time,
and truncating the dimension from $2m$ to $m$ at each iteration.

For simplicity, we assume that, at the start of the algorithm, the transfer matrix already
has dimension $m$. We call this transfer matrix $P_N$.

We note that this initial $P_N$ for a system with a free boundary can be obtained by contracting $a$-tensors, until
$P_N$ becomes of size $m \times m$. See \autoref{fig:tmrg_initial_half_row_transfer_matrix}.

To specify fixed instead of open boundary conditions, we may use as boundary tensor a
slightly modified version of the three-legged version of $a$, namely
\begin{equation}\label{eq:boundary_tensor_three_legged}
  a_{i j k}^{\sigma} = \sum_{a b c} \delta_{\sigma a b c} P_{i a}P_{j b}P_{k c},
\end{equation}
that represents an edge site with spin fixed at $\sigma$.

We enlarge the system with one site by contracting with an additional $a$-tensor,
obtaining $P_{N + 1}$. See the first network in
\autoref{fig:tmrg_add_site_and_renormalize}.

In order to find the best projection from $2m$ basis states back to $m$, we embed the
system in an environment that is the mirror image of the system we presently have. We call
this matrix $T_{2N + 2}$. It represents the transfer matrix of $2N + 2$ sites. We find the
largest eigenvalue and corresponding eigenvector, as shown in
\autoref{fig:tmrg_eigenvalue_equation}.

The equivalent of the \textit{reduced density matrix of a
block} in the classical case is:
\begin{equation}\label{eq:reduced_density_matrix_classical_case}
  \rho_{N + 1} = \sum_{\sigma_B} \braket{\sigma_B | \lambda_0}\braket{\lambda_0 |
  \sigma_B},
\end{equation}
where we have summed over all the degrees of freedom of one of the half-row transfer
matrices $P_{N+1}$. See the first step of \autoref{fig:tmrg_reduced_density_matrix}.

The optimal renormalization
\begin{equation}
  \widetilde{P}_{N+1} = O P_{N + 1} O^{\dagger}
\end{equation}
is obtained by diagonalizing $\rho_{N + 1}$ and keeping the eigenvectors corresponding
to the $m$ largest eigenvalues. See the second step of \autoref{fig:tmrg_reduced_density_matrix}.

With this blocking procedure, we can successively find
\begin{equation}
    P_{N + 1} \rightarrow P_{N + 2} \rightarrow \dots,
\end{equation}
until we have reached some termination condition.\footnote{The termination condition for the infinite-system algorithm
is discussed in \autoref{sec:convergence_criteria}.}

\begin{figure}
  \centering
  \begin{tikzpicture}
	\begin{pgfonlayer}{nodelayer}
		\node [style=a-tensor] (0) at (-4, 0) {};
		\node [style=a-tensor] (1) at (-3.25, 0) {};
		\node [style=a-tensor] (2) at (-2.5, 0) {};
		\node [style=a-tensor] (3) at (-1.75, 0) {};
		\node [style=none] (4) at (-4, 0.5) {};
		\node [style=none] (5) at (-4, -0.5) {};
		\node [style=none] (6) at (-3.25, 0.5) {};
		\node [style=none] (7) at (-3.25, -0.5) {};
		\node [style=none] (8) at (-2.5, 0.5) {};
		\node [style=none] (9) at (-2.5, -0.5) {};
		\node [style=none] (10) at (-1.75, 0.5) {};
		\node [style=none] (11) at (-1.75, -0.5) {};
		\node [style=none] (12) at (-1, 0) {$=$};
		\node [style=generic2] (13) at (-0.25, 0) {};
		\node [style=none] (14) at (-0.25, 0.5) {};
		\node [style=none] (15) at (-0.25, -0.5) {};
	\end{pgfonlayer}
	\begin{pgfonlayer}{edgelayer}
		\draw [style=simple] (0) to (1);
		\draw [style=simple] (1) to (2);
		\draw [style=simple] (2) to (3);
		\draw [style=simple] (3) to (10.center);
		\draw [style=simple] (3) to (11.center);
		\draw [style=simple] (2) to (8.center);
		\draw [style=simple] (2) to (9.center);
		\draw [style=simple] (1) to (6.center);
		\draw [style=simple] (1) to (7.center);
		\draw [style=simple] (0) to (4.center);
		\draw [style=simple] (0) to (5.center);
		\draw [style=thick leg] (13) to (14.center);
		\draw [style=thick leg] (13) to (15.center);
	\end{pgfonlayer}
\end{tikzpicture}
  \caption{A good starting point for the half-row transfer $P_N$ is obtained by
  contracting a couple of $a$-tensors, until $P_N$ reaches dimension $m$.}
  \label{fig:tmrg_initial_half_row_transfer_matrix}
\end{figure}

\begin{figure}
  \centering
  \begin{tikzpicture}
	\begin{pgfonlayer}{nodelayer}
		\node [style=generic2] (0) at (-3.25, 0) {};
		\node [style=none] (1) at (-3.25, 0.5) {};
		\node [style=none] (2) at (-3.25, -0.5) {};
		\node [style=none] (3) at (-2, 0) {$\longrightarrow$};
		\node [style=none] (4) at (-1, 0.5) {};
		\node [style=generic2] (5) at (-1, 0) {};
		\node [style=none] (6) at (-1, -0.5) {};
		\node [style=a-tensor] (7) at (0, 0) {};
		\node [style=none] (8) at (0, 0.5) {};
		\node [style=none] (9) at (0, -0.5) {};
		\node [style=none] (10) at (-2.75, 0) {};
		\node [style=none] (11) at (0.5, 0) {};
		\node [style=none] (12) at (0.5, 0) {};
		\node [style=none] (13) at (3.75, 0) {};
		\node [style=none] (14) at (2.25, -0.5) {};
		\node [style=none] (15) at (3.75, 0) {};
		\node [style=none] (16) at (3.25, -0.5) {};
		\node [style=none] (17) at (2.25, 0.5) {};
		\node [style=a-tensor] (18) at (3.25, 0) {};
		\node [style=none] (19) at (3.25, 0.5) {};
		\node [style=generic2] (20) at (2.25, 0) {};
		\node [style=none] (21) at (1.25, 0) {$\longrightarrow$};
		\node [style=isometry, rotate=90, anchor=west, minimum width=1.25cm] (22) at (2.75, 0.5) {};
		\node [style=none] (23) at (2.75, 1.25) {};
		\node [style=none] (24) at (2.75, -1.25) {};
		\node [style=isometry, rotate=-90, anchor=west, minimum width=1.25cm] (25) at (2.75, -0.5) {};
	\end{pgfonlayer}
	\begin{pgfonlayer}{edgelayer}
		\draw [style=thick leg] (0) to (1.center);
		\draw [style=thick leg] (0) to (2.center);
		\draw [style=thick leg] (5) to (4.center);
		\draw [style=thick leg] (5) to (6.center);
		\draw [style=simple] (5) to (7);
		\draw [style=simple] (7) to (8.center);
		\draw [style=simple] (7) to (9.center);
		\draw [style=simple] (0) to (10.center);
		\draw [style=simple] (7) to (11.center);
		\draw [style=thick leg] (20) to (17.center);
		\draw [style=thick leg] (20) to (14.center);
		\draw [style=simple] (20) to (18);
		\draw [style=simple] (18) to (19.center);
		\draw [style=simple] (18) to (16.center);
		\draw [style=simple] (18) to (13.center);
		\draw [style=thick leg] (22.center) to (23.center);
		\draw [style=thick leg] (25.center) to (24.center);
	\end{pgfonlayer}
\end{tikzpicture}
  \caption{In the first step, $P_{N + 1}$ is obtained by contracting the current half-row
  transfer matrix $P_N$ with an additional $a$-tensor.
  In the second step, $P_{N + 1}$ is truncated back to an $m$-dimensional matrix, with the
  optimal low-rank approximation given by keeping the basis states corresponding to the
  $m$ largest eigenvalues of the density matrix. See
  \autoref{fig:tmrg_reduced_density_matrix}. }
  \label{fig:tmrg_add_site_and_renormalize}
\end{figure}

\begin{figure}
  \centering
  \begin{tikzpicture}
	\begin{pgfonlayer}{nodelayer}
		\node [style=generic2] (0) at (-2.25, 1) {};
		\node [style=a-tensor] (1) at (-2.25, 0) {};
		\node [style=a-tensor] (2) at (-2.25, -1) {};
		\node [style=generic2] (3) at (-2.25, -2) {};
		\node [style=none] (4) at (-2.75, -2) {};
		\node [style=none] (5) at (-1.5, -2) {};
		\node [style=none] (6) at (-1.5, -1) {};
		\node [style=none] (7) at (-2.75, -1) {};
		\node [style=none] (8) at (-2.75, 0) {};
		\node [style=none] (9) at (-1.5, 0) {};
		\node [style=none] (10) at (-1.5, 1) {};
		\node [style=none] (11) at (-2.75, 1) {};
		\node [style={largest_eigenvector}] (12) at (-1.5, -0.5) {};
		\node [style=none] (13) at (-0.75, -0.5) {$=$};
		\node [style=none] (14) at (0, -0.5) {$\lambda_0$};
		\node [style={largest_eigenvector}] (15) at (0.75, -0.5) {};
		\node [style=none] (16) at (0.25, 1) {};
		\node [style=none] (17) at (0.25, 0) {};
		\node [style=none] (18) at (0.25, -1) {};
		\node [style=none] (19) at (0.25, -2) {};
		\node [style=none] (20) at (0.75, 1) {};
		\node [style=none] (21) at (0.75, 0) {};
		\node [style=none] (22) at (0.75, -1) {};
		\node [style=none] (23) at (0.75, -2) {};
	\end{pgfonlayer}
	\begin{pgfonlayer}{edgelayer}
		\draw [style=simple] (1) to (9.center);
		\draw [style=simple] (1) to (8.center);
		\draw [style=simple] (2) to (7.center);
		\draw [style=simple] (2) to (6.center);
		\draw [style=simple] (2) to (3);
		\draw [style=simple] (1) to (2);
		\draw [style=simple] (1) to (0);
		\draw [style=thick leg] (0) to (10.center);
		\draw [style=thick leg] (0) to (11.center);
		\draw [style=thick leg] (3) to (5.center);
		\draw [style=thick leg] (3) to (4.center);
		\draw [style=thick leg] (19.center) to (23.center);
		\draw [style=simple] (18.center) to (22.center);
		\draw [style=simple] (17.center) to (21.center);
		\draw [style=thick leg] (16.center) to (20.center);
	\end{pgfonlayer}
\end{tikzpicture}
  \caption{Equation for the lowest-lying eigenvector of the row-to-row transfer matrix $T_{2N +
  2}$.}
  \label{fig:tmrg_eigenvalue_equation}
\end{figure}

\begin{figure}
  \centering
  \begin{tikzpicture}
	\begin{pgfonlayer}{nodelayer}
		\node [style={largest_eigenvector}, rotate=90] (0) at (-4.5, 1) {};
		\node [style=none] (1) at (-5, 1) {};
		\node [style=none] (2) at (-4, 1) {};
		\node [style=none] (3) at (-6, 1) {};
		\node [style=none] (4) at (-3, 1) {};
		\node [style={largest_eigenvector}, rotate=90] (5) at (-4.5, 0) {};
		\node [style=none] (6) at (-6, 0) {};
		\node [style=none] (7) at (-5, 0) {};
		\node [style=none] (8) at (-4, 0) {};
		\node [style=none] (9) at (-3, 0) {};
		\node [style=white no border small] (10) at (-6, 0.5) {};
		\node [style=white no border small] (11) at (-5, 0.5) {};
		\node [style=none] (12) at (-2, 0.5) {$=$};
		\node [style=square] (13) at (-1, 0.5) {};
		\node [style=white no border small] (14) at (-1, 0.75) {};
		\node [style=white no border small] (15) at (-1, 0.25) {};
		\node [style=none] (16) at (-1.5, 0.75) {};
		\node [style=none] (17) at (-1.5, 0.25) {};
		\node [style=none] (18) at (-0.5, 0.25) {};
		\node [style=none] (19) at (-0.5, 0.75) {};
		\node [style=none] (20) at (0, 0.5) {$\longrightarrow$};
		\node [style=isometry] (21) at (1, 0.5) {};
		\node [style=singular values matrix] (22) at (1.75, 0.5) {};
		\node [style=none] (23) at (1, 0.75) {};
		\node [style=none] (24) at (0.5, 0.75) {};
		\node [style=none] (25) at (0.5, 0.25) {};
		\node [style=none] (26) at (1, 0.25) {};
		\node [style=none] (27) at (3, 0.25) {};
		\node [style=isometry, rotate=180] (28) at (2.5, 0.5) {};
		\node [style=none] (29) at (2.5, 0.75) {};
		\node [style=none] (30) at (3, 0.75) {};
		\node [style=none] (31) at (2.5, 0.25) {};
	\end{pgfonlayer}
	\begin{pgfonlayer}{edgelayer}
		\draw [style=thick leg] (4.center) to (9.center);
		\draw [style=thick leg] (6.center) to (10);
		\draw [style=simple] (2.center) to (8.center);
		\draw [style=simple] (11) to (7.center);
		\draw [style=simple] (1.center) to (11);
		\draw [style=thick leg] (3.center) to (10);
		\draw [style=simple] (17.center) to (15);
		\draw [style=simple] (18.center) to (15);
		\draw [style=thick leg] (14) to (19.center);
		\draw [style=thick leg] (14) to (16.center);
		\draw [style=thick leg] (21.center) to (22.center);
		\draw [style=thick leg] (24.center) to (23.center);
		\draw [style=simple] (25.center) to (26.center);
		\draw [style=thick leg] (30.center) to (29.center);
		\draw [style=simple] (27.center) to (31.center);
		\draw [style=thick leg] (22.center) to (28.center);
	\end{pgfonlayer}
\end{tikzpicture}
  \caption{Graphical form of
  \autoref{eq:reduced_density_matrix_classical_case}.
  In the second step, $\rho_{N + 1}$ is diagonalized and only the eigenvectors
  corresponding to the $m$ largest eigenvalues are kept.
  }
  \label{fig:tmrg_reduced_density_matrix}
\end{figure}

\subsection{Physical interpretation of the reduced density matrix}
Generalizing the remarks from \autoref{sec:fixed_boundary_conditions} to the
two-dimensional case, we see that the normalized lowest-lying eigenvector of the transfer
matrix $T_N$ contains the Boltzmann weights of spin configurations on the boundary of a
half-infinite $N \times \infty$ lattice.

Therefore, the classical equivalent of the quantum mechanical reduced density matrix,
given by \autoref{eq:reduced_density_matrix_classical_case}, and by the first network in
\autoref{fig:tmrg_reduced_density_matrix}, represents the Boltzmann weights of
configurations along a cut in an $N \times \infty$ lattice.

Nishino and Okunishi \cite{nishino1996corner}, drawing on ideas from Baxter, realized the
Boltzmann weights of configurations along this cut could be obtained by employing
\textit{corner transfer matrices}, making it unneccessary to solve the eigenvalue problem
in \autoref{fig:tmrg_eigenvalue_equation}. Their method, called the Corner Transfer Matrix
Renormalization Group method, consumes far less resources while maintaining precision. For
this reason, it is the method of choice for most of the simulations in this thesis.


\section{Corner transfer matrix renormalization group}
\subsection{Corner transfer matrices}

The concept of corner transfer matrices for 2D lattices was first introduced by
Baxter \cite{baxter1968dimers, baxter1978variational, baxter1982exactly_ctm}.
Whereas the row-to-row transfer matrix \autoref{eq:row_to_row_transfer_matrix}
corresponds to adding a row to the lattice, the corner transfer matrix adds
a quadrant of spins. It was originally defined by Baxter as

\begin{equation}\label{eq:corner_transfer_matrix}
  A_{\bm{\sigma}, \bm{\sigma'}} =
  \begin{cases}
    \sum \prod_{\langle i, j, k, l \rangle} W(\sigma_i, \sigma_j, \sigma_k, \sigma_l) & \text{if } \sigma_{1} = \sigma_{1}' \\
    0 & \text{if } \sigma_{1} \neq \sigma_{1}'.
  \end{cases}
\end{equation}
Here, the product runs over groups of four spins that share the same face, and
the sum is over all spins in the interior of the quadrant.

In a symmetric and isotropic model such as the Ising model, we have
\begin{equation}
  W(a, b, c, d) = W(b, a, d, c) = W(c, a, d, b) = W(d, c, b, a)
\end{equation}
and the partition of an $N \times N$ lattice is expressed as
\begin{equation}\label{eq:z_n_times_n}
  Z_{N \times N} = \tr A^4 = \sum_{\alpha = 1}^{2^N} \nu_{\alpha}^4,
\end{equation}
where $\nu_{\alpha}$ are the eigenvalues of $A$.

In the thermodynamic limit, this partition function is equal to
the partition function of an $N \times \infty$ lattice, given by
\autoref{eq:z_n_times_infty}.


\subsection{Corner transfer matrix as a tensor network}
Similarly to how we redefined the row-to-row transfer matrix
(\autoref{eq:row_to_row_transfer_matrix}) as the tensor network in
\autoref{fig:transfer_matrix_as_tensor_network}, we can redefine the corner transfer
matrix (\autoref{eq:corner_transfer_matrix}) as the tensor network in
\autoref{fig:corner_transfer_matrix_as_tensor_network}. Again, the new and old definitions
of $A$ are related by a basis tranformation
\begin{equation}
  A_{\text{new}} = P A_{\text{old}} P^T.
\end{equation}

The partition function, as in \autoref{eq:z_n_times_n}, is given by the tensor network in
\autoref{fig:partition_function_tensor_network}.

\begin{figure}
  \centering
  \begin{tikzpicture}
	\begin{pgfonlayer}{nodelayer}
		\node [style=a-tensor] (0) at (-3, 2) {};
		\node [style=a-tensor] (1) at (-2.25, 2) {};
		\node [style=a-tensor] (2) at (-3, 1.25) {};
		\node [style=a-tensor] (3) at (-2.25, 1.25) {};
		\node [style=white no border] (4) at (-1.25, 2) {$\dots$};
		\node [style=white no border, rotate=90] (5) at (-2.25, 0.25) {$\dots$};
		\node [style=white no border] (6) at (-1.25, 1.25) {$\dots$};
		\node [style=white no border, rotate=90] (7) at (-3, 0.25) {$\dots$};
		\node [style=a-tensor] (8) at (0.5, 1.25) {};
		\node [style=a-tensor] (9) at (-0.25, 2) {};
		\node [style=a-tensor] (10) at (0.5, 2) {};
		\node [style=white no border, rotate=90] (11) at (0.5, 0.25) {$\dots$};
		\node [style=white no border, rotate=90] (12) at (-0.25, 0.25) {$\dots$};
		\node [style=a-tensor] (13) at (-0.25, 1.25) {};
		\node [style=a-tensor] (14) at (-3, -1.5) {};
		\node [style=a-tensor] (15) at (-0.25, -1.5) {};
		\node [style=a-tensor] (16) at (-2.25, -0.75) {};
		\node [style=white no border] (17) at (-1.25, -1.5) {$\dots$};
		\node [style=white no border] (18) at (-1.25, -0.75) {$\dots$};
		\node [style=a-tensor] (19) at (-0.25, -0.75) {};
		\node [style=a-tensor] (20) at (0.5, -1.5) {};
		\node [style=a-tensor] (21) at (0.5, -0.75) {};
		\node [style=a-tensor] (22) at (-3, -0.75) {};
		\node [style=a-tensor] (23) at (-2.25, -1.5) {};
		\node [style=white no border] (24) at (1.5, 2) {};
		\node [style=white no border] (25) at (1.5, 1.25) {};
		\node [style=white no border] (26) at (1.5, -0.75) {};
		\node [style=white no border] (27) at (1.5, -1.5) {};
		\node [style=white no border] (28) at (0.5, -2.5) {};
		\node [style=white no border] (29) at (-0.25, -2.5) {};
		\node [style=white no border] (30) at (-2.25, -2.5) {};
		\node [style=white no border] (31) at (-3, -2.5) {};
	\end{pgfonlayer}
	\begin{pgfonlayer}{edgelayer}
		\draw [style=simple] (0) to (1);
		\draw [style=simple] (0) to (2);
		\draw [style=simple] (2) to (3);
		\draw [style=simple] (3) to (1);
		\draw [style=simple] (1) to (4);
		\draw [style=simple] (3) to (6);
		\draw [style=simple] (3) to (5);
		\draw [style=simple] (2) to (7);
		\draw [style=simple] (10) to (9);
		\draw [style=simple] (10) to (8);
		\draw [style=simple] (8) to (13);
		\draw [style=simple] (13) to (9);
		\draw [style=simple] (13) to (12);
		\draw [style=simple] (8) to (11);
		\draw [style=simple] (4) to (9);
		\draw [style=simple] (13) to (6);
		\draw [style=simple] (22) to (16);
		\draw [style=simple] (22) to (14);
		\draw [style=simple] (14) to (23);
		\draw [style=simple] (23) to (16);
		\draw [style=simple] (16) to (18);
		\draw [style=simple] (23) to (17);
		\draw [style=simple] (21) to (19);
		\draw [style=simple] (21) to (20);
		\draw [style=simple] (20) to (15);
		\draw [style=simple] (15) to (19);
		\draw [style=simple] (18) to (19);
		\draw [style=simple] (15) to (17);
		\draw [style=simple] (22) to (7);
		\draw [style=simple] (16) to (5);
		\draw [style=simple] (19) to (12);
		\draw [style=simple] (21) to (11);
		\draw [style=simple] (10) to (24);
		\draw [style=simple] (8) to (25);
		\draw [style=simple] (21) to (26);
		\draw [style=simple] (20) to (27);
		\draw [style=simple] (20) to (28);
		\draw [style=simple] (15) to (29);
		\draw [style=simple] (23) to (30);
		\draw [style=simple] (14) to (31);
	\end{pgfonlayer}
\end{tikzpicture}
  \caption{Corner transfer matrix expressed as a contraction of $a$-tensors.}
  \label{fig:corner_transfer_matrix_as_tensor_network}
\end{figure}

\begin{figure}
  \centering
  \begin{tikzpicture}
	\begin{pgfonlayer}{nodelayer}
		\node [style=a-tensor] (0) at (0, 0) {};
		\node [style=generic2] (1) at (0, 1) {};
		\node [style=generic2] (2) at (1, 0) {};
		\node [style=generic2] (3) at (0, -1) {};
		\node [style=generic2] (4) at (-1, 0) {};
		\node [style=ctm] (5) at (-1, 1) {};
		\node [style=ctm] (6) at (1, 1) {};
		\node [style=ctm] (7) at (1, -1) {};
		\node [style=ctm] (8) at (-1, -1) {};
	\end{pgfonlayer}
	\begin{pgfonlayer}{edgelayer}
		\draw [style=simple] (1) to (0);
		\draw [style=simple] (0) to (3);
		\draw [style=simple] (0) to (2);
		\draw [style=simple] (0) to (4);
		\draw [style=thick leg] (1) to (6);
		\draw [style=thick leg] (6) to (2);
		\draw [style=thick leg] (2) to (7);
		\draw [style=thick leg] (7) to (3);
		\draw [style=thick leg] (3) to (8);
		\draw [style=thick leg] (8) to (4);
		\draw [style=thick leg] (4) to (5);
		\draw [style=thick leg] (5) to (1);
	\end{pgfonlayer}
\end{tikzpicture}

  \caption{Tensor network approximation to $Z_{N \times N}$ in the CTMRG method.}
  \label{fig:partition_function_tensor_network}
\end{figure}

\subsection{Corner transfer matrix renormalization group method}
The corner transfer matrix renormalization group iteratively adds a layer to $A$,
while keeping only $m$ basis states at each step.
It was first employed by Baxter \cite{baxter1982exactly_ctm, baxter1978variational}.

As can be seen from \autoref{eq:z_n_times_n}, the best approximation to the partition function within a restricted
number of basis states $m$ is obtained by keeping the eigenvectors corresponding to the $m$ largest eigenvalues of
$A^4$.

The algorithm proceeds very much like the transfer matrix renormalization group method.
In addition to renormalizing the half-row transfer matrix $P$, we also renormalize the corner transfer matrix $A$ at
each step, using the projector obtained from diagonalizing $A^4$ or equivalently $A$.

We first initialize $P_N$ and $A_N$.
A free or fixed boundary may be imposed in the same way as in the transfer matrix renormalization group.
See \autoref{eq:boundary_tensor_three_legged}.

We then obtain the unrenormalized $A_{N+1}$ by adding a layer of spins to the quadrant
represented by $A_N$. This is done by contracting with two half-row transfer matrices
$P_N$ and a single $a$-tensor, as shown in the first step of
\autoref{fig:ctmrg_add_site_and_renormalize}. We obtain the unnormalized $P_{N+1}$ as
before, as shown in the first step of \autoref{fig:tmrg_add_site_and_renormalize}.

% To find the optimal projector from $2m$ to $m$ basis states, we can directly diagonalize
% $A_{N + 1}^4$, or, equivalently, $A_{N + 1}$. As always, we keep the basis states
% corresponding to the $m$ largest eigenvectors. This is shown in
% \autoref{fig:ctmrg_reduced_density_matrix}.

We use the projector to obtain the
renormalized versions of $A_{N + 1}$ and $T_{N + 1}$
\begin{align}
    \widetilde{A}_{N + 1} &= OA_{N + 1}O^{\dagger}, \\
    \widetilde{T}_{N + 1} &= OT_{N + 1}O^{\dagger}.
\end{align}
shown in the second steps of \autoref{fig:ctmrg_add_site_and_renormalize} and
\autoref{fig:tmrg_add_site_and_renormalize}.

We repeat the above procedure to successively obtain
\begin{align}
  A_{N + 1} & \rightarrow A_{N + 2} \rightarrow \dots, \\
  T_{N + 1} & \rightarrow T_{N + 2} \rightarrow \dots
\end{align}
until a convergence criterion is reached.

\subsubsection{Equivalence to TMRG and DMRG in the thermodynamic limit}
$A^4$ contains the Boltzmann weights of spins along a cut in the middle of the $N \times N$ system.
In contrast, the matrix in \autoref{eq:reduced_density_matrix_classical_case} contains the Boltzmann
weights of spins along a cut down the middle of an $N \times \infty$ system.

In the thermodynamic limit, both become the same, and we can make the identification
\begin{equation}
  \rho = A^4.
\end{equation}
See \autoref{fig:ctmrg_reduced_density_matrix}.

Hence, DMRG and TMRG are equivalent to CTMRG in the thermodynamic limit.
This was first realized by Nishino \cite{nishino1997corner}.

\begin{figure}
  \centering
  \begin{tikzpicture}
	\begin{pgfonlayer}{nodelayer}
		\node [style=ctm] (0) at (-4, 0) {};
		\node [style=none] (1) at (-3.5, 0) {};
		\node [style=none] (2) at (-4, -0.5) {};
		\node [style=none] (3) at (-2.75, 0) {$\longrightarrow$};
		\node [style=ctm] (4) at (-1.75, 0) {};
		\node [style=generic2] (5) at (-1, 0) {};
		\node [style=generic2] (6) at (-1.75, -0.75) {};
		\node [style=a-tensor] (7) at (-1, -0.75) {};
		\node [style=none] (8) at (-0.5, 0) {};
		\node [style=none] (9) at (-0.5, -0.75) {};
		\node [style=none] (10) at (-1, -1.25) {};
		\node [style=none] (11) at (-1.75, -1.25) {};
		\node [style=none] (12) at (0.25, 0) {$\longrightarrow$};
		\node [style=none] (13) at (1.25, -1.25) {};
		\node [style=generic2] (14) at (1.25, -0.75) {};
		\node [style=a-tensor] (15) at (2, -0.75) {};
		\node [style=ctm] (16) at (1.25, 0) {};
		\node [style=none] (17) at (2.5, 0) {};
		\node [style=none] (18) at (2, -1.25) {};
		\node [style=none] (19) at (2.5, -0.75) {};
		\node [style=generic2] (20) at (2, 0) {};

    \node [style=isometry, anchor = west, minimum width = 1.25cm] (west isometry) at (2.5, -0.375) {};
    \node [style=isometry] (south isometry) at (1.25, -1.25) {};
	\end{pgfonlayer}
	\begin{pgfonlayer}{edgelayer}
		\draw [style=thick leg] (0) to (1.center);
		\draw [style=thick leg] (0) to (2.center);
		\draw [style=thick leg] (4) to (5);
		\draw [style=thick leg] (4) to (6);
		\draw [style=thick leg] (5) to (8.center);
		\draw [style=thick leg] (6) to (11.center);
		\draw [style=simple] (5) to (7);
		\draw [style=simple] (7) to (6);
		\draw [style=simple] (7) to (9.center);
		\draw [style=simple] (7) to (10.center);
		\draw [style=thick leg] (16) to (20);
		\draw [style=thick leg] (16) to (14);
		\draw [style=thick leg] (20) to (17.center);
		\draw [style=thick leg] (14) to (13.center);
		\draw [style=simple] (20) to (15);
		\draw [style=simple] (15) to (14);
		\draw [style=simple] (15) to (19.center);
		\draw [style=simple] (15) to (18.center);
	\end{pgfonlayer}
\end{tikzpicture}

  \caption{In the first step, the unrenormalized $A_{N+1}$ is obtained by contracting with
  two copies of $P_N$ and a single $a$-tensor. This corresponds to adding a layer of spins
  to the quadrant, thus enlarging it from $N \times N$ to $N + 1 \times N + 1$. In the
  second step, $A_{N + 1}$ is renormalized with the projector obtained from diagonalizing
  $A_{N+1}^4$ and keeping the basis states corresponding to the $m$ largest eigenvalues.  }
  \label{fig:ctmrg_add_site_and_renormalize}
\end{figure}

\begin{figure}
  \centering
  \begin{tikzpicture}
	\begin{pgfonlayer}{nodelayer}
		\node [style=generic2] (0) at (-5, 2) {};
		\node [style=a-tensor] (1) at (-5, 1) {};
		\node [style=ctm] (2) at (-6, 2) {};
		\node [style=generic2] (3) at (-6, 1) {};
		\node [style=ctm] (4) at (-3, 2) {};
		\node [style=generic2] (5) at (-3, 1) {};
		\node [style=a-tensor] (6) at (-4, 1) {};
		\node [style=generic2] (7) at (-4, 2) {};
		\node [style=a-tensor] (8) at (-4, 0) {};
		\node [style=ctm] (9) at (-6, -1) {};
		\node [style=generic2] (10) at (-6, 0) {};
		\node [style=a-tensor] (11) at (-5, 0) {};
		\node [style=ctm] (12) at (-3, -1) {};
		\node [style=generic2] (13) at (-3, 0) {};
		\node [style=generic2] (14) at (-4, -1) {};
		\node [style=generic2] (15) at (-5, -1) {};
		\node [style=white no border small] (16) at (-6, 0.5) {};
		\node [style=white no border small] (17) at (-5, 0.5) {};
		\node [style=none] (18) at (-2, 0.75) {};
		\node [style=none] (19) at (-2.5, 0.5) {$=$};
		\node [style=square] (20) at (-1.5, 0.5) {};
		\node [style=none] (21) at (0.5, 0.75) {};
		\node [style=none] (22) at (2, 0.75) {};
		\node [style=none] (23) at (2, 0.25) {};
		\node [style=none] (24) at (-1, 0.25) {};
		\node [style=isometry, rotate=180] (25) at (2, 0.5) {};
		\node [style=none] (26) at (0, 0.25) {};
		\node [style=none] (27) at (2.5, 0.75) {};
		\node [style=none] (28) at (0, 0.75) {};
		\node [style=none] (29) at (-1, 0.75) {};
		\node [style=white no border small] (30) at (-1.5, 0.75) {};
		\node [style=singular values matrix] (31) at (1.25, 0.5) {};
		\node [style=none] (32) at (-0.5, 0.5) {$\longrightarrow$};
		\node [style=none] (33) at (2.5, 0.25) {};
		\node [style=none] (34) at (-2, 0.25) {};
		\node [style=white no border small] (35) at (-1.5, 0.25) {};
		\node [style=none] (36) at (0.5, 0.25) {};
		\node [style=isometry] (37) at (0.5, 0.5) {};
	\end{pgfonlayer}
	\begin{pgfonlayer}{edgelayer}
		\draw [style=thick leg] (2) to (0);
		\draw [style=thick leg] (2) to (3);
		\draw [style=simple] (0) to (1);
		\draw [style=simple] (1) to (3);
		\draw [style=thick leg] (4) to (7);
		\draw [style=thick leg] (4) to (5);
		\draw [style=simple] (7) to (6);
		\draw [style=simple] (6) to (5);
		\draw [style=simple] (1) to (6);
		\draw [style=thick leg] (0) to (7);
		\draw [style=thick leg] (9) to (15);
		\draw [style=thick leg] (9) to (10);
		\draw [style=simple] (15) to (11);
		\draw [style=simple] (11) to (10);
		\draw [style=thick leg] (12) to (14);
		\draw [style=thick leg] (12) to (13);
		\draw [style=simple] (14) to (8);
		\draw [style=simple] (8) to (13);
		\draw [style=simple] (11) to (8);
		\draw [style=thick leg] (15) to (14);
		\draw [style=thick leg] (5) to (13);
		\draw [style=simple] (6) to (8);
		\draw [style=simple] (11) to (17);
		\draw [style=simple] (1) to (17);
		\draw [style=thick leg] (10) to (16);
		\draw [style=thick leg] (3) to (16);
		\draw [style=simple] (34.center) to (35);
		\draw [style=simple] (24.center) to (35);
		\draw [style=thick leg] (30) to (29.center);
		\draw [style=thick leg] (30) to (18.center);
		\draw [style=thick leg] (37) to (31);
		\draw [style=thick leg] (28.center) to (21.center);
		\draw [style=simple] (26.center) to (36.center);
		\draw [style=thick leg] (27.center) to (22.center);
		\draw [style=simple] (33.center) to (23.center);
		\draw [style=thick leg] (31) to (25);
	\end{pgfonlayer}
\end{tikzpicture}
  \caption{The matrix $A_{N+1}^4$ is approximately equal to $\rho_{N+1}$ in
  \autoref{eq:reduced_density_matrix_classical_case}. Compare the graphical forms of this
  network and the one shown in \autoref{fig:tmrg_reduced_density_matrix}. We obtain the
  optimal projector by diagonalizing $A_{N + 1}^4$, or equivalently $A_{N+1}$.}
  \label{fig:ctmrg_reduced_density_matrix}
\end{figure}



\section{Calculation of observable quantities}
\subsection{Free energy per site}
Baxter \cite{baxter1978variational, baxter1982exactly_ctm} showed that the partition function per site
\begin{equation}
  \kappa = Z^{1/N^2}
\end{equation}
is, within the corner transfer matrix renormalization group method, written as
\begin{equation}\label{eq:partition_function_per_site_variational_approximation}
  \kappa = \frac{r_1 r_4}{r_2 r_3},
\end{equation}
with $r_2$, $r_3$ and $r_4$ as in \autoref{fig:partition_function_per_site_tensor_networks} and $r_1 = Z_{N \times N}$
as in \autoref{fig:partition_function_tensor_network}. The free energy per site is then simply
\begin{equation}
  \frac{F}{N^2} = -T \log \kappa.
\end{equation}

\begin{figure}
  \centering
  \begin{subfigure}{.33\linewidth}
    \begin{tikzpicture}
	\begin{pgfonlayer}{nodelayer}
		\node [style=ctm] (0) at (0, 1) {};
		\node [style=ctm] (1) at (0, -1) {};
		\node [style=ctm] (2) at (1, -1) {};
		\node [style=ctm] (3) at (1, 1) {};
		\node [style=generic2] (4) at (1, 0) {};
		\node [style=generic2] (5) at (0, 0) {};
	\end{pgfonlayer}
	\begin{pgfonlayer}{edgelayer}
		\draw [style=simple] (5) to (4);
		\draw [style=thick leg] (0) to (3);
		\draw [style=thick leg] (3) to (4);
		\draw [style=thick leg] (4) to (2);
		\draw [style=thick leg] (2) to (1);
		\draw [style=thick leg] (1) to (5);
		\draw [style=thick leg] (5) to (0);
	\end{pgfonlayer}
\end{tikzpicture}
    % \caption{A subfigure}\label{fig:1a}
  \end{subfigure}%
  \begin{subfigure}{.33\linewidth}
    \begin{tikzpicture}
	\begin{pgfonlayer}{nodelayer}
		\node [style=ctm] (0) at (-1, 1) {};
		\node [style=ctm] (1) at (1, 1) {};
		\node [style=ctm] (2) at (1, 0) {};
		\node [style=ctm] (3) at (-1, 0) {};
		\node [style=generic2] (4) at (0, 1) {};
		\node [style=generic2] (5) at (0, 0) {};
	\end{pgfonlayer}
	\begin{pgfonlayer}{edgelayer}
		\draw [style=thick leg] (0) to (4);
		\draw [style=thick leg] (0) to (3);
		\draw [style=thick leg] (3) to (5);
		\draw [style=thick leg] (5) to (2);
		\draw [style=thick leg] (4) to (1);
		\draw [style=thick leg] (1) to (2);
		\draw [style=simple] (4) to (5);
	\end{pgfonlayer}
\end{tikzpicture}
    % \caption{Another subfigure}\label{fig:1b}
  \end{subfigure}
  \begin{subfigure}{.33\linewidth}
    \begin{tikzpicture}
	\begin{pgfonlayer}{nodelayer}
		\node [style=ctm] (0) at (0, 1) {};
		\node [style=ctm] (1) at (1, 1) {};
		\node [style=ctm] (2) at (0, 0) {};
		\node [style=ctm] (3) at (1, 0) {};
	\end{pgfonlayer}
	\begin{pgfonlayer}{edgelayer}
		\draw [style=thick leg] (0) to (1);
		\draw [style=thick leg] (1) to (3);
		\draw [style=thick leg] (3) to (2);
		\draw [style=thick leg] (2) to (0);
	\end{pgfonlayer}
\end{tikzpicture}
    % \caption{Another subfigure}\label{fig:1b}
  \end{subfigure}
  \caption{From left to right:
  $r_2$, $r_3$ and $r_4$ as in
  \autoref{eq:partition_function_per_site_variational_approximation}.}
  \label{fig:partition_function_per_site_tensor_networks}
\end{figure}


\subsection{Magnetization per site}\label{sec:magnetization_per_site}
The magnetization per site may be calculated as
\begin{equation}
  M = T \frac{\partial (\log \kappa)}{\partial h},
\end{equation}
but this involves a numerical derivative and a numerical limit $h \to 0$ in the case of the spontaneous magnetization.
A more practical method, that is employed in this thesis, is to use the magnetization of the central spin
\begin{equation}
  \langle \sigma_0 \rangle = \frac{\sum_{ \{ \bm{\sigma} \} } \sigma_0 \exp \left( -\beta H(\bm{\sigma}) \right) }{Z}
\end{equation}
as a proxy quantity to the magnetization per site.

In the original definition of the corner transfer matrix by Baxter (\autoref{eq:corner_transfer_matrix}),
it is written as
\begin{equation}
  \langle \sigma_0 \rangle = \frac{\tr A_{+}^4 - \tr A_{-}^4}{\tr A^4}.
\end{equation}
Here, $A_{\pm}$ is the corner transfer matrix with the central spin fixed to $\pm$.

In tensor network notation, $\tr A_{+}^4 - \tr A_{-}^4$ is written as the tensor network in
\autoref{fig:magnetization_central_spin_tensor_network}, with the tensor $b_{i j k l}$ defined as
\begin{equation}\label{eq:b_tensor}
  b_{i j k l} = \sum_{\sigma \in \{ -1, 1 \} } \sigma \delta_{\sigma i j k l}.
\end{equation}

All numerical results in this thesis involving the magnetization per site are actually obtained by calculating $\langle
\sigma_0 \rangle$, which shall be referred to simply as $M$ from now on.

\begin{figure}
  \centering
  \begin{tikzpicture}
	\begin{pgfonlayer}{nodelayer}
		\node [style=ctm] (0) at (1, -1) {};
		\node [style=generic2] (1) at (-1, 0) {};
		\node [style=ctm] (2) at (1, 1) {};
		\node [style=ctm] (3) at (-1, -1) {};
		\node [style=generic2] (4) at (1, 0) {};
		\node [style=ctm] (5) at (-1, 1) {};
		\node [style=generic2] (6) at (0, -1) {};
		\node [style=generic2] (7) at (0, 1) {};
		\node [style=generic] (8) at (0, 0) {$b$};
	\end{pgfonlayer}
	\begin{pgfonlayer}{edgelayer}
		\draw [style=simple] (7) to (8);
		\draw [style=simple] (8) to (6);
		\draw [style=simple] (8) to (4);
		\draw [style=simple] (8) to (1);
		\draw [style=thick leg] (7) to (2);
		\draw [style=thick leg] (2) to (4);
		\draw [style=thick leg] (4) to (0);
		\draw [style=thick leg] (0) to (6);
		\draw [style=thick leg] (6) to (3);
		\draw [style=thick leg] (3) to (1);
		\draw [style=thick leg] (1) to (5);
		\draw [style=thick leg] (5) to (7);
	\end{pgfonlayer}
\end{tikzpicture}
  \caption{Unnormalized expectation value of central spin, with the tensor $b_{i j k l}$ defined in \autoref{eq:b_tensor}.}
  \label{fig:magnetization_central_spin_tensor_network}
\end{figure}

\subsection{Analogy to entanglement entropy for classical systems}\label{sec:analogy_to_entropy}
The key point of the corner transfer matrix renormalization group method \cite{nishino1997corner, nishino1996corner} is
that it unifies White's density matrix renormalization group method \cite{white1992density} with Baxter's corner
transfer matrix approach \cite{baxter1968dimers, baxter1978variational}, through the identification (in the isotropic
case)
\begin{equation}\label{eq:correspondence_density_matrix_ctm}
  \rho_{\text{half-chain}} = A^4.
\end{equation}

This allows one to define a 2D classical analogue to the half-chain entanglement entropy of a 1D quantum system
\begin{equation}\label{eq:classical_entropy}
  S_{\text{classical}} = - \tr A^4 \log A^4 = - \sum_{\alpha=1}^{m} \nu_{\alpha}^4 \log \nu_{\alpha}^4,
\end{equation}
where $\nu_{\alpha}$ are the eigenvalues of the corner transfer matrix $A$.
In the CTMRG algorithm, $A$ is kept in diagonal form, making $S_{\text{classical}}$ trivial to compute.

In \cite{huang2017holographic}, numerical evidence is given for the validity of \autoref{eq:classical_entropy} for a
wide range of models, and the concept is generalized to higher dimensions. For an overview of applying corner transfer
matrices in higher dimensions and to quantum systems, see \cite{orus2012exploring}.

\section{Spectrum of the corner transfer matrix}\label{sec:spectrum_of_ctm}

\subsection{Analytical results for the Ising model}
In what follows, we present results established in \cite{peschel2009reduced, peschel1999density}.

For the off-critical Ising model on a square lattice, we have \cite{davies1988corner}
\begin{equation}\label{eq:rho_exact_expression_off_critical}
  \hat{\rho} = \hat{A}^4 = \exp(-\hat{H}_{\text{CTM}}),
\end{equation}
where
\begin{equation}
  \hat{H}_{\text{CTM}} = \sum_{l = 0}^{\infty} \epsilon_l(T) c^{\dagger}_l c_l,
\end{equation}
with $c_l$ and $c^{\dagger}_l$ fermionic annihilation and creation operators and
\begin{equation}
  \epsilon_l =
  \begin{cases}
    (2l + 1)\epsilon(T) & \text{if } T > T_c, \\
    2l\epsilon(T) & \text{if } T < T_c.
  \end{cases}
\end{equation}
with $\epsilon(T)$ a model-specific factor that only depends on temperature.

In other words, the reduced density matrix (or equivalently, the corner transfer matrix $A$) can be written as a
density matrix of an effective free fermionic Hamiltonian with equally spaced excitations.

What does this mean for the spectrum of $A$?
If we assume a free boundary, we have to distinguish between the ordered and disordered phase.

In the disordered phase, we have $\epsilon_l = (2l + 1)\epsilon(T)$.
The ground state, $E = 0$, corresponds to the vacuum state of the effective system described by $H_{\text{CTM}}$.
The single-fermion excitations give $\epsilon, 3\epsilon, 5\epsilon,
\dots$, while two-fermion excitations give $4\epsilon$ ($c^{\dagger}_0 c^{\dagger}_1 \ket{0}$),
$6\epsilon$ ($c^{\dagger}_0 c^{\dagger}_2 \ket{0}$) and $8\epsilon$ ($c^{\dagger}_0 c^{\dagger}_3 \ket{0}$ \emph{or}
$c^{\dagger}_1 c^{\dagger}_2 \ket{0}$).
So the first degeneracy appears at $8\epsilon$.
$9\epsilon$ is also degenerate:
it can be constructed with a single-fermion excitation ($ c^{\dagger}_4 \ket{0} $) and a three-fermion
excitation ($ c^{\dagger}_2 c^{\dagger}_1 c^{\dagger}_0 \ket{0} $).

The numerical results from the CTMRG algorithm exactly confirm this picture.
See the $T = 2.6$ line in the left panel of \autoref{fig:spectrum_ctm}. The gap after the first two eigenvalues is due
to the absence of the level $2\epsilon$. The $\epsilon_l$ are linear and the degeneracies are correct.

In the ordered phase, we have a two-fold degeneracy for every state due to symmetry and
ground state energy $E = 0$.
After that, the only available levels are $2\epsilon, 4\epsilon, 6\epsilon,
\dots$.
The degeneracy of the $n$th energy level is given by $2p(n)$, twice the number of partitions of $n$ into distinct integers
\cite{okunishi1999universal}, with the factor of two coming from symmetry.

To illustrate:
$c^{\dagger}_1 c^{\dagger}_2 \ket{0}$ and $c^{\dagger}_3 \ket{0}$ both have $E = 6\epsilon$,
the third energy level (counting the vacuum as the zeroth energy level),
which is to say $p(3) = 2$ since $\{3, 2 + 1 \}$ are the ways to write $3$.
The line $T = 2$ in the left panel of \autoref{fig:spectrum_ctm} confirms these results.

With a fixed boundary, the spectrum in the disordered phase doesn't change.
In the ordered phase however, the two-fold degeneracy due to symmetry is lifted,
so the degeneracy of the $n$th energy level becomes $p(n)$.
As a consequence, the spectrum decays much faster. See the right panel of \autoref{fig:spectrum_ctm}.

At or close to criticality, the expression in \autoref{eq:rho_exact_expression_off_critical} breaks down,
and the spectrum of $\hat{\rho}$ is smoothened out.
In general, below and at criticality, the spectrum decays slower for a free boundary.
This is to be expected, since $A$ preserves the symmetry when the boundary is free.
At $T = 0$, $A$ has two equally large non-zero eigenvalues, representing either all up or all down spins on the boundary
of the quadrant, while for a fixed boundary, $A$ has one non-zero eigenvalue:
it represents a completely polarized state.

\begin{figure}
  \centering
  \includegraphics[]{spectrum_ctm.tikz}
  \caption{First part of the spectrum of $A$ after $n = 1000$ steps with a bond dimension of $m =
  250$.}\label{fig:spectrum_ctm}
\end{figure}

\subsection{Implications for finite-$m$ simulations}\label{sec:implications_for_finite_m_simulations}

When approximating the corner transfer matrix with a free boundary in the ordered phase,
it is crucial to retain all basis states corresponding to an energy level \cite{okunishi1999universal}.
Failure to do so will lead to a symmetry-broken state.

Near criticality, however, even when all degenerate basis states are kept,
the algorithm is still prone to symmetry breaking.

\section{Equivalence to variational approximation in the space of matrix product states.}

In closing this chapter, we note that it has been widely established that DMRG produces a ground state that
corresponds to a variational optimization within a matrix-product structure \cite{ostlund1995thermodynamic,
rommer1997class}.

CTMRG and TMRG, by the classical-quantum equivalence, find transfer matrices with similar structure.
This was first noted by Baxter \cite{baxter1982exactly_ctm}.

For an introduction to these algorithms from this variational point of view, see \cite{nishino1999density}.


\chapter{Critical behaviour and finite-size scaling}
\chapterprecishere{
In this chapter, we introduce the central concepts in critical phenomena and finite-size scaling,
largely following the review by Barber \cite{barber1983finite} and the less technical overview by Kadanoff
\cite{kadanoff2009more}.
}

\section{Phase transitions}
When matter exhibits a sudden change in behaviour, often characterized by a discontinuity or divergence of one or more
thermodynamic quantities, we say it undergoes a \emph{phase transition}.

A quantity that signifies this change is called an \emph{order parameter},
which can take vastly different forms across systems and transitions.
For example, for the transition of a ferromagnet, the order parameter is the net magnetization of the system,
while for a percolation transition, it is the size of the largest connected graph.

For a historical account of the classification of phase transitions,
see \cite{jaeger1998ehrenfest}.
At the present time, we distinguish between two different types \cite{kadanoff2009more}.

When some thermodynamic quantity changes discontinuously, i.e.
shows a jump, we call the transition \emph{first order}.
In contrast, during a \emph{continuous} phase transition a variable undergoes change continuously.
The point at which a continuous phase transition occurs, is called the critical point.

The two-dimensional Ising model in a magnetic field shows both types of transition.
At zero magnetic field and $T = T_c = 1 / (\log(1 + \sqrt{2}))$, the magnetization changes from zero for $T > T_c$ to
a finite value for $T < T_c$ in a continuous manner.

Below the critical temperature $T_c$, when the magnetic field $h$ tends to zero from $h > 0$,
the magnetization tends to a positive value.
Conversely, when the magnetic field tends to zero from $h < 0$,
the magnetization tends to a negative value.
Thus, across the region $h = 0, T < T_c$ the system undergoes a first-order phase transition.

\subsection{Finite systems}

We will now argue that a phase transition cannot occur in a finite system,
but only happens when the number of particles tends to infinity.

Because thermodynamic quantities are averages over all possible
microstates of a system, those quantities are completely defined in terms
of the system's partition function, or equivalently its free energy.

Since in a finite system, the partition function is a finite sum of
exponentials, it is analytic (infinitely differentiable). Hence, thermodynamic
quantities cannot show true discontinuities and the phase transitions
described in the above section do not occur.

\section{Critical behaviour}
We will now focus our attention on continuous phase transitions, more specifically the one that occurs in the
two-dimensional Ising model.
Before we discuss the behaviour of the free energy around the critical point,
we briefly summarize how the thermodynamic limit is approached far away from it.
Here, we largely follow \cite{barber1983finite}.

We assume that the free energy per site in the thermodynamic limit
\begin{equation}
  f_{\infty}(T) = \lim_{N \to \infty} \frac{F(T, N)}{N}
\end{equation}
exists, and is not dependent on boundary conditions.
By definition, it is not analytic in a region around the critical point.

Outside that region, however, we can write
\begin{equation}\label{eq:free_energy_finite_system_outside_critical_region}
  F(T, N) = N f_{\infty}(T) + o(N),
\end{equation}
where correction terms $g(N)$ of $o(N)$ (little-o of $N$) obey
\begin{equation}
  \lim_{N \to \infty} \frac{g(N)}{N} = 0.
\end{equation}
These corrections, of course, do depend on boundary conditions.

\autoref{eq:free_energy_finite_system_outside_critical_region} is valid only outside the critical region precisely
because $F(T, N)$ is analytic \emph{everywhere}, and $f_{\infty}(T)$ is only analytic away from the critical point.

The behaviour of $F(T, N)$ (and hence, all thermodynamic quantities) at criticality is approached is described by
\emph{finite-size scaling}.

\subsection{Finite-size scaling}

\autoref{fig:order_parameter_finite_N_exact} shows the behaviour of the order parameter obtained by exact
diagonalization of the partition function of small lattices.
It is clear that far from the critical point, the order parameter is essentially not dependent on system size,
while in critical region there are significant deviations from the thermodynamic behaviour.

\begin{figure}
  \includegraphics[]{order_parameter_finite_N_exact.tikz}
  \caption{The magnetization of the central spin for small lattices with boundary spins fixed to
  $+1$. The black line is the exact solution in the thermodynamic limit.}\label{fig:order_parameter_finite_N_exact}
\end{figure}

One can now define two characteristic temperatures \cite{fisher1967interfacial,
barber1983finite}.
The first being the cross-over temperature $T_X$ at which finite-size effects become important,
which is predicted to scale as
\begin{equation}\label{eq:cross_over_temperature_scaling}
  |T_X - T_c| \propto N^{-\theta}.
\end{equation}
$\theta$ is called the cross-over or rounding exponent.

The second characteristic temperature is the pseudocritical temperature,
denoted by $T^{\star}$.
It can be defined in several ways, one being the point where the order parameter becomes almost zero,
or the point where the heat capacity
\begin{equation}
  C = T^2 \frac{\partial^2 F}{\partial T^2}
\end{equation}
reaches its maximum.
$T^{\star}$ can be regarded as the point where the finite system in some sense comes closest to undergoing a transition.

Generally $T^{\star}$ will not equal $T_X$. Furthermore, $T^{\star}$ depends on boundary conditions:
periodic or fixed boundary conditions will nudge the system into an ordered state,
therefore $T^{\star} > T_c$.
Free boundary conditions will cause the system to favor disorder and the pseudocritical temperature to be lowered.

In any case, it is predicted that
\begin{equation}\label{eq:scaling_law_T_star}
  |T^{\star} - T_c| \propto N^{-\lambda}.
\end{equation}

It is generally accepted that \cite{barber1983finite}
\begin{equation}
  \lambda = \theta.
\end{equation}

Furthermore, if one assumes that finite-size effects become important once the correlation length of the system becomes of order of the system size, i.e. \cite{fisher1967interfacial}
\begin{equation}\label{eq:correlation_length_propto_N}
  \xi(T_X(N)) \propto N,
\end{equation}
then the correlation length exponent $\nu$, given by
\begin{equation}
  \xi(T) \propto |T - T_c|^{-\nu}
\end{equation}
is, by using \autoref{eq:cross_over_temperature_scaling}, related to $\theta$ as
\begin{equation}
  \theta = \frac{1}{\nu}.
\end{equation}

\subsubsection{The finite-size scaling ansatz}

The behaviour of a system of finite size $N$ is expected to be a function of the ratio
\begin{equation}
  y = \frac{N}{\xi(T)},
\end{equation}
where $\xi(T)$ is the correlation length of the thermodynamic system \cite{fisher1972scaling}.

With the assumption in \autoref{eq:correlation_length_propto_N}, this means that in the limit $y \gg 1$,
we expect to see thermodynamic behaviour, while for $y \ll 1$,
the finite system size should enter in the analysis.

To see exactly how this happens, consider as an example the order parameter $M$,
which in the thermodynamic limit, close to the critical point obeys
\begin{equation}
  M(T) \propto
  \begin{cases}
    (-t)^{\beta} & \text{if } T \leq T_c, \\
    0 & \text{if } T \geq T_c,
  \end{cases}
\end{equation}
where we have defined the reduced temperature
\begin{equation}
  t = \frac{T - T_c}{T_c}.
\end{equation}

Assuming the correlation length diverges algebraically
\begin{equation}\label{eq:xi_propto_t}
  \xi(T) \propto |t|^{-\nu},
\end{equation}
for $T < T_c$ we have
\begin{equation}\label{eq:order_parameter_propto_correlation_length}
  M(T) \propto \xi(T)^{-\beta / \nu}.
\end{equation}

The \emph{finite-size scaling ansatz} now says that for finite systems
\begin{equation}\label{eq:finite_size_scaling_ansatz}
  M(T, N) = N^{-\beta/\nu}\mathcal{F}(y),
\end{equation}
with the requirement that for $N \to \infty$, it should reproduce the thermodynamic behaviour
in \autoref{eq:order_parameter_propto_correlation_length}, leading to
\begin{equation}
  \lim_{y \to \infty} \mathcal{F}(y) \propto y^{\beta/\nu}.
\end{equation}
At the critical point, however, the bulk correlation length diverges and the only relevant length scale is $N$,
so that we must have
\begin{equation}
  M(T = T_c, N) \propto N^{-\beta / \nu},
\end{equation}
from which it follows that
\begin{equation}
  \lim_{y \to 0} \mathcal{F}(y) = \text{const}.
\end{equation}

\subsubsection{Extracting exponents from numerical simulation}\label{sec:extracting_exponents_from_numerical_simulation}

To extract critical exponents from (finite) numerical simulations, \autoref{eq:finite_size_scaling_ansatz} may be
written as
\begin{equation}\label{eq:finite_size_scaling_ansatz_tN}
  M(T, N) = N^{-\beta/\nu} \mathcal{G}(t N^{1/\nu})
\end{equation}
where it is used that (per \autoref{eq:xi_propto_t})
\begin{equation}
  y = \frac{N}{\xi(T)} \propto t^{\nu} N,
\end{equation}
and the new scaling function is customarily written as having argument $t N^{1/\nu} = (t^{\nu} N)^{1/\nu}$.

The critical exponents $\beta$ and $\nu$ and the critical temperature can now be extracted by asserting that the
numerical data for different system sizes should collapse on a single curve
\begin{equation}
  \mathcal{G}(t N^{1/\nu}) = M(T, N) N^{\beta/\nu}.
\end{equation}

The authors of \cite{bhattacharjee2001measure} propose a measure of the fitness $P(\beta, \nu, T_c)$ of such a data collapse
\begin{equation}\label{eq:fitness_data_collapse}
  P(\beta, \nu, T_c) = \frac{1}{\mathcal{N}_{\text{overlap}}} \sum_p \sum_{j \neq p} \sum_{i_{\text{overlap}}}
    |  M(t_{i j}, N_j) N_{j}^{\beta/\nu} - \mathcal{E}_{p}(t_{i j} N_{j}^{1/\nu} ) |,
\end{equation}
where for each system size $N_p$, the data points collected for the other system sizes $N_j$ that overlap (that is,
fall between any two data points collected for $N_p$) are compared with the interpolation $\mathcal{E}_{p}(t_{i j}
N_{j}^{1/\nu})$ between those two data points. $\mathcal{N}_{\text{overlap}}$ is the number of overlapping pairs.

It is clear that
\begin{equation}
  P(\beta, \nu, T_c) \geq 0
\end{equation}
and the optimal values for $\beta$, $\nu$ and $T_c$ minimize $P(\beta, \nu, T_c)$.

This measure for the data collapse is found, for data collected for this thesis,
to work significantly better than other proposed measures such as fitting a polynomial or order 3-8 through all data
points.


\chapter{Finite-$m$ scaling in the CTMRG algorithm}
\chapterprecishere{The connection between finite-size scaling, as introduced in the last chapter,
and finite-size effects as a consequence of the finite bond dimension $m$ within the CTMRG algorithm is made.

We discuss ideas by Nishino \cite{nishino1996numerical}, who was the first to investigate these effects,
by linking the finite-bond dimension $m$ to the inherently finite correlation length of the approximated system at the
critical point.
These ideas were later made more precise via the connection of the CTMRG algorithm in the thermodynamic limit to matrix
product states \cite{baxter1978variational, ostlund1995thermodynamic}, which inherently have a finite correlation length
\cite{wolf2006quantum, rommer1997class}.

Then, we discuss a more recent theory of finite-entropy scaling, developed in \cite{pollmann2009theory} (earlier
numerical evidence was given in \cite{tagliacozzo2008scaling, andersson1999density}),
which implies a scaling of the correlation length for a matrix product state with finite bond dimension of the form $\xi
\propto m^{\kappa}$.}

\section{Introduction}

Up until now, we have developed our scaling analysis in terms of a finite system size $N$.
But the approximation of the infinite-system partition function with the CTMRG algorithm depends on two parameters;
the system size $N$ and the bond dimension $m$.

A finite bond dimension $m$ carries a characteristic length scale.
Baxter \cite{baxter1978variational}, and later Östlund and Rommer \cite{ostlund1995thermodynamic} (in the context of
one-dimensional quantum systems) showed that in the thermodynamic limit,
CTMRG and DMRG are variational optimizations in the space of matrix product states.
\todo[inline]{Can extend this idea a bit.}

It is known that an MPS-ansatz with finite bond dimension inherently limits the
correlation length of the system to a finite value \cite{wolf2006quantum, rommer1997class}. Hence,
thermodynamic quantities obtained from the CTMRG algorithm with finite $m$, in the limit
$N \to \infty$, cannot diverge and must show finite-size effects similar to those of some
effective finite system of size $N_{\text{eff}}(m)$ depending on the bond dimension $m$.

\autoref{fig:order_parameter_vs_T} shows the behaviour of the order parameter of the
two-dimensional Ising model for systems of finite-size,
where the result is converged in $m$, and for systems of finite $m$, where
the result is converged in the system size $N$. The results look very similar and support
the claim that there are two relevant length scales in the critical region, namely the system size $N$ and
the length scale associated to the finite bond dimension $m$.

\begin{figure}
\includegraphics[]{order_parameter_vs_T.tikz}
\caption{Upper panel: expectation value of the central spin $\langle \sigma_0 \rangle$
  after $n$ CTMRG steps. $m$ is chosen such that the truncation error is smaller than
  $10^{-6}$. Lower panel: $\langle \sigma_0 \rangle$ for systems with bond dimension $m$.}\label{fig:order_parameter_vs_T}
\end{figure}

\section{Definition of the effective length scale in terms of the correlation length at $T_c$}\label{sec:definition_effective_length_scale_in_terms_of_xi}

The first direct comparison of finite-size scaling in the system size $N$ with scaling in
the bond dimension of the CTMRG method $m$ was done
in \cite{nishino1996numerical}.

In the thermodynamic limit (corresponding to infinite $m$ and $N$), we have the following
expression for the correlation length of a classical system
\cite{baxter1982exactly_correlation_length}
\begin{equation}\label{eq:correlation_length_row_to_row_transfer_matrix}
  \xi(T) = \frac{1}{\log\left(\frac{T_0}{T_1}\right)}.
\end{equation}
Here, $T_0$ and $T_1$ are the largest and second-largest eigenvalues of the row-to-row
transfer matrix $T$, respectively. With $N$ tending towards infinity and finite $m$, near
the critical point $\xi(T)$ should obey a scaling law of the form
\begin{equation}
  \xi(T, m) = N_{\text{eff}}(m) \mathcal{F}(N_{\text{eff}}(m) / \xi(T))
\end{equation}
with
\begin{equation}
  \mathcal{F}(x) = \begin{cases}
      \text{const} & \text{if } x \to 0, \\
      x^{-1} & \text{if } x \to \infty.
    \end{cases}
\end{equation}

Hence, the effective length scale corresponding to the finite bond dimension $m$ is
proportional to the correlation length of the system at the critical point $t = 0$.
\begin{equation}
  N_{\text{eff}}(m) \propto \xi(T = T_c, m).
\end{equation}

\todo[inline]{Look ahead to replicating this in results section?}

\section{Relation to finite-entropy scaling and the exponent $\kappa$.}


The first numerical evidence of a law for the correlation length at the critical point of the form
\begin{equation}\label{eq:xi_propto_m_kappa}
  \xi(m) \propto m^{\kappa}
\end{equation}
was given by the authors of \cite{andersson1999density}, who found
\begin{equation}
  \kappa \approx 1.3
\end{equation}
for a gapless system of free fermions, using DMRG calculations. Later, using the iTEBD algorithm
\cite{vidal2007classical}, the authors of \cite{tagliacozzo2008scaling} presented numerical evidence for such a relation
for the Ising model with transverse field and the Heisenberg model, with
\begin{align}
  \kappa_{\text{Ising}} & \approx 2, \\
  \kappa_{\text{Heisenberg}} & \approx 1.37.
\end{align}

\subsection{Quantitative theory for $\kappa$}
A quantitative theory of the existence of an exponent $\kappa$ was given in \cite{pollmann2009theory}.
We reproduce the argument, which is presented in the language of one-dimensional quantum systems, below.

We start by noting that in the critical region, the entanglement of a half-infinite subsystem $A$ diverges as
\begin{equation}\label{eq:entropy_scaling_near_criticality}
  S_A \propto \mathcal{A}(c/6)\log(\xi),
\end{equation}
where $\mathcal{A}$ is the number of boundary points of $A$ and $c$ is the central charge of the conformal field theory
at the critical point \cite{calabrese2004entanglement, vidal2003entanglement, ercolessi2010exact}.

Recalling the definition of the entanglement entropy
\begin{equation}
  S_A = - \tr(\rho_A \log \rho_A) = - \sum_{\alpha} \omega_{\alpha} \log \omega_{\alpha},
\end{equation}
it is trivially seen that the entropy of a state given by the DMRG (or any other MPS), which only
retains $m$ basis states of $\rho_A$, is limited by
\begin{equation}
  S^{\text{max}}_A(m) = \log m
\end{equation}
by putting $\omega_{\alpha} = 1/m$ for $\alpha = 1, \dots, m$.

This is, incidentally, another way to see that DMRG or CTMRG, or any other algorithm which produces ground states with a
matrix-product structure have an inherently finite correlation length.

The leading energy correction to the free energy per site of a one-dimensional quantum system at a conformally invariant
critical point at finite temperature $T$ in the thermodynamic limit is \cite{affleck1986universal}
\begin{equation}\label{eq:correction_free_energy_critical_point_finite_temperature}
  f(T) = f_0 + aT^2 + \mathcal{O}(T^3).
\end{equation}

Due to the quantum-classical correspondence, this is equivalent to a two-dimensional classical $N \times \infty$ lattice
with strip width $N = 1/T$.
This implies also that the correlation length of a critical one-dimensional quantum system at finite temperature cannot
diverge and goes as $\xi \propto 1/T$.
In terms of this finite correlation length, \autoref{eq:correction_free_energy_critical_point_finite_temperature} is
written as
\begin{equation}\label{eq:correction_free_energy_critical_point_finite_correlation_length}
  f(\xi) = f_{\infty} + \frac{A}{\xi^2} + \mathcal{O(\frac{1}{\xi^3})}.
\end{equation}

Empirically, optimized ground states with a matrix-product structure at criticality do not simply maximize their
entropy, as they should if we take \autoref{eq:correction_free_energy_critical_point_finite_correlation_length} to be
true for ground states with a matrix-product structure.

We will now show that \autoref{eq:correction_free_energy_critical_point_finite_correlation_length} needs,
in fact, an additional term due to the matrix-product structure with finite bond dimension $m$.

The ground state with finite correlation length and energy density as in
\autoref{eq:correction_free_energy_critical_point_finite_correlation_length} has a Schmidt decomposition that in
principle can have infinitely many terms
\begin{equation}\label{eq:ground_state_infinite_schmidt_decomposition}
  \ket{\psi_0} = \sum_{n = 1}^{\infty} \lambda_n \ket{\psi_{n}^{L}}\ket{\psi_{n}^{R}},
\end{equation}
where $\ket{\psi_{n}^{L}}$ and $\ket{\psi_{n}^{R}}$ are states of the left and right infinite half-chains. Normalization
requires
\begin{equation}
  \sum_{n}^{\infty} \lambda_{n}^2 = 1.
\end{equation}

The ground state with a matrix-product structure with finite bond dimension $m$ has an additional constraint:
its Schmidt decomposition carries only the $m$ $\ket{\psi_n}$ with largest $\lambda_n$.
It is written as
\begin{equation}
  \ket{\psi_{0}^{\text{MPS}}} = \frac{\sum_{n = 1}^{m} \lambda_n
  \ket{\psi_{n}^{L}}\ket{\psi_{n}^{R}}}{\sqrt{\sum_{n=1}^{m} \lambda_{n}^2}}.
\end{equation}

To find the extra energy cost of only keeping the first $m$ terms in the Schmidt decomposition,
note that in the limit of $m$ large, $\ket{\psi_{0}^{\text{MPS}}}$ almost entirely overlaps with $\ket{\psi_0}$,
hence can be written as
\begin{equation}
  \ket{\psi_{0}^{\text{MPS}}} = \sqrt{1 - \epsilon^2} \ket{\psi_0} + \epsilon \ket{\psi_{\text{ex}}},
\end{equation}
where $\ket{\psi_{\text{ex}}}$ is some excited state and $\epsilon \ll 1$. This leads to an energy of
\begin{equation}
  E_{0}^{\text{MPS}} = \braket{\psi_{0}^{\text{MPS}} | \hat{H} | \psi_{0}^{\text{MPS}}} = E_0 + \epsilon^2 (E_{\text{ex}} - E_0),
\end{equation}
with
\begin{equation}
  \epsilon^2 = \left(1 - \braket{\psi_0 | \psi_{0}^{\text{MPS}}}^2 \right) = 1 - \sum_{n = 1}^{m} \lambda_{n}^2 \equiv
  P_{\text{res}}(m).
\end{equation}
Here, we have defined the residual probability $P_{\text{res}}$, also known as the truncation error,
as the part of the spectrum that is thrown away.

If we now assume that $E_0 - E_{\text{ex}}$ is proportional to the energy gap $\Delta$, which scales as \cite{lieb1961two, mata1989energy, pfeuty1970one}
\begin{equation}
  \Delta \propto \frac{1}{\xi},
\end{equation}
we arrive at
\begin{equation}\label{eq:correction_energy_mps_ground_state}
  E_{0}^{\text{MPS}} = E_{\infty} + \frac{A}{\xi^2} + \frac{B P_{\text{res}}(m)}{\xi}.
\end{equation}

It is clear that when the correlation length is very large, by \autoref{eq:entropy_scaling_near_criticality} the entropy
and $P_{\text{res}}(m)$ must be too.
So, the third term in \autoref{eq:correction_energy_mps_ground_state} dominates.

If the correlation length is small, the second term dominates.
The correlation length that belongs to the MPS ground state with fixed $m$ is the optimum that minimizes this
expression.

The details of the calculation, which can be found in the supplementary material of \cite{pollmann2009theory},
depend on the asymptotic form of $P_{\text{res}}$, found in \cite{calabrese2008entanglement}. In the limit $m \to \infty$, the correlation is indeed of the form in \autoref{eq:xi_propto_m_kappa} with
\begin{equation}\label{eq:exact_value_kappa}
  \kappa = \frac{6}{c \left( \sqrt{12/c} + 1 \right) },
\end{equation}
which is in good agreement with the values found in \cite{tagliacozzo2008scaling}.

\todo[inline]{Refer back to chapter on spectrum of CTM}

\section{Locating the critical point with the entanglement spectrum}\label{sec:locating_critical_point_entanglement}
Since phase transitions of quantum systems can be located by studying their entanglement spectrum
\cite{huang2017holographic, osborne2002entanglement}, classical systems may be investigated in the same way through the
correspondence in \autoref{eq:correspondence_density_matrix_ctm}.
This is an alternative to the usual approach of studying an order parameter or derivatives of thermodynamical
observables.

Examples of studies using the spectrum of the corner transfer matrix to analyze two-dimensional classical systems are
\cite{krvcmar2015reentrant, PhysRevE.94.022134, krvcmar2016phase}.

At the critical point, the entropy must diverge (cf.
\autoref{eq:entropy_scaling_near_criticality}).
For finite systems the entropy will remain finite, but the pseudocritical temperature $T^{\star}$ is defined as the
point of maximum entropy.
The critical point is then located by fitting the scaling law in \autoref{eq:scaling_law_T_star}.

%
\chapter{Methods}
\section{Abstract}

We describe the technical details of the algorithms used to compute quantities of interest.
We report the convergence behaviour of the algorithms and discuss validity and sources of error.

\section{Technical details}
For the models treated in this thesis, the corner transfer matrix $A$ and the row-to-row transfer matrix $T$ are
symmetric. But due to the accumulation of machine-precision sized errors in the matrix multiplication and singular value
decomposition, this will, after many algorithm steps, no longer be the case. In order for results to remain valid, we
manually enforce symmetricity after each step.

The tensor network contractions at each algorithm step will cause the elements of $A$ and $T$ to tend to infinity, which
means that they will at some point exceed the maximum value of a floating point number as it can be stored in memory.
But because the elements of $A$ and $T$ represent Boltzmann weights, they can be scaled by a constant factor, which
allows us to prevent this overflow if we use a suitable scaling. For example by requiring that
\begin{equation}
  \tr A^4 = 1,
\end{equation}
so that the interpretation of $A^4$ as a reduced density matrix of an effective one-dimensional quantum is valid.

\section{Convergence criteria}

\subsection{Simulations with finite bond dimension}
The convergence of the CTMRG algorithm with fixed bond dimension $m$ (the infinite system algorithm) can be defined
in multiple ways (\emph{cite}). In this thesis, the convergence after step $i$ of the algorithm is defined as
\begin{equation}
  c_i = \sum_{\alpha = 1}^{m} | s_{\alpha}^{(i)} - s_{\alpha}^{(i - 1)} |,
\end{equation}
where $s_{\alpha}$ are the singular values of the corner transfer matrix $A$. If the convergence falls below some
threshold $\epsilon$, the algorithm terminates.

The assumption is that once the singular values stop changing to some precision, the optimal projection is sufficiently
close to its fixed point and the transfer matrices $A$ and $T$ represent an environment only limited by the length scale
given by $m$, i.e.
\begin{equation}
  \xi(m) \ll N
\end{equation}
is satisfied.

The convergence of the order parameter of the Ising model is shown in
\autoref{fig:convergence_order_parameter_finite_chi}.

\todo[inline]{Cross check with correlation length, report on boundary conditions}

\begin{figure}
  \includegraphics[]{convergence_order_parameter_finite_chi.tikz}
  \caption{hallootjes}\label{fig:convergence_order_parameter_finite_chi}
\end{figure}


\subsection{Simulations with finite system size}
In the finite-system algorithm, we require
\begin{equation}
  N \ll \xi(m),
\end{equation}
so the question becomes at which $m$ the results are sufficiently converged in $m$. Throughout this work, we have used
the residual probability (also called truncation error)
\begin{equation}
  P(m)^{(i)} = \frac{\sum_{\alpha = m + 1}^{dm} (s_{\alpha}^{(i)})^2 }{ \sum_{\alpha = 1}^{dm} (s_{\alpha}^{(i)})^2 },
\end{equation}
which quantifies the fraction of the spectrum of the corner transfer matrix that is thrown away, as a measure of how
accurate the transfer matrices represent the finite system of size $N$. Here, $d$ is the
dimension of the local tensors ($d = 2$ for the Ising model).

If, for a given $m$, we have
\begin{equation}
  P(m) < P_{\text{max}}
\end{equation}
we deem the result accurate enough.

In the limit $m \to d^n$, with
\begin{equation}
  n = \frac{N - 1}{2}
\end{equation}
the number of algorithm steps, we obtain the exact result for the transfer matrices and hence the partition function,
i.e. $P(m) \to 0$.

To justify that for small enough $P_{\max}$, we obtain good results for a wide range of $N$, figure bla bla.

% %
\chapter{Numerical results for the Ising model}
\begin{abstract}
We present numerical results of finite-$m$ and finite-size scaling within the CTMRG method on the
Ising model.
\end{abstract}

\section{At the critical point}

\subsection{Existence of two length scales}

First, we reproduce the results presented in \cite{nishino1996numerical} to validate the assumption that at the critical
point, the only relevant length scales are the system size $N$ and the length scale associated to a finite dimension $m$
of the corner transfer matrix $\xi(m)$.
Here, we assume that $\xi(m)$ is given by the correlation length at the critical point,
see \autoref{sec:definition_effective_length_scale_in_terms_of_xi}.

The order parameter\footnote{It is worth stressing that the order parameter and the magnetization per site are used
interchangeably for the Ising model, and that the magnetization per site is approximated,
within the CTMRG algorithm, by the expectation value of the central spin.
See \autoref{sec:magnetization_per_site}.} should obey the following scaling relation at the critical temperature
\begin{equation}\label{eq:order_param_scaling_relation_finite_m}
  M(T = T_c,m) \propto \xi(T = T_c, m)^{-\beta/\nu}.
\end{equation}
The left panel of \autoref{fig:order_parameter_power_law_fit} shows that this scaling relation holds.
The fit yields $\frac{\beta}{\nu} \approx 0.125(5)$, close to the true value of $\frac{1}{8}$.

The right panel shows the conventional finite-size scaling relation
\begin{equation}\label{eq:order_param_scaling_relation_finite_N}
  M(T = T_c, N) \propto N^{-\beta/\nu},
\end{equation}
yielding $\beta/\nu \approx 0.1249(1)$.

The correlation length $\xi(m)$ shows characteristic half-moon patterns on a log-log scale,
stemming from the smeared-out stepwise pattern in the corner transfer matrix spectrum (see
\autoref{sec:spectrum_of_ctm}).
This makes the data harder to interpret, since the effect of increasing $m$ depends on how much of the spectrum is
currently retained.

\begin{figure}
  % \includegraphics[width=\textwidth, axisratio=1]{order_parameter_power_law_fit.tikz}
  \includegraphics[]{order_parameter_power_law_fit.tikz}
  \caption{Left panel: fit to the relation in
  \autoref{eq:order_param_scaling_relation_finite_m}, yielding $\frac{\beta}{\nu} \approx
  0.125(5)$. The data points are obtained from simulations with $m = 2, 4, \dots, 64$. The
  smallest 10 values of $m$ have not been used for fitting, to diminish correction terms
  to the basic scaling law. Right panel: fit to conventional finite-size scaling law
  given in \autoref{eq:order_param_scaling_relation_finite_N}, fitted with $n = 1500, 1750, \dots, 4000$, calculated with a truncation error no larger than $10^{-7}$, yielding $\beta/\nu \approx 0.1249$.
  }
  \label{fig:order_parameter_power_law_fit}
\end{figure}

To further test the hypothesis that $N$ and $\xi(m)$ are the only relevant length scales,
the authors of \cite{nishino1996numerical} propose a scaling relation for the order
parameter $M$ at the critical temperature of the form
\begin{equation}\label{eq:order_param_scaling_relation}
  M(N, m) = N^{-\beta/\nu} \mathcal{G}(\xi(m) / N)
\end{equation}
with
\begin{equation}
  \mathcal{G}(x) =
  \begin{cases}
    \text{const} & \text{if } x \to \infty, \\
    x^{-\beta/\nu} & \text{if } x \to 0,
  \end{cases}
\end{equation}
meaning that \autoref{eq:order_param_scaling_relation} reduces to
\autoref{eq:order_param_scaling_relation_finite_N} in the limit $\xi(m) \gg N$ and to
\autoref{eq:order_param_scaling_relation_finite_m} in the limit $N \gg \xi(m)$.
\autoref{fig:data_collapse_nishino} shows that the scaling relation of \autoref{eq:order_param_scaling_relation}
is justified.

\autoref{fig:crossover_nishino} shows the cross-over behaviour from the $N$-limiting regime, where
$M(N, m) \propto N^{-\beta/\nu}$ to the $\xi(m)$-limiting regime, where $M(N, m)$ does not depend on $N$.

\begin{figure}
  \includegraphics[]{data_collapse_nishino.tikz}
  \caption{Scaling function $\mathcal{G}(\xi(m)/N)$ given in
  \autoref{eq:order_param_scaling_relation}.}\label{fig:data_collapse_nishino}
\end{figure}

\begin{figure}
  \includegraphics[]{crossover_nishino.tikz}
  \caption{Behaviour of the order parameter at fixed $m$ as function of
  the number of renormalization steps $n$. For small $n$, all curves coincide, since the system size is the only
  limiting length scale. For large enough $n$, the order parameter is only limited by the length scale
  $\xi(m)$. In between, there is a cross-over described by $\mathcal{G}(\xi(m)/N)$, given in
  \autoref{eq:order_param_scaling_relation}.}\label{fig:crossover_nishino}
\end{figure}

\subsection{Central charge}
We may directly verify the value of the central charge $c$ associated with the conformal field theory at the critical
point by fitting to
\begin{equation}\label{eq:entropy_vs_correlation_length}
  S_{\text{classical}} \propto \frac{c}{6} \log \xi(m),
\end{equation}
which yields $c = 0.501$, shown in the left panel of \autoref{fig:entropy_vs_correlation_length}.

The right panel of \autoref{fig:entropy_vs_correlation_length} shows the fit to the scaling relation in $N$ (or,
equivalently the number of CTMRG steps $n$)
\begin{equation}\label{eq:entropy_vs_N}
  S_{\text{classical}} \propto \frac{c}{6} \log N,
\end{equation}
which yields $c = 0.499$.

\begin{figure}
  \includegraphics[]{entropy_vs_correlation_length.tikz}

  \caption{Left panel:
numerical fit to \autoref{eq:entropy_vs_correlation_length}, yielding $c = 0.501$.
Here, $m \in \{ 8, 10, \dots, 70 \}$ and the convergence threshold $\epsilon = 10^{-9}$.
Right panel:
numerical fit to \autoref{eq:entropy_vs_N}, yielding $c = 0.499$,
with the fit made to $n \in \{ 1500, 1550, \dots 2500 \}$, such that the truncation error is smaller than
$10^{-7}$.}
\label{fig:entropy_vs_correlation_length}
\end{figure}

\subsection{Using the entropy to define the correlation length}\label{sec:entropy_to_define_correlation_length}
Via \autoref{eq:entropy_scaling_near_criticality}, the correlation length is expressed as
\begin{equation}\label{eq:correlation_length_as_function_of_entropy}
  \xi \propto \exp(\frac{6}{c}S).
\end{equation}

\autoref{fig:order_parameter_power_law_fit_entropy} shows the results of fitting the relation in
\autoref{eq:order_param_scaling_relation_finite_m} with this definition of the correlation length. The fit is an order
of magnitude better in the least-squares sense, and the half-moon shapes have almost disappeared,
yielding a much more robust exponent of $\beta/\nu = 0.12498$.

The entropy uses all eigenvalues of the corner transfer matrix, making it apparently less prone to structure in the
spectrum than the correlation length as defined in \autoref{eq:correlation_length_row_to_row_transfer_matrix},
which uses only two eigenvalues of the row-to-row transfer matrix.
Furthermore, the corner transfer matrix $A$ is kept diagonal in the CTMRG algorithm,
so $S$ is much cheaper to compute than $\xi$.

\begin{figure}
  \includegraphics[]{order_parameter_power_law_fit_entropy.tikz}
  \caption{Fit to
  \autoref{eq:order_param_scaling_relation_finite_m}, using \autoref{eq:correlation_length_as_function_of_entropy} as the
  definition of the correlation length.
  For the fit, we have used $m \in \{ 10, 11, \dots, 66 \}$, calculated with convergence threshold $\epsilon = 10^{-9}$, yielding $\beta/\nu = 0.12498$.}
  \label{fig:order_parameter_power_law_fit_entropy}
\end{figure}

\subsection{Exponent $\kappa$}

We now check the validity of the relation
\begin{equation}\label{eq:xi_propto_kappa_2}
  \xi(m) \propto m^{\kappa}
\end{equation}
in the context of the CTMRG method for two-dimensional
classical systems. Similar checks were done for one-dimensional quantum systems in \cite{tagliacozzo2008scaling}.

Let us first state that boundary conditions are relevant.
From \autoref{sec:spectrum_of_ctm} we expect that for fixed boundary conditions,
the entropy and therefore the correlation length is lower for a given bond dimension $m$.

There are various ways of extracting the exponent $\kappa$.
\autoref{fig:support_for_kappa} shows the results for fixed boundary conditions and \autoref{fig:support_for_kappa_free}
for free boundary conditions.

Directly checking \autoref{eq:xi_propto_kappa_2} yields $\kappa = 1.93$ for a fixed boundary
and $\kappa = 1.96$ for a free boundary.

Under the assumption of \autoref{eq:xi_propto_kappa_2}, we have the following scaling laws at the critical point
\begin{align}\label{eq:scaling_laws_order_param_free_energy_kappa}
  M(m) & \propto m^{-\beta \kappa / \nu} \\
  f(m) - f_{\text{exact}} & \propto m^{(2-\alpha)\kappa / \nu}
\end{align}
for the order parameter and the singular part of the free energy, respectively.
With a fixed boundary, a fit to $M(m)$ yields $\kappa = 1.93$.
For a free boundary we cannot extract any exponent, since $M = 0$ for every temperature.
A fit to $f(m) - f_{\text{exact}}$ yields $\kappa = 1.90$ for a fixed boundary and $\kappa = 1.93$ for a free boundary.
\autoref{fig:support_for_kappa}. Here, we have used $\beta = 1/8$, $\nu = 1$ and $\alpha = 0$ for the Ising model.

We may use \autoref{eq:entropy_scaling_near_criticality} and \autoref{eq:classical_entropy} to check the
relation
\begin{equation}\label{eq:scaling_law_entropy_kappa}
  S_{\text{classical}} \propto \frac{c\kappa}{6}\log m,
\end{equation}
which yields $\kappa = 1.93$ for a fixed boundary and $\kappa = 1.96$ for a free boundary,
with $c = 1/2$ for the Ising model.

\begin{figure}
  \includegraphics[]{support_for_kappa.tikz}
  \caption{Numerical evidence for \autoref{eq:xi_propto_kappa_2}, \autoref{eq:scaling_laws_order_param_free_energy_kappa},
  \autoref{eq:scaling_law_entropy_kappa} with fixed boundary, yielding, from left to right and top to bottom, $\kappa = \{ 1.93, 1.93, 1.90,
  1.93 \}$. These values have been calculated from simulations with $m \in \{8, 10, \dots, 70\} $ and convergence threshold $\epsilon = 10^{-9}$. }\label{fig:support_for_kappa}
\end{figure}

\begin{figure}
  \includegraphics[]{support_for_kappa_free.tikz}
  \caption{Numerical evidence for
\autoref{eq:xi_propto_kappa_2} with free boundary, yielding from left to right and then bottom $\kappa = \{ 1.96,
1.93, 1.96 \}$.
These values have been calculated from simulations with $m \in \{10,
11, \dots, 66 \}$, but with $m \in \{13, 19, 28, 29, 40, 41, 42,
59 \}$ left out, because for those values $m$ the system breaks its symmetry (see
\autoref{sec:implications_for_finite_m_simulations}).
The convergence threshold was chosen to be $\epsilon = 10^{-7}$.
It is not lower since more values $m$ break symmetry as machine precision is approached.}
\label{fig:support_for_kappa_free}
\end{figure}

\subsubsection{Comparison with exact result in asymptotic limit}

The predicted value for $\kappa$ \cite{pollmann2009theory} is $2.034\dots$ (see also \autoref{eq:exact_value_kappa}).
With the CTMRG method, we extract the slightly lower value of $1.96$ (corresponding to free boundary conditions).
But, the structure in the quantities as function of $m$ makes it hard to get an accurate fit to $\kappa$.

It is interesting to note that for fixed boundary conditions, the relation in \autoref{eq:xi_propto_kappa_2} holds,
but with a lower exponent $\kappa$.
This is to be expected, since half the spectrum of the corner transfer matrix is missing.

\section{Locating the critical point}\label{sec:locating_the_critical_point}

In general, the critical point is not known, but it may be located by extrapolating the position of the pseudocritical temperature at finite system sizes.

The pseudocritical point can be defined in a variety of ways.
In this chapter, we will define the pseudocritical point as the point of maximum entropy,
as described in \autoref{sec:locating_critical_point_entanglement}.
\autoref{fig:entropy_vs_T} shows the classical analogue to the entanglement entropy as a function of temperature for
different values of $m$.

The critical point is located by fitting the scaling law in \autoref{eq:scaling_law_T_star}.

\subsection{Finite $m$}
For approximations with finite bond dimension $m$,
it is not clear what length scale should be used to fit the scaling behaviour of $T^{\star}(m)$.
\autoref{fig:T_pseudocrit_chi_power_law_fit} shows the fits for different choices of this length scale.
The results are tabulated in \autoref{table:T_star_nu_results}.
To obtain $T^{\star}$, we have calculated $T^{\star}(m)$ for $m \in \{10,
11, \dots, 60\}$ with a convergence threshold of $10^{-8}$ and a temperature tolerance of $10^{-8}$.
The boundaries are fixed to $+1$.

We denote the estimated value of the critical temperature as $\widetilde{T_c}$. Recall that the exact value is
\begin{equation}
  T_c = 2.2691853\dots
\end{equation}
and
\begin{equation}
  \nu = 1.
\end{equation}

When using $\xi(T_c, m)$, the correlation length at the exact critical point,
the result shows a lot of structure, yielding $\widetilde{T_c} = 2.269172$ and $\nu = 1.057$.

If, instead, the correlation length at the estimated pseudocritical temperature $\xi(T^{\star}(m))$ is used,
the data shows less structure and we obtain the much more precise results $\widetilde{T_c} = 2.269183$ and $\nu =
1.002$.

Another option is to use the entropy to define the correlation length,
via \autoref{eq:correlation_length_as_function_of_entropy}, which gave more accurate results than using the transfer
matrix definition in \autoref{sec:entropy_to_define_correlation_length}.
In this case, the results are slightly worse than the transfer matrix definition:
$T_c = 2.269183$ and $\nu = 1.02$.

Finally, we may directly fit the law
\begin{equation}
  |T_c - T^{\star}(m)| \propto m^{-\kappa/\nu},
\end{equation}
yielding $T_c = 2.269181$ and $\kappa/\nu = 1.91$.
Incidentally, this is another way to confirm $\kappa \approx 1.9$ for systems with a fixed boundary.


\subsection{Finite $N$}

As a cross check, we can instead use systems of finite size to extract $T_c$ and $\nu$.
We have calculated $T^{\star}(n)$ for $n \in \{ 2300, 2500, \dots, 7900 \}$,
with $m$ big enough such that the truncation error is no larger than $10^{-6}$.
This yields $T_c = 2.269185$ and $\nu = 0.98$.

\begin{table}[]
\centering
\begin{tabular}{@{}lll@{}} \toprule
$N_{\text{eff}}$                  & $T_c$   & $\nu$   \\ \midrule
$\xi(T_c, m)$                     & 2.269172  & 1.057         \\
$\xi(T^{\star}(m))$               & 2.269183   & 1.002        \\
$\exp((6/c)S(T^{\star}(m))$       & 2.269183   & 1.02       \\
$m^{\kappa}$                      & 2.269181  & --        \\
$N$                               & 2.269185  & 0.98          \\ \bottomrule
\end{tabular}
  \caption{Results for fits to the scaling law \autoref{eq:scaling_law_T_star} using different length scales.
  When using $m^{\kappa}$, $\kappa \approx 1.91$ was found to give the best fit.} \label{table:T_star_nu_results}
\end{table}

\begin{figure}
  \includegraphics[]{entropy_vs_T.tikz}
  \caption{Classical analogue to the entanglement entropy, as in \autoref{eq:classical_entropy},
  near the critical point (shown as dashed line).}\label{fig:entropy_vs_T}
\end{figure}

\begin{figure}
  \includegraphics[]{T_pseudocrit_chi_power_law_fit.tikz}
  \caption{Fits to the scaling law \autoref{eq:scaling_law_T_star}.
  Results for the critical temperature and exponent $\nu$ are tabulated in
  \autoref{table:T_star_nu_results}.}\label{fig:T_pseudocrit_chi_power_law_fit}
\end{figure}

\section{Away from the critical point}

We may also verify the validity of the different length scales by asserting that the data for different values of $m$
should collapse on a single curve
\begin{equation}
  \mathcal{G}(t \xi(m)^{1/\nu}) = M(T, m) N_{\xi(m)^{\beta/\nu}}.
\end{equation}

All data points were calculated with a convergence threshold of $10^{-7}$.
The values of the pseudocritical temperatures are taken from the results in \autoref{sec:locating_the_critical_point}.
No temperatures beyond $T_c$ is considered because the order parameter drops off sharply,
causing the curve $\mathcal{G}(x)$ to tend to zero almost vertically, making the fitness $P$ unreliable.

\autoref{fig:data_collapse_chi} shows that for all length scales, the results more or less fall on one curve.
\autoref{table:fitness_data_collapse_different_length_scales} shows the fitness of the data collapse
\cite{bhattacharjee2001measure} (given by \autoref{eq:fitness_data_collapse}) for all length scales used.

\todo[inline]{Say which length scales apparently don't work so well}

Using $m^{\kappa}$ as a length scale for optimized fitness $P(\kappa)$ yields $\kappa \approx 1.98$,
substantially higher than found previously for fixed boundary conditions.

As a cross-check, the bottom-right panel of \autoref{fig:data_collapse_chi} shows data points for finite-$N$
simulations. Here, the bond dimension is chosen such that the truncation error is smaller than $10^{-6}$.

\begin{figure}
  \includegraphics[]{data_collapse_chi.tikz}
  \caption{Data collapses using different length scales.
  For the bottom-right plot, approximations with finite $N$ instead of finite $m$ have been used,
  with $n = \{160, 480, 1000, 1500 \}$ ($n = \frac{N - 1}{2}$ is the number of algorithm
  steps).}\label{fig:data_collapse_chi}
\end{figure}

\begin{table}[]
\centering
\begin{tabular}{@{}ll@{}} \toprule
$N_{\text{eff}}$                  & fitness $P$ \\ \midrule
$\xi(T_c, m)$                     & 0.0075      \\
$\xi(T^{\star}(m))$               & 0.066       \\
$\exp((6/c)S(T_c, m))$            & 0.057       \\
$\exp((6/c)S(T^{\star}(m))$       & 0.087       \\
$m^{\kappa}$                      & 0.0080      \\
$N$                               & 0.0075      \\ \bottomrule
\end{tabular}
  \caption{Fitness of data collapse (\autoref{eq:fitness_data_collapse}) for different length scales.
  $\kappa \approx 1.98$ was found to be optimal for the length scale $m^{\kappa}$.}
  \label{table:fitness_data_collapse_different_length_scales}
\end{table}


\section{Discussion}

\todo[inline]{Why do length scales defined at $T^{\star}$ work better??
It is fortunate that we don't need the length scales at $T_c$, since we don't know it.}


\chapter{Numerical results for the clock model}
\chapterprecishere{We present results of scaling in bond dimension and system size with the CTMRG algorithm for the
five- and six-state clock model.}

\todo[inline]{Not finished yet.}

\section{Introduction}
In the field of phase transitions and critical phenomena, the two-dimensional topological phase transition discovered by
Kosterlitz and Thouless \cite{kosterlitz1973ordering, kosterlitz1974critical} receives much attention. This phase
transition is characterized not by an order parameter which indicates a breaking of symmetry, but by the proliferation
of topological defects.

In the low-temperature phase, the two-point correlation functions decay with a power-law with varying
exponent $\eta(T)$. At the transition, the correlation length diverges as
\begin{equation}\label{eq:corr_length_divergence_kt}
  \xi \propto \exp(A |T - T_c|^{-1/2}),
\end{equation}
with $A$ a non-universal constant. Above the transition, the two-point correlators decay exponentially.

The XY model consists of planar rotors on the square lattice. It exhibits the Kosterlitz-Thouless (KT) phase transition
and by the Mermin-Wagner-Hohenberg theorem the symmetry of the ground state is broken for all temperatures, due to
the $O(2)$ (planar rotational) symmetry of the potential \cite{mermin1966absence, hohenberg1967existence}.

The $q$-state clock model possesses the discrete $\mathbb{Z}_q$ symmetry and is an interpolation between the Ising
model, which corresponds to $q = 2$, and the XY model, which corresponds to $q \to \infty$. Its energy function is given
by
\begin{equation}\label{eq:hamiltonian_clock_model}
  H_q = -\sum_{\langle i j \rangle} \cos(\theta_i - \theta_j),
\end{equation}
with the spins taking the values
\begin{equation}
  \theta = \frac{2 \pi n}{q} \qquad n \in \{ 0, \dots, q-1 \}.
\end{equation}

It has been proven that for high enough $q$, this model indeed exhibits a Kosterlitz-Thouless transition
\cite{frohlich1981kosterlitz}. Furthermore, it has been proven that for $q \geq 5$, a general model with $\mathbb{Z}_q$
symmetry (of which \autoref{eq:hamiltonian_clock_model} is a special case) has three phases: a symmetry broken phase for
$T < T_1$, an intermediate phase with power law decay of the correlation function, and a high-temperature phase with
exponential decay of the correlation function for $T > T_2$ \cite{cardy1980general}.

In the Villain formulation of the potential \cite{villain1975theory}, it has been proven that the transition at $T_2$ is
a BK-transition \cite{jose1977renormalization}, and numerical results suggest that for a broad range of temperatures,
the thermodynamic behaviour becomes identical to the XY model for high enough $q$ \cite{lapilli2006universality}.

Furthermore, in the Villain formulation it is known that \cite{elitzur1979phase, nienhuis1984critical}
\begin{equation}\label{eq:eta_villain}
  \eta(T_1) =\frac{4}{q^2}, \qquad \eta(T_2) = \frac{1}{4},
\end{equation}
where $\frac{\eta}{2} = \frac{\beta}{\nu}$, the magnetization exponent in the finite-size regime.

For the cosine model in \autoref{eq:hamiltonian_clock_model}, the value $q_c$ for which it first exhibits a
BK-transition is not precisely known.
There is some disagreement about whether the cases $q = 5, 6$ exhibit KT-type transitions (see previous numerical
results below).

In our simulations we will focus on the cases $q = 5, 6$, to (i) study the nature of the phase transition from a new
perspective and (ii) compare the accuracy of finite-$m$ and finite-$N$ scaling within the CTMRG
method to other established numerical methods.

We briefly summarise previous numerical results, then present results obtained with the CTMRG algorithm.

\section{Previous numerical results}
\subsection{The $q = 5$ clock model}

The general consensus is that the two transitions of the $q = 5$ clock model with cosine potential are of the KT-type,
though there are no rigorous results.
It is also assumed that the critical indices are the same as those in the Villain formulation.

The disagreement about the nature of the phase transitions
stems from numerical results for the helicity modulus
\cite{fisher1973helicity}.

Most notably, Baek and Minnhagen \cite{baek2010non} claim that since the helicity modulus does
not vanish in the high-temperature phase, the upper transition is not of the KT-type.

It was shown by Kumano et al.
in \cite{kumano2013response}, however, that the definition used by Baek and Minnhagen is not suitable for systems with a
discrete symmetry.
The correct discrete definition yields the expected result, namely that the helicity modulus does vanish and the
three-phase KT-picture holds.

The conclusion of Kumano et al, which was obtained by a Monte Carlo study,
was verified by Chatelain \cite{chatelain2014dmrg} using the TMRG algorithm \cite{nishino1995density} (see also
\autoref{sec:tmrg}).
Chatelain also found that the critical indices match those of the Villain model (\autoref{eq:eta_villain}),
implying the cosine model is in the same universality class as the Villain model.

After the rebuttal by Kumano et al., Baek et al.
published another work \cite{baek2013residual} in which they again use the (in the eyes of Kumano et al.) wrong
definition of the helicity modulus, yet calculated in a different way.
Again they conclude the transition is not of the KT-type.

Meanwhile, Borisenko et al.
\cite{borisenko2011numerical} carried out a very detailed Monte Carlo study confirming the KT-picture,
using Binder-cumulants to find the critical points and the magnetization and susceptability to find the critical
indices.

Brito et al.
\cite{brito2010twodimensional} conclude from a Monte Carlo study that while the transition is of KT-type,
the resolution of their numerical method is not high enough to distinguish between $T_1$ and $T_2$.

\autoref{table:q5_previous_results} shows the results for the transition temperatures found by other authors.

\subsection{The $q = 6$ clock model}

Here, there is overwhelming consensus that both transitions are of the KT-type.
The only exceptions are Lapilli et al. \cite{lapilli2006universality} and Hwang \cite{hwang2009six}.

Lapilli et al. use the incorrect definition of the helicity modulus.

Hwang asserts that the transition is not of KT-type because the data,
which was obtained from systems of rather small size ($L \times L$-systems with $L = 20,
\dots, 28$), also agrees with a power-law divergence of the correlation length. We will get back to this point.

The previous results for the transition temperatures are listed in \autoref{table:q6_previous_results}.
For an overview that goes further back, see \cite{krvcmar2016phase}.

We note that \cite{tomita2002probability, brito2010twodimensional, kumano2013response} use Monte Carlo methods,
while \cite{krvcmar2016phase} uses the CTMRG algorithm (combined with finite-size scaling,
but not with finite-$m$ scaling).

\begin{table}[]
\centering
\begin{tabular}{@{}lll@{}}
\toprule
 & $T_1$ & $T_2$ \\ \midrule
Brito et al.\tablefootnote{These authors found $T_1 > T_2$, which is not an error in the text, but due to the low resolution of the methods used.} (2010) \cite{brito2010twodimensional} & 0.91 & 0.90 \\
Borisenko et al. (2011) \cite{borisenko2011numerical} & 0.9056 & 0.9432 \\
Kumano et al. (2013) \cite{kumano2013response} & 0.908  & 0.944  \\
Chatelain (2014) \cite{chatelain2014dmrg} & 0.914 & 0.945  \\ \midrule
This work (finite-$N$ scaling) & 0.915 & 0.935 \\
This work (finite-$m$ scaling) & - & 0.944 \\ \bottomrule
\end{tabular}
\caption{Previous results for the transition temperatures for $q = 5$.}
\label{table:q5_previous_results}
\end{table}

\begin{table}[]
\centering
\begin{tabular}{@{}lll@{}}
\toprule
 & $T_1$ & $T_2$ \\ \midrule
Tomita and Okabe (2002) \cite{tomita2002probability} & 0.7014 & 0.9008 \\
Hwang\tablefootnote{To obtain these values, the author assumed an algebraic divergence of the correlation length.} (2009) \cite{hwang2009six} & 0.632 & 0.997 \\
Brito et al. (2010) \cite{brito2010twodimensional} & 0.68 & 0.90 \\
Kumano et al. (2013) \cite{kumano2013response} & 0.700 & 0.904 \\
Krčmár et al. (2016) \cite{krvcmar2016phase} & 0.70 & 0.88 \\ \midrule
This work (finite-$N$ scaling) & 0.700 & 0.883 \\
This work (finite-$m$ scaling) & - & 0.901 \\ \bottomrule
\end{tabular}
\caption{Previous results for the transition temperatures for $q = 6$.}
\label{table:q6_previous_results}
\end{table}

\section{Spectrum of the corner transfer matrix}

In order to get an idea of the accuracy that we might expect, we have plotted the spectrum of the $q = \{5,
6\}$ clock models in \autoref{fig:spectrum_ctm_clock}.

It is clear that the spectra of both clock models fall off at about the same pace,
if we compare points in the ordered, massless and disordered phase.
The $q = 6$ clock model has a slightly more degenerate spectrum, as might be expected from its larger symmetry group,
but there is no clear pattern.

As compared to the Ising model (see \autoref{sec:spectrum_of_ctm}), the spectra of the $q = \{5,
6\}$ clock models fall off much more slowly\footnote{For the calculation of the spectrum of the Ising model in this
work, a bond dimension of $m = 250$ was used, as opposed to $m = 100$ for the clock model.
This means that, in small part, the slower decay of the spectrum is due to the normalization $\tr A^4 = 1$.
But this does not change the general picture that the spectra of the $q = \{5,
6\}$ clock models decay more slowly.}.
This implies that a much larger bond dimension is needed to obtain the same accuracy for quantities in the thermodynamic
limit.

\begin{figure}
  \includegraphics[]{spectrum_ctm_clock.tikz}
  \caption{First part of the spectrum of $A$ with fixed boundary,
  calculated with $m = 100$ and a convergence threshold of $10^{-8}$,
  at temperatures corresponding to the ordered phase, approximate midpoint of the massless phase and disordered phase,
  respectively.}\label{fig:spectrum_ctm_clock}
\end{figure}

\section{Magnetization}

For the clock model, we define the magnetization per site as
\begin{equation}\label{eq:magnetization_clock_model}
  M = \langle \cos \theta_0 \rangle,
\end{equation}
where $\theta_0$ is the central spin.

This quantity can be computed in the same way as for the Ising model (see \autoref{sec:magnetization_per_site}) by
generalizing the tensor $b_{i j k l}$ to
\begin{equation}
  b_{i j k l} = \sum_{n \in \{ 0, \dots, q-1 \}} \cos \left( \frac{2\pi n}{q} \right) \delta_{n i j k l}.
\end{equation}

\section{Classical analogue to the entanglement entropy}

The classical analogue to the half-chain entanglement entropy $S$ is defined in \autoref{sec:analogy_to_entropy}.
Its definition remains valid.

In the limit $T \to \infty$, for both a fixed and free boundary, we have
\begin{equation}
  S(T \to \infty) = 0.
\end{equation}

To see this, consider that all $2^{2N}$ configurations on the inner edges of the $N \times N$ quadrant represented by
the corner transfer matrix are equally likely in this limit, hence
\begin{equation}
  A_{i j} = \frac{1}{2^{2N}},
\end{equation}
which has one eigenvalue of 1 and the others 0\footnote{One can also make the argument that the corresponding
quantum state tends to a product state in the limit $T \to 0$,
yielding the same conclusion.}.

In the limit $T \to 0$, there is only one non-zero matrix element in the case of a fixed boundary (namely all inner
spins having the same value as the outer boundary), and $q$ equally likely configurations in the case of a free
boundary, yielding
\begin{align*}
  S^{\text{fixed}}(T = 0) &= 0, \\
  S^{\text{free}}(T = 0)  &= \log q.
\end{align*}

For a fixed boundary, the point of maximum entropy approaches the massless phase from the high-temperature region,
hence tending towards $T_2$.
In contrast, the point of maximum entropy approaches $T_1$ for systems with a free boundary.
\todo[inline]{Maybe make intuitive why?}

\autoref{figure:entropy_and_order_param_vs_T_clock5} and \autoref{figure:entropy_and_order_param_vs_T_clock6} show
these quantities for $q = 5$ and $q = 6$ for systems with a fixed boundary, clearly confirming the three-phase picture.

\begin{figure}
  \includegraphics[]{entropy_and_order_param_vs_T_clock5.tikz}
  \caption{Classical analogue to half chain entanglement entropy (\autoref{sec:analogy_to_entropy}) and magnetization
  (\autoref{eq:magnetization_clock_model}) computed for systems with a fixed boundary for the $q = 5$ clock model.
  Simulations were done with a convergence threshold of $10^{-7}$.}\label{figure:entropy_and_order_param_vs_T_clock5}
\end{figure}

\begin{figure}
  \includegraphics[]{entropy_and_order_param_vs_T_clock6.tikz}
  \caption{Classical analogue to half chain entanglement entropy (\autoref{sec:analogy_to_entropy}) and magnetization
  (\autoref{eq:magnetization_clock_model}) computed for systems with a fixed boundary for the $q = 6$ clock model.
  Simulations were done with a convergence threshold of $10^{-7}$.}\label{figure:entropy_and_order_param_vs_T_clock6}
\end{figure}

\section{Transition temperatures}

Since we expect an essential singularity of the form in \autoref{eq:corr_length_divergence_kt} for both transitions,
for finite systems we have
\begin{equation}\label{eq:kt_t_pseudocrit}
  N = a \exp \left( b |T^{\star}(N) - T_c|^{-1/2}  \right),
\end{equation}
where $N$ is an effective finite length scale of the system and $a$ and $b$ are non-universal constants.

$N$ is the system size in the case of finite-size scaling and a length scale derived $\xi(m)$ from the bond dimension
$m$ in the case of finite-$m$ scaling.
Throughout this chapter, we have defined $\xi(m)$ through the relation
\begin{equation}\label{eq:entropy_as_function_of_xi2}
  S \propto \frac{c}{6} \log \xi(m)
\end{equation}
where $c = 1$ is expected, since the massless phase corresponds to a Gaussian model \cite{kosterlitz1974critical}.
These assumptions are validated in \autoref{sec:central_charge_massless_phase}.

We define $T^{\star}(N)$ as the point of maximum entanglement entropy, as discussed in
\autoref{sec:locating_critical_point_entanglement}.

Inverting \autoref{eq:kt_t_pseudocrit} gives the following relations for the pseudocritical transition temperatures
\begin{align}
  T^{\star}_{1}(N) &= -\frac{\alpha_1}{(\log \beta_1 N)^2} + T_1 \\
  T^{\star}_{2}(N) &= \frac{\alpha_2}{(\log \beta_2 N)^2} + T_2
\end{align}
where $\alpha = b^2$ and $\beta = 1/a$ (we drop the subscripts denoting the transition).

For convenience, we define the scaled length variable
\begin{equation}\label{eq:scaled_length_scale_kt}
  \ell = (\log \beta N)^2,
\end{equation}
such that
\begin{equation}
  T^{\star}(N) - T_c \propto \frac{1}{\ell}.
\end{equation}

\subsection{Numerical difficulties with finite-$m$ simulations around $T_1$}

For both the $q = \{5, 6\}$ clock models, it has been found that locating $T_1^{\star}(m)$ is not possible,
since for systems with a free boundary, numerical errors cause the matrices $A$ and $P$ to lose their symmetry and
converge to a fixed boundary fixed point instead.
This happens after a modest amount of steps, especially near $T_1(m)^{\star}$,
making it impossible to reach any feasible convergence threshold such as $10^{-6}$.

This means that for locating $T_1$, we must rely on finite-size scaling,
whereas for locating $T_2$ we can rely on both finite-size and finite-$m$ scaling.

\subsection{Transition from the ordered to the massless phase $T_1$}

\autoref{figure:t1_fit_q5} and \autoref{figure:t1_fit_q6} show the fits to \autoref{eq:kt_t_pseudocrit} for $q = \{ 5,
6\}$, yielding
\begin{equation}
  T_1^{q = 5} = 0.915, \qquad T_1^{q = 6} = 0.700.
\end{equation}

Conform to the Kosterlitz-Thouless divergence of the correlation length,
the pseudocritical temperatures indeed become linear in $\frac{1}{\ell}$,
with $\ell$ defined in \autoref{eq:scaled_length_scale_kt}.

It is interesting to note that finite-size effects are much more pronounced for $q = 5$.

\begin{figure}
  \centering
  \includegraphics[]{t1_fit_q5.tikz}
  \caption{We find $T_1 = 0.915$ for the $q = 5$ clock model.
  We have fitted the final 8 points $n = \{ 60, 65, 70, 80, 90,
  100, 110, 120 \}$. Not included in the fit are $n = \{ 10, 15, \dots, 55 \}$.
  $m$ was chosen such that the truncation error was smaller than $10^{-6}$ for $n \leq 70$ and smaller than $10^{-5}$
  for $n > 70$.
  In finding the maximum of the entropy, a tolerance in temperature of $10^{-5}$ was used.
  }\label{figure:t1_fit_q5}
\end{figure}

\begin{figure}
  \centering
  \includegraphics[]{t1_fit_q6.tikz}
  \caption{We find $T_1 = 0.700$ for the $q = 6$ clock model.
  We have fitted the final 8 points $n = \{ 35, 40, 45, 50, 55,
  60, 70, 80 \}$.
  Not included in the fit are $n = \{ 15, 20, 25, 30 \}$.
  $m$ was chosen such that the truncation error was smaller than $10^{-6}$ for $n \leq 60$ and smaller than $10^{-5}$
  for $n > 60$.
  In finding the maximum of the entropy, a tolerance in temperature of $10^{-5}$ was used.
  }\label{figure:t1_fit_q6}
\end{figure}

\subsection{Transition from the massless to the disordered phase $T_2$}

\subsubsection{Finite-size scaling}
Finite-size scaling, shown in \autoref{figure:t2_fit_q5_finite_N} for $q = 5$ and \autoref{figure:t2_fit_q6_finite_N}
for $q = 6$ yields
\begin{equation}
  T_2^{q = 5} = 0.935, \qquad T_2^{q = 6} = 0.883.
\end{equation}

For both clock models, finite-size effects are large. For $q = 6$, the finite-size effects are more pronounced
than at $T_1$.

\subsubsection{Finite-$m$ scaling}
Finite-$m$ scaling, shown in \autoref{figure:t2_fit_q5_finite_m} for $q = 5$ and \autoref{figure:t2_fit_q6_finite_m}
for $q = 6$ yields
\begin{equation}
  T_2^{q = 5} = 0.944, \qquad T_2^{q = 6} = 0.901.
\end{equation}

It is seen that with finite-$m$ simulations,
systems of significantly larger effective size can be simulated.
From the finite-size fits to $T_2^{\star}(N)$, it can be estimated that a system of $m = 90$ approximately corresponds
to a $2700 \times 2700$ lattice for $q = 5$ and a $2400 \times 2400$ lattice for $q = 6$.

There is some structure in the data, but as long as a wide range of $m$ values is included,
the estimation of $T_2$ is robust.

\begin{figure}
  \centering
  \includegraphics[]{t2_fit_q5_finite_N.tikz}
  \caption{We find $T_2 = 0.935$ for the $q = 5$ clock model with finite-size scaling.
  We have fitted the final 6 points $n = \{ 85, 90, \dots, 110 \}$.
  $m$ was chosen such that the truncation error was smaller than $10^{-6}$.
  In finding the maximum of the entropy, a tolerance in temperature of $10^{-6}$ was used.}
  \label{figure:t2_fit_q5_finite_N}
\end{figure}

\begin{figure}
  \centering
  \includegraphics[]{t2_fit_q5_finite_m.tikz}
  \caption{
  For the finite-$m$ simulations, the fit yields $T_2 = 0.944$.
  using $m = 20, 25, \dots,
  90$ with a convergence threshold of $10^{-7}$.
  The pseudocritical temperature belonging to $m = 10$ is also shown,
  but is not included in the fit.}\label{figure:t2_fit_q5_finite_m}
\end{figure}

\begin{figure}
  \centering
  \includegraphics[]{t2_fit_q6_finite_N.tikz}
  \caption{We find $T_2 = 0.883$ for the $q = 6$ clock model with finite-size scaling.
  We have fitted the final 6 points $n = \{ 60, 65, \dots, 85 \}$.
  The points $n = \{ 10, 15, \dots, 55 \}$ were not included.
  $m$ was chosen such that the truncation error was smaller than $10^{-6}$.
  In finding the maximum of the entropy, a tolerance in temperature of $10^{-6}$ was used.}
  \label{figure:t2_fit_q6_finite_N}
\end{figure}

\begin{figure}
  \centering
  \includegraphics[]{t2_fit_q6_finite_m.tikz}
  \caption{
  For the finite-$m$ simulations, the fit yields $T_2 = 0.901$.
  using $m = 20, 25, \dots,
  90$ with a convergence threshold of $10^{-7}$.
  The pseudocritical temperature belonging to $m = 10$ is also shown,
  but is not included in the fit.}\label{figure:t2_fit_q6_finite_m}
\end{figure}

\begin{figure}
  \centering
  \includegraphics[]{t2_fit_q5_finite_m.tikz}
  \caption{
  For the finite-$m$ simulations, the fit yields $T_2 = 0.944$.
  using $m = 20, 25, \dots,
  90$ with a convergence threshold of $10^{-7}$.
  The pseudocritical temperature belonging to $m = 10$ is also shown,
  but is not included in the fit.}\label{figure:t2_fit_q5_finite_m}
\end{figure}

\section{The massless phase}

\subsection{Central charge of the massless phase}\label{sec:central_charge_massless_phase}

By fitting the relation in \autoref{eq:entropy_as_function_of_xi2},
where $\xi(m)$ is calculated as in \autoref{sec:definition_effective_length_scale_in_terms_of_xi},
we can directly extract the central charge in this region.

The result is shown in the top panel of \autoref{figure:c_and_magnetic_exponent_vs_T}.
It is seen to precisely agree with $c = 1$ in the massless phase.

Outside the massless phase, a good fit to \autoref{eq:entropy_as_function_of_xi2} can no longer be obtained.
This is consistent with the location of $T_1$ and $T_2$ that are found in this work.

\subsection{Varying exponent for the magnetization}
We may verify the exponent with which the magnetization goes to zero in the massless phase by fitting
\begin{equation}\label{eq:magnetization_scaling_massless_phase}
  M(m, T) = \xi(m)^{-\frac{\eta(T)}{2}},
\end{equation}
where $\xi(m)$ again defined via \autoref{eq:entropy_as_function_of_xi2}.

The result is shown in the middle panel of \autoref{figure:c_and_magnetic_exponent_vs_T}.
It agree very well with bla bla.
Refer to villain exponents.

\todo[inline]{Finish this when $T_1$ is definite}

\begin{figure}
  \centering
  \includegraphics[]{c_and_magnetic_exponent_vs_T.tikz}
  \caption{$\eta = \frac{2\beta}{\nu}$ reaches $\frac{4}{25}$ around $T = 0.906$ and $\frac{1}{4}$ around $T = 0.952$.
  First 4 values were left out of the fit.
}\label{figure:c_and_magnetic_exponent_vs_T}
\end{figure}

% \begin{figure}
%   \centering
%   \includegraphics[]{c_and_magnetic_exponent_vs_T_q6.tikz}
%   \caption{$\frac{2\beta}{\nu}$ reaches $\frac{4}{25}$ around $T = 0.906$ and $\frac{1}{4}$ around $T = 0.952$.
%   First 4 values were left out of the fit.
% }\label{figure:c_and_magnetic_exponent_vs_T_q6}
% \end{figure}

\begin{figure}
  \centering
  \includegraphics[]{fits_magnetization_massless_phase.tikz}
  \caption{Fits to \autoref{eq:magnetization_scaling_massless_phase} for temperatures a little bit to the left of,
  in the middle of and a little bit to the right of the massless phase,
  respectively.
}\label{figure:fits_magnetization_massless_phase}
\end{figure}

\section{Discussion}

\begin{itemize}
  \item doesn't apparently make sense to take many consecutive $m$-values. Rather take sparse, but bigger $m$-values.
  \item For finite-$N$: tradeoff between truncation error and finite-size effects are not clear.
  \item finite-$\chi$ reaches much bigger system sizes, but has structure that needs to be overcome while fitting.
  \item fix bug where I attached an $a$-tensor with wrong temperature.
  \item omitting $m$ values seems to give worse results than just taking them all, except for the really small ones.
  \item second-order phase transition cannot be ruled on grounds of this data, but is there theory that forbids it?
  \item If you make convergence too low, it can get impossible to reach, even for fixed boundary?
\end{itemize}


\appendix
\chapter{Correspondence of quantum and classical lattice
systems}\label{chapter:correspondence_quantum_classical}
The partition of a discrete quantum mechanical system is given by
\begin{equation}\label{eq:quantum_partition_function}
  Z_{q} = \tr \exp(-\beta H_{q}) =
  \sum_{\sigma} \bra{\sigma} \exp(-\beta H_{q}) \ket{\sigma}
\end{equation}
Imagine splitting the imaginary time interval $\beta$ into $N$ smaller steps:
\begin{align}
  \beta &= N \delta \tau, \\
  \exp(-\beta H_q) &= \exp(-\delta \tau H_q)^N.
\end{align}
Recall that for any orthonormal basis, the identity can be expressed as a sum over
projectors onto the basis states
\begin{equation}
    \mathbb{1} = \sum_{\sigma} \ket{\sigma}\bra{\sigma}.
\end{equation}
If we insert $N - 1$ resolutions of indentity into
\autoref{eq:quantum_partition_function}, we obtain
\begin{equation}
  Z_q = \sum_{\sigma} \sum_{\sigma_1, \dots, \sigma_{N-1}}
  \braket{\sigma | \exp(-\beta \delta \tau) | \sigma_1} \dots
  \braket{\sigma_{N-1} | \exp(-\beta \delta \tau) | \sigma}.
\end{equation}

This is the imaginary time path integral formulation of quantum mechanics. Similar to the
real-time path integral, an evolution in the imaginary time direction is expressed as a
sum over all paths connecting the initial and final state, which are the same here, since
we are taking the trace.

We turn to the partition function of a classical system, written as a product of
its transfer matrix, as in \autoref{eq:partition_function_transfer_matrix_1d}:
\begin{equation}
  Z_{\text{cl}} = \tr T^N.
\end{equation}
There is a striking similarity between a quantum mechanical partition function in $d$
dimensions and a classical partition function in $d + 1$ dimensions.
Adding a row to the classical lattice by applying the transfer matrix corresponds to time
evolution of a quantum system:
\begin{equation}
  T \longleftrightarrow \exp(-\delta \tau H_q).
\end{equation}
The classical temperature corresponds to the coupling constants in the Hamiltonian
$H_{\text{q}}$.

Letting $\beta \to \infty$ (or equivalently $T \to 0$) amounts to taking $N \to \infty$,
while keeping $\delta \tau$ fixed. In this limit, analogously to the transfer matrix for
the classical system (cf. \autoref{eq:largest_eigenvector}), the operator $\exp(-\beta
H_{\text{q}})$ becomes a projector onto the ground state.

Thus, a quantum lattice in the ground state corresponds to a
classical lattice that is infinite in its additional dimension.

\todo[inline]{What is the role $\delta \tau$? Why should it be small?
If we want to make the correspondence $T \equiv H_q$, we need $\delta \tau \to 0$.
But what does this imply for the lattice? How does the 'scaling limit' come into play?
Also: explain the correspondence between the energy scale of the quantum system and the
correlation length of the classical system. Maybe do an example of 0D quantum to 1D
classical and 1D quantum to 2D classical (Ising model?).}


\chapter{Introduction to tensor networks}\label{chapter:introduction_to_tensor_networks}
\section{Tensors, or multidimensional arrays}

\todo[inline]{Make clear that there is a difference between tensors that are
defined with a metric.}

In the field of tensor networks, a tensor is a multidimensional table with numbers -- a
convenient way to organize information. It is the generalization of a vector

\begin{equation}
  v_i =
  \begin{bmatrix}
    v_1 \\
    \vdots \\
    v_n
  \end{bmatrix}
\end{equation}

which has one index, and a matrix

\begin{equation}
  M_{i j} =
  \begin{bmatrix}
  M_{1 1} & \dots & M_{1 n} \\
  \vdots  & & \vdots \\
  M_{m 1} & \dots & M_{m n}
  \end{bmatrix}
\end{equation}

which has two. A tensor of rank $N$ has $N$ indices:\footnote{The definition of rank in this
context is not to be confused with the rank of a matrix, which is the number of
linearly independent columns. Synonyms of tensor rank are tensor degree and
tensor order.}

\begin{equation}
  T_{i_1 \dots i_N}
\end{equation}

A tensor of rank zero is just a scalar.

\section{Tensor contraction}

Tensor contraction is the higher-dimensional generalization of the dot product
\begin{equation}
  \bm{a} \cdot \bm{b} = \sum_i a_i b_i
\end{equation}
where a lower-dimensional tensor (in this case, a scalar, which is a
zero-dimensional tensor) is obtained by summing over all values of a repeated
index.

Examples are matrix-vector multiplication
\begin{equation}
  (M \bm{a})_{i} = \sum_j M_{i j} a_j
\end{equation}
and matrix-matrix multiplication
\begin{equation}
  (A B)_{i j} = \sum_k A_{i k} B_{k j},
\end{equation}
but a more elaborate tensor multiplication could look like
\begin{equation}
  w_{a b c} = \sum_{d, e, f} T_{a b c d e f} v_{d e f}.
\end{equation}

As with the dot product between vectors, matrix-vector multiplication and
matrix-matrix multiplication, a contraction between tensors is only defined if
the dimensions of the indices match.

\section{Tensor networks}

A tensor network is specified by a set of tensors, together with a set of contractions to be performed. For example:

\begin{equation}
  M_{a b} = \sum_{i, j, k} A_{a i} B_{i j} C_{j k} D_{k b}
\end{equation}

which corresponds to the matrix product $A B C D$.

\subsection{Graphical notation}
It is highly convenient to introduce a graphical notation that is common in the
tensor network community. It greatly simplifies expressions and makes certain
properties manifest.

Each tensor is represented by a shape. Open-ended lines, called legs, represent
unsummed indices. See \autoref{fig:tensors_graphical_notation}. Each contracted
index is represented by a connected line. See \autoref{fig:contracted_tensors}.

Many tensor equations, while burdensome when written out, are readily
understood in this graphical way. As an example, consider the matrix trace in
\autoref{fig:contracted_tensors}, where its cyclic property is manifest.

\begin{figure}
  \includestandalone{images/tensors_graphical_notation}
  \caption{Open-ended lines, called legs, represent unsummed indices. A tensor
  with no open legs is a scalar.}
  \label{fig:tensors_graphical_notation}
\end{figure}

\begin{figure}
  \includestandalone{images/contracted_tensors}
  \caption{Connected legs represent contracted indices. The networks in the
  figure represent $\sum_i a_i b_i$ (dot product),
  $\sum_j M_{i j} a_j$ (matrix-vector product), $\sum_{k} A_{i k} B_{k
  j}$ (matrix-matrix product) and $\tr A B C D$, respectively.}
  \label{fig:contracted_tensors}
\end{figure}



\subsection{Computational complexity of contraction}
\todo[inline]{Computational complexity.}


% \chapter{Singular value decomposition}
% Hallo



\backmatter
\printbibliography

\end{document}
