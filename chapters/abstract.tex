\noindent This thesis investigates scaling in the number of basis states
kept (the \emph{bond dimension} $m$) in approximating the partition function
of two-dimensional classical models with the corner transfer matrix
renormalization group (CTMRG) method.

For the Ising model, it is shown that exponents and the transition temperature may be approximated with a scaling
analysis in the correlation length defined in terms of the row-to-row transfer matrix at the (pseudo)critical point,
as was suggested by Nishino et al.
However, the calculated quantities show inherent deviations from the basic scaling laws,
due to the spectrum of the underlying corner transfer matrix (CTM).
These deviations are mitigated to some extent when we define the correlation length in terms of the classical analogue
of the entanglement entropy.
Scaling directly in the bond dimension $m$ is also possible, but less accurate since the law for the correlation length
$\xi \propto m^{\kappa}$ holds only in the limit $m \to \infty$ and does not take into account the spectrum of the CTM
that is obtained.

It is found that finite-$m$ scaling and finite-size scaling yield comparable accuracy for critical exponents and the
transition temperature.
With finite-$m$ scaling larger effective system sizes are obtainable,
but finite-size approximations do not suffer from the deviations due to the CTM spectrum and are
consequently of higher quality. Therefore it is plausible that finite-size results will improve significantly if
corrections to scaling are included in the fits.

We also present a numerical analysis of the clock model with $q = \{5,
6\}$ states, concluding that the Kosterlitz-Thouless picture is plausible.
We find values of the transition temperatures that are in agreement with values found by other authors.
Results for the exponent $\eta$ indicate that the critical temperatures found in both this study and previous work might
be too close together.
It is conceivable that, after considering larger systems and taking into account finite-size corrections,
both critical temperatures and the values of $\eta$ will be adjusted outwards towards their true values,
thereby completely reconciling the results.

Overall, we conclude that finite-$m$ scaling is a valuable alternative to finite-size scaling within CTMRG,
since larger system sizes are accessible.
The CTMRG analysis is itself a valuable addition to other approximation methods such as Monte Carlo,
yielding comparable results, while obtaining estimates from completely different principles.
Furthermore it reveals information, such as the the spectrum of the transfer matrices and the central charge of the
massless phase, that is not accessible otherwise.
