In this thesis, we investigated finite bond dimension scaling with the corner transfer matrix renormalization group
(CTMRG) method for two-dimensional classical models.
The two main questions posed at the outset were
\begin{enumerate}[(i)]
  \item how finite bond dimension scaling works within the CTMRG method
  and relatedly, how it compares to finite-size scaling within the same method, and\label{enumerate:1}
  \item how finite bond dimension scaling with CTMRG compares to other numerical approaches.\label{enumerate:2}
\end{enumerate}

To address (i), we studied the square-lattice Ising model in \autoref{chapter:results_ising_model}.
It is shown that exponents and the transition temperature may be approximated with an effective length scale defined in
terms of the row-to-row transfer matrix at the (pseudo-)critical point,
as was suggested by \cite{nishino1996numerical}.
However, the calculated quantities show inherent deviations from the basic scaling laws,
due to the spectrum of the underlying corner transfer matrix (CTM).
These deviations are mitigated to some extent when we define the correlation length in terms of the classical analogue
of the entanglement entropy.

Scaling directly in the bond dimension $m$ is also possible, but less accurate since the law for the correlation
length
\begin{equation*}
  \xi \propto m^{\kappa}
\end{equation*}
holds only in the limit $m \to \infty$ and does not take into account the spectrum of the CTM that is obtained.
Our results indicate that $\kappa$ is between 1.93 and 1.96, while a theoretical analysis \cite{pollmann2009theory}
predicts $\kappa \approx 2.03$, but the discrepancy can be explained by finite-$m$ effects and low quality of the fit
because of to deviations due to the CTM spectrum.

It was found that finite-$m$ scaling and finite-size scaling yield comparable accuracy for critical exponents and the
transition temperature.
With finite-$m$ scaling larger effective system sizes are obtainable,
but finite-size approximations do not suffer from the deviations due to the CTM spectrum and are
consequently of much higher quality. Therefore it is plausible that finite-size results will improve significantly if
corrections to scaling are included in the fits.

In \autoref{chapter:methods}, the convergence properties of both finite-$m$ and finite-size calculations were studied.
For finite-size simulations, it is not entirely clear how the chosen bond dimension (and correspondingly,
the truncation error) influences the precision of quantities and the position of the pseudocritical point.
It might be possible to simulate larger system sizes without much loss of accuracy,
but it seems unlikely that the same system sizes as in the finite-$m$ regime are accessible.

For finite-$m$ results it is much easier to assess the convergence of quantities.
The quality of the fit remains limited by the structure that is inherent in the data.

In \autoref{chapter:results_clock_model}







\section{Outlook}

Lorem ipsum dolor sit amet, consectetur adipisicing elit, sed do eiusmod tempor incididunt ut labore et dolore magna
aliqua.
Ut enim ad minim veniam, quis nostrud exercitation ullamco laboris nisi ut aliquip ex ea commodo consequat.
Duis aute irure dolor in reprehenderit in voluptate velit esse cillum dolore eu fugiat nulla pariatur.
Excepteur sint occaecat cupidatat non proident, sunt in culpa qui officia deserunt mollit anim id est laborum.
