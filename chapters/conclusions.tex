In this thesis, we investigated finite bond dimension scaling with the corner transfer matrix renormalization group
(CTMRG) method for two-dimensional classical models.
The two main questions posed at the outset were
\begin{enumerate}[(i)]
  \item how finite bond dimension scaling works within the CTMRG method
  and relatedly, how it compares to finite-size scaling within the same method, and
  \item how finite bond dimension scaling with CTMRG compares to other numerical approaches in a difficult scenario.
\end{enumerate}

To address (i), we studied the square-lattice Ising model in \autoref{chapter:results_ising_model}.
It is shown that exponents and the transition temperature may be approximated with a scaling analysis in an effective
length scale defined in terms of the row-to-row transfer matrix at the (pseudo-)critical point,
as was suggested by \cite{nishino1996numerical}.
However, the calculated quantities show inherent deviations from the basic scaling laws,
due to the spectrum of the underlying corner transfer matrix (CTM).
These deviations are mitigated to some extent when we define the correlation length in terms of the classical analogue
of the entanglement entropy.
Scaling directly in the bond dimension $m$ is also possible, but less accurate since the law for the correlation
length
\begin{equation*}
  \xi \propto m^{\kappa}
\end{equation*}
holds only in the limit $m \to \infty$ and does not take into account the spectrum of the CTM that is obtained.
Our results indicate that $\kappa$ is between 1.93 and 1.96, while a theoretical analysis \cite{pollmann2009theory}
predicts $\kappa \approx 2.03$, but the discrepancy can be explained by finite-$m$ effects and low quality of the fit
because of deviations due to the CTM spectrum.

It was found that finite-$m$ scaling and finite-size scaling yield comparable accuracy for critical exponents and the
transition temperature.
With finite-$m$ scaling larger effective system sizes are obtainable,
but finite-size approximations do not suffer from the deviations due to the CTM spectrum and are
consequently of much higher quality. Therefore it is plausible that finite-size results will improve significantly if
corrections to scaling are included in the fits.

In \autoref{chapter:methods}, the convergence properties of both finite-$m$ and finite-size calculations were studied.
For finite-size simulations, it is not entirely clear how the chosen bond dimension (and correspondingly,
the truncation error) influences the precision of quantities and the position of the pseudocritical point.
It might be possible to simulate larger system sizes without much loss of accuracy,
but it seems unlikely that the same system sizes as in the finite-$m$ regime are accessible.
For finite-$m$ results it is much easier to assess the convergence of quantities.
The quality of the fit remains limited by the structure that is inherent in the data.

To compare finite bond dimension scaling with CTMRG to other numerical approaches (ii),
we studied the two phase transitions of the clock model with $q = \{5,
6\}$ states in \autoref{chapter:results_clock_model}.
This model is subject to some controversy, because while the renormalization group analysis suggests that the
transitions are of the Kosterlitz-Thouless (KT) type, recent papers claim that this may not be the case.

We find a divergence of the correlation length that is consistent with the KT picture.
Furthermore, we find a varying value for the correlator exponent $\eta$ that comes close to exact values for the model
in the Villain formulation, which has two KT transitions.
From this, we conclude that it is very implausible that the phase diagram
found from the renormalization group analysis needs to be altered.

The values that we found for $T_1$ and $T_2$ for $q = \{ 5, 6 \}$ agree reasonably well with the values found by other
authors.
We were able to study $T_1$ only with finite-size scaling, since it requires approximating systems with a free boundary,
which is not possible near $T_1$ due to numerical instability of the tensors in finite-$m$ approximations.

Finite-size scaling within the CTMRG suffers more from finite-size effects than finite-$m$ scaling does,
since with finite-size simulations smaller system sizes are obtainable.
It is plausible that this is the reason that finite-size scaling yields values of the critical temperatures that differ
somewhat from previous results and the results of finite-$m$ scaling.

The remarks made for the Ising model are applicable here as well. In summary,
the most straightforward way of improving finite-size results is to include finite-size corrections,
which should be feasible since the data is of high quality.
In conjunction with this, a more systematic study of the convergence of quantities and the location of the
pseudocritical points may reveal that
simulating larger system sizes is in fact possible without too much loss of accuracy.
For finite-$m$ scaling, inherent deviations from the basic scaling laws, stemming from the underlying CTM spectrum, make
including finite-size corrections less feasible.

Results for the exponent $\eta$ indicate that the critical temperatures found in both this study and previous work might
be too close together.
It is conceivable that, after considering larger systems and taking into account finite-size corrections,
both critical temperatures and the values of $\eta$ will be adjusted outwards towards their true values,
thereby completely reconciling the results.

Overall, we conclude that finite-$m$ scaling is a valuable alternative to finite-size scaling within CTMRG,
since much larger system sizes are accessible.
The CTMRG analysis is itself a valuable addition to Monte Carlo methods,
yielding comparable results, while establishing the KT picture from completely different principles.
Furthermore it reveals information, such as the the spectrum of the transfer matrices and the central charge of the
massless phase, that is not accessible otherwise.

\section{Outlook}

\begin{itemize}
  \item study 2D quantum systems (mention sign problem)
  \item apply symmetries
\end{itemize}
