\begin{abstract}
We present numerical results of finite-$m$ and finite-size scaling within the CTMRG method on the
Ising model.
\end{abstract}

\section{At the critical point}

\subsection{Existence of two length scales}

First, we reproduce the results presented in \cite{nishino1996numerical} to validate the assumption that at the critical
point, the only relevant length scales are the system size $N$ and the length scale associated to a finite dimension $m$
of the corner transfer matrix $\xi(m)$.
Here, we assume that $\xi(m)$ is given by the correlation length at the critical point,
see \autoref{sec:definition_effective_length_scale_in_terms_of_xi}.

The order parameter\footnote{It is worth stressing that the order parameter and the magnetization per site are used
interchangeably for the Ising model, and that the magnetization per site is approximated,
within the CTMRG algorithm, by the expectation value of the central spin.
See \autoref{sec:magnetization_per_site}.} should obey the following scaling relation at the critical temperature
\begin{equation}\label{eq:order_param_scaling_relation_finite_m}
  M(T = T_c,m) \propto \xi(T = T_c, m)^{-\beta/\nu}.
\end{equation}
The left panel of \autoref{fig:order_parameter_power_law_fit} shows that this scaling relation holds.
The fit yields $\frac{\beta}{\nu} \approx 0.125(5)$, close to the true value of $\frac{1}{8}$.

The right panel shows the conventional finite-size scaling relation
\begin{equation}\label{eq:order_param_scaling_relation_finite_N}
  M(T = T_c, N) \propto N^{-\beta/\nu},
\end{equation}
yielding $\beta/\nu \approx 0.1249(1)$.

The correlation length $\xi(m)$ shows characteristic half-moon patterns on a log-log scale,
stemming from the smeared-out stepwise pattern in the corner transfer matrix spectrum (see
\autoref{sec:spectrum_of_ctm}).
This makes the data harder to interpret, since the effect of increasing $m$ depends on how much of the spectrum is
currently retained.

\begin{figure}
  % \includegraphics[width=\textwidth, axisratio=1]{order_parameter_power_law_fit.tikz}
  \includegraphics[]{order_parameter_power_law_fit.tikz}
  \caption{Left panel: fit to the relation in
  \autoref{eq:order_param_scaling_relation_finite_m}, yielding $\frac{\beta}{\nu} \approx
  0.125(5)$. The data points are obtained from simulations with $m = 2, 4, \dots, 64$. The
  smallest 10 values of $m$ have not been used for fitting, to diminish correction terms
  to the basic scaling law. Right panel: fit to conventional finite-size scaling law
  given in \autoref{eq:order_param_scaling_relation_finite_N}, fitted with $n = 1500, 1750, \dots, 4000$, calculated with a truncation error no larger than $10^{-7}$, yielding $\beta/\nu \approx 0.1249$.
  }
  \label{fig:order_parameter_power_law_fit}
\end{figure}

To further test the hypothesis that $N$ and $\xi(m)$ are the only relevant length scales,
the authors of \cite{nishino1996numerical} propose a scaling relation for the order
parameter $M$ at the critical temperature of the form
\begin{equation}\label{eq:order_param_scaling_relation}
  M(N, m) = N^{-\beta/\nu} \mathcal{G}(\xi(m) / N)
\end{equation}
with
\begin{equation}
  \mathcal{G}(x) =
  \begin{cases}
    \text{const} & \text{if } x \to \infty, \\
    x^{-\beta/\nu} & \text{if } x \to 0,
  \end{cases}
\end{equation}
meaning that \autoref{eq:order_param_scaling_relation} reduces to
\autoref{eq:order_param_scaling_relation_finite_N} in the limit $\xi(m) \gg N$ and to
\autoref{eq:order_param_scaling_relation_finite_m} in the limit $N \gg \xi(m)$.
\autoref{fig:data_collapse_nishino} shows that the scaling relation of \autoref{eq:order_param_scaling_relation}
is justified.

\autoref{fig:crossover_nishino} shows the cross-over behaviour from the $N$-limiting regime, where
$M(N, m) \propto N^{-\beta/\nu}$ to the $\xi(m)$-limiting regime, where $M(N, m)$ does not depend on $N$.

\begin{figure}
  \includegraphics[]{data_collapse_nishino.tikz}
  \caption{Scaling function $\mathcal{G}(\xi(m)/N)$ given in
  \autoref{eq:order_param_scaling_relation}.}\label{fig:data_collapse_nishino}
\end{figure}

\begin{figure}
  \includegraphics[]{crossover_nishino.tikz}
  \caption{Behaviour of the order parameter at fixed $m$ as function of
  the number of renormalization steps $n$. For small $n$, all curves coincide, since the system size is the only
  limiting length scale. For large enough $n$, the order parameter is only limited by the length scale
  $\xi(m)$. In between, there is a cross-over described by $\mathcal{G}(\xi(m)/N)$, given in
  \autoref{eq:order_param_scaling_relation}.}\label{fig:crossover_nishino}
\end{figure}

\subsection{Central charge}
We may directly verify the value of the central charge $c$ associated with the conformal field theory at the critical
point by fitting to
\begin{equation}\label{eq:entropy_vs_correlation_length}
  S_{\text{classical}} \propto \frac{c}{6} \log \xi(m),
\end{equation}
which yields $c = 0.501$, shown in the left panel of \autoref{fig:entropy_vs_correlation_length}.

The right panel of \autoref{fig:entropy_vs_correlation_length} shows the fit to the scaling relation in $N$ (or,
equivalently the number of CTMRG steps $n$)
\begin{equation}\label{eq:entropy_vs_N}
  S_{\text{classical}} \propto \frac{c}{6} \log N,
\end{equation}
which yields $c = 0.499$.

\begin{figure}
  \includegraphics[]{entropy_vs_correlation_length.tikz}

  \caption{Left panel:
numerical fit to \autoref{eq:entropy_vs_correlation_length}, yielding $c = 0.501$.
Here, $m \in \{ 8, 10, \dots, 70 \}$ and the convergence threshold $\epsilon = 10^{-9}$.
Right panel:
numerical fit to \autoref{eq:entropy_vs_N}, yielding $c = 0.499$,
with the fit made to $n \in \{ 1500, 1550, \dots 2500 \}$, such that the truncation error is smaller than
$10^{-7}$.}
\label{fig:entropy_vs_correlation_length}
\end{figure}

\subsection{Using the entropy to define the correlation length}\label{sec:entropy_to_define_correlation_length}
Via \autoref{eq:entropy_scaling_near_criticality}, the correlation length is expressed as
\begin{equation}\label{eq:correlation_length_as_function_of_entropy}
  \xi \propto \exp(\frac{6}{c}S).
\end{equation}

\autoref{fig:order_parameter_power_law_fit_entropy} shows the results of fitting the relation in
\autoref{eq:order_param_scaling_relation_finite_m} with this definition of the correlation length. The fit is an order
of magnitude better in the least-squares sense, and the half-moon shapes have almost disappeared,
yielding a much more robust exponent of $\beta/\nu = 0.12498$.

The entropy uses all eigenvalues of the corner transfer matrix, making it apparently less prone to structure in the
spectrum than the correlation length as defined in \autoref{eq:correlation_length_row_to_row_transfer_matrix},
which uses only two eigenvalues of the row-to-row transfer matrix.
Furthermore, the corner transfer matrix $A$ is kept diagonal in the CTMRG algorithm,
so $S$ is much cheaper to compute than $\xi$.

\begin{figure}
  \includegraphics[]{order_parameter_power_law_fit_entropy.tikz}
  \caption{Fit to
  \autoref{eq:order_param_scaling_relation_finite_m}, using \autoref{eq:correlation_length_as_function_of_entropy} as the
  definition of the correlation length.
  For the fit, we have used $m \in \{ 10, 11, \dots, 66 \}$, calculated with convergence threshold $\epsilon = 10^{-9}$, yielding $\beta/\nu = 0.12498$.}
  \label{fig:order_parameter_power_law_fit_entropy}
\end{figure}

\subsection{Exponent $\kappa$}

We now check the validity of the relation
\begin{equation}\label{eq:xi_propto_kappa_2}
  \xi(m) \propto m^{\kappa}
\end{equation}
in the context of the CTMRG method for two-dimensional
classical systems. Similar checks were done for one-dimensional quantum systems in \cite{tagliacozzo2008scaling}.

Let us first state that boundary conditions are relevant.
From \autoref{sec:spectrum_of_ctm} we expect that for fixed boundary conditions,
the entropy and therefore the correlation length is lower for a given bond dimension $m$.

There are various ways of extracting the exponent $\kappa$.
\autoref{fig:support_for_kappa} shows the results for fixed boundary conditions and \autoref{fig:support_for_kappa_free}
for free boundary conditions.

Directly checking \autoref{eq:xi_propto_kappa_2} yields $\kappa = 1.93$ for a fixed boundary
and $\kappa = 1.96$ for a free boundary.

Under the assumption of \autoref{eq:xi_propto_kappa_2}, we have the following scaling laws at the critical point
\begin{align}\label{eq:scaling_laws_order_param_free_energy_kappa}
  M(m) & \propto m^{-\beta \kappa / \nu} \\
  f(m) - f_{\text{exact}} & \propto m^{(2-\alpha)\kappa / \nu}
\end{align}
for the order parameter and the singular part of the free energy, respectively.
With a fixed boundary, a fit to $M(m)$ yields $\kappa = 1.93$.
For a free boundary we cannot extract any exponent, since $M = 0$ for every temperature.
A fit to $f(m) - f_{\text{exact}}$ yields $\kappa = 1.90$ for a fixed boundary and $\kappa = 1.93$ for a free boundary.
\autoref{fig:support_for_kappa}. Here, we have used $\beta = 1/8$, $\nu = 1$ and $\alpha = 0$ for the Ising model.

We may use \autoref{eq:entropy_scaling_near_criticality} and \autoref{eq:classical_entropy} to check the
relation
\begin{equation}\label{eq:scaling_law_entropy_kappa}
  S_{\text{classical}} \propto \frac{c\kappa}{6}\log m,
\end{equation}
which yields $\kappa = 1.93$ for a fixed boundary and $\kappa = 1.96$ for a free boundary,
with $c = 1/2$ for the Ising model.

\begin{figure}
  \includegraphics[]{support_for_kappa.tikz}
  \caption{Numerical evidence for \autoref{eq:xi_propto_kappa_2}, \autoref{eq:scaling_laws_order_param_free_energy_kappa},
  \autoref{eq:scaling_law_entropy_kappa} with fixed boundary, yielding, from left to right and top to bottom, $\kappa = \{ 1.93, 1.93, 1.90,
  1.93 \}$. These values have been calculated from simulations with $m \in \{8, 10, \dots, 70\} $ and convergence threshold $\epsilon = 10^{-9}$. }\label{fig:support_for_kappa}
\end{figure}

\begin{figure}
  \includegraphics[]{support_for_kappa_free.tikz}
  \caption{Numerical evidence for
\autoref{eq:xi_propto_kappa_2} with free boundary, yielding from left to right and then bottom $\kappa = \{ 1.96,
1.93, 1.96 \}$.
These values have been calculated from simulations with $m \in \{10,
11, \dots, 66 \}$, but with $m \in \{13, 19, 28, 29, 40, 41, 42,
59 \}$ left out, because for those values $m$ the system breaks its symmetry (see
\autoref{sec:implications_for_finite_m_simulations}).
The convergence threshold was chosen to be $\epsilon = 10^{-7}$.
It is not lower since more values $m$ break symmetry as machine precision is approached.}
\label{fig:support_for_kappa_free}
\end{figure}

\subsubsection{Comparison with exact result in asymptotic limit}

The predicted value for $\kappa$ \cite{pollmann2009theory} is $2.034\dots$ (see also \autoref{eq:exact_value_kappa}).
With the CTMRG method, we extract the slightly lower value of $1.96$ (corresponding to free boundary conditions).
But, the structure in the quantities as function of $m$ makes it hard to get an accurate fit to $\kappa$.

It is interesting to note that for fixed boundary conditions, the relation in \autoref{eq:xi_propto_kappa_2} holds,
but with a lower exponent $\kappa$.
This is to be expected, since half the spectrum of the corner transfer matrix is missing.

\section{Locating the critical point}\label{sec:locating_the_critical_point}

In general, the critical point is not known, but it may be located by extrapolating the position of the pseudocritical temperature at finite system sizes.

The pseudocritical point can be defined in a variety of ways.
In this chapter, we will define the pseudocritical point as the point of maximum entropy,
as described in \autoref{sec:locating_critical_point_entanglement}.
\autoref{fig:entropy_vs_T} shows the classical analogue to the entanglement entropy as a function of temperature for
different values of $m$.

The critical point is located by fitting the scaling law in \autoref{eq:scaling_law_T_star}.

\subsection{Finite $m$}
For approximations with finite bond dimension $m$,
it is not clear what length scale should be used to fit the scaling behaviour of $T^{\star}(m)$.
\autoref{fig:T_pseudocrit_chi_power_law_fit} shows the fits for different choices of this length scale.
The results are tabulated in \autoref{table:T_star_nu_results}.
To obtain $T^{\star}$, we have calculated $T^{\star}(m)$ for $m \in \{10,
11, \dots, 60\}$ with a convergence threshold of $10^{-8}$ and a temperature tolerance of $10^{-8}$.
The boundaries are fixed to $+1$.

We denote the estimated value of the critical temperature as $\widetilde{T_c}$. Recall that the exact value is
\begin{equation}
  T_c = 2.2691853\dots
\end{equation}
and
\begin{equation}
  \nu = 1.
\end{equation}

When using $\xi(T_c, m)$, the correlation length at the exact critical point,
the result shows a lot of structure, yielding $\widetilde{T_c} = 2.269172$ and $\nu = 1.057$.

If, instead, the correlation length at the estimated pseudocritical temperature $\xi(T^{\star}(m))$ is used,
the data shows less structure and we obtain the much more precise results $\widetilde{T_c} = 2.269183$ and $\nu =
1.002$.

Another option is to use the entropy to define the correlation length,
via \autoref{eq:correlation_length_as_function_of_entropy}, which gave more accurate results than using the transfer
matrix definition in \autoref{sec:entropy_to_define_correlation_length}.
In this case, the results are slightly worse than the transfer matrix definition:
$T_c = 2.269183$ and $\nu = 1.02$.

Finally, we may directly fit the law
\begin{equation}
  |T_c - T^{\star}(m)| \propto m^{-\kappa/\nu},
\end{equation}
yielding $T_c = 2.269181$ and $\kappa/\nu = 1.91$.
Incidentally, this is another way to confirm $\kappa \approx 1.9$ for systems with a fixed boundary.


\subsection{Finite $N$}

As a cross check, we can instead use systems of finite size to extract $T_c$ and $\nu$.
We have calculated $T^{\star}(n)$ for $n \in \{ 2300, 2500, \dots, 7900 \}$,
with $m$ big enough such that the truncation error is no larger than $10^{-6}$.
This yields $T_c = 2.269185$ and $\nu = 0.98$.

\begin{table}[]
\centering
\begin{tabular}{@{}lll@{}} \toprule
$N_{\text{eff}}$                  & $T_c$   & $\nu$   \\ \midrule
$\xi(T_c, m)$                     & 2.269172  & 1.057         \\
$\xi(T^{\star}(m))$               & 2.269183   & 1.002        \\
$\exp((6/c)S(T^{\star}(m))$       & 2.269183   & 1.02       \\
$m^{\kappa}$                      & 2.269181  & --        \\
$N$                               & 2.269185  & 0.98          \\ \bottomrule
\end{tabular}
  \caption{Results for fits to the scaling law \autoref{eq:scaling_law_T_star} using different length scales.
  When using $m^{\kappa}$, $\kappa \approx 1.91$ was found to give the best fit.} \label{table:T_star_nu_results}
\end{table}

\begin{figure}
  \includegraphics[]{entropy_vs_T.tikz}
  \caption{Classical analogue to the entanglement entropy, as in \autoref{eq:classical_entropy},
  near the critical point (shown as dashed line).}\label{fig:entropy_vs_T}
\end{figure}

\begin{figure}
  \includegraphics[]{T_pseudocrit_chi_power_law_fit.tikz}
  \caption{Fits to the scaling law \autoref{eq:scaling_law_T_star}.
  Results for the critical temperature and exponent $\nu$ are tabulated in
  \autoref{table:T_star_nu_results}.}\label{fig:T_pseudocrit_chi_power_law_fit}
\end{figure}

\section{Away from the critical point}

We may also verify the validity of the different length scales by asserting that the data for different values of $m$
should collapse on a single curve
\begin{equation}
  \mathcal{G}(t \xi(m)^{1/\nu}) = M(T, m) N_{\xi(m)^{\beta/\nu}}.
\end{equation}

All data points were calculated with a convergence threshold of $10^{-7}$.
The values of the pseudocritical temperatures are taken from the results in \autoref{sec:locating_the_critical_point}.
No temperatures beyond $T_c$ is considered because the order parameter drops off sharply,
causing the curve $\mathcal{G}(x)$ to tend to zero almost vertically, making the fitness $P$ unreliable.

\autoref{fig:data_collapse_chi} shows that for all length scales, the results more or less fall on one curve.
\autoref{table:fitness_data_collapse_different_length_scales} shows the fitness of the data collapse
\cite{bhattacharjee2001measure} (given by \autoref{eq:fitness_data_collapse}) for all length scales used.

\todo[inline]{Say which length scales apparently don't work so well}

Using $m^{\kappa}$ as a length scale for optimized fitness $P(\kappa)$ yields $\kappa \approx 1.98$,
substantially higher than found previously for fixed boundary conditions.

As a cross-check, the bottom-right panel of \autoref{fig:data_collapse_chi} shows data points for finite-$N$
simulations. Here, the bond dimension is chosen such that the truncation error is smaller than $10^{-6}$.

\begin{figure}
  \includegraphics[]{data_collapse_chi.tikz}
  \caption{Data collapses using different length scales.
  For the bottom-right plot, approximations with finite $N$ instead of finite $m$ have been used,
  with $n = \{160, 480, 1000, 1500 \}$ ($n = \frac{N - 1}{2}$ is the number of algorithm
  steps).}\label{fig:data_collapse_chi}
\end{figure}

\begin{table}[]
\centering
\begin{tabular}{@{}ll@{}} \toprule
$N_{\text{eff}}$                  & fitness $P$ \\ \midrule
$\xi(T_c, m)$                     & 0.0075      \\
$\xi(T^{\star}(m))$               & 0.066       \\
$\exp((6/c)S(T_c, m))$            & 0.057       \\
$\exp((6/c)S(T^{\star}(m))$       & 0.087       \\
$m^{\kappa}$                      & 0.0080      \\
$N$                               & 0.0075      \\ \bottomrule
\end{tabular}
  \caption{Fitness of data collapse (\autoref{eq:fitness_data_collapse}) for different length scales.
  $\kappa \approx 1.98$ was found to be optimal for the length scale $m^{\kappa}$.}
  \label{table:fitness_data_collapse_different_length_scales}
\end{table}


\section{Discussion}

\todo[inline]{Why do length scales defined at $T^{\star}$ work better??
It is fortunate that we don't need the length scales at $T_c$, since we don't know it.}
