\section{Abstract}

\todo[inline]{Cite more}

We present finite-size scaling results using the corner transfer matrix renormalization
group method on two-dimensional classical square lattices \cite{nishino1996corner}. We
compare the results of conventional finite-size scaling in the system size $N$ with
scaling in the number of states kept during the renormalization step of the algorithm,
denoted by $m$. Such a comparison was first done in \cite{nishino1996numerical}. We
highlight the areas in which method excels over the other.

Calculate critical temperature and exponents using information that is directly
extractable from the corner transfer matrix.


\section{Introduction}

The first direct comparison of finite-size scaling in the system size $N$ with scaling in
the bond dimension of the corner transfer matrix renormalization group method $m$ was done
in \cite{nishino1996numerical}. In explaining the basic concepts, we largely follow this
paper.

The error in the approximation of the partition function (and thus all thermodynamic
quantities) in the thermodynamic limit with the corner transfer matrix method depends on
two characteristic length
scales. The first is the size of the system $N$. After $n$ steps of the infinite-system
algorithm, we have
\begin{equation}
  N = 2n + 1.
\end{equation}

The second length scale is related to the finite bond dimension $m$. Baxter
\cite{baxter1978variational}, and later Östlund and Rommer \cite{ostlund1995thermodynamic}
(in the context of one-dimensional quantum systems) showed that in the thermodynamic
limit, CTMRG and DMRG are variational optimizations in the space of matrix product states.
\todo[inline]{Can extend this idea a bit.}

\begin{figure}
\includegraphics[]{order_parameter_vs_T.tikz}
\caption{Upper panel: expectation value of the central spin $\langle \sigma_0 \rangle$
  after $n$ CTMRG steps. $m$ is chosen such that the truncation error is smaller than
  $10^{-6}$. Lower panel: $\langle \sigma_0 \rangle$ for systems with bond dimension $m$.}\label{fig:order_parameter_vs_T}
\end{figure}

It is known that an MPS-ansatz with finite bond dimension inherently limits the
correlation length of the system to a finite value \cite{wolf2006quantum}. Hence,
thermodynamic quantities obtained from the CTMRG algorithm with finite $m$, in the limit
$N \to \infty$, cannot diverge and must show finite-size effects similar to those of some
effective finite system of size $N_{\text{eff}}(m)$ depending on the bond dimension $m$.

\autoref{fig:order_parameter_vs_T} shows the behaviour of the order parameter of the
two-dimensional Ising model for systems of finite-size, where $m$ has been chosen such
that the truncation error is smaller than $10^{-6}$, and for systems of finite $m$, where
the result is converged in the system size $N$. The results look very similar and support
the above claim.

\todo[inline]{Order parameter is not the same as magnetization central spin. Where to
explain this?}

\subsection{Definition of  in terms of the correlation length at $T_c$}

In the thermodynamic limit (corresponding to infinite $m$ and $N$), we have the following
expression for the correlation length of a classical system
\cite{baxter1982exactly_correlation_length}
\begin{equation}
  \xi(T) = \frac{1}{\log\left(\frac{T_0}{T_1}\right)}.
\end{equation}
Here, $T_0$ and $T_1$ are the largest and second-largest eigenvalues of the row-to-row
transfer matrix $T$, respectively. With $N$ tending towards infinity and finite $m$, near
the critical point $\xi(T)$ should obey a scaling law of the form
\begin{equation}
  \xi(T, m) = N_{\text{eff}}(m) \mathcal{F}(N_{\text{eff}}(m) / \xi(T))
\end{equation}
with
\begin{equation}
  \mathcal{F}(x) = \begin{cases}
      \text{const} & \text{if } x \to 0, \\
      x^{-1} & \text{if } x \to \infty.
    \end{cases}
\end{equation}

Hence, the effective length scale corresponding to the finite bond dimension $m$ is
proportional to the correlation length of the system at the critical point $t = 0$.
\begin{equation}
  N_{\text{eff}}(m) \propto \xi(T = T_c, m).
\end{equation}

Under this assumption, the order parameter should obey the following scaling relation at
the critical temperature
\begin{equation}\label{eq:order_param_scaling_relation_finite_m}
  M(T = T_c, m) \propto \xi(T = T_c, m)^{-\beta/\nu}.
\end{equation}
The left panel of \autoref{fig:order_parameter_power_law_fit} shows that this scaling
relation holds. The fit yields $\frac{\beta}{\nu} \approx 0.125(5)$, close to the true
value of $\frac{1}{8}$.

The right panel shows the conventional finite-size scaling relation
\begin{equation}\label{eq:order_param_scaling_relation_finite_N}
  M(T = T_c, N) \propto N^{-\beta/\nu},
\end{equation}
yielding $\beta/\nu \approx 0.1249(1)$, which can be systematically improved by
fitting to larger system sizes, obtained with a fixed truncation error.

In the case of scaling in correlation length $\xi(m)$, the exponent does not improve when
taking bigger values of $m$, while keeping the termination criterion (relative change of
singular values) fixed. This points to a flaw in the termination criterion of the
algorithm.

Furthermore, the correlation length $\xi(m)$ shows characteristic half-moon patterns on a
log-log scale, stemming from the degeneracies in the corner transfer matrix spectrum. This
makes the data harder to interpret, since the effect of increasing $m$ depends on how much
of the spectrum is currently retained.

\todo[inline]{Talk about how to alleviate this partially by using entropy $S$ as
length scale.}

\begin{figure}
  % \includegraphics[width=\textwidth, axisratio=1]{order_parameter_power_law_fit.tikz}
  \includegraphics[]{order_parameter_power_law_fit.tikz}
  \caption{Left panel: fit to the relation in
  \autoref{eq:order_param_scaling_relation_finite_m}, yielding $\frac{\beta}{\nu} \approx
  0.125(5)$. The data points are obtained from simulations with $m = 2, 4, \dots, 64$. The
  smallest 10 values of $m$ have not been used for fitting, to diminish correction terms
  to the basic scaling law. Right panel: fit to conventional finite-size scaling law
  given in \autoref{eq:order_param_scaling_relation_finite_N}.
  }
  \label{fig:order_parameter_power_law_fit}
\end{figure}

To further test the hypothesis that $N$ and $\xi(m)$ are the only relevant length scales,
the authors of \cite{nishino1996numerical} propose a scaling relation for the order
parameter $M$ at the critical temperature of the form
\begin{equation}\label{eq:order_param_scaling_relation}
  M(N, m) = N^{-\beta/\nu} \mathcal{G}(\xi(m) / N)
\end{equation}
with
\begin{equation}
  \mathcal{G}(x) =
  \begin{cases}
    \text{const} & \text{if } x \to \infty, \\
    x^{-\beta/\nu} & \text{if } x \to 0,
  \end{cases}
\end{equation}
meaning that \autoref{eq:order_param_scaling_relation} reduces to
\autoref{eq:order_param_scaling_relation_finite_N} in the limit $\xi(m) \gg N$ and to
\autoref{eq:order_param_scaling_relation_finite_m} in the limit $N \gg \xi(m)$.
\autoref{fig:data_collapse_nishino} shows that the scaling relation of \autoref{eq:order_param_scaling_relation}
is justified.

\autoref{fig:crossover_nishino} shows the cross-over behaviour from the $N$-limiting regime, where
$M(N, m) \propto N^{-\beta/\nu}$ to the $\xi(m)$-limiting regime, where $M(N, m)$ does not depend on $N$.

\begin{figure}
  \includegraphics[]{data_collapse_nishino.tikz}
  \caption{Scaling function $\mathcal{G}(\xi(m)/N)$ given in
  \autoref{eq:order_param_scaling_relation}.}\label{fig:data_collapse_nishino}
\end{figure}

\begin{figure}
  \includegraphics[]{crossover_nishino.tikz}
  \caption{Behaviour of the order parameter at fixed $m$ as function of
  the number of renormalization steps $n$. For small $n$, all curves coincide, since the system size is the only
  limiting length scale. For large enough $n$, the order parameter is only limited by the length scale
  $\xi(m)$. In between, there is a cross-over described by $\mathcal{G}(\xi(m)/N)$, given in
  \autoref{eq:order_param_scaling_relation}.}\label{fig:crossover_nishino}
\end{figure}

\subsection{Scaling relations away from the critical point}
In general, the position of the critical point is not known. In that situation, the scaling relation in
\autoref{eq:order_param_scaling_relation} cannot be used to calculate thermodynamic information. Instead, in the limit
$N \to \infty$, we should have
\begin{equation}\label{eq:order_param_scaling_relation_finite_chi_t}
  M(t, m) \propto \xi(m)^{-\beta/\nu}\mathcal{P}(t \xi(m)^{1/\nu}),
\end{equation}
which is confirmed in \autoref{fig:data_collapse_chi_correlation_length}.

However, in practice this is still problematic, since $\xi(m)$ is defined at the critical point. Thus, we must find a
way to define the length scale corresponding to a finite bond dimension $m$ without making use of the position of the
critical point.

\begin{figure}
  \includegraphics[]{data_collapse_chi_correlation_length.tikz}
  \caption{Scaling function $\mathcal{P}(t \xi(m)^{1/\nu})$ in
  \autoref{eq:order_param_scaling_relation_finite_chi_t}.}\label{fig:data_collapse_chi_correlation_length}
\end{figure}

\section{Finite-entanglement scaling and its relation to two-dimensional classical lattices}

Another way to understand the fact that the CTMRG method with finite $m$ can never accurately represent systems at
criticality, is by looking at the entanglement properties of the ground state of the corresponding one-dimensional
quantum systems.

It is known that near the critical point, when the correlation length $\xi$ is large but finite, the entanglement of a
subsystem $A$ scales as
\begin{equation}\label{eq:entropy_scaling_near_criticality}
  S_A \propto \mathcal{A}(c/6)\log(\xi)
\end{equation}
where $\mathcal{A}$ is the number of boundary points of $A$ and $c$ is the central charge of the conformal field theory
at the critical point \cite{calabrese2004entanglement, vidal2003entanglement, ercolessi2010exact}.

Recalling the definition of the entanglement entropy
\begin{equation}
  S_A = - \tr(\rho_A \log \rho_A) = - \sum_{\alpha} \omega_{\alpha} \log \omega_{\alpha},
\end{equation}
it is trivially seen that the entropy of a state given by the DMRG (or any other MPS), which only
retains $m$ basis states of $\rho_A$, is limited by
\begin{equation}
  S^{\text{max}}_A(m) = \log m
\end{equation}
by putting $\omega_{\alpha} = 1/m$ for $\alpha = 1, \dots, m$.

Empirically, MPS ground states of critical systems do not reach their maximum entropy $\log m$, but one may still assume
\begin{equation}
  S_A \propto S^{\text{max}}_A(m)
\end{equation}
close to criticality, which directly implies the relationship
\begin{equation}\label{eq:xi_propto_kappa}
  \xi(m) \propto m^{\kappa}.
\end{equation}

Numerical evidence of this fact was first given by the authors of \cite{andersson1999density}, who found
\begin{equation}
  \xi(m) \propto m^{1.3}
\end{equation}
for a gapless system of free fermions, using DMRG calculations. Later, using the iTEBD algorithm
\cite{vidal2007classical}, the authors of \cite{tagliacozzo2008scaling} presented numerical evidence for such a relation
for the Ising model with transverse field and the Heisenberg model, with
\begin{align}
  \kappa_{\text{Ising}} & \approx 2, \\
  \kappa_{\text{Heisenberg}} & \approx 1.37.
\end{align}

A quantitative theory of this behaviour was given in \cite{pollmann2009theory}.
Assuming the energy density as function of the effective correlation length $\xi$ takes the form
\begin{equation}
  E(\xi) = E_{\infty} + \frac{A}{\xi^2} + \frac{B}{\xi}P_r(m),
\end{equation}
where
\begin{equation}
  P_r(m) = \sum_{\alpha = m}^{\infty} \omega_{\alpha}
\end{equation}
is the residual probability.
Using results in \cite{calabrese2008entanglement} for the spectrum of $\rho_A$ in limit $m \to \infty$, $\xi$ is found
to obey \autoref{eq:xi_propto_kappa} with
\begin{equation}
  \kappa = \frac{6}{c(\sqrt{12/c} + 1)} + \mathcal{O}(1/\log m),
\end{equation}
yielding
\begin{align}
  \kappa_{\text{Ising}} & = 2.034\dots, \\
  \kappa_{\text{Heisenberg}} & = 1.344\dots,
\end{align}
which are in good agreement with the findings in \cite{tagliacozzo2008scaling}.


\subsection{Classical analogue of entanglement entropy}

The key point of the corner transfer matrix renormalization group method \cite{nishino1997corner, nishino1996corner} is
that it unifies White's density matrix renormalization group method \cite{white1992density} with Baxter's corner
transfer matrix approach \cite{baxter1968dimers, baxter1978variational}, through the identification (in the isotropic
case)
\begin{equation}\label{eq:correspondence_density_matrix_ctm}
  \rho_{\text{half-chain}} = A^4.
\end{equation}

This allows one to define a 2D classical analogue to the half-chain entanglement entropy of a 1D quantum system
\begin{equation}\label{eq:classical_entropy}
  S_{\text{classical}} = - \tr A^4 \log A^4 = - \sum_{\alpha=1}^{m} \nu_{\alpha}^4 \log \nu_{\alpha}^4,
\end{equation}
where $\nu_{\alpha}$ are the eigenvalues of the corner transfer matrix $A$.
In the CTMRG algorithm, $A$ is kept in diagonal form, making $S_{\text{classical}}$ trivial to compute.

In \cite{huang2017holographic}, numerical evidence is given for the validity of \autoref{eq:classical_entropy} for a
wide range of models, and the concept is generalized to higher dimensions. For an overview of applying corner transfer
matrices in higher dimensions and to quantum systems, see \cite{orus2012exploring}.

\subsection{Locating the critical point with the entanglement spectrum}
Since phase transitions of quantum systems can be located by studying their entanglement spectrum (\emph{cite here}),
classical systems may be investigated in the same way through the correspondence in
\autoref{eq:correspondence_density_matrix_ctm}. This is an alternative to the usual approach of studying an order
parameter or derivatives of thermodynamical observables (\emph{cite here?}).

Examples of studies using the spectrum of the corner transfer matrix to analyze two-dimensional classical systems are
\cite{krvcmar2015reentrant, PhysRevE.94.022134, krvcmar2016phase}.


\begin{itemize}
  \item \cite{huang2017holographic}: refs [42] and [12] contain many papers which study the phenomenon of pinpointing
  a phase transition without using physical observables (i.e. entanglement, spectrum, fidelity instead.)
  \item \cite{osborne2002entanglement}: for XY and Ising model, proves that next-to-nearest neighbor entanglement peaks
  at critical point (though not nearest-neighbor entanglement.)
\end{itemize}


\subsection{Numerical results}

We now check the validity of \autoref{eq:xi_propto_kappa} in the context of the CTMRG method for two-dimensional
classical systems. Similar checks were done for one-dimensional quantum systems in \cite{tagliacozzo2008scaling}.

Directly checking \autoref{eq:xi_propto_kappa} yields $\kappa = 1.93$, see top left panel of
\autoref{fig:support_for_kappa}.
Under the assumption of \autoref{eq:xi_propto_kappa}, we have the following scaling laws at the critical point
\begin{align}\label{eq:scaling_laws_order_param_free_energy_kappa}
  M(m) & \propto m^{-\beta \kappa / \nu} \\
  f(m) - f_{\text{exact}} & \propto m^{(2-\alpha)\kappa / \nu}
\end{align}
for the order parameter and the singular part of the free energy, respectively. A fit to $M(m)$ yields $\kappa = 1.93$
and a fit to $f(m) - f_{\text{exact}}$ yields $\kappa = 1.90$. See the top right and bottom left panels of
\autoref{fig:support_for_kappa}. Here, we have used $\beta = 1/8$, $\nu = 1$ and $\alpha = 0$ for the Ising model.
\todo[inline]{Tell that the $\kappa$ law is indeed valid, since it is a good fit.}

We may use \autoref{eq:entropy_scaling_near_criticality} and \autoref{eq:classical_entropy} to check the
relation
\begin{equation}\label{eq:scaling_law_entropy_kappa}
  S_{\text{classical}} \propto \frac{c\kappa}{6}\log m,
\end{equation}
which also yields $\kappa = 1.93$, where $c = 1/2$ for the Ising model. See bottom right panel of
\autoref{fig:support_for_kappa}.

We may directly verify the value of the central charge $c$ associated with the conformal field theory at the critical
point by fitting to
\begin{equation}\label{eq:entropy_vs_correlation_length}
  S_{\text{classical}} \propto \frac{c}{6} \log \xi(m),
\end{equation}
which yields $c = 0.501$, shown in the left panel of \autoref{fig:entropy_vs_correlation_length}.

The right panel of \autoref{fig:entropy_vs_correlation_length} shows the fit to the scaling relation in $N$ (or,
equivalently the number of CTMRG steps $n$)
\begin{equation}\label{eq:entropy_vs_N}
  S_{\text{classical}} \propto \frac{c}{6} \log N,
\end{equation}
which yields $c = 0.498$.

To verify if the point of maximum entropy
\begin{equation}
  T^{\star}(m) = \max_{T} S(T, m)
\end{equation}
is a good definition of the pseudocritical point, we fit the relation
\begin{equation}\label{eq:T_pseudocrit_scaling_law_chi}
  T^{\star} - T_c \propto \xi(m)^{-1 / \nu}.
\end{equation}
which yields $\widetilde{T}_c = 2.2692$ and $\nu = 0.997$ when the length scale $\xi(T^{\star}, m)$ is used, shown in
the left panel of \autoref{fig:T_pseudocrit_chi_power_law_fit}. Here, $\widetilde{T}_c$ denotes the critical temperature
found by minimising the norm of squares of a fit of the form given in \autoref{eq:T_pseudocrit_scaling_law_chi}. In
finding the position of the pseudoccritical temperature $T^{\star}$, a tolerance of $10^{-6}$ was used.

If, however, the length scale $\xi(m, T_c)$ at the actual critical point is used, a much worse fit is obtained, yielding
$\widetilde{T}_c = 2.2691$ and $\nu = 0.90$, shown in the right panel of
\autoref{fig:T_pseudocrit_chi_power_law_fit}.

This signifies the value of $\xi(T^{\star}, m)$ is heavily dependent on
$T^{\star}$, and using the length scale at the actual pseudocritical temperature found somehow offsets the error on its
position.
\todo[inline]{this is unclear.}

Assuming \autoref{eq:xi_propto_kappa}, \autoref{eq:T_pseudocrit_scaling_law_chi} becomes
\begin{equation}
  T^{\star} - T_c \propto m^{-\kappa / \nu},
\end{equation}
which yields \emph{values}, shown in the bottom left panel of \autoref{fig:T_pseudocrit_chi_power_law_fit}.

As a cross check, we can fit instead to scaling relation of the pseudocritical temperature for finite $N$
\begin{equation}
  T^{\star} - T_c \propto N^{-1/\nu},
\end{equation}
yielding \emph{values}. See the bottom right panel of \autoref{fig:T_pseudocrit_chi_power_law_fit}.




\begin{figure}
  \includegraphics[]{support_for_kappa.tikz}
  \caption{Numerical evidence for \autoref{eq:xi_propto_kappa}, \autoref{eq:scaling_laws_order_param_free_energy_kappa},
  \autoref{eq:scaling_law_entropy_kappa}, yielding, from left to right and top to bottom, $\kappa = \{ 1.93, 1.93, 1.90,
  1.93 \}$.}\label{fig:support_for_kappa}
\end{figure}

\begin{figure}
  \includegraphics[]{entropy_vs_correlation_length.tikz}
  \caption{Left panel: numerical fit to \autoref{eq:entropy_vs_correlation_length}, yielding $c = 0.501$. Right panel:
  numerical fit to \autoref{eq:entropy_vs_N}, yielding $c = 0.498$. }\label{fig:entropy_vs_correlation_length}
\end{figure}

\begin{figure}
  \includegraphics[]{T_pseudocrit_chi_power_law_fit.tikz}
  \caption{Left panel: numerical fit to \autoref{eq:T_pseudocrit_scaling_law_chi} with $\xi(T^{\star}(m), m)$ used as
  relevant length scale. Right panel: same fit but using $\xi(T_c, m)$, the correlation length at the exact critical
  point.}\label{fig:T_pseudocrit_chi_power_law_fit}
\end{figure}

% \begin{figure}
%   \includegraphics[]{T_pseudocrit_N_power_law_fit.tikz}
%   \caption{}\label{fig:T_pseudocrit_N_power_law_fit}
% \end{figure}


\begin{itemize}
  \item validate pseudocritical point by matching it to pseudocritical point given by correlation length and
  magnetization (how?)
  \item scaling of pseudocritical point $T^{\star} - T_c \propto m^{-\kappa / \nu}$.
\end{itemize}

\section{To do}

Articles to cite:

\begin{itemize}
\item \cite{pirvu2012matrix}: assumes existence of $\kappa$ and compares finite-size scaling with finite-$m$ scaling
for 1D quantum systems with periodic boundary conditions.
\item \cite{ercolessi2010exact}: proves relation for classical eight vertex model
\end{itemize}

Things to check

\begin{itemize}
  \item does $S(T^{\star}, m) \propto \log \xi(T^{\star}, m)$ hold?
  \item does $S(T^{\star}, N) \propto \log N$ hold better than $S(T_c, N) \propto \log N$?
  \item does fitting $T_c - T^{\star}(N)$ to $N$ give better results than fitting against
  $S(T^{\star}, N)$?
  \item how does $T^{\star}(m)$ from entropy compare against $T^{\star}$ found from max correlation length, or
  vanishing magnetization?
  \item optimize $\kappa$ for scaling of pseudo critical point?
  \item find $T^{\star, N}$ for larger $N$ for ising model.

\end{itemize}

Plots to make:

\begin{itemize}
  \item $T^{\star} - T_c$ vs $N$
  \item $S(T_c, N) \propto \log N$
  \item generally, why should there be a difference between using entropy at critical point
  vs using entropy at pseudocritical point? How does each one scale?
\end{itemize}
