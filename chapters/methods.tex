\section{Abstract}

We describe the technical details of the algorithms used to compute quantities of interest.
We report the convergence behaviour of the algorithms and discuss validity and sources of error.

\section{Technical details}
For the models treated in this thesis, the corner transfer matrix $A$ and the row-to-row transfer matrix $T$ are
symmetric. But due to the accumulation of machine-precision sized errors in the matrix multiplication and singular value
decomposition, this will, after many algorithm steps, no longer be the case. In order for results to remain valid, we
manually enforce symmetricity after each step.

The tensor network contractions at each algorithm step will cause the elements of $A$ and $T$ to tend to infinity, which
means that they will at some point exceed the maximum value of a floating point number as it can be stored in memory.
But because the elements of $A$ and $T$ represent Boltzmann weights, they can be scaled by a constant factor, which
allows us to prevent this overflow if we use a suitable scaling. For example by requiring that
\begin{equation}
  \tr A^4 = 1,
\end{equation}
so that the interpretation of $A^4$ as a reduced density matrix of an effective one-dimensional quantum is valid.

\section{Convergence criteria}

\subsection{Simulations with finite bond dimension}
The convergence of the CTMRG algorithm with fixed bond dimension $m$ (the infinite system algorithm) can be defined
in multiple ways (\emph{cite}). In this thesis, the convergence after step $i$ of the algorithm is defined as
\begin{equation}
  c_i = \sum_{\alpha = 1}^{m} | s_{\alpha}^{(i)} - s_{\alpha}^{(i - 1)} |,
\end{equation}
where $s_{\alpha}$ are the singular values of the corner transfer matrix $A$. If the convergence falls below some
threshold $\epsilon$, the algorithm terminates.

The assumption is that once the singular values stop changing to some precision, the optimal projection is sufficiently
close to its fixed point and the transfer matrices $A$ and $T$ represent an environment only limited by the length scale
given by $m$, i.e.
\begin{equation}
  \xi(m) \ll N
\end{equation}
is satisfied.

The convergence of the order parameter of the Ising model is shown in
\autoref{fig:convergence_order_parameter_finite_chi}.

\todo[inline]{Cross check with correlation length, report on boundary conditions}

\begin{figure}
  \includegraphics[]{convergence_order_parameter_finite_chi.tikz}
  \caption{hallootjes}\label{fig:convergence_order_parameter_finite_chi}
\end{figure}


\subsection{Simulations with finite system size}
In the finite-system algorithm, we require
\begin{equation}
  N \ll \xi(m),
\end{equation}
so the question becomes at which $m$ the results are sufficiently converged in $m$. Throughout this work, we have used
the residual probability (also called truncation error)
\begin{equation}
  P(m)^{(i)} = \frac{\sum_{\alpha = m + 1}^{dm} (s_{\alpha}^{(i)})^2 }{ \sum_{\alpha = 1}^{dm} (s_{\alpha}^{(i)})^2 },
\end{equation}
which quantifies the fraction of the spectrum of the corner transfer matrix that is thrown away, as a measure of how
accurate the transfer matrices represent the finite system of size $N$. Here, $d$ is the
dimension of the local tensors ($d = 2$ for the Ising model).

If, for a given $m$, we have
\begin{equation}
  P(m) < P_{\text{max}}
\end{equation}
we deem the result accurate enough.

In the limit $m \to d^n$, with
\begin{equation}
  n = \frac{N - 1}{2}
\end{equation}
the number of algorithm steps, we obtain the exact result for the transfer matrices and hence the partition function,
i.e. $P(m) \to 0$.

To justify that for small enough $P_{\max}$, we obtain good results for a wide range of $N$, figure bla bla.
