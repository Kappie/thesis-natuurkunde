\section{Abstract}

We describe the technical details of the algorithms used to compute quantities of interest.
We report the convergence behaviour of the algorithms and discuss validity and sources of error.

\section{Technical details}
For the models treated in this thesis, the corner transfer matrix $A$ and the row-to-row transfer matrix $T$ are
symmetric. But due to the accumulation of machine-precision sized errors in the matrix multiplication and singular value
decomposition, this will, after many algorithm steps, no longer be the case. In order for results to remain valid, we
manually enforce symmetricity after each step.

The tensor network contractions at each algorithm step will cause the elements of $A$ and $T$ to tend to infinity, which
means that they will at some point exceed the maximum value of a floating point number as it can be stored in memory.
But because the elements of $A$ and $T$ represent Boltzmann weights, they can be scaled by a constant factor, which
allows us to prevent this overflow if we use a suitable scaling. For example by requiring that
\begin{equation}
  \tr A^4 = 1,
\end{equation}
so that the interpretation of $A^4$ as a reduced density matrix of an effective one-dimensional quantum is valid.

\section{Convergence criteria}

\subsection{Simulations with finite bond dimension}
The convergence of the CTMRG algorithm with fixed bond dimension $m$ (the infinite system algorithm) can be defined
in multiple ways (\emph{cite}). In this thesis, the convergence after step $i$ of the algorithm is defined as
\begin{equation}\label{eq:convergence}
  c_i = \sum_{\alpha = 1}^{m} | s_{\alpha}^{(i)} - s_{\alpha}^{(i - 1)} |,
\end{equation}
where $s_{\alpha}$ are the singular values of the corner transfer matrix $A$. If the convergence falls below some
threshold $\epsilon$, the algorithm terminates.

The assumption is that once the singular values stop changing to some precision, the optimal projection is sufficiently
close to its fixed point and the transfer matrices $A$ and $T$ represent an environment only limited by the length scale
given by $m$, i.e.
\begin{equation}
  \xi(m) \ll N
\end{equation}
is satisfied.

\subsubsection{Convergence at the critical point of the Ising model}
The convergence is shown in \autoref{fig:convergence_vs_n}. It is clear that the
phenomenological law
\begin{equation}\label{eq:convergence_vs_n_semilogarithmic_law}
  \log c_n \propto n
\end{equation}
holds to high precision, with the slope depending on $m$.
Deviations only occur at values of $c$ of around $10^{-12}$.

The convergence of the various quantities as function of the number of algorithm steps is shown in
\autoref{fig:convergence_finite_chi}. For all quantities $Q$, the absolute relative difference
with the final algorithm step
\begin{equation}\label{eq:abs_rel_diff}
  \Delta Q_{\text{rel}}(n) = \left| \frac{Q(n) - Q(n = 10^5)}{Q(n = 10^5)} \right|
\end{equation}
is shown. Again, a law of the form
\begin{equation}\label{eq:abs_rel_diff_vs_n_semilogarithmic_law}
  \log(\Delta Q_{\text{rel}}) \propto n
\end{equation}
seems to hold.

To make an estimate of a quantity in the limit $N \to \infty$, or equivalently $\epsilon \to 0$,
we can study the change in a quantity as function of the convergence threshold $\epsilon$. We define
\begin{equation}\label{eq:stepwise_difference_epsilon}
  \Delta Q(\epsilon) = M(\epsilon) - M(10\epsilon),
\end{equation}
i.e. the change of quantity $Q$ when we decrease the threshold $\epsilon$ by an order of magnitude. The results in
\autoref{fig:stepwise_diffs_vs_tolerance} show that, remarkably, the order parameter, entropy and correlation length
to high precision follow the linear relationship
\begin{equation}\label{eq:stepwise_difference_linear}
  \Delta Q(\epsilon) = \alpha_1(m) \epsilon,
\end{equation}
whereas the free energy follows a quadratic relationship
\begin{equation}\label{eq:stepwise_difference_quadratic}
  \Delta f(\epsilon) = \alpha_2(m) \epsilon^2.
\end{equation}

This means that we can confidently extrapolate the value of a quantity in the fully converged limit as
\begin{equation}
  Q(\epsilon \to 0) = Q(\epsilon_{\text{min}}) + \sum_{\epsilon = \frac{\epsilon_{\text{min}}}{10},
  \frac{\epsilon_{\text{min}}}{100}, \dots} \Delta Q(\epsilon),
\end{equation}
where $\epsilon_{\text{min}}$ is the lowest threshold used in simulation, and $\Delta Q(\epsilon)$ is determined
by fitting to suitable higher values of the threshold.

% Another way to define the error in the convergence of a quantity is directly as function of the threshold $\epsilon$,
% as shown in \autoref{fig:quantities_vs_tolerance}. Here, the absolute relative error in a quantity is defined as
% \begin{equation}
%   \Delta M_{\text{rel}} = \left| \frac{Q(\epsilon) - Q(\epsilon = 10^{-11})}{Q(\epsilon = 10^{-11})} \right|.
% \end{equation}

\todo[inline]{technically this is not correct for the already converged values of $m$}


\begin{figure}
  \includegraphics[]{convergence_vs_n.tikz}
  \caption{Convergence as defined in \autoref{eq:convergence} versus $n$, the number of CTMRG
  steps.
  Up until very small values of $c_n$, the convergence is monotonically decreasing and obeys
  a logarithmic law with slope depending on $m$.}\label{fig:convergence_vs_n}
\end{figure}

\begin{figure}
  \includegraphics[]{convergence_finite_chi.tikz}
  \caption{Absolute relative difference of quantities (see \autoref{eq:abs_rel_diff}).
  Same legend as \autoref{fig:convergence_vs_n}.}\label{fig:convergence_finite_chi}
\end{figure}

% \begin{figure}
%   \includegraphics[]{quantities_vs_tolerance.tikz}
%   \caption{hallootjes}\label{fig:quantities_vs_tolerance}
% \end{figure}

\begin{figure}
  \includegraphics[]{stepwise_diffs_vs_tolerance.tikz}
  \caption{Stepwise differences upon decreasing the threshold $\epsilon$ by an order of magnitude,
  as in \autoref{eq:stepwise_difference_epsilon}.
  For the order parameter, entropy and correlation length, a linear relationship holds to high precision,
  while for the free energy the relationship is quadratic.}\label{fig:stepwise_diffs_vs_tolerance}
\end{figure}

\todo[inline]{Cross check with correlation length, report on boundary conditions}



\subsection{Simulations with finite system size}
In the finite-system algorithm, we want to reliably extrapolate quantities in the bond dimension $m$.
The convergence behaviour is shown in \autoref{fig:quantities_vs_chi}.
For each quantity $Q$, we plot the absolute relative difference with the value at the highest $m$
\begin{equation}\label{eq:abs_rel_diff_quantity_m}
  \Delta Q_{\text{rel}}(m) = \left| \frac{Q(m) - Q(m = 200)}{Q(m = 200)} \right|
\end{equation}
versus the bond dimension $m$.

The plateaus of $m$-values that barely increase the precision are due to the degeneracies in the spectrum of the reduced
density matrix. Apart from this structure, the law
\begin{equation}\label{eq:abs_rel_diff_power_law_m}
  \Delta Q_{\text{rel}}(m) \propto m^{\alpha(N)}
\end{equation}
is seen to hold for high enough $m$ for the order parameter, free energy and entropy.

\todo[inline]{What is $\alpha$? Refer to section where you discuss the spectrum of the reduced density matrix.
log log law doesn't seem to work out for correlation length. why??}

\begin{figure}
  \includegraphics[]{quantities_vs_chi.tikz}
  \caption{The absolute relative difference of quantities, as defined in \autoref{eq:abs_rel_diff_quantity_m}.
  For high enough $m$, it obeys a power law with varying exponent $\alpha(N)$.
  The sharp drop for the highest values of $m$ is an artefact of the definition of $\Delta Q_{\text{rel}}$ and the
  plateau-like fashion in which the value of a quantity converges,
  owing to the spectrum of the reduced density matrix approximated by the CTMRG algorithm.
  Like in the finite-$m$ case, the free energy converges much faster than the other quantities,
  and does so with little $n$-dependence.
  Note that $\Delta \xi_{\text{rel}}$ does not obey a power law.}\label{fig:quantities_vs_chi}
\end{figure}


\section{Spectrum of the corner transfer matrix}

\subsection{Analytical results for the Ising model}
In what follows, we largely follow \cite{peschel2009reduced, peschel1999density}.

For the off-critical Ising model on a square lattice, we have \cite{davies1988corner}
\begin{equation}
  \hat{\rho} = \hat{A}^4 = \exp(-\hat{H}_{\text{CTM}}),
\end{equation}
where
\begin{equation}
  \hat{H}_{\text{CTM}} = \sum_{l = 0}^{\infty} \epsilon_l(T) c^{\dagger}_l c_l,
\end{equation}
with $c_l$ and $c^{\dagger}_l$ fermionic annihilation and creation operators and
\begin{equation}
  \epsilon_l =
  \begin{cases}
    (2l + 1)\epsilon(T) & \text{if } T > T_c, \\
    2l\epsilon(T) & \text{if } T < T_c.
  \end{cases}
\end{equation}

In other words, the reduced density matrix (or equivalently, the corner transfer matrix $A$) can be written as a
density matrix of an effective free fermionic Hamiltonian with equally spaced excitations.

What does this mean for the spectrum of $A$?
If we assume a free boundary, we have to distinguish between the ordered and disordered phase.

In the disordered phase, we have $\epsilon_l = (2l + 1)\epsilon(T)$.
The ground state, $E = 0$, corresponds to the vacuum state of the effective system described by $H_{\text{CTM}}$.
The single-fermion excitations give $\epsilon, 3\epsilon, 5\epsilon,
\dots$, while two-fermion excitations give $4\epsilon$ ($c^{\dagger}_0 c^{\dagger}_1 \ket{0}$),
$6\epsilon$ ($c^{\dagger}_0 c^{\dagger}_2 \ket{0}$) and $8\epsilon$ ($c^{\dagger}_0 c^{\dagger}_3 \ket{0}$ \emph{or}
$c^{\dagger}_1 c^{\dagger}_2 \ket{0}$).
So the first degeneracy appears at $8\epsilon$.
$9\epsilon$ is triply degenerate:
it can be constructed with a single-fermion excitation and in two ways with a three-fermion excitation.

In the ordered phase, we have a two-fold degeneracy for every state due to symmetry.
The ground state energy $E = 0$.
After that, we can only make the levels $2\epsilon, 4\epsilon, 6\epsilon,
\dots$.
The degeneracy of the $n$th energy level is given by $2p(n)$, twice the number of partitions of $n$ into distinct integers
\cite{okunishi1999universal}, with the factor of two coming from symmetry.

To illustrate:
$c^{\dagger}_1 c^{\dagger}_2 \ket{0}$ and $c^{\dagger}_3 \ket{0}$ both have $E = 6\epsilon$,
the third energy level (counting the vacuum as the zeroth energy level),
which is to say $p(3) = 2$ since $\{3, 2 + 1 \}$ are the ways to write $3$.

\todo[inline]{Boundary conditions, plots.}
