\chapterprecishere{We describe the technical details of the algorithms used to compute quantities of interest.
We report the convergence behaviour of the algorithms and discuss validity and sources of error.}

\section{Symmetricity and normalization of tensors}
For the models treated in this thesis, the corner transfer matrix $A$ and the row-to-row transfer matrix $T$ are
symmetric. But due to the accumulation of machine-precision sized errors in the matrix multiplication and singular value
decomposition, this will, after many algorithm steps, no longer be the case. In order for results to remain valid, we
manually enforce symmetricity after each step.

The tensor network contractions at each algorithm step will cause the elements of $A$ and $T$ to tend to infinity, which
means that they will at some point exceed the maximum value of a floating point number as it can be stored in memory.
But because the elements of $A$ and $T$ represent Boltzmann weights, they can be scaled by a constant factor, which
allows us to prevent this overflow if we use a suitable scaling. For example by requiring that
\begin{equation}
  \tr A^4 = 1,
\end{equation}
so that the interpretation of $A^4$ as a reduced density matrix of an effective one-dimensional quantum is valid.

\section{Boundary conditions}
Unless otherwise stated, all simulations are run with the boundary spins of the lattice fixed to $+1$.
For specific purposes, free boundary conditions are used.

\section{Convergence criteria}\label{sec:convergence_criteria}

\subsection{Simulations with finite bond dimension}
The convergence of the CTMRG algorithm with fixed bond dimension $m$ (the infinite system algorithm) can be defined
in multiple ways (\emph{cite}). In this thesis, the convergence after step $i$ of the algorithm is defined as
\begin{equation}\label{eq:convergence}
  c_i = \sum_{\alpha = 1}^{m} | s_{\alpha}^{(i)} - s_{\alpha}^{(i - 1)} |,
\end{equation}
where $s_{\alpha}$ are the singular values of the corner transfer matrix $A$. If the convergence falls below some
threshold $\epsilon$, the algorithm terminates.

The assumption is that once the singular values stop changing to some precision, the optimal projection is sufficiently
close to its fixed point and the transfer matrices $A$ and $T$ represent an environment only limited by the length scale
given by $m$, i.e.
\begin{equation}
  \xi(m) \ll N
\end{equation}
is satisfied.

\subsubsection{Convergence at the critical point of the Ising model}
The convergence is shown in \autoref{fig:convergence_vs_n}. It is clear that the
phenomenological law
\begin{equation}\label{eq:convergence_vs_n_semilogarithmic_law}
  \log c_n \propto n
\end{equation}
holds to high precision, with the slope depending on $m$.
Deviations only occur at values of $c$ of around $10^{-12}$.

The convergence of the various quantities as function of the number of algorithm steps is shown in
\autoref{fig:convergence_finite_chi}. For all quantities $Q$, the absolute value of the relative difference
with the final algorithm step
\begin{equation}\label{eq:abs_rel_diff}
  \Delta Q_{\text{rel}}(n) = \left| \frac{Q(n) - Q(n = 10^5)}{Q(n = 10^5)} \right|
\end{equation}
is shown. Again, a law of the form
\begin{equation}\label{eq:abs_rel_diff_vs_n_semilogarithmic_law}
  \log(\Delta Q_{\text{rel}}) \propto n
\end{equation}
seems to hold.

To make an estimate of a quantity in the limit $N \to \infty$, or equivalently $\epsilon \to 0$,
we can study the change in a quantity as function of the convergence threshold $\epsilon$. We define
\begin{equation}\label{eq:stepwise_difference_epsilon}
  \Delta Q_{\text{step}}(\epsilon) = Q(\epsilon) - Q(10\epsilon),
\end{equation}
i.e. the change of quantity $Q$ when we decrease the threshold $\epsilon$ by an order of magnitude. The results in
\autoref{fig:stepwise_diffs_vs_tolerance} show that, the order parameter, entropy and correlation length
to high precision follow the linear relationship
\begin{equation}\label{eq:stepwise_difference_linear}
  \Delta Q_{\text{step}}(\epsilon) = \alpha_{Q}(m) \epsilon,
\end{equation}
whereas the free energy follows a quadratic relationship
\begin{equation}\label{eq:stepwise_difference_quadratic}
  \Delta f(\epsilon) = \alpha_{f}(m) \epsilon^2.
\end{equation}

\todo[inline]{Why is this case?
It must have something to do with the fact that in the definition of the free energy the log is taken.}

This means that we can confidently extrapolate the value of a quantity in the fully converged limit as
\begin{equation}\label{eq:extrapolation_fully_converged_limit_finite_m}
  Q(\epsilon \to 0) = Q(\epsilon_{\text{min}}) + \sum_{\epsilon = \frac{\epsilon_{\text{min}}}{10},
  \frac{\epsilon_{\text{min}}}{100}, \dots} \Delta Q_{\text{step}}(\epsilon),
\end{equation}
where $\epsilon_{\text{min}}$ is the lowest threshold used in simulation, and $\Delta Q_{\text{step}}(\epsilon)$ is determined
by fitting to suitable higher values of the threshold.

\begin{figure}
  \includegraphics[]{convergence_vs_n.tikz}
  \caption{Convergence as defined in \autoref{eq:convergence} versus $n$, the number of CTMRG
  steps.
  Up until very small values of $c_n$, the convergence is monotonically decreasing and obeys
  a logarithmic law with slope depending on $m$.}\label{fig:convergence_vs_n}
\end{figure}

\begin{figure}
  \includegraphics[]{convergence_finite_chi.tikz}
  \caption{Absolute value of relative difference of quantities (see \autoref{eq:abs_rel_diff}).
  Same legend as \autoref{fig:convergence_vs_n}.}\label{fig:convergence_finite_chi}
\end{figure}

\begin{figure}
  \includegraphics[]{stepwise_diffs_vs_tolerance.tikz}
  \caption{Stepwise differences upon decreasing the threshold $\epsilon$ by an order of magnitude,
  as in \autoref{eq:stepwise_difference_epsilon}.
  For the order parameter, entropy and correlation length, a linear relationship holds to high precision,
  while for the free energy the relationship is quadratic.}\label{fig:stepwise_diffs_vs_tolerance}
\end{figure}

\subsection{Simulations with finite system size}
In the finite-system algorithm, we want to reliably extrapolate quantities in the bond dimension $m$.
The convergence behaviour is shown in \autoref{fig:quantities_vs_chi}.
For each quantity $Q$, we plot the absolute value of the relative difference with the value at the highest $m$
\begin{equation}\label{eq:abs_rel_diff_quantity_m}
  \Delta Q_{\text{rel}}(m) = \left| \frac{Q(m) - Q(m = 200)}{Q(m = 200)} \right|
\end{equation}
versus the bond dimension $m$.

The plateaus of $m$-values that barely increase the precision are due to the structure in the spectrum of the reduced
density matrix. Apart from this noise, the law
\begin{equation}\label{eq:abs_rel_diff_power_law_m}
  \Delta Q_{\text{rel}}(m) \propto m^{\alpha(N)}
\end{equation}
is seen to hold for high enough $m$ for the order parameter, free energy and entropy.

To extrapolate to $m \to \infty$, analogously to the finite-$m$ case we define
\begin{equation}\label{eq:delta_q_step_finite_N}
  \Delta Q_{\text{step}}(m) = Q(m) - Q(m - 1),
\end{equation}
which is plotted in the left panel of \autoref{fig:stepwise_diffs_vs_m}.
As expected from \autoref{eq:abs_rel_diff_power_law_m}, the overall trend looks like a power law,
but the noise makes it hard to determine it accurately.
It is anyway neither practical nor needed to calculate a system of finite size for many consecutive values of $m$,
in order to get a good approximation to \autoref{eq:delta_q_step_finite_N}.
Instead, one might compute the quantity for equally spaced values of $m$ with a difference of say,
10 or 20.

As described in \autoref{sec:spectrum_of_ctm} and in \cite{okunishi1999universal,
davies1988corner, peschel2009reduced}, for the off-critical Ising model the degeneracies in the corner transfer matrix
$A$ are exactly known.
These degeneracies are smoothed out, but not completely lost for finite systems near criticality. This is directly related to the convergence of quantities in $m$.

As an alternative to \autoref{eq:delta_q_step_finite_N}, it might be conjectured that not a fixed increase of $m$,
but taking all basis states corresponding to the next energy level will increase accuracy in a predictable way.

This suspicion is confirmed in the right panel of \autoref{fig:stepwise_diffs_vs_m},
which shows the convergence using only the values of $m$ for which all basis states with a certain energy level are
present, i.e.
\cite{okunishi1999universal}
\begin{equation}
  m(j) = \sum_{k = 0}^{k = j} p(k).
\end{equation}
The first few values of $m(j)$ are
\begin{equation}\label{eq:m_values_all_degenerate_basis_states}
  1, 2, 3, 5, 7, 10, 14, 19, 25, 33, 43, 55, 70, 88, 110, 137, 169, 207, \dots
\end{equation}
Using these values of $m$, the result is again a power law, and the noise largely disappears, but the convergence
behaviour is still not devoid of structure entirely.

The error on a quantity calculated at bond dimension $m_{\text{max}}$ can be approximated as
\begin{equation}\label{eq:extrapolation_finite_N}
  M(m \to \infty) = M(m_{\text{max}}) + \sum_{\text{higher values of $m$}} \Delta Q_{\text{step}}(m),
\end{equation}
where $\Delta Q_{\text{step}}$ is fitted from aptly chosen lower values of $m$.

\begin{figure}
  \includegraphics[]{quantities_vs_chi.tikz}
  \caption{The absolute value of the relative difference of quantities, as defined in \autoref{eq:abs_rel_diff_quantity_m}.
  For high enough $m$, it obeys a power law with varying exponent $\alpha(N)$.
  The sharp drop for the highest values of $m$ is an artefact of the definition of $\Delta Q_{\text{rel}}$ and the
  plateau-like fashion in which the value of a quantity converges,
  owing to the spectrum of the reduced density matrix approximated by the CTMRG algorithm.
  Like in the finite-$m$ case, the free energy converges much faster than the other quantities,
  and does so with little $n$-dependence.
  Note that $\Delta \xi_{\text{rel}}$ does not obey a power law.}\label{fig:quantities_vs_chi}
\end{figure}

\begin{figure}
  \includegraphics[]{stepwise_diffs_vs_m.tikz}
  \caption{Same legend as \autoref{fig:quantities_vs_chi}}\label{fig:stepwise_diffs_vs_m}
\end{figure}


\section{Values of hyperparameters for the Ising model}
We have found that for practical purposes, in the infinite-system algorithm the convergence threshold $\epsilon$ may be
set to $10^{-7}$ for values of $m$ below, say, 30, and to $10^{-8}$ for $m$-values above 40.
For the highest value of $m = 120$, this leads to a relative error,
as compared with the theoretically fully converged limit $\epsilon \to 0$,
of about $0.1\%$ in the correlation length, and less than $0.02\%$ in the order parameter,
entropy and free energy (the relative error on the free energy being basically zero).
For lower values of $m$, the relative errors are substantially lower.
See \autoref{table:rel_error_correlation_length_finite_m_T_crit}.

\todo[inline]{What about convergence at $T^{\star}$?}

For simulations of finite systems, we have found that choosing $m$ such that the truncation error $P_r$ (for
\emph{residual probability}, its other common name) is smaller than $10^{-6}$ is sufficient for most purposes.
This leads to a relative error with the fully converged limit $m \to \infty$ of at most $0.06\%$ in the order parameter,
entropy and free energy for $n = 8000$.
For lower values of $N$, the relative error is lower.
Here, we have taken the order parameter as the main benchmark, since the correlation length is not used for finite-size
calculations. See \autoref{table:rel_error_order_param_finite_N_T_crit}.

\begin{table}[]
\centering
\begin{tabular}{@{}lll@{}}
$m$ & rel. error \% ($\epsilon = 10^{-7}$) & rel. error \% ($\epsilon = 10^{-8}$) \\ \midrule
40  & 0.02                              & 0.02                              \\
80  & 0.6                               & 0.05                              \\
120 & 1.1                               & 0.1                               \\ \bottomrule
\end{tabular}
\caption{Relative errors of $\xi(T_c, m, \epsilon)$, as compared with the fully converged limit $\epsilon \to 0$,
approximated by
\autoref{eq:extrapolation_fully_converged_limit_finite_m}.}\label{table:rel_error_correlation_length_finite_m_T_crit}
\end{table}

\begin{table}[]
\centering
\begin{tabularx}{\linewidth}{cXXX}
$n$ & rel. error \% ($ P_r < 10^{-5} $)  & rel. error \% ($P_r < 10^{-6}$) & rel. error \% ($P_r < 10^{-7}$) \\ \midrule
1000  & 0.08 (40)                         & 0.005 (80)    & 0.001 (120)                               \\
2000  & 0.08 (60)                         & 0.008 (100)   & 0.002 (160)                             \\
8000  & 0.6  (80)                         & 0.07  (140)   & 0.005 (280)                              \\ \bottomrule
\end{tabularx}
\caption{Relative errors of $M(T_c, n)$, calculated such that the truncation error $P_r$ does not exceed listed threshold,
as compared with the fully converged limit of $m \to \infty$ (zero truncation error),
approximated by \autoref{eq:extrapolation_finite_N}. The number in brackets is the value of $m$ used. To approximate the error, steps in $m$-value of 5, 10 or 20 were used, depending on $m_{\text{max}}$.} \label{table:rel_error_order_param_finite_N_T_crit}
\end{table}
