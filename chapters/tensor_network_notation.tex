\section{Tensors, or multidimensional arrays}

In the field of tensor networks, a tensor is a multidimensional table with numbers -- a convenient
way to organize information. It is the generalization of a vector

\begin{equation}
  v_i =
  \begin{bmatrix}
    v_1 \\
    \vdots \\
    v_n
  \end{bmatrix}
\end{equation}

which has one index, and a matrix

\begin{equation}
  M_{i j} = 
  \begin{bmatrix}
  M_{1 1} & \dots & M_{1 n} \\
  \vdots  & & \vdots \\
  M_{m 1} & \dots & M_{m n} 
  \end{bmatrix}
\end{equation}

which has two. A tensor of rank $N$ has $N$ indices\footnote{The definition of rank in this
context is not to be confused with the rank of a matrix, which is the number of
linearly independent columns. Synonyms of tensor rank are tensor degree and
tensor order.}

\begin{equation}
  T_{i_1 \dots i_N}
\end{equation}

\section{Tensor contraction}

Tensor contraction is the higher-dimensional generalization of the dot product

\begin{equation}
  \bm{a} \cdot \bm{b} = \sum_i a_i b_i
\end{equation}

where a lower-dimensional tensor (in this case, a scalar, which is a
zero-dimensional tensor) is obtained by summing over all values of a repeated
index. Examples are matrix-vector multiplication

\begin{equation}
  (M \bm{a})_{i} = \sum_j M_{i j} a_j
\end{equation}

and matrix-matrix multiplication

\begin{equation}
  (A B)_{i j} = \sum_k A_{i k} B_{k j}
\end{equation}

But a more elaborate tensor multiplication could look like

\begin{equation}
  w_{a b c} = \sum_{d, e, f} T_{a b c d e f} v_{d e f}
\end{equation}

As with the dot product between vectors, matrix-vector multiplication and
matrix-matrix multiplication, a contraction between tensors is only defined if
the dimensions of the indices match.

\section{Tensor networks}

A tensor network is specified by a set of tensors, together with a set of contractions to be performed. For example:

\begin{equation}
  M_{a b} = \sum_{i, j, k} A_{a i} B_{i j} C_{j k} D_{k b}
\end{equation}

which corresponds to the matrix product $A B C D$.

\subsection{Graphical notation}
It is highly convenient to introduce a graphical notation, which
greatly simplifies expressions and makes certain properties manifest.


