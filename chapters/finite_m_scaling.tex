\chapterprecishere{abstract bla bla}

Up until now, we have developed our scaling analysis in terms of a finite system size $N$.
But the approximation of the infinite-system partition function with the CTMRG algorithm depends on two parameters;
the system size $N$ and the bond dimension $m$.

A finite bond dimension $m$ carries a characteristic length scale.
Baxter \cite{baxter1978variational}, and later Östlund and Rommer \cite{ostlund1995thermodynamic} (in the context of
one-dimensional quantum systems) showed that in the thermodynamic limit,
CTMRG and DMRG are variational optimizations in the space of matrix product states.
\todo[inline]{Can extend this idea a bit.}

It is known that an MPS-ansatz with finite bond dimension inherently limits the
correlation length of the system to a finite value \cite{wolf2006quantum, rommer1997class}. Hence,
thermodynamic quantities obtained from the CTMRG algorithm with finite $m$, in the limit
$N \to \infty$, cannot diverge and must show finite-size effects similar to those of some
effective finite system of size $N_{\text{eff}}(m)$ depending on the bond dimension $m$.

\autoref{fig:order_parameter_vs_T} shows the behaviour of the order parameter of the
two-dimensional Ising model for systems of finite-size,
where the result is converged in $m$, and for systems of finite $m$, where
the result is converged in the system size $N$. The results look very similar and support
the claim that there are two relevant length scales in the critical region, namely the system size $N$ and
the length scale associated to the finite bond dimension $m$.

\begin{figure}
\includegraphics[]{order_parameter_vs_T.tikz}
\caption{Upper panel: expectation value of the central spin $\langle \sigma_0 \rangle$
  after $n$ CTMRG steps. $m$ is chosen such that the truncation error is smaller than
  $10^{-6}$. Lower panel: $\langle \sigma_0 \rangle$ for systems with bond dimension $m$.}\label{fig:order_parameter_vs_T}
\end{figure}

\section{Definition of the effective length scale in terms of the correlation length at $T_c$}\label{sec:definition_effective_length_scale_in_terms_of_xi}

The first direct comparison of finite-size scaling in the system size $N$ with scaling in
the bond dimension of the CTMRG method $m$ was done
in \cite{nishino1996numerical}.

In the thermodynamic limit (corresponding to infinite $m$ and $N$), we have the following
expression for the correlation length of a classical system
\cite{baxter1982exactly_correlation_length}
\begin{equation}\label{eq:correlation_length_row_to_row_transfer_matrix}
  \xi(T) = \frac{1}{\log\left(\frac{T_0}{T_1}\right)}.
\end{equation}
Here, $T_0$ and $T_1$ are the largest and second-largest eigenvalues of the row-to-row
transfer matrix $T$, respectively. With $N$ tending towards infinity and finite $m$, near
the critical point $\xi(T)$ should obey a scaling law of the form
\begin{equation}
  \xi(T, m) = N_{\text{eff}}(m) \mathcal{F}(N_{\text{eff}}(m) / \xi(T))
\end{equation}
with
\begin{equation}
  \mathcal{F}(x) = \begin{cases}
      \text{const} & \text{if } x \to 0, \\
      x^{-1} & \text{if } x \to \infty.
    \end{cases}
\end{equation}

Hence, the effective length scale corresponding to the finite bond dimension $m$ is
proportional to the correlation length of the system at the critical point $t = 0$.
\begin{equation}
  N_{\text{eff}}(m) \propto \xi(T = T_c, m).
\end{equation}

\todo[inline]{Look ahead to replicating this in results section?}

\section{Relation to finite-entropy scaling and the exponent $\kappa$.}


The first numerical evidence of a law for the correlation length at the critical point of the form
\begin{equation}\label{eq:xi_propto_m_kappa}
  \xi(m) \propto m^{\kappa}
\end{equation}
was given by the authors of \cite{andersson1999density}, who found
\begin{equation}
  \kappa \approx 1.3
\end{equation}
for a gapless system of free fermions, using DMRG calculations. Later, using the iTEBD algorithm
\cite{vidal2007classical}, the authors of \cite{tagliacozzo2008scaling} presented numerical evidence for such a relation
for the Ising model with transverse field and the Heisenberg model, with
\begin{align}
  \kappa_{\text{Ising}} & \approx 2, \\
  \kappa_{\text{Heisenberg}} & \approx 1.37.
\end{align}

\subsection{Quantitative theory for $\kappa$}
A quantitative theory of the existence of an exponent $\kappa$ was given in \cite{pollmann2009theory}.
We reproduce the argument, which is presented in the language of one-dimensional quantum systems, below.

We start by noting that in the critical region, the entanglement of a half-infinite subsystem $A$ diverges as
\begin{equation}\label{eq:entropy_scaling_near_criticality}
  S_A \propto \mathcal{A}(c/6)\log(\xi),
\end{equation}
where $\mathcal{A}$ is the number of boundary points of $A$ and $c$ is the central charge of the conformal field theory
at the critical point \cite{calabrese2004entanglement, vidal2003entanglement, ercolessi2010exact}.

Recalling the definition of the entanglement entropy
\begin{equation}
  S_A = - \tr(\rho_A \log \rho_A) = - \sum_{\alpha} \omega_{\alpha} \log \omega_{\alpha},
\end{equation}
it is trivially seen that the entropy of a state given by the DMRG (or any other MPS), which only
retains $m$ basis states of $\rho_A$, is limited by
\begin{equation}
  S^{\text{max}}_A(m) = \log m
\end{equation}
by putting $\omega_{\alpha} = 1/m$ for $\alpha = 1, \dots, m$.

This is, incidentally, another way to see that DMRG or CTMRG, or any other algorithm which produces ground states with a
matrix-product structure have an inherently finite correlation length.

The leading energy correction to the free energy per site of a one-dimensional quantum system at a conformally invariant
critical point at finite temperature $T$ in the thermodynamic limit is \cite{affleck1986universal}
\begin{equation}\label{eq:correction_free_energy_critical_point_finite_temperature}
  f(T) = f_0 + aT^2 + \mathcal{O}(T^3).
\end{equation}

Due to the quantum-classical correspondence, this is equivalent to a two-dimensional classical $N \times \infty$ lattice
with strip width $N = 1/T$.
This implies also that the correlation length of a critical one-dimensional quantum system at finite temperature cannot
diverge and goes as $\xi \propto 1/T$.
In terms of this finite correlation length, \autoref{eq:correction_free_energy_critical_point_finite_temperature} is
written as
\begin{equation}\label{eq:correction_free_energy_critical_point_finite_correlation_length}
  f(\xi) = f_{\infty} + \frac{A}{\xi^2} + \mathcal{O(\frac{1}{\xi^3})}.
\end{equation}

Empirically, optimized ground states with a matrix-product structure at criticality do not simply maximize their
entropy, as they should if we take \autoref{eq:correction_free_energy_critical_point_finite_correlation_length} to be
true for ground states with a matrix-product structure.

We will now show that \autoref{eq:correction_free_energy_critical_point_finite_correlation_length} needs,
in fact, an additional term due to the matrix-product structure with finite bond dimension $m$.

The ground state with finite correlation length and energy density as in
\autoref{eq:correction_free_energy_critical_point_finite_correlation_length} has a Schmidt decomposition that in
principle can have infinitely many terms
\begin{equation}\label{eq:ground_state_infinite_schmidt_decomposition}
  \ket{\psi_0} = \sum_{n = 1}^{\infty} \lambda_n \ket{\psi_{n}^{L}}\ket{\psi_{n}^{R}},
\end{equation}
where $\ket{\psi_{n}^{L}}$ and $\ket{\psi_{n}^{R}}$ are states of the left and right infinite half-chains. Normalization
requires
\begin{equation}
  \sum_{n}^{\infty} \lambda_{n}^2 = 1.
\end{equation}

The ground state with a matrix-product structure with finite bond dimension $m$ has an additional constraint:
its Schmidt decomposition carries only the $m$ $\ket{\psi_n}$ with largest $\lambda_n$.
It is written as
\begin{equation}
  \ket{\psi_{0}^{\text{MPS}}} = \frac{\sum_{n = 1}^{m} \lambda_n
  \ket{\psi_{n}^{L}}\ket{\psi_{n}^{R}}}{\sqrt{\sum_{n=1}^{m} \lambda_{n}^2}}.
\end{equation}

To find the extra energy cost of only keeping the first $m$ terms in the Schmidt decomposition,
note that in the limit of $m$ large, $\ket{\psi_{0}^{\text{MPS}}}$ almost entirely overlaps with $\ket{\psi_0}$,
hence can be written as
\begin{equation}
  \ket{\psi_{0}^{\text{MPS}}} = \sqrt{1 - \epsilon^2} \ket{\psi_0} + \epsilon \ket{\psi_{\text{ex}}},
\end{equation}
where $\ket{\psi_{\text{ex}}}$ is some excited state and $\epsilon \ll 1$. This leads to an energy of
\begin{equation}
  E_{0}^{\text{MPS}} = \braket{\psi_{0}^{\text{MPS}} | \hat{H} | \psi_{0}^{\text{MPS}}} = E_0 + \epsilon^2 (E_{\text{ex}} - E_0),
\end{equation}
with
\begin{equation}
  \epsilon^2 = \left(1 - \braket{\psi_0 | \psi_{0}^{\text{MPS}}}^2 \right) = 1 - \sum_{n = 1}^{m} \lambda_{n}^2 \equiv
  P_{\text{res}}(m).
\end{equation}
Here, we have defined the residual probability $P_{\text{res}}$, also known as the truncation error,
as the part of the spectrum that is thrown away.

If we now assume that $E_0 - E_{\text{ex}}$ is proportional to the energy gap $\Delta$, which scales as \cite{lieb1961two, mata1989energy, pfeuty1970one}
\begin{equation}
  \Delta \propto \frac{1}{\xi},
\end{equation}
we arrive at
\begin{equation}\label{eq:correction_energy_mps_ground_state}
  E_{0}^{\text{MPS}} = E_{\infty} + \frac{A}{\xi^2} + \frac{B P_{\text{res}}(m)}{\xi}.
\end{equation}

It is clear that when the correlation length is very large, by \autoref{eq:entropy_scaling_near_criticality} the entropy
and $P_{\text{res}}(m)$ must be too.
So, the third term in \autoref{eq:correction_energy_mps_ground_state} dominates.

If the correlation length is small, the second term dominates.
The correlation length that belongs to the MPS ground state with fixed $m$ is the optimum that minimizes this
expression.

The details of the calculation, which can be found in the supplementary material of \cite{pollmann2009theory},
depend on the asymptotic form of $P_{\text{res}}$, found in \cite{calabrese2008entanglement}. In the limit $m \to \infty$, the correlation is indeed of the form in \autoref{eq:xi_propto_m_kappa} with
\begin{equation}\label{eq:exact_value_kappa}
  \kappa = \frac{6}{c \left( \sqrt{12/c} + 1 \right) },
\end{equation}
which is in good agreement with the values found in \cite{tagliacozzo2008scaling}.

\todo[inline]{Refer back to chapter on spectrum of CTM}

\section{Locating the critical point with the entanglement spectrum}
Since phase transitions of quantum systems can be located by studying their entanglement spectrum (\emph{cite here}),
classical systems may be investigated in the same way through the correspondence in
\autoref{eq:correspondence_density_matrix_ctm}. This is an alternative to the usual approach of studying an order
parameter or derivatives of thermodynamical observables (\emph{cite here?}).

Examples of studies using the spectrum of the corner transfer matrix to analyze two-dimensional classical systems are
\cite{krvcmar2015reentrant, PhysRevE.94.022134, krvcmar2016phase}.


\begin{itemize}
  \item \cite{huang2017holographic}: refs [42] and [12] contain many papers which study the phenomenon of pinpointing
  a phase transition without using physical observables (i.e. entanglement, spectrum, fidelity instead.)
  \item \cite{osborne2002entanglement}: for XY and Ising model, proves that next-to-nearest neighbor entanglement peaks
  at critical point (though not nearest-neighbor entanglement.)
\end{itemize}
