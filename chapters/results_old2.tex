\section{Abstract}
We present the results of finite-size scaling in the bond dimension of the transfer
matrices on the two-dimensional Ising model.

\section{Two-dimensional Ising model}

The corner transfer matrix renormalization group method was first used to perform a
finite-size scaling analysis in \cite{nishino1996numerical}.

\todo[inline]{This is a bad explanation. Cite Nishino more, since this is basically a
review of his article}

The error in the approximation of the partition function (and thus all thermodynamic
quantities) in the thermodynamic limit with the corner transfer matrix method depends on
two characteristic length
scales. The first is the size of the system $N$. After $n$ steps of the infinite-system
algorithm, we have
\begin{equation}
  N = 2n + 1.
\end{equation}
The second length scale is related to the finite bond dimension $m$. In the thermodynamic
limit (corresponding to infinite $N$ and $m$)



\cite{baxter1982exactly_correlation_length}
\begin{equation}
  \xi(T) = \frac{1}{\log\left(\frac{T_0}{T_1}\right)}.
\end{equation}
where $T_0$ and $T_1$ are the largest and second-largest eigenvalues of the row-to-row
transfer matrix $T$, respectively.

Let us denote by $T_{i}(N, m)$ the $i$th eigenvalue of the row-to-row transfer matrix of
the system of size $N$ with bond dimension $m$ obtained through the corner transfer
matrix. A natural way to define the characteristic length scale associated with the bond
dimension $m$ is \footnote{This length scale can also be viewed as the finite correlation
length inherent to a matrix product state of a 1D quantum system, see REF.}
\begin{equation}
  \xi(m) = \lim_{N \to \infty} \frac{1}{\log\left(\frac{T_0(N, m)}{T_1(N, m)}\right)}.
\end{equation}

\todo[inline]{How do we explain the box size, i.e. that we should take the maximum of
$\xi(m)$ to be the effective length scale?}

For off-critical systems, the true correlation length $\xi$ is finite, so we expect to be
able to approximate thermodynamic quantities well if
\begin{equation}\label{eq:correlation_lengths_big_enough}
\begin{aligned}
\xi(m) & \gg \xi, \\
N & \gg \xi
\end{aligned}
\end{equation}
are both satisfied.

For critical systems, the correlation length diverges, so
\autoref{eq:correlation_lengths_big_enough} is never satisfied. Instead, the correlation
length is either limited by $N$ or $\xi(m)$, and we expect to see cross-over behaviour
between the regions $N \gg \xi(m)$ and $N \ll \xi(m)$. This cross-over is studied in
\cite{nishino1996numerical}. We reproduce some of the results of this work using the
two-dimensional Ising model.

For the magnetization at the critical temperature, we expect a scaling law of the form
\begin{equation}
  M(N, m) \thicksim
  \begin{cases}
    N^{\frac{\beta}{\nu}} &\text{if } \xi(m) \ll N \\
    \xi(m)^{\frac{\beta}{\nu}} &\text{if } N \ll \xi(m)
  \end{cases}.
\end{equation}
In \autoref{fig:crossover}, this crossover is clearly visible. For small $N$, $M \thicksim
N^{\frac{-\beta}{\nu}} = N^{-\frac{1}{8}}$, for all values of $m$, because $N$ is the only
important length scale. For large $N$, the magnetization tends towards a non-zero value
set by the length scale $\xi(m)$.

\autoref{fig:xi_vs_m_loglogfit} shows that in the limit $N \to
\infty$, we indeed have

\begin{equation}
  M(N, m) \thicksim \xi(m)^{-\frac{\beta}{\nu}}.
\end{equation}
A least-squares fitting of this power law gives the fraction
\begin{equation}
\frac{\beta}{\nu} \approx 0.12499.
\end{equation}
which differs only very slightly from the exact value $\frac{\beta}{\nu} = \frac{1}{8}$.

\todo[inline]{Tell how I got this (selecting m values etc)}

\begin{figure}
\includegraphics[]{N_vs_m_at_T_crit_chi4-148.tikz}
\caption{Cross-over between $N = 2n + 1$, the system size and $\xi(m)$, the correlation
length due to finite $m$.}
\label{fig:crossover}
\end{figure}

\begin{figure}
\includegraphics[]{xi_vs_m_loglogfit.tikz}
\caption{The value of $M(N, m)$ in the limit $N \to \infty$ depends only on the length
scale $\xi(m)$.}
\label{fig:xi_vs_m_loglogfit}
\end{figure}

\subsection{Data collapse}

\subsection{Determination of the critical point by the corner transfer matrix spectrum}
