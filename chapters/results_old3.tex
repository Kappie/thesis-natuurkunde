\section{Abstract}

\todo[inline]{Cite more}

We present finite-size scaling results using the corner transfer matrix renormalization
group method on two-dimensional classical square lattices \cite{nishino1996corner}. We
compare the results of conventional finite-size scaling in the system size $N$ with
scaling in the number of states kept during the renormalization step of the algorithm,
denoted by $m$. Such a comparison was first done in \cite{nishino1996numerical}. We
highlight the areas in which method excels over the other.

Calculate critical temperature and exponents using information that is directly
extractable from the corner transfer matrix.


\section{Introduction}

The first direct comparison of finite-size scaling in the system size $N$ with scaling in
the bond dimension of the corner transfer matrix renormalization group method $m$ was done
in \cite{nishino1996numerical}. In explaining the basic concepts, we largely follow this
paper.

The error in the approximation of the partition function (and thus all thermodynamic
quantities) in the thermodynamic limit with the corner transfer matrix method depends on
two characteristic length
scales. The first is the size of the system $N$. After $n$ steps of the infinite-system
algorithm, we have
\begin{equation}
  N = 2n + 1.
\end{equation}

The second length scale is related to the finite bond dimension $m$. Baxter
\cite{baxter1978variational}, and later Östlund and Rommer \cite{ostlund1995thermodynamic}
(in the context of one-dimensional quantum systems) showed that in the thermodynamic
limit, CTMRG and DMRG are variational optimizations in the space of matrix product states.
\todo[inline]{Can extend this idea a bit.}

\begin{figure}
\includegraphics[]{order_parameter_vs_T.tikz}
\caption{Upper panel: expectation value of the central spin $\langle \sigma_0 \rangle$
  after $n$ CTMRG steps. $m$ is chosen such that the truncation error is smaller than
  $10^{-6}$. Lower panel: $\langle \sigma_0 \rangle$ for systems with bond dimension $m$.}\label{fig:order_parameter_vs_T}
\end{figure}

It is known that an MPS-ansatz with finite bond dimension inherently limits the
correlation length of the system to a finite value \cite{wolf2006quantum}. Hence,
thermodynamic quantities obtained from the CTMRG algorithm with finite $m$, in the limit
$N \to \infty$, cannot diverge and must show finite-size effects similar to those of some
effective finite system of size $N_{\text{eff}}(m)$ depending on the bond dimension $m$.

\autoref{fig:order_parameter_vs_T} shows the behaviour of the order parameter of the
two-dimensional Ising model for systems of finite-size, where $m$ has been chosen such
that the truncation error is smaller than $10^{-6}$, and for systems of finite $m$, where
the result is converged in the system size $N$. The results look very similar and support
the above claim.

\todo[inline]{Order parameter is not the same as magnetization central spin. Where to
explain this?}

\subsection{Definition of  in terms of the correlation length at $T_c$}

In the thermodynamic limit (corresponding to infinite $m$ and $N$), we have the following
expression for the correlation length of a classical system
\cite{baxter1982exactly_correlation_length}
\begin{equation}
  \xi(T) = \frac{1}{\log\left(\frac{T_0}{T_1}\right)}.
\end{equation}
Here, $T_0$ and $T_1$ are the largest and second-largest eigenvalues of the row-to-row
transfer matrix $T$, respectively. With $N$ tending towards infinity and finite $m$, near
the critical point $\xi(T)$ should obey a scaling law of the form
\begin{equation}
  \xi(T, m) = N_{\text{eff}}(m) \mathcal{F}(N_{\text{eff}}(m) / \xi(T))
\end{equation}
with
\begin{equation}
  \mathcal{F}(x) = \begin{cases}
      \text{const} & \text{if } x \to 0, \\
      x^{-1} & \text{if } x \to \infty.
    \end{cases}
\end{equation}

Hence, the effective length scale corresponding to the finite bond dimension $m$ is
proportional to the correlation length of the system at the critical point $t = 0$.
\begin{equation}
  N_{\text{eff}}(m) \propto \xi(T = T_c, m).
\end{equation}


\subsection{Scaling relations away from the critical point}
In general, the position of the critical point is not known. In that situation, the scaling relation in
\autoref{eq:order_param_scaling_relation} cannot be used to calculate thermodynamic information. Instead, in the limit
$N \to \infty$, we should have
\begin{equation}\label{eq:order_param_scaling_relation_finite_chi_t}
  M(t, m) \propto \xi(m)^{-\beta/\nu}\mathcal{P}(t \xi(m)^{1/\nu}),
\end{equation}
which is confirmed in \autoref{fig:data_collapse_chi_correlation_length}.

However, in practice this is still problematic, since $\xi(m)$ is defined at the critical point. Thus, we must find a
way to define the length scale corresponding to a finite bond dimension $m$ without making use of the position of the
critical point.

\begin{figure}
  \includegraphics[]{data_collapse_chi_correlation_length.tikz}
  \caption{Scaling function $\mathcal{P}(t \xi(m)^{1/\nu})$ in
  \autoref{eq:order_param_scaling_relation_finite_chi_t}.}\label{fig:data_collapse_chi_correlation_length}
\end{figure}


\subsection{Numerical results}

We now check the validity of \autoref{eq:xi_propto_kappa} in the context of the CTMRG method for two-dimensional
classical systems. Similar checks were done for one-dimensional quantum systems in \cite{tagliacozzo2008scaling}.

Directly checking \autoref{eq:xi_propto_kappa} yields $\kappa = 1.93$, see top left panel of
\autoref{fig:support_for_kappa}.
Under the assumption of \autoref{eq:xi_propto_kappa}, we have the following scaling laws at the critical point
\begin{align}\label{eq:scaling_laws_order_param_free_energy_kappa}
  M(m) & \propto m^{-\beta \kappa / \nu} \\
  f(m) - f_{\text{exact}} & \propto m^{(2-\alpha)\kappa / \nu}
\end{align}
for the order parameter and the singular part of the free energy, respectively. A fit to $M(m)$ yields $\kappa = 1.93$
and a fit to $f(m) - f_{\text{exact}}$ yields $\kappa = 1.90$. See the top right and bottom left panels of
\autoref{fig:support_for_kappa}. Here, we have used $\beta = 1/8$, $\nu = 1$ and $\alpha = 0$ for the Ising model.
\todo[inline]{Tell that the $\kappa$ law is indeed valid, since it is a good fit.}

We may use \autoref{eq:entropy_scaling_near_criticality} and \autoref{eq:classical_entropy} to check the
relation
\begin{equation}\label{eq:scaling_law_entropy_kappa}
  S_{\text{classical}} \propto \frac{c\kappa}{6}\log m,
\end{equation}
which also yields $\kappa = 1.93$, where $c = 1/2$ for the Ising model. See bottom right panel of
\autoref{fig:support_for_kappa}.

We may directly verify the value of the central charge $c$ associated with the conformal field theory at the critical
point by fitting to
\begin{equation}\label{eq:entropy_vs_correlation_length}
  S_{\text{classical}} \propto \frac{c}{6} \log \xi(m),
\end{equation}
which yields $c = 0.501$, shown in the left panel of \autoref{fig:entropy_vs_correlation_length}.

The right panel of \autoref{fig:entropy_vs_correlation_length} shows the fit to the scaling relation in $N$ (or,
equivalently the number of CTMRG steps $n$)
\begin{equation}\label{eq:entropy_vs_N}
  S_{\text{classical}} \propto \frac{c}{6} \log N,
\end{equation}
which yields $c = 0.498$.

To verify if the point of maximum entropy
\begin{equation}
  T^{\star}(m) = \max_{T} S(T, m)
\end{equation}
is a good definition of the pseudocritical point, we fit the relation
\begin{equation}\label{eq:T_pseudocrit_scaling_law_chi}
  T^{\star} - T_c \propto \xi(m)^{-1 / \nu}.
\end{equation}
which yields $\widetilde{T}_c = 2.2692$ and $\nu = 0.997$ when the length scale $\xi(T^{\star}, m)$ is used, shown in
the left panel of \autoref{fig:T_pseudocrit_chi_power_law_fit}. Here, $\widetilde{T}_c$ denotes the critical temperature
found by minimising the norm of squares of a fit of the form given in \autoref{eq:T_pseudocrit_scaling_law_chi}. In
finding the position of the pseudoccritical temperature $T^{\star}$, a tolerance of $10^{-6}$ was used.

If, however, the length scale $\xi(m, T_c)$ at the actual critical point is used, a much worse fit is obtained, yielding
$\widetilde{T}_c = 2.2691$ and $\nu = 0.90$, shown in the right panel of
\autoref{fig:T_pseudocrit_chi_power_law_fit}.

This signifies the value of $\xi(T^{\star}, m)$ is heavily dependent on
$T^{\star}$, and using the length scale at the actual pseudocritical temperature found somehow offsets the error on its
position.
\todo[inline]{this is unclear.}

Assuming \autoref{eq:xi_propto_kappa}, \autoref{eq:T_pseudocrit_scaling_law_chi} becomes
\begin{equation}
  T^{\star} - T_c \propto m^{-\kappa / \nu},
\end{equation}
which yields \emph{values}, shown in the bottom left panel of \autoref{fig:T_pseudocrit_chi_power_law_fit}.

As a cross check, we can fit instead to scaling relation of the pseudocritical temperature for finite $N$
\begin{equation}
  T^{\star} - T_c \propto N^{-1/\nu},
\end{equation}
yielding \emph{values}. See the bottom right panel of \autoref{fig:T_pseudocrit_chi_power_law_fit}.


\begin{itemize}
  \item xi vs m: 1.96
  \item entropy vs m: 1.96
  \item free energy diff vs m: 1.93
\end{itemize}


\begin{figure}
  \includegraphics[]{support_for_kappa.tikz}
  \caption{Numerical evidence for \autoref{eq:xi_propto_kappa}, \autoref{eq:scaling_laws_order_param_free_energy_kappa},
  \autoref{eq:scaling_law_entropy_kappa}, yielding, from left to right and top to bottom, $\kappa = \{ 1.93, 1.93, 1.90,
  1.93 \}$.}\label{fig:support_for_kappa}
\end{figure}

\begin{figure}
  \includegraphics[]{entropy_vs_correlation_length.tikz}
  \caption{Left panel: numerical fit to \autoref{eq:entropy_vs_correlation_length}, yielding $c = 0.501$. Right panel:
  numerical fit to \autoref{eq:entropy_vs_N}, yielding $c = 0.498$. }\label{fig:entropy_vs_correlation_length}
\end{figure}

\begin{figure}
  \includegraphics[]{T_pseudocrit_chi_power_law_fit.tikz}
  \caption{Left panel: numerical fit to \autoref{eq:T_pseudocrit_scaling_law_chi} with $\xi(T^{\star}(m), m)$ used as
  relevant length scale. Right panel: same fit but using $\xi(T_c, m)$, the correlation length at the exact critical
  point.}\label{fig:T_pseudocrit_chi_power_law_fit}
\end{figure}

% \begin{figure}
%   \includegraphics[]{T_pseudocrit_N_power_law_fit.tikz}
%   \caption{}\label{fig:T_pseudocrit_N_power_law_fit}
% \end{figure}


\begin{itemize}
  \item validate pseudocritical point by matching it to pseudocritical point given by correlation length and
  magnetization (how?)
  \item scaling of pseudocritical point $T^{\star} - T_c \propto m^{-\kappa / \nu}$.
\end{itemize}

\section{To do}

Articles to cite:

\begin{itemize}
\item \cite{pirvu2012matrix}: assumes existence of $\kappa$ and compares finite-size scaling with finite-$m$ scaling
for 1D quantum systems with periodic boundary conditions.
\item \cite{ercolessi2010exact}: proves relation for classical eight vertex model
\end{itemize}

Things to check

\begin{itemize}
  \item does $S(T^{\star}, m) \propto \log \xi(T^{\star}, m)$ hold?
  \item does $S(T^{\star}, N) \propto \log N$ hold better than $S(T_c, N) \propto \log N$?
  \item does fitting $T_c - T^{\star}(N)$ to $N$ give better results than fitting against
  $S(T^{\star}, N)$?
  \item how does $T^{\star}(m)$ from entropy compare against $T^{\star}$ found from max correlation length, or
  vanishing magnetization?
  \item optimize $\kappa$ for scaling of pseudo critical point?
  \item find $T^{\star, N}$ for larger $N$ for ising model.

\end{itemize}

Plots to make:

\begin{itemize}
  \item $T^{\star} - T_c$ vs $N$
  \item $S(T_c, N) \propto \log N$
  \item generally, why should there be a difference between using entropy at critical point
  vs using entropy at pseudocritical point? How does each one scale?
\end{itemize}
