This thesis investigates a numerical approximation method put forth by
Baxter in 1978 \cite{baxter1978variational, baxter1982exactly_ctm,
tsang1979square} based on the corner transfer matrix formulation of the
partition function for two-dimensional classical lattice models.

The method rose to prominence in 1996, under the name \emph{corner
transfer matrix renormalization group} (CTMRG), when Nishino showed
\cite{nishino1996corner} that in the thermodynamic limit, it is equivalent
to the hugely successful density matrix renormalization group (DMRG)
method for one-dimensional quantum systems \cite{white1992density},
discovered a few years earlier by White.

The error in the CTMRG method comes from the fact that the corner transfer
matrices, whose dimension diverges exponentially in the lattice size, have
to be truncated at a maximum dimension $m$ in order to make numerical
manipulation possible. This finite \emph{bond dimension} $m$ (also denoted
by $\chi$ or $D$ in the literature) introduces
finite-size effects, comparable to those observed for systems that are
finite in one or more spatial dimensions. This was already realized by
Nishino \cite{nishino1996numerical}.

The main objective of this thesis is to study how \emph{finite bond
dimension scaling} may be performed with the CTMRG method.

Before the structure of this thesis is laid out, I will first make some
general remarks on statistical mechanics and phase transitions, and on how
the CTMRG method relates to the class of newer methods for simulating
many-body systems that grew out of White's breakthrough, known as
\emph{tensor network algorithms}.

\section{Statistical mechanics and phase transitions}

Statistical mechanics is concerned with describing the average properties
of systems consisting of many particles. Examples of such systems are the
atoms making up a bar magnet, the water molecules in a glass of water, or
virtually any other instance of matter around us.

Matter can arrange itself in various structures with fundamentally
different properties. We call these distinct states of matter
\emph{phases}. When matter changes from one phase to another, we say it
undergoes a phase transition.

Physics has made great strides in understanding these transitions. The
complete history of the field is beyond the scope of this introduction and this
thesis, but the reader may wish to consult \cite{kadanoff2009more,
domb1996critical} to get an idea.

Only as late as 1936, the occurrence of a phase transition
within the framework of statistical physics was established by R. Peierls
\cite{peierls1936on_ising}. He showed that the two-dimensional Ising model
has a non-zero magnetization for sufficiently low temperatures. Since for
high enough temperatures the Ising model loses its magnetization, it
follows that there must be phase transition in between.

The effort to understand the Ising model culminated with Onsager's exact solution in 1944
\cite{onsager1944two_dimensional}, which rigorously established a sharp transition point in the thermodynamic limit.
In the years that followed, more models were exactly solved.
Examples are the spherical model \cite{berlin1952spherical} (1952),
ice-type models \cite{sutherland1967exact} (1967) and the eight-vertex model \cite{baxter1971eight-vertex} (1971).
A pedagogical treatment of these and other models can be found in \cite{baxter1982exactly}.

In the meantime, various approximation methods were developed.
With rapidly improving computers, the method of statistical sampling (Monte Carlo) became one of the most prominent and
its significance has not waned.
Other methods are, among others, series expansions and variational approximations.
A variational approximation put forth by Kramers and Wannier (1941) \cite{kramers1941statistics} was later seen to be
equivalent to Baxter's corner transfer matrix method with $m = 1$.

\subsection{Universality}
One may question the relevance of studying very simple models such as the
Ising model. As it turns out, systems that are at first sight vastly
different may show qualitatively similar behaviour near a phase
transition. For example, exponents that characterize the divergence of
quantities near a transition are conjectured to be independent on
microscopic details of the interactions between particles, but instead
fall into distinct \emph{universality classes}
\cite{griffiths1970dependence, fisher1966quantum}. Thus, studying the very
simplest model may yield universal results.

\section{Baxter's method as a precursor to tensor network methods}

Baxter showed that the optimal truncation of corner transfer matrices
corresponds, in the thermodynamic limit, to a variational optimization of
the row-to-row transfer matrix within a certain subspace, now known as the
subspace of \emph{matrix product states} (MPSs) \cite{baxter1968dimers,
baxter1982exactly_ctm}.

After the success of White's DMRG, which, as Nishino pointed out, is equivalent
to Baxter's method, the underlying matrix-product structure was rediscovered in
the context of one-dimensional quantum systems by Östlund and Rommer
\cite{ostlund1995thermodynamic, rommer1997class}.

It is historically significant but little known that Nightingale, in
a footnote of a 1986 paper \cite{nightingale1986gap}, already made the
remark that \enquote{The generalization [of Baxter's method] to quantum
mechanical systems is straightforward.}

After Östlund and Rommer, it was realized that reformulating White's
algorithm directly in terms of matrix product states provided the
explanation of the algorithm's shortcomings around phase transitions. An
MPS-ansatz fundamentally limits the entropy of the ground state
approximation and since the entropy diverges at a conformally invariant
critical point \cite{calabrese2004entanglement}, DMRG gives inaccurate
results.

This gave rise to other ansätze, formulated in the language of tensor
networks \cite{orus2014practical}, specifically designed to represent
states with a certain amount of entropy. Examples are multi-scale
entanglement renormalization ansatz (MERA) for critical one-dimensional
quantum systems \cite{vidal2007entanglement} and projected entangled-pair
states (PEPS) \cite{verstraete2004renormalization}
for two-dimensional quantum systems.

Other tensor network algorithms, such as infinite time-evolving block
decimation \cite{vidal2007classical} in one dimension and iPEPS (infinite
PEPS) \cite{jordan2008classical} in two dimensions made it possible to
directly approximate quantum systems in the thermodynamic limit.

iTEBD was used to study finite bond dimension scaling (under the slightly
different name of \emph{finite-entropy scaling})
\cite{tagliacozzo2008scaling}. Some theoretical predictions were later
made in \cite{pollmann2009theory}.

The goal of this thesis is twofold: (i) investigate how finite bond
dimension scaling works in the CTMRG algorithm for classical systems,
where we can directly compare it with finite-size scaling, and (ii)
investigate how it compares to other numerical approaches, such as
Monte Carlo, in a difficult scenario.

For (i), I have studied the Ising model, for which all results may be checked
against the exact solution. For (ii), I have studied the clock model with $q
= \{5,
6 \}$ states, which is regarded as difficult numerically and subject to some
controverse.

\section{Structure of this thesis}

It is in the mostly quantum-oriented research field sketched above that the
work for this thesis was done. Therefore, I have chosen to
begin by introducing White's algorithm in its original description
(chapter two), before making the connection to two-dimensional classical
lattices and properly introducing the corner transfer matrix formulation
(chapter three).

In chapter four, the concepts of critical behaviour and finite-size
scaling are introduced and in chapter five these concepts are connected to
the work already done on finite bond dimension (or finite-entropy)
scaling.

Technical details and convergence behaviour of the CTMRG algorithm are reported in chapter six.
It is found that the values of observables may be accurately extrapolated in the chosen convergence criterion of the
algoritm.

Results for the Ising model are presented and analyzed in chapter seven.
With finite-$m$ simulations, it is much easier to reach large system sizes,
but thermodynamic quantities do not grow smoothly as a function of the bond dimension,
as a result of the underlying spectrum of the corner transfer matrix.

Quantities calculated with finite-size simulations do not suffer this unsmooth behaviour.
Results for both methods are comparable, but it is plausible that finite-size data turns out to be more accurate when
corrections to scaling are included.

A numerical analysis of the clock model with $q = \{5, 6\}$ states is given in chapter eight.
The model has a low-temperature ordered phase, a massless phase and a high-temperature disordered phase.
The transition temperatures $T_1$ and $T_2$ are located by extrapolating the positions of pseudocritical temperatures,
assuming the transitions are of the Kosterlitz-Thouless type.
I find slightly contradictory results, based on exact results in a related formulation of the model,
but argue it is plausible that this is due to finite-size effects.
