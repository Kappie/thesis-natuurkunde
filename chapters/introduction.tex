\section{Strongly interacting 1D quantum lattice models}

\subsection{The Hilbert space is enormous}

Consider the problem of numerically finding the ground state $\ket{\Psi_0}$ of the
$N$-site 1-dimensional transverse-field Ising-model, given by

\begin{equation} 
  H = -J \sum_{i = 1}^{N} \sigma_{i}^{z}\sigma_{i+1}^{z}
  - h \sum_{i=1}^{N} \sigma_{i}^{x}
\end{equation}

The underlying Hilbert space of the system is a tensor product of the
local Hilbert spaces $\mathcal{H}_{\text{site}}$, which are spanned by the
states $\{\ket{\uparrow}, \ket{\downarrow}\}$. Thus, a general state of the system is a unit vector in
a $2^N$-dimensional space.

\begin{equation}
  \ket{\Psi} = \sum_{\sigma_1, \sigma_2, \ldots \in \{\ket{\uparrow}, \ket{\downarrow}\}} c_{\sigma_1, \sigma_2, \ldots, \sigma_N} \ket{\sigma_1} \otimes \ket{\sigma_2} \otimes \ldots \otimes \ket{\sigma_N}
\end{equation}

So, for a system with 1000 particles, the dimensionality of the Hilbert
space comes in at about $10^{301}$, some 220 orders of magnitude larger than
the number of atoms in the observable universe. As it turns out, nature is
very well described by Hamiltonians that are local -- that do not contain
interactions between an arbitrary number of bodies. And for these
Hamiltonians, only an exponentially small subset of states can be explored
in the lifetime of the universe \cite{poulin2011quantum}. That is, only
exponentially few states are physical. \todo[inline]{Expand a bit. Refer
back to problem of finding ground state.}
