\section{Abstract}
We present results of scaling in bond dimension and system size with the CTMRG algorithm for the five- and six-state
clock model.

\section{Introduction}
In the field of phase transitions and critical phenomena, the two-dimensional topological phase transition discovered by
Kosterlitz and Thouless \cite{kosterlitz1973ordering, kosterlitz1974critical} received much attention. This phase
transition is characterized not by an order parameter which indicates a breaking of symmetry, but by the proliferation
of topological defects.

The XY model consists of planar rotors on the square lattice. It exhibits the Kosterlitz-Thouless (BK) phase transition
and by the Mermin-Wagner-Hohenberg theorem the symmetry of the ground state is broken for all temperatures, due to
the $O(2)$ (planar rotational) symmetry of the potential \cite{mermin1966absence, hohenberg1967existence}.

The $q$-state clock model possesses the discrete $\mathbb{Z}_q$ symmetry and is an interpolation between the Ising
model, which corresponds to $q = 2$, and the XY model, which corresponds to $q \to \infty$. Its energy function is given
by
\begin{equation}
  H_q = -\sum_{\langle i j \rangle} \cos(\theta_i - \theta_j).
\end{equation}

It has been proven that for high enough $q$, this model indeed exhibits a Kosterlitz-Thouless transition
\cite{frohlich1981kosterlitz}.

In the Villain formulation of the potential \cite{villain1975theory}, it has been proven that the model has two phase
transitions for $q \geq 5$: a symmetry broken phase for $T < T_1$, an intermediate phase with power law decay of the
correlation function, and a high-temperature phase with exponential decay of the correlation function for $T > T_2$. The
transition at $T_2$ is a BK-transition, and for a broad range of temperatures, the thermodynamic behaviour becomes
identical to the XY model for high enough $q$ \cite{lapilli2006universality}.


\section{Previous work}

\section{Results}
