\chapterprecishere{We present results of scaling in bond dimension and system size with the CTMRG algorithm for the
five- and six-state clock model.}

\todo[inline]{Not finished yet.}

\section{Introduction}
In the field of phase transitions and critical phenomena, the two-dimensional topological phase transition discovered by
Kosterlitz and Thouless \cite{kosterlitz1973ordering, kosterlitz1974critical} receives much attention. This phase
transition is characterized not by an order parameter which indicates a breaking of symmetry, but by the proliferation
of topological defects.

In the low-temperature phase, the two-point correlation functions decay with a power-law with varying
exponent $\eta(T)$. At the transition, the correlation length diverges as
\begin{equation}\label{eq:corr_length_divergence_kt}
  \xi \propto \exp(A |T - T_c|^{-1/2}),
\end{equation}
with $A$ a non-universal constant. Above the transition, the two-point correlators decay exponentially.

The XY model consists of planar rotors on the square lattice. It exhibits the Kosterlitz-Thouless (KT) phase transition
and by the Mermin-Wagner-Hohenberg theorem the symmetry of the ground state is broken for all temperatures, due to
the $O(2)$ (planar rotational) symmetry of the potential \cite{mermin1966absence, hohenberg1967existence}.

The $q$-state clock model possesses the discrete $\mathbb{Z}_q$ symmetry and is an interpolation between the Ising
model, which corresponds to $q = 2$, and the XY model, which corresponds to $q \to \infty$. Its energy function is given
by
\begin{equation}\label{eq:hamiltonian_clock_model}
  H_q = -\sum_{\langle i j \rangle} \cos(\theta_i - \theta_j),
\end{equation}
with the spins taking the values
\begin{equation}
  \theta = \frac{2 \pi n}{q} \qquad n \in \{ 0, \dots, q-1 \}.
\end{equation}

It has been proven that for high enough $q$, this model indeed exhibits a Kosterlitz-Thouless transition
\cite{frohlich1981kosterlitz}. Furthermore, it has been proven that for $q \geq 5$, a general model with $\mathbb{Z}_q$
symmetry (of which \autoref{eq:hamiltonian_clock_model} is a special case) has three phases: a symmetry broken phase for
$T < T_1$, an intermediate phase with power law decay of the correlation function, and a high-temperature phase with
exponential decay of the correlation function for $T > T_2$ \cite{cardy1980general}.

In the Villain formulation of the potential \cite{villain1975theory}, it has been proven that the transition at $T_2$ is
a BK-transition \cite{jose1977renormalization}, and numerical results suggest that for a broad range of temperatures,
the thermodynamic behaviour becomes identical to the XY model for high enough $q$ \cite{lapilli2006universality}.

Furthermore, in the Villain formulation it is known that \cite{elitzur1979phase, nienhuis1984critical}
\begin{equation}\label{eq:eta_villain}
  \eta(T_1) =\frac{4}{q^2}, \qquad \eta(T_2) = \frac{1}{4},
\end{equation}
where $\frac{\eta}{2} = \frac{\beta}{\nu}$, the magnetization exponent in the finite-size regime.

For the cosine model in \autoref{eq:hamiltonian_clock_model}, the value $q_c$ for which it first exhibits a
BK-transition is not precisely known.
There is some disagreement about whether the cases $q = 5, 6$ exhibit KT-type transitions (see previous numerical
results below).

In our simulations we will focus on the cases $q = 5, 6$, to (i) study the nature of the phase transition from a new
perspective and (ii) compare the accuracy of finite-$m$ and finite-$N$ scaling within the CTMRG
method to other established numerical methods.

We briefly summarise previous numerical results, then present results obtained with the CTMRG algorithm.

\section{Previous numerical results}
\subsection{The $q = 5$ clock model}

The general consensus is that the two transitions of the $q = 5$ clock model with cosine potential are of the KT-type,
though there are no rigorous results.
It is also assumed that the critical indices are the same as those in the Villain formulation.

The disagreement about the nature of the phase transitions
stems from numerical results for the helicity modulus
\cite{fisher1973helicity}.

Most notably, Baek and Minnhagen \cite{baek2010non} claim that since the helicity modulus does
not vanish in the high-temperature phase, the upper transition is not of the KT-type.

It was shown by Kumano et al.
in \cite{kumano2013response}, however, that the definition used by Baek and Minnhagen is not suitable for systems with a
discrete symmetry.
The correct discrete definition yields the expected result, namely that the helicity modulus does vanish and the
three-phase KT-picture holds.

The conclusion of Kumano et al, which was obtained by a Monte Carlo study,
was verified by Chatelain \cite{chatelain2014dmrg} using the TMRG algorithm \cite{nishino1995density} (see also
\autoref{sec:tmrg}).
Chatelain also found that the critical indices match those of the Villain model (\autoref{eq:eta_villain}),
implying the cosine model is in the same universality class as the Villain model.

After the rebuttal by Kumano et al., Baek et al.
published another work \cite{baek2013residual} in which they again use the (in the eyes of Kumano et al.) wrong
definition of the helicity modulus, yet calculated in a different way.
Again they conclude the transition is not of the KT-type.

Meanwhile, Borisenko et al.
\cite{borisenko2011numerical} carried out a very detailed Monte Carlo study confirming the KT-picture,
using Binder-cumulants to find the critical points and the magnetization and susceptability to find the critical
indices.

Brito et al.
\cite{brito2010twodimensional} conclude from a Monte Carlo study that while the transition is of KT-type,
the resolution of their numerical method is not high enough to distinguish between $T_1$ and $T_2$.

\autoref{table:q5_previous_results} shows the results for the transition temperatures found by other authors.

\subsection{The $q = 6$ clock model}

Here, there is overwhelming consensus that both transitions are of the KT-type.
The only exceptions are Lapilli et al. \cite{lapilli2006universality} and Hwang \cite{hwang2009six}.

Lapilli et al. use the incorrect definition of the helicity modulus.

Hwang asserts that the transition is not of KT-type because the data,
which was obtained from systems of rather small size ($L \times L$-systems with $L = 20,
\dots, 28$), also agrees with a power-law divergence of the correlation length. We will get back to this point.

The previous results for the transition temperatures are listed in \autoref{table:q6_previous_results}.
For an overview that goes further back, see \cite{krvcmar2016phase}.

We note that \cite{tomita2002probability, brito2010twodimensional, kumano2013response} use Monte Carlo methods,
while \cite{krvcmar2016phase} uses the CTMRG algorithm (combined with finite-size scaling,
but not with finite-$m$ scaling).

\begin{table}[]
\centering
\begin{tabular}{@{}lll@{}}
\toprule
 & $T_1$ & $T_2$ \\ \midrule
Brito et al.\tablefootnote{These authors found $T_1 > T_2$, which is not an error in the text, but due to the low resolution of the methods used.} (2010) \cite{brito2010twodimensional} & 0.91 & 0.90 \\
Borisenko et al. (2011) \cite{borisenko2011numerical} & 0.9056 & 0.9432 \\
Kumano et al. (2013) \cite{kumano2013response} & 0.908  & 0.944  \\
Chatelain (2014) \cite{chatelain2014dmrg} & 0.914 & 0.945  \\ \bottomrule
\end{tabular}
\caption{Previous results for the transition temperatures for $q = 5$.}
\label{table:q5_previous_results}
\end{table}

\begin{table}[]
\centering
\begin{tabular}{@{}lll@{}}
\toprule
 & $T_1$ & $T_2$ \\ \midrule
Tomita and Okabe (2002) \cite{tomita2002probability} & 0.7014 & 0.9008 \\
Hwang\tablefootnote{To obtain these values, the author assumed an algebraic divergence of the correlation length.} (2009) \cite{hwang2009six} & 0.632 & 0.997 \\
Brito et al. (2010) \cite{brito2010twodimensional} & 0.68 & 0.90 \\
Kumano et al. (2013) \cite{kumano2013response} & 0.700 & 0.904 \\
Krčmár et al. (2016) \cite{krvcmar2016phase} & 0.70 & 0.88 \\ \bottomrule
\end{tabular}
\caption{Previous results for the transition temperatures for $q = 6$.}
\label{table:q6_previous_results}
\end{table}

\section{Spectrum of the corner transfer matrix}

In order to get an idea of the accuracy that we might expect, we have plotted the spectrum of the $q = \{5,
6\}$ clock models in \autoref{fig:spectrum_ctm_clock}.

It is clear that the spectra of both clock models fall off at about the same pace,
if we compare points in the ordered, massless and disordered phase.
The $q = 6$ clock model has a slightly more degenerate spectrum, as might be expected from its larger symmetry group,
but there is no clear pattern.

As compared to the Ising model (see \autoref{sec:spectrum_of_ctm}), the spectra of the $q = \{5,
6\}$ clock models fall off much more slowly\footnote{For the calculation of the spectrum of the Ising model in this
work, a bond dimension of $m = 250$ was used, as opposed to $m = 100$ for the clock model.
This means that, in small part, the slower decay of the spectrum is due to the normalization $\tr A^4 = 1$.
But this does not change the general picture that the spectra of the $q = \{5,
6\}$ clock models decay more slowly.}.
This implies that a much larger bond dimension is needed to obtain the same accuracy for quantities in the thermodynamic
limit.

\begin{figure}
  \includegraphics[]{spectrum_ctm_clock.tikz}
  \caption{First part of the spectrum of $A$ with fixed boundary,
  calculated with $m = 100$ and a convergence threshold of $10^{-8}$,
  at temperatures corresponding to the ordered phase, approximate midpoint of the massless phase and disordered phase,
  respectively.}\label{fig:spectrum_ctm_clock}
\end{figure}

\section{Magnetization}

For the clock model, we define the magnetization per site as
\begin{equation}\label{eq:magnetization_clock_model}
  M = \langle \cos \theta_0 \rangle,
\end{equation}
where $\theta_0$ is the central spin.

This quantity can be computed in the same way as for the Ising model (see \autoref{sec:magnetization_per_site}) by
generalizing the tensor $b_{i j k l}$ to
\begin{equation}
  b_{i j k l} = \sum_{n \in \{ 0, \dots, q-1 \}} \cos \left( \frac{2\pi n}{q} \right) \delta_{n i j k l}.
\end{equation}

\section{Classical analogue to the entanglement entropy}

The classical analogue to the half-chain entanglement entropy $S$ is defined in \autoref{sec:analogy_to_entropy}.
Its definition remains valid.

In the limit $T \to \infty$, for both a fixed and free boundary, we have
\begin{equation}
  S(T \to \infty) = 0.
\end{equation}

To see this, consider that all $2^2N$ configurations on the inner edges of the $N \times N$ quadrant represented by the
corner transfer matrix are equally likely in this limit, hence
\begin{equation}
  A_{i j} = \frac{1}{2^{2N}}
\end{equation}
which is matrix with one eigenvalue of 1 and the others 0\footnote{One can also make the argument that the corresponding
quantum state tends to a product state in the limit $T \to 0$,
yielding the same conclusion.}.

In the limit $T \to 0$, there is only one non-zero matrix element in the case of a fixed boundary (namely all inner
spins having the same value as the outer boundary), and $q$ equally likely configurations in the case of a free
boundary, yielding
\begin{align*}
  S^{\text{fixed}}(T = 0) &= 0, \\
  S^{\text{free}}(T = 0)  &= \log q.
\end{align*}

For a fixed boundary, the point of maximum entropy approaches the massless phase from the high-temperature region,
hence tending towards $T_2$.
In contrast, the point of maximum entropy approaches $T_1$ for systems with a free boundary.
\todo[inline]{Maybe make intuitive why?}

\autoref{figure:entropy_and_order_param_vs_T_clock5} shows these quantities for $q = 5$ for systems with a fixed
boundary.
\todo[inline]{Finish this paragraph when results of $q = 6$ are in.}

\begin{figure}
  \includegraphics[]{entropy_and_order_param_vs_T_clock5.tikz}
  \caption{Classical analogue to half chain entanglement entropy (\autoref{sec:analogy_to_entropy}) and magnetization
  (\autoref{eq:magnetization_clock_model}) computed for systems with a fixed boundary.
  Simulations were done with a convergence threshold of $2.5 \times
  10^{-8}$.}\label{figure:entropy_and_order_param_vs_T_clock5}
\end{figure}

\section{Transition temperatures}

Since we expect an essential singularity of the form in \autoref{eq:corr_length_divergence_kt} for both transition
temperatures, for finite systems we have
\begin{equation}\label{eq:kt_t_pseudocrit}
  N = a \exp \left( b |T^{\star}(N) - T_c|^{-1/2}  \right),
\end{equation}
where $N$ is an effective finite length scale of the system and $a$ and $b$ are non-universal constants.
$N$ is the system size in the case of finite-size scaling and a length scale derived from the bond dimension $m$ in the
case of finite-$m$ scaling.

We define $T^{\star}(N)$ as the point of maximum entanglement entropy, as discussed in
\autoref{sec:locating_critical_point_entanglement}.

Inverting \autoref{eq:kt_t_pseudocrit} gives the following relations for the pseudocritical transition temperatures
\begin{align}
  T^{\star}_{1}(N) &= -\frac{\alpha_1}{(\log \beta_1 N)^2} + T_1 \\
  T^{\star}_{2}(N) &= \frac{\alpha_2}{(\log \beta_2 N)^2} + T_2
\end{align}
where $\alpha = b^2$ and $\beta = 1/a$ (we drop the subscripts denoting the transition).

For convenience, we define the scaled length variable
\begin{equation}
  \ell = (\log \beta N)^2,
\end{equation}
such that
\begin{equation}
  T^{\star}(N) - T_c \propto \frac{1}{\ell}.
\end{equation}

\subsection{Transition from the ordered to the massless phase $T_1$}

\begin{figure}
  \includegraphics[]{t1_fit.tikz}
  \caption{For $q = 5$, we find $T_1 = 0.917$, fitting the final 6 points.
  Max truncation error = $10^{-6}$, TolX = $10^{-6}$}\label{figure:t1_fit}
\end{figure}

\subsection{Transition from the massless to the disordered phase $T_2$}

\begin{figure}
  \includegraphics[]{t2_fit_q5.tikz}
  \caption{}\label{figure:t2_fit_q5}
\end{figure}


\section{Central charge of the massless phase}

\section{Varying exponent for the magnetization}
