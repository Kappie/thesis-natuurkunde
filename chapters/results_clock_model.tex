\chapterprecishere{We present results of scaling in bond dimension and system size with the CTMRG algorithm for the
five- and six-state clock model.}

\todo[inline]{Not finished yet.}

\section{Introduction}
In the field of phase transitions and critical phenomena, the two-dimensional topological phase transition discovered by
Kosterlitz and Thouless \cite{kosterlitz1973ordering, kosterlitz1974critical} receives much attention. This phase
transition is characterized not by an order parameter which indicates a breaking of symmetry, but by the proliferation
of topological defects.

In the low-temperature phase, the two-point correlation functions decay with a power-law with varying
exponent $\eta(T)$. At the transition, the correlation length diverges as
\begin{equation}
  \xi \propto \exp(A |T - T_c|^{-1/2}),
\end{equation}
with $A$ a non-universal constant. Above the transition, the two-point correlators decay exponentially.

The XY model consists of planar rotors on the square lattice. It exhibits the Kosterlitz-Thouless (KT) phase transition
and by the Mermin-Wagner-Hohenberg theorem the symmetry of the ground state is broken for all temperatures, due to
the $O(2)$ (planar rotational) symmetry of the potential \cite{mermin1966absence, hohenberg1967existence}.

The $q$-state clock model possesses the discrete $\mathbb{Z}_q$ symmetry and is an interpolation between the Ising
model, which corresponds to $q = 2$, and the XY model, which corresponds to $q \to \infty$. Its energy function is given
by
\begin{equation}\label{eq:hamiltonian_clock_model}
  H_q = -\sum_{\langle i j \rangle} \cos(\theta_i - \theta_j),
\end{equation}
with the spins taking the values
\begin{equation}
  \theta = \frac{2 \pi n}{q} \qquad n \in \{ 0, \dots, q-1 \}.
\end{equation}

It has been proven that for high enough $q$, this model indeed exhibits a Kosterlitz-Thouless transition
\cite{frohlich1981kosterlitz}. Furthermore, it has been proven that for $q \geq 5$, a general model with $\mathbb{Z}_q$
symmetry (of which \autoref{eq:hamiltonian_clock_model} is a special case) has three phases: a symmetry broken phase for
$T < T_1$, an intermediate phase with power law decay of the correlation function, and a high-temperature phase with
exponential decay of the correlation function for $T > T_2$ \cite{cardy1980general}.

In the Villain formulation of the potential \cite{villain1975theory}, it has been proven that the transition at $T_2$ is
a BK-transition \cite{jose1977renormalization}, and numerical results suggest that for a broad range of temperatures,
the thermodynamic behaviour becomes identical to the XY model for high enough $q$ \cite{lapilli2006universality}.

Furthermore, in the Villain formulation it is known that \cite{elitzur1979phase, nienhuis1984critical}
\begin{equation}\label{eq:eta_villain}
  \eta(T_1) =\frac{4}{q^2}, \qquad \eta(T_2) = \frac{1}{4},
\end{equation}
where $\frac{\eta}{2} = \frac{\beta}{\nu}$, the magnetization exponent in the finite-size regime.

For the cosine model in \autoref{eq:hamiltonian_clock_model}, the value $q_c$ for which it first exhibits a
BK-transition and critical indices are not precisely known, though it is generally accepted that $q = 6$ exhibits two
BK-transitions. The $q = 5$ case is contested (see previous numerical results below).

In our simulations we will focus on the cases $q = 5, 6$, to (i) study the nature of the phase transition from a new
perspective in the case of $q = 5$ and (ii) compare the accuracy of finite-$m$ and finite-$N$ scaling within the CTMRG
method to other established methods for both cases.

We briefly summarise previous numerical results, then present results from the CTMRG algorithm.

\section{Previous numerical results}
\subsection{The $q = 5$ clock model}

The general consensus is that the two transitions of the $q = 5$ clock model with cosine potential are of the KT-type,
though there are no rigorous results.
It is also assumed that the critical indices are the same as those in the Villain formulation.

The disagreement about the nature of the phase transitions
stems from numerical results for the helicity modulus
\cite{fisher1973helicity}.

Most notably, Baek and Minnhagen \cite{baek2010non} claim that since the helicity modulus does
not vanish in the high-temperature phase, the upper transition is not of the KT-type.

It was shown by Kumano et al.
in \cite{kumano2013response}, however, that the definition used by Baek and Minnhagen is not suitable for systems with a
discrete symmetry.
The correct discrete definition yields the expected result, namely that the helicity modulus does vanish and the
three-phase KT-picture holds.

The conclusion of Kumano et al, which was obtained by a Monte Carlo study,
was verified by Chatelain \cite{chatelain2014dmrg} using the TMRG algorithm \cite{nishino1995density} (see also
\autoref{sec:tmrg}).
Chatelain also found that the critical indices match those of the Villain model (\autoref{eq:eta_villain}),
implying the cosine model is in the same universality class as the Villain model.

After the rebuttal by Kumano et al., Baek et al.
published another work \cite{baek2013residual} in which they again use the (in the eyes of Kumano et al.) wrong
definition of the helicity modulus, yet calculated in a different way.
Again they conclude the transition is not of the KT-type.

Meanwhile, Borisenko et al.
\cite{borisenko2011numerical} carried out a very detailed Monte Carlo study confirming the KT-picture,
using Binder-cumulants to find the critical points and the magnetization and susceptability to find the critical
indices.

Brito et al.
\cite{brito2010twodimensional} conclude from a Monte Carlo study that while the transition is of KT-type,
the resolution of their numerical method is not high enough to distinguish between $T_1$ and $T_2$.

\begin{table}[]
\centering
\begin{tabular}{@{}lll@{}}
\toprule
 & $T_1$ & $T_2$ \\ \midrule
Brito et al. (2010) & 0.91(2) & 0.90(2) \\
Borisenko et al. (2011) &  &  \\
Kumano et al. (2013) &  &  \\
Chatelain (2014) &  &  \\ \bottomrule
\end{tabular}
\caption{My caption}
\label{my-label}
\end{table}

\todo[inline]{List critical points found by other authors}

\subsection{The $q = 6$ clock model}


For $q = 5$:
\begin{itemize}
  \item \cite{chatelain2014dmrg} 2014 claims BK-transition for $q = 5$ by discrete helicity modulus using TMRG.
  \item \cite{kumano2013response} 2013 claims BK-transition for $q = 5$ by discrete helicity modelus using Monte Carlo.
  Shows infinitesimal helicity modulus does not vanish for finite $q$.
  \item \cite{borisenko2011numerical} 2011 most comprehensive study? Claims BK-transition for $q = 5$ by calculating
  Binder cumulants + stuff I don't understand using Monte Carlo.
  \item \cite{borisenko2012phase} 2012 don't understand. Some numerical and theoretical validations for certain
  assumptions for high $q$.
  \item \cite{brito2010twodimensional} 2010 claims BK-transition for $q = 5$ based on Monte Carlo simulation and mapping
  to solid-on-solid growth model.
  \item \cite{baek2010non} 2010 claims that $q = 5$ exhibits non-KT-transition, based on the fact that their (wrong)
  definition of the helicity modulus does not vanish.
  \item \cite{baek2013residual} 2013 claims that $q = 5$ exhibits weaker cousin of KT-transition, based on things I do
  not fully understand, but among others on the correct, discrete definition of the helicity modulus (true?).
  \item \cite{kim2017partition} 2017 uses partition function zeros (don't understand that concept) to show that the
  behaviour of zeros of the $q = 5$ model significantly departs from the behaviour of $q \geq 6$.
\end{itemize}

For $q = 6$:
\begin{itemize}
  \item \cite{krvcmar2016phase} 2016 KT-transition from entropy.
  \item \cite{baek2013residual}
  \item \cite{kumano2013response}
\end{itemize}

\section{Results}

What are the results that really should be in there? Focus on $q = 5, 6$.

\begin{itemize}
  \item show that scaling relation holds for magnetization and correlation length and calculate exponent. Should vary as
  $\eta \propto T$ in the massless phase and probably (?) $\eta(T_2) = 1/4$ and $\eta(T_1) \propto 1/q^2$ or something.
  \item show that $c = 1$ massless phase exists.
  \item show that $T^{\star} - T_c$ is compatible with correlation length at essential singularity (but what about
  logarithmic corrections...?)
  \item data collapse in magnetization around Kosterlitz-transition
\end{itemize}
