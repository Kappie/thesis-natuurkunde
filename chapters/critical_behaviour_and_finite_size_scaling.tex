\section{Abstract}

In this chapter, we introduce the central concepts in critical phenomena
and finite-size scaling.

We follow the excellent review by Barber \cite{barber1983finite} and
chapter five of Cardy's book \cite{cardy1996scaling}.

\section{Introduction}

\section{Phase transitions}
When matter exhibits a sudden change in behaviour, characterized by a sudden change in one or more thermodynamic
quantities, we say it undergoes a \emph{phase transition}.
A quantity that signifies this change is called an \emph{order parameter},
that can take vastly different forms across systems and transitions.
For example, for the transition of a ferromagnet, the order parameter is the net magnetization of the system,
while for a percolation transition, it is the size of the largest connected graph.

For a historical account of the classification of phase transitions,
see \cite{jaeger1998ehrenfest}.
At the present time, we distinguish between two different types \cite{kadanoff2009more}.

When some thermodynamic quantity changes discontinuously, i.e.
shows a jump, we call the transition \emph{first order}.
In contrast, during a \emph{continuous} phase transition a variable undergoes change continuously.
The point at which a continuous phase transition occurs, is called the critical point.

The two-dimensional Ising model (\emph{ref here}) in a magnetic field shows both types of transition.
At zero magnetic field and $T = T_c = 1 / (\log(1 + \sqrt{2}))$, the magnetization changes from zero for $T > T_c$ to
a finite value for $T < T_c$ in a continuous manner.

Below the critical temperature $T_c$, when the magnetic field $h$ tends to zero from $h > 0$,
the magnetization tends to a positive value.
Conversely, when the magnetic field tends to zero from $h < 0$,
the magnetization tends to a negative value.
Thus, across the region $h = 0, T < T_c$ the system undergoes a first-order phase transition.

\subsection{Finite systems}

We will now argue that a phase transition cannot occur in a finite system,
but only happens when the number of particles tends to infinity.

Because thermodynamic quantities are averages over all possible
microstates of a system, those quantities are completely defined in terms
of the system's partition function, or equivalently its free energy.

Since in a finite system, the partition function is a finite sum of
exponentials, it is analytic (infinitely differentiable). Hence, thermodynamic
quantities cannot show true discontinuities and the phase transitions
described in the above section do not occur.

\todo[inline]{pictures?}

\section{Critical behaviour}
We will now focus our attention on continuous phase transitions, more specifically the one that occurs in the
two-dimensional Ising model.
Before we discuss the behaviour of the free energy around the critical point,
we briefly summarize how the thermodynamic limit is approached far away from it.
Here, we largely follow \cite{barber1983finite}.

We assume that the free energy per site in the thermodynamic limit
\begin{equation}
  f_{\infty}(T) = \lim_{N \to \infty} \frac{F(T, N)}{N}
\end{equation}
exists, and is not dependent on boundary conditions.
By definition, it is not analytic in a region around the critical point.

Outside that region, however, we can write
\begin{equation}\label{eq:free_energy_finite_system_outside_critical_region}
  F(T, N) = N f_{\infty}(T) + o(N),
\end{equation}
where corrections $f(N)$ of $o(N)$ (little-o of $N$) mean that
\begin{equation}
  \lim_{N \to \infty} \frac{f(N)}{N} = 0.
\end{equation}
These corrections, of course, do depend on boundary conditions.

\autoref{eq:free_energy_finite_system_outside_critical_region} is valid only outside the critical region precisely
because $F(T, N)$ is analytic \emph{everywhere}, and $f_{\infty}(T)$ is only analytic away from the critical point.

The behaviour of $F(T, N)$ (and hence, all thermodynamic quantities) at criticality is approached is described by
\emph{finite-size scaling}.

\subsection{Finite-size scaling}

\autoref{fig:order_parameter_finite_N_exact} shows the behaviour of the order parameter obtained by exact
diagonalization of the partition function of small lattices.
It is clear that far from the critical point, the order parameter is essentially not dependent on system size,
while in critical region there are significant deviations from the thermodynamic behaviour.

One can now define two characteristic temperatures \cite{fisher1967interfacial,
barber1983finite}.
The first being the cross-over temperature $T_X$ at which finite-size effects become important,
which is predicted to scale as
\begin{equation}
  |T_X - T_c| \propto N^{-\theta}.
\end{equation}
$\theta$ is called the cross-over or rounding exponent.

The second characteristic temperature is the pseudocritical temperature, denoted by $T^{\star}$.
It can be defined in several ways, one being the point where the order parameter becomes almost zero (in which case it is the same as the cross-over temperature), or the point where the heat capacity
\begin{equation}
  C = T^2 \frac{\partial^2 F}{\partial T^2}
\end{equation}
reaches its maximum.
$T^{\star}$ can be regarded as the point where the finite system in some sense comes closest to undergoing a transition.
Generally $T^{\star}$ will not equal $T_X$, if only because $T^{\star}$ depends on boundary conditions:
periodic or fixed boundary conditions will nudge the system into an ordered state,
therefore $T^{\star} > T_c$.
Free boundary conditions will cause the system to favor disorder and the pseudocritical temperature to be lowered.

In any case, it is predicted that
\begin{equation}
  |T^{\star} - T_c| \propto N^{-\lambda}.
\end{equation}

It is generally accepted that \cite{barber1983finite}
\begin{equation}
  \lambda = \theta.
\end{equation}

Furthermore, if one assumes that finite-size effects become important once the correlation length of the system becomes of order of the system size, i.e. \cite{fisher1967interfacial}
\begin{equation}
  \xi(T_X(N)) \propto N,
\end{equation}
then the correlation length exponent $\nu$, given by
\begin{equation}
  \xi(T) \propto |T - T_c|^{-\nu}
\end{equation}
can be related to $\theta$ as
\begin{equation}
  \theta = \frac{1}{\nu}.
\end{equation}


\begin{figure}
  \includegraphics[]{order_parameter_finite_N_exact.tikz}
  \caption{The magnetization of the central spin for small lattices with boundary spins fixed to
  $+1$. The black line is the exact solution in the thermodynamic limit.}\label{fig:order_parameter_finite_N_exact}
\end{figure}




\section{Effective length scale related to the bond dimension $m$}

To cite

\begin{itemize}
  \item \cite{fisher1972scaling} original physical derivation of finite-size scaling
  \item \cite{fisher1967interfacial} rounding and displacement exponents
  \item \cite{kadanoff2009more} Basic (philosophical) introduction to phase transitions
  \item \cite{jaeger1998ehrenfest} Historical account of classification of phase transitions
\end{itemize}

Todo
\begin{itemize}
  \item finite-size scaling ansatz, basic scaling laws, data collapse, etc.
  \item renormalization group derivation? (in appendix?)
\end{itemize}
