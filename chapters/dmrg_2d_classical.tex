\section{Partition functions of classical lattices}
The central quantity in equilibrium statistical mechanics is the partition
function $Z$, which, for a discrete system such as a lattice, is defined as

\begin{equation}
  Z = \sum_{s} \exp{(-\beta H(s))}
\end{equation}

where the sum is over all microstates $s$, $H$ is the energy function, $J > 0$ and
$\beta = T^{-1}$, the inverse temperature.

\section{Transfer matrices of lattice models}

\subsection{1D Ising model}

\todo[inline]{refer to Ising, talk a bit about model (magnetism etc).}

Consider the 1D zero-field Ising model, defined by the energy function

\begin{equation}\label{ising_energy_function}
  H(\sigma) = -J \sum_{\langle i j \rangle} \sigma_i \sigma_j
\end{equation}

Here, we sum over nearest neighbors $\langle i j \rangle$ and the spins
$\sigma_i$ take the values $\pm 1$. Assume, for the moment, that the chain
consists of $N$ spins, and apply periodic boundary conditions.
The partition function of this system is given by

\begin{equation}
  Z_{N} = \sum_{\sigma_1, \dotsc, \sigma_N \in \{-1, 1\}} \exp (-\beta H(\sigma))
\end{equation}

Exploiting the local nature of the interaction between spins, we can write

\begin{equation}
  Z_{N} = \sum_{\sigma_1, \cdots, \sigma_N \in \{-1, 1\}} \prod_{\langle i, j \rangle} e^{K\sigma_i \sigma_j}
\end{equation}

where we defined $K \equiv \beta J$. Now, we can define
the $2 \times 2$ matrix $T_{\sigma \sigma'} = \exp(K \sigma \sigma')$ to get

\begin{equation}
  Z_N = \sum_{\sigma_1, \cdots, \sigma_N} T_{\sigma_1 \sigma_2} \dotsm T_{\sigma_N \sigma_1} = \tr T^N
\end{equation}

$T$ is called the transfer matrix. Since $T^N = P D^N P^{-1}$, where $P$
consists of the eigenvectors of $T$, and by the cyclic property of the trace, we
have

\begin{equation}
  Z = \lambda_{1}^{N} + \lambda_{2}^{N}
\end{equation}

Thus, we have reduced the problem of finding the partition function to an
eigenvalue problem, which is quite easy in this case. Also, note that in the thermodynamic limit $N \to \infty$

\begin{equation}
  Z = \lim_{N \to \infty} \lambda_{1}^{N}
\end{equation}

where $\lambda_1$ is the non-degenerate largest eigenvalue (in absolute value) of $T$.

\subsection{2D Ising model}

Next, we treat the two-dimensional, square-lattice Ising model. In two
dimensions, the energy function is still written as in
\eqref{ising_energy_function}, but now every lattice site has four neighbors.
Let $N$ be the number of columns and $l$ be the number of rows of the lattice, and assume 
$l \gg N$. In the vertical direction, we apply periodic boundary conditions, as in the one-dimensional case.
In the horizontal direction, we keep an open boundary.

Similarly to the 1D case, the partition function can be written as 

\begin{equation}
  \sum_{\bm{\sigma}} \prod_{\langle i, j, k, l \rangle} W(\sigma_i, \sigma_j, \sigma_k, \sigma_l)
\end{equation}

where the product runs over all groups of four spins sharing the same face. The Boltzmann weight of such a face is given by

\begin{equation}
  W(\sigma_i, \sigma_j, \sigma_k, \sigma_l) = \exp \left\{ \frac{K}{2} (\sigma_i \sigma_j + \sigma_j \sigma_k + \sigma_k \sigma_l + \sigma_l \sigma_i) \right\}
\end{equation}

We can express the Boltzmann weight of a configuration of the whole lattice as a product of the Boltzmann weights of the rows

\begin{equation}
  Z = \sum_{\bm{\sigma}} \prod_{r = 1}^{l - 1} \prod_{c = 1}^{N - 1} W(\sigma_{r, c}, \sigma_{r, c + 1}, \sigma_{r + 1, c}, \sigma_{r + 1, c + 1})
\end{equation}

Now, we can generalize the definition of the transfer matrix to two dimensions, by defining it as the Boltzmann weight of an entire row

\begin{equation} 
  T_{N}(\bm{\sigma} | \bm{\sigma'}) = \prod_{\text{lol}}
\end{equation}









\section{DMRG applied to 2D classical lattice models}





\todo[inline]{More about connection between 1D quantum and 2D classical}
\todo[inline]{Pictures}

Nishino \cite{nishino1995density} was the first one to apply DMRG in its modern
form to the problem of finding the partition function $Z$ of 2D classical
lattices. Consider the zero-field Ising model on a square lattice, with energy function

