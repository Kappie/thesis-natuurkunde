\section{Partition functions of classical lattices}
The central quantity in equilibrium statistical mechanics is the partition
function $Z$, which, for a discrete system such as a lattice, is defined as
\begin{equation}
  Z = \sum_{s} \exp{(-\beta H(s))}
\end{equation}
where the sum is over all microstates $s$, $H$ is the energy function, and
$\beta = T^{-1}$ the inverse temperature.

\section{Transfer matrices of lattice models}

\subsection{1D Ising model}

\todo[inline]{refer to Ising, talk a bit about model (magnetism etc).}

Consider the 1D zero-field ferromagnetic Ising model \cite{ising1925beitrag}, defined by the energy function
\begin{equation}\label{ising_energy_function}
  H(\sigma) = -J \sum_{\langle i j \rangle} \sigma_i \sigma_j
\end{equation}
Here, we sum over nearest neighbors $\langle i j \rangle$ and the spins
$\sigma_i$ take the values $\pm 1$. $J > 0$.

Assume, for the moment, that the chain
consists of $N$ spins, and apply periodic boundary conditions.
The partition function of this system is given by
\begin{equation}
  Z_{N} = \sum_{\sigma_1, \dotsc, \sigma_N \in \{-1, 1\}} \exp (-\beta H(\sigma))
\end{equation}
Exploiting the local nature of the interaction between spins, we can write
\begin{equation}
  Z_{N} = \sum_{\sigma_1, \cdots, \sigma_N \in \{-1, 1\}} \prod_{\langle i, j \rangle} e^{K\sigma_i \sigma_j}
\end{equation}
where we defined $K \equiv \beta J$.

Now, we define the $2 \times 2$ matrix
\begin{equation}\label{eq:transfer_matrix_1d_ising}
  T_{\sigma \sigma'} = \exp(K \sigma \sigma')
\end{equation}

A possible
choice of basis is
\begin{equation}\label{eq:basis_transfer_matrix_1d}
  \bigl( \ket{\uparrow} = 1, \ket{\downarrow} = -1 \bigr) =
  \bigl(
  \begin{bmatrix}
    1 \\
    0
  \end{bmatrix},
  \begin{bmatrix}
    0 \\
    1
  \end{bmatrix}
  \bigr)
\end{equation}

In terms of this matrix, $Z_N$ is written as
\begin{equation}\label{eq:partition_function_transfer_matrix_1d}
  Z_N = \sum_{\sigma_1, \cdots, \sigma_N} T_{\sigma_1 \sigma_2} \dotsm T_{\sigma_N \sigma_1} = \tr T^N
\end{equation}

$T$ is called the transfer matrix. Since $T$ is, in fact, diagonalizable, $T^N = P D^N
P^{-1}$, where $P$ consists of the eigenvectors of $T$. By the cyclic property of the
trace, we have
\begin{equation}
  Z_N = \lambda_{1}^{N} + \lambda_{2}^{N}
\end{equation}

Thus, we have reduced the problem of finding the partition function to an
eigenvalue problem, which is quite easy in this case.

Note that in the thermodynamic limit $N \to \infty$
\begin{equation}
  Z = \lim_{N \to \infty} \lambda_{1}^{N}
\end{equation}
where $\lambda_1$ is the non-degenerate largest eigenvalue (in absolute value) of $T$.

\subsubsection{Fixed boundary conditions}
We may also apply fixed boundary conditions. The partition function is then written as
\begin{equation}
  Z_N = \bra{\sigma'}T^N\ket{\sigma},
\end{equation}
where $\ket{\sigma}$ and $\ket{\sigma'}$ are the right and left boundary spins.

In the large-$N$ limit, $T^N$ tends towards the projector onto the eigenspace spanned by
the eigenvector belonging to the largest eigenvalue
\begin{equation}\label{eq:largest_eigenvector}
  \ket{\lambda_1} = \lim_{N \to \infty} \frac{T^N \ket{\sigma}}{\left\lVert T^N \ket{\sigma} \right\rVert}.
\end{equation}

\autoref{eq:largest_eigenvector} is true for any $\ket{\sigma}$ that is not orthogonal to $\ket{\lambda_1}$.

The physical significance of the normalized lowest-lying eigenvector $\ket{\lambda_1}$ is
that
$\braket{\lambda_1 | \uparrow}$ and
$\braket{\lambda_1 | \downarrow}$ represent the Boltzmann weight of $\ket{\uparrow}$ and
$\ket{\downarrow}$ at the boundary of a half-infinite chain.

\subsection{2D Ising model}
\todo[inline]{Talk about exact solution (Onsager). Why is it important? Maybe
star-triangle relation (Baxter). Not all IRF models solvable.}

Next, we treat the two-dimensional, square-lattice Ising model. In two
dimensions, the energy function is still written as in
\autoref{ising_energy_function}, but now every lattice site has four neighbors.

Let $N$ be the number of columns and $l$ be the number of rows of the lattice, and assume
$l \gg N$. In the vertical direction, we apply periodic boundary conditions, as in the
one-dimensional case. In the horizontal direction, we keep an open boundary. We refer to
$N$ as the system size.

Similarly as in the 1D case, the partition function can be written as
\begin{equation}
  Z_N = \sum_{\bm{\sigma}} \prod_{\langle i, j, k, l \rangle} W(\sigma_i, \sigma_j, \sigma_k, \sigma_l)
\end{equation}
where the product runs over all groups of four spins sharing the same face. The Boltzmann weight of such a face is given by
\begin{equation}\label{eq:boltzmann_weight_face_ising_model}
  W(\sigma_i, \sigma_j, \sigma_k, \sigma_l) = \exp \left\{ \frac{K}{2} (\sigma_i \sigma_j + \sigma_j \sigma_k + \sigma_k \sigma_l + \sigma_l \sigma_i) \right\}
\end{equation}

We can express the Boltzmann weight of a configuration of the whole lattice as
a product of the Boltzmann weights of the rows
\begin{equation}
  Z_N = \sum_{\bm{\sigma}} \prod_{r = 1}^{l} W(\sigma_{1}^{r}, \sigma_{2}^{r}, \sigma_{1}^{r+1}, \sigma_{2}^{r+1}) \dots W(\sigma_{N-1}^{r}, \sigma_{N}^{r}, \sigma_{N-1}^{r+1}, \sigma_{N}^{r+1})
\end{equation}
where $\sigma_{i}^{r}$ denotes the value of the $i$th spin of row $r$.

Now, we can generalize the definition of the transfer matrix to two dimensions, by
defining it as the Boltzmann weight of an entire row
\begin{equation}\label{eq:row_to_row_transfer_matrix}
  T_{N}(\bm{\sigma}, \bm{\sigma'}) = W(\sigma_1, \sigma_2, \sigma_1', \sigma_2') \dots W(\sigma_{N-1}, \sigma_N, \sigma_{N-1}', \sigma_{N}')
\end{equation}
If we take the spin configurations of an entire row as basis vectors, $T_N$ can be written
as a matrix of dimensions $2^N \times 2^N$.

Similarly as in the one-dimensional case, the partition function now becomes
\begin{equation}\label{eq:z_n_times_infty}
  Z_N = \sum_{\bm{\sigma}} \prod_{r = 1}^{l} T_{N}(\bm{\sigma}^r, \bm{\sigma}^{r+1}) = \tr T_{N}^l
\end{equation}

In the limit of an $N \times \infty$ cylinder, the partition function is once again
determined by the largest eigenvalue\footnote{As in the 1D case, $T$ is symmetric, so it
is orthogonally diagonalizable.}.
\begin{equation}\label{largest_eigenvalue_transfer_matrix}
  Z_N = \lim_{l \to \infty} T_{N}^{l} = \lim_{l \to \infty} (\lambda_0)_{N}^{l}
\end{equation}

The partition function in the thermodynamic limit is given by
\begin{equation}
  Z = \lim_{N \to \infty} Z_N
\end{equation}

\section{Partition function of the 2D Ising model as a tensor network}
In calculating the partition function of 1D and 2D lattices, matrices of Boltzmann weights
like $W$ and $T$ play a crucial role. We have formulated them in a way that is valid for
any interaction-round-a-face (IRF) model, defined by
\begin{equation}
  H \thicksim \sum_{\langle i, j, k, l \rangle} W(\sigma_i, \sigma_j, \sigma_k,
  \sigma_l)
\end{equation}
where the summation is over all spins sharing a face. $W$ can contain 4-spin,
3-spin, 2-spin and 1-spin interaction terms. The Ising model is a special case of the IRF
model, with $W$ given by \autoref{eq:boltzmann_weight_face_ising_model}.

We will now express the partition function of the 2D Ising model as a tensor network. The
transfer matrix $T$ is redefined in the process. This allows us to visualize the equations
in a way that is consistent with the many other tensor network algorithms under research
today.

\todo[inline]{
    For any $l$, the trace over the transfer matrix is the same in both definitions of
    $T$. It is symmetric, so they should be related by a basis transformation. Is it,
    perhaps, exactly the basis in which the CTM is diagonal? }

\subsection{A system of four spins}

We define
\begin{equation}
  Q(\sigma_i, \sigma_j) = \exp(K \sigma_i \sigma_j)
\end{equation}
as the Boltzmann weight of the bond between $\sigma_i$ and $\sigma_j$. It is the
same as the 1D transfer matrix in \autoref{eq:transfer_matrix_1d_ising}.

The Boltzmann weight of a face $W$ decomposes into a product of Boltzmann weights of
bonds
\begin{equation}
  W(\sigma_i, \sigma_j, \sigma_k, \sigma_l) =
  Q(\sigma_i, \sigma_j)Q(\sigma_j, \sigma_l)Q(\sigma_l, \sigma_k)Q(\sigma_k, \sigma_i)
\end{equation}

It is now easy to see that the partition function is equal to the contracted tensor
network in \autoref{fig:tensor_network_4_sites}:
\begin{equation}\label{eq:tensor_network_4_sites}
  \begin{split}
    Z_{2 \times 2} & =
    \sum_{\sigma_1, \sigma_2, \sigma_3, \sigma_4} \sum_{a, b, c, d}
    \delta_{\sigma_1, a} Q(a, b) \delta_{\sigma_2, b} Q(b, c)
    \delta_{\sigma_3, c} Q(c, d) \delta_{\sigma_4, d} Q(d, a) \\
    & =
    \sum_{\sigma_1, \sigma_2, \sigma_3, \sigma_4} W(\sigma_1, \sigma_2, \sigma_3, \sigma_4)
  \end{split}
\end{equation}
where the Kronecker delta is defined as usual:
\begin{equation}
  \delta_{i j} =
  \begin{cases}
    1 & \text{if } i = j \\
    0 & \text{if } i \neq j
  \end{cases}
\end{equation}

\begin{figure}
  \begin{tikzpicture}
	\begin{pgfonlayer}{nodelayer}
		\node [style=delta-tensor] (0) at (-1, 0) {};
		\node [style=delta-tensor] (1) at (1, 0) {};
		\node [style=delta-tensor] (2) at (1, 2) {};
		\node [style=delta-tensor, label={$\delta$}] (3) at (-1, 2) {};
		\node [style=q-tensor] (4) at (1, 1) {};
		\node [style=q-tensor, label={$Q$}] (5) at (0, 2) {};
		\node [style=q-tensor] (6) at (-1, 1) {};
		\node [style=q-tensor] (7) at (0, 0) {};
		\node [style=white no border] (8) at (-2, 1) {$Z_{2 \times 2} = $};
	\end{pgfonlayer}
	\begin{pgfonlayer}{edgelayer}
		\draw [style=simple] (3) to (6);
		\draw [style=simple] (6) to (0);
		\draw [style=simple] (0) to (7);
		\draw [style=simple] (7) to (1);
		\draw [style=simple] (1) to (4);
		\draw [style=simple] (4) to (2);
		\draw [style=simple] (2) to (5);
		\draw [style=simple] (5) to (3);
	\end{pgfonlayer}
\end{tikzpicture}

  \caption{A tensor network representation of the partition function of the Ising model on
  a $2 \times 2$ lattice. See \autoref{eq:tensor_network_4_sites}.}
  \label{fig:tensor_network_4_sites}
\end{figure}


\subsection{Thermodynamic limit}
\begin{figure}
  \begin{tikzpicture}
	\begin{pgfonlayer}{nodelayer}
		\node [style=q-tensor, label={$Q$}] (0) at (0, 0) {};
		\node [style=white no border] (1) at (-1, 0) {};
		\node [style=white no border] (2) at (1, 0) {$=$};
		\node [style=p-tensor-right, label={$P$}] (3) at (2, 0) {};
		\node [style=p-tensor-left, label={$P$}] (4) at (2.75, 0) {};
		\node [style=white no border] (5) at (3.75, 0) {};
	\end{pgfonlayer}
	\begin{pgfonlayer}{edgelayer}
		\draw [style=simple] (0) to (1);
		\draw [style=simple] (0) to (2);
		\draw [style=simple] (3) to (4);
		\draw [style=simple] (4) to (5);
		\draw [style=simple] (3) to (2);
	\end{pgfonlayer}
\end{tikzpicture}

  \caption{Graphical form of \autoref{eq:q_to_p}.}
  \label{fig:q_to_p}
\end{figure}

\begin{figure}
  \begin{tikzpicture}
	\begin{pgfonlayer}{nodelayer}
		\node [style=delta-tensor] (0) at (-2, 0) {};
		\node [style=p-tensor-up] (1) at (-2, -0.75) {};
		\node [style=p-tensor-left] (2) at (-2.75, 0) {};
		\node [style=p-tensor-down] (3) at (-2, 0.75) {};
		\node [style=p-tensor-right] (4) at (-1.25, 0) {};
		\node [style=white no border] (5) at (-0.5, 0) {$=$};
		\node [style=white no border] (6) at (-2, 1.5) {};
		\node [style=white no border] (7) at (-3.5, 0) {};
		\node [style=white no border] (8) at (-2, -1.5) {};
		\node [style=a-tensor] (9) at (0.5, 0) {$a$};
		\node [style=white no border] (10) at (0.5, 1) {};
		\node [style=white no border] (11) at (1.5, 0) {};
		\node [style=white no border] (12) at (0.5, -1) {};
	\end{pgfonlayer}
	\begin{pgfonlayer}{edgelayer}
		\draw [style=simple] (9) to (11);
		\draw [style=simple] (9) to (10);
		\draw [style=simple] (9) to (12);
		\draw [style=simple] (9) to (5);
		\draw [style=simple] (5) to (4);
		\draw [style=simple] (4) to (0);
		\draw [style=simple] (0) to (3);
		\draw [style=simple] (0) to (1);
		\draw [style=simple] (0) to (2);
		\draw [style=simple] (2) to (7);
		\draw [style=simple] (1) to (8);
		\draw [style=simple] (3) to (6);
	\end{pgfonlayer}
\end{tikzpicture}

  \caption{Graphical form of \autoref{eq:a_tensor}.}
  \label{fig:a_tensor}
\end{figure}

We define the matrix $P$ by
\begin{equation}\label{eq:q_to_p}
  P^2 = Q
\end{equation}
as in \autoref{fig:q_to_p}. This allows us to write the partition function of an arbitrary
$N \times l$ square lattice as a tensor network of a single recurrent tensor $a_{i j k
l}$, given by
\begin{equation}\label{eq:a_tensor}
  a_{i j k l} = \sum_{a, b, c, d} \delta_{a b c d} P_{i a} P_{j b} P_{k c} P_{l d}
\end{equation}
where the generalization of the Kronecker delta is defined as
\begin{equation}
  \delta_{i_1 \dots i_n} =
  \begin{cases}
    1 & \text{if } i_1 = \ldots = i_n \\
    0 & \text{otherwise}
  \end{cases}
\end{equation}

See \autoref{fig:a_tensor} and \autoref{fig:2d_ising_as_tensor_network}. At the edges and
corners, we define suitable tensors of rank 3 and 2, which we will also denote by $a$.
\begin{align*}
  a_{i j k} &= \sum_{a b c} \delta_{a b c} P_{i a} P_{j b} P_{k c} \\
  a_{i j} &= \sum_{a b} \delta_{a b} P_{i a} P_{j b}
\end{align*}

The challenge is to approximate this tensor network in the thermodynamic limit.

\begin{figure}
  \begin{tikzpicture}
	\begin{pgfonlayer}{nodelayer}
		\node [style=delta-tensor] (0) at (2, -1) {};
		\node [style=white no border, rotate=90] (1) at (2, -2.5) {$\dots$};
		\node [style=p-tensor-down] (2) at (2, -0.25) {};
		\node [style=p-tensor-up] (3) at (2, -1.75) {};
		\node [style=p-tensor-left] (4) at (1.25, -1) {};
		\node [style=p-tensor-right] (5) at (2.75, -1) {};
		\node [style=white no border] (6) at (0.5, -1) {$\dots$};
		\node [style=p-tensor-left] (7) at (1.25, 1) {};
		\node [style=p-tensor-right] (8) at (2.75, 1) {};
		\node [style=p-tensor-down] (9) at (2, 1.75) {};
		\node [style=delta-tensor] (10) at (2, 1) {};
		\node [style=p-tensor-up] (11) at (2, 0.25) {};
		\node [style=p-tensor-left] (12) at (3.25, -1) {};
		\node [style=p-tensor-right] (13) at (4.75, -1) {};
		\node [style=p-tensor-down] (14) at (4, -0.25) {};
		\node [style=delta-tensor] (15) at (4, -1) {};
		\node [style=p-tensor-up] (16) at (4, -1.75) {};
		\node [style=p-tensor-left] (17) at (3.25, 1) {};
		\node [style=p-tensor-right] (18) at (4.75, 1) {};
		\node [style=p-tensor-down] (19) at (4, 1.75) {};
		\node [style=delta-tensor] (20) at (4, 1) {};
		\node [style=p-tensor-up] (21) at (4, 0.25) {};
		\node [style=white no border, rotate=90] (22) at (4, -2.5) {$\dots$};
		\node [style=white no border] (23) at (5.5, -1) {$\dots$};
		\node [style=white no border] (24) at (5.5, 1) {$\dots$};
		\node [style=white no border, rotate=90] (25) at (4, 2.5) {$\dots$};
		\node [style=white no border, rotate=90] (26) at (2, 2.5) {$\dots$};
		\node [style=white no border] (27) at (0.5, 1) {$\dots$};
		\node [style=delta-tensor] (28) at (-1.5, -1) {};
		\node [style=delta-tensor] (29) at (-1.5, 1) {};
		\node [style=delta-tensor] (30) at (-3.5, -1) {};
		\node [style=delta-tensor] (31) at (-3.5, 1) {};
		\node [style=q-tensor] (32) at (-2.5, -1) {};
		\node [style=q-tensor] (33) at (-1.5, 0) {};
		\node [style=q-tensor] (34) at (-2.5, 1) {};
		\node [style=q-tensor] (35) at (-3.5, 0) {};
		\node [style=white no border] (36) at (-0.5, 1) {$\dots$};
		\node [style=white no border] (37) at (-0.5, -1) {$\dots$};
		\node [style=white no border, rotate=90] (38) at (-1.5, -2) {$\dots$};
		\node [style=white no border, rotate=90] (39) at (-3.5, -2) {$\dots$};
		\node [style=white no border, rotate=90] (40) at (-3.5, 2) {$\dots$};
		\node [style=white no border, rotate=90] (41) at (-1.5, 2) {$\dots$};
		\node [style=white no border] (42) at (-4.5, 1) {$\dots$};
		\node [style=white no border] (43) at (-4.5, -1) {$\dots$};
		\node [style=white no border] (44) at (0, 0) {$=$};
		\node [style=a-tensor] (45) at (-3, -4.25) {$a$};
		\node [style=a-tensor] (46) at (-2, -4.25) {$a$};
		\node [style=a-tensor] (47) at (-3, -5.25) {$a$};
		\node [style=a-tensor] (48) at (-2, -5.25) {$a$};
		\node [style=white no border] (49) at (-1, -4.25) {$\dots$};
		\node [style=white no border, rotate=90] (50) at (-2, -3.25) {$\dots$};
		\node [style=white no border, rotate=90] (51) at (-3, -3.25) {$\dots$};
		\node [style=white no border] (52) at (-4, -4.25) {$\dots$};
		\node [style=white no border] (53) at (-4, -5.25) {$\dots$};
		\node [style=white no border, rotate=90] (54) at (-3, -6.25) {$\dots$};
		\node [style=white no border, rotate=90] (55) at (-2, -6.25) {$\dots$};
		\node [style=white no border] (56) at (-1, -5.25) {$\dots$};
		\node [style=white no border] (57) at (-4.5, -4.75) {$=$};
	\end{pgfonlayer}
	\begin{pgfonlayer}{edgelayer}
		\draw [style=simple] (5) to (0);
		\draw [style=simple] (0) to (2);
		\draw [style=simple] (0) to (3);
		\draw [style=simple] (0) to (4);
		\draw [style=simple] (4) to (6);
		\draw [style=simple] (3) to (1);
		\draw [style=simple] (8) to (10);
		\draw [style=simple] (10) to (9);
		\draw [style=simple] (10) to (11);
		\draw [style=simple] (10) to (7);
		\draw [style=simple] (13) to (15);
		\draw [style=simple] (15) to (14);
		\draw [style=simple] (15) to (16);
		\draw [style=simple] (15) to (12);
		\draw [style=simple] (18) to (20);
		\draw [style=simple] (20) to (19);
		\draw [style=simple] (20) to (21);
		\draw [style=simple] (20) to (17);
		\draw [style=simple] (9) to (26);
		\draw [style=simple] (19) to (25);
		\draw [style=simple] (18) to (24);
		\draw [style=simple] (7) to (27);
		\draw [style=simple] (13) to (23);
		\draw [style=simple] (16) to (22);
		\draw [style=simple] (11) to (2);
		\draw [style=simple] (21) to (14);
		\draw [style=simple] (5) to (12);
		\draw [style=simple] (8) to (17);
		\draw [style=simple] (30) to (32);
		\draw [style=simple] (32) to (28);
		\draw [style=simple] (28) to (33);
		\draw [style=simple] (33) to (29);
		\draw [style=simple] (29) to (34);
		\draw [style=simple] (34) to (31);
		\draw [style=simple] (31) to (35);
		\draw [style=simple] (35) to (30);
		\draw [style=simple] (28) to (37);
		\draw [style=simple] (28) to (38);
		\draw [style=simple] (30) to (39);
		\draw [style=simple] (30) to (43);
		\draw [style=simple] (31) to (42);
		\draw [style=simple] (31) to (40);
		\draw [style=simple] (29) to (41);
		\draw [style=simple] (29) to (36);
		\draw [style=simple] (45) to (46);
		\draw [style=simple] (45) to (47);
		\draw [style=simple] (47) to (48);
		\draw [style=simple] (48) to (46);
		\draw [style=simple] (46) to (49);
		\draw [style=simple] (48) to (56);
		\draw [style=simple] (48) to (55);
		\draw [style=simple] (47) to (54);
		\draw [style=simple] (47) to (53);
		\draw [style=simple] (45) to (52);
		\draw [style=simple] (45) to (51);
		\draw [style=simple] (46) to (50);
	\end{pgfonlayer}
\end{tikzpicture}
  \caption{$Z_{N \times l}$ can be written as a contracted tensor network of $N \times l$
  copies of the tensor $a$.}
  \label{fig:2d_ising_as_tensor_network}
\end{figure}

\subsection{The transfer matrix as a tensor network}
\todo[inline]{Say something about reshaping legs.}

With our newfound representation of the partition function as a tensor network, we can
redefine the row-to-row transfer matrix from
\autoref{eq:row_to_row_transfer_matrix} as the tensor network expressed in
\autoref{fig:transfer_matrix_as_tensor_network}. For all $l$, it is still true that
\begin{equation}
  Z_{N \times l} = \tr T_{N}^{l} = \sum_{i = 1}^{2^N} \lambda_{i}^{l}
\end{equation}
so the eigenvalues must be the same. That means that the new definition of the transfer
matrix is related to the old one by a basis transformation
\begin{equation}
  T_{\text{new}} = P T_{\text{old}} P^{-1}
\end{equation}

\begin{figure}
  \begin{tikzpicture}
	\begin{pgfonlayer}{nodelayer}
		\node [style={a-tensor}] (0) at (-1, -0) {$a$};
		\node [style={a-tensor}] (1) at (0, -0) {$a$};
		\node [style={white no border}] (2) at (1, -0) {$\dots$};
		\node [style={a-tensor}] (3) at (2, -0) {$a$};
		\node [style={a-tensor}] (4) at (3, -0) {$a$};
		\node [style={white no border}] (5) at (3, 1) {};
		\node [style={white no border}] (6) at (2, 1) {};
		\node [style={white no border}] (7) at (0, 1) {};
		\node [style={white no border}] (8) at (-1, 1) {};
		\node [style={white no border}] (9) at (-1, -1) {};
		\node [style={white no border}] (10) at (0, -1) {};
		\node [style={white no border}] (11) at (2, -1) {};
		\node [style={white no border}] (12) at (3, -1) {};
	\end{pgfonlayer}
	\begin{pgfonlayer}{edgelayer}
		\draw [style=simple] (0) to (1);
		\draw [style=simple] (1) to (2);
		\draw [style=simple] (2) to (3);
		\draw [style=simple] (3) to (4);
		\draw [style=simple] (4) to (5);
		\draw [style=simple] (3) to (6);
		\draw [style=simple] (1) to (7);
		\draw [style=simple] (0) to (8);
		\draw [style=simple] (0) to (9);
		\draw [style=simple] (1) to (10);
		\draw [style=simple] (3) to (11);
		\draw [style=simple] (4) to (12);
	\end{pgfonlayer}
\end{tikzpicture}

  \caption{The definition of $T_N$ as a network of $N$ copies of the tensor $a$.}
  \label{fig:transfer_matrix_as_tensor_network}
\end{figure}

\section{Transfer matrix renormalization group}
There is a deep connection between quantum mechanical lattice systems in $d$ dimensions
and classical lattice systems in $d + 1$ dimensions. Via the imaginary time path integral
formulation, the partition function of a one-dimensional quantum system can be written as
the partition function of an effective two-dimensional classical system. The ground state
of the quantum system corresponds to the largest eigenvector of the transfer matrix of the
classical system.

For more on the quantum-classical correspondence, see
\autoref{chapter:correspondence_quantum_classical}.
\todo[inline]{I want to refer to Baxter's variational
approach anno 1968 somehwere.}

Nishino \cite{nishino1995density, nishino1996corner} was the first to apply density matrix
renormalization group methods in the context of two-dimensional classical
lattices.

\subsection{The infinite system algorithm for the transfer matrix}

Analogous to the infinite system DMRG algorithm for quantum spin chains, our goal is to
approximate the transfer matrix in the thermodynamic limit as well as possible within a
restriced number of basis states $m$. We will do this by adding a single site at a time,
and truncating the dimension from $2m$ to $m$ at each iteration.

We start the algorithm from a transfer matrix that already has dimension $m$, perhaps by
exactly diagonalizing the transfer matrix of a couple of sites. We call this transfer
matrix $P_N$. FIGURE.

We enlarge the system with one site by contracting with an additional $a$-tensor,
obtaining $P_{N + 1}$. FIGURE

In order to find the best projection from $2m$ basis states back to
$m$, we embed the system in an environment that is the mirror image of the system we
presently have. We call this matrix $T_{2N + 2}$. It represents the transfer matrix of $2N +
2$ sites. We find the largest eigenvalue and corresponding eigenvector, as shown in
FIGURE.

Having identified the lowest-lying eigenvector of the transfer matrix with the ground
state of a superblock in DMRG, the equivalent of the \textit{reduced density matrix of a
block} in the classical case is:
\begin{equation}
  \rho_{N + 1} = \sum_{\sigma_B} \braket{\sigma_B | \lambda_0}\braket{\lambda_0 |
  \sigma_B}
\end{equation}
where we have summed over all the degrees of freedom of one of the half-row transfer
matrices $P_{N+1}$. FIGURE. The optimal renormalization
\begin{equation}
  \widetilde{P}_{N+1} = O P_{N + 1} O^{\dagger}
\end{equation}
is obtained by diagonalizing $\rho_{N + 1}$ and keeping the eigenvectors corresponding
to the $m$ largest eigenvalues. FIGURE.
With this blocking procedure, we can successively find
\begin{equation}
    P_{N + 1} \rightarrow P_{N + 2} \rightarrow \dots
\end{equation}
until we have reached some termination procedure.



% Generalizing \autoref{eq:largest_eigenvector} to two dimensions, we know that the
% normalized eigenvector corresponding to
% largest (non-degenerate) eigenvalue of the transfer matrix contains the probabilities
% of the different possible configurations of a row of spins $\bm{\sigma}$ at the boundary
% of the half-infinite lattice.

% Now, the connection to ground state DMRG is clear. Recall the full-system
% density matrix of an $N$-site 1D quantum lattice system (the superblock) in the
% ground state (cf. \eqref{eq:density_matrix_superblock})
%
% \begin{equation}
%   \rho_N = \ket{\Psi_0}_{N}\bra{\Psi_0}_{N}
% \end{equation}



\section{Corner transfer matrices}

The concept of corner transfer matrices for 2D lattices was first introduced by
Baxter \cite{baxter1968dimers, baxter1978variational, baxter1982exactly}.
Whereas the row-to-row transfer matrix \eqref{eq:row_to_row_transfer_matrix}
corresponds to adding a row to the lattice, the corner transfer matrix adds
a quadrant of spins. It is defined as

\begin{equation}
  A_{\bm{\sigma}, \bm{\sigma'}} =
  \begin{cases}
    \sum \prod_{\langle i, j, k, l \rangle} W(\sigma_i, \sigma_j, \sigma_k, \sigma_l) & \text{if } \sigma_{1} = \sigma_{1}' \\
    0 & \text{if } \sigma_{1} \neq \sigma_{1}'
  \end{cases}
\end{equation}

\todo[inline]{Picture.}
Here, the product runs over groups of four spins that share the same face, and
the sum is over all spins in the interior of the quadrant.
In a symmetric and isotropic model, we have

\begin{equation}
  W(a, b, c, d) = W(b, a, d, c) = W(c, a, d, b) = W(d, c, b, a)
\end{equation}

and the partition of an $N \times N$ lattice is expressed as

\begin{equation}\label{z_n_times_n}
  Z_{N \times N} = \tr A^4
\end{equation}

In the thermodynamic limit, \eqref{z_n_times_n} is equal to \eqref{eq:z_n_times_infty}.



\subsection{Corner transfer matrix renormalization group}

\todo[inline]{More about Baxter's variational approach.}
Nishino and Okunishi combined ideas from Baxter and White to formulate the
corner transfer matrix renormalization group \cite{nishino1996corner}.
