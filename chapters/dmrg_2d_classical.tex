\section{Partition functions of classical lattices}
The central quantity in equilibrium statistical mechanics is the partition
function $Z$, which, for a discrete system such as a lattice, is defined as

\begin{equation}
  Z = \sum_{s} \exp{(-\beta H(s))}
\end{equation}

where the sum is over all microstates $s$, $H$ is the energy function, and
$\beta = T^{-1}$ the inverse temperature.

\section{Transfer matrices of lattice models}

\subsection{1D Ising model}

\todo[inline]{refer to Ising, talk a bit about model (magnetism etc).}

Consider the 1D zero-field ferromagnetic Ising model, defined by the energy function

\begin{equation}\label{ising_energy_function}
  H(\sigma) = -J \sum_{\langle i j \rangle} \sigma_i \sigma_j
\end{equation}

Here, we sum over nearest neighbors $\langle i j \rangle$ and the spins
$\sigma_i$ take the values $\pm 1$. $J > 0$. Assume, for the moment, that the chain
consists of $N$ spins, and apply periodic boundary conditions.
The partition function of this system is given by

\begin{equation}
  Z_{N} = \sum_{\sigma_1, \dotsc, \sigma_N \in \{-1, 1\}} \exp (-\beta H(\sigma))
\end{equation}

Exploiting the local nature of the interaction between spins, we can write

\begin{equation}
  Z_{N} = \sum_{\sigma_1, \cdots, \sigma_N \in \{-1, 1\}} \prod_{\langle i, j \rangle} e^{K\sigma_i \sigma_j}
\end{equation}

where we defined $K \equiv \beta J$. Now, we can define
the $2 \times 2$ matrix $T_{\sigma \sigma'} = \exp(K \sigma \sigma')$ to get

\begin{equation}
  Z_N = \sum_{\sigma_1, \cdots, \sigma_N} T_{\sigma_1 \sigma_2} \dotsm T_{\sigma_N \sigma_1} = \tr T^N
\end{equation}

$T$ is called the transfer matrix. Since $T^N = P D^N P^{-1}$, where $P$
consists of the eigenvectors of $T$, and by the cyclic property of the trace, we
have

\begin{equation}
  Z_N = \lambda_{1}^{N} + \lambda_{2}^{N}
\end{equation}

Thus, we have reduced the problem of finding the partition function to an
eigenvalue problem, which is quite easy in this case. Also, note that in the thermodynamic limit $N \to \infty$

\begin{equation}
  Z = \lim_{N \to \infty} \lambda_{1}^{N}
\end{equation}

where $\lambda_1$ is the non-degenerate largest eigenvalue (in absolute value) of $T$.

\subsection{2D Ising model}
\todo[inline]{Talk about exact solution (Onsager). Why is it important? Maybe
star-triangle relation (Baxter). Not all IRF models solvable.}

Next, we treat the two-dimensional, square-lattice Ising model. In two
dimensions, the energy function is still written as in
\eqref{ising_energy_function}, but now every lattice site has four neighbors.
Let $N$ be the number of columns and $l$ be the number of rows of the lattice, and assume 
$l \gg N$. In the vertical direction, we apply periodic boundary conditions, as in the one-dimensional case.
In the horizontal direction, we keep an open boundary. We refer to $N$ as the system size.

Similarly to the 1D case, the partition function can be written as 

\begin{equation}
  Z_N = \sum_{\bm{\sigma}} \prod_{\langle i, j, k, l \rangle} W(\sigma_i, \sigma_j, \sigma_k, \sigma_l)
\end{equation}

where the product runs over all groups of four spins sharing the same face. The Boltzmann weight of such a face is given by

\begin{equation}
  W(\sigma_i, \sigma_j, \sigma_k, \sigma_l) = \exp \left\{ \frac{K}{2} (\sigma_i \sigma_j + \sigma_j \sigma_k + \sigma_k \sigma_l + \sigma_l \sigma_i) \right\}
\end{equation}

We can express the Boltzmann weight of a configuration of the whole lattice as
a product of the Boltzmann weights of the rows

\begin{equation}
  Z_N = \sum_{\bm{\sigma}} \prod_{r = 1}^{l} W(\sigma_{1}^{r}, \sigma_{2}^{r}, \sigma_{1}^{r+1}, \sigma_{2}^{r+1}) \dots W(\sigma_{N-1}^{r}, \sigma_{N}^{r}, \sigma_{N-1}^{r+1}, \sigma_{N}^{r+1})
\end{equation}

Where $\sigma_{i}^{r}$ denotes the value of the $i$th spin of row $r$. Now, we
can generalize the definition of the transfer matrix to two dimensions, by
defining it as the Boltzmann weight of an entire row

\begin{equation}\label{row_to_row_transfer_matrix} 
  T_{N}(\bm{\sigma}, \bm{\sigma'}) = W(\sigma_1, \sigma_2, \sigma_1', \sigma_2') \dots W(\sigma_{N-1}, \sigma_N, \sigma_{N-1}', \sigma_{N}')
\end{equation}

The dimensions of $T_N$ are $2^N \times 2^N$. Similarly as in the
one-dimensional case, the partition function now becomes

\begin{equation}\label{z_n_times_infty}
  Z_N = \sum_{\bm{\sigma}} \prod_{r = 1}^{l} T_{N}(\bm{\sigma}^r, \bm{\sigma}^{r+1}) = \tr T_{N}^l
\end{equation}

In the limit of an $N \times \infty$ cylinder, the partition function is once again determined by the largest eigenvalue.

\begin{equation}\label{largest_eigenvalue_transfer_matrix}
  Z_N = \lim_{l \to \infty} T_{N}^{l} = \lim_{l \to \infty} (\lambda_0)_{N}^{l}
\end{equation}

The partition function in the thermodynamic limit is given by

\begin{equation}
  Z = \lim_{N \to \infty} Z_N
\end{equation}



\section{Partition function of 2D lattice as a tensor network}
In calculating the partition function of 1D and 2D lattices, matrices of Boltzmann weights like $W$
and $T$ play a crucial role. Up until now, we have formulated them in a way
that serves us analytically, but makes it less clear how we should use them to
implement a partition sum in a computer program, when calculations typically
have to be written as matrix multiplications in order to be fast. Now, we
rewrite the partition sum of 2D classical lattices in the language of tensor
networks to make the computational steps more manifest, while retaining the
intuitions of the earlier formulation.



\begin{figure}
  \includestandalone{images/2d_classical_partition_function_as_tensor_network}
  \caption{2D lattice to as a tensor network. bla bla.}
  \label{fig:2d_classical_partition_function_as_tensor_network}
\end{figure}


\section{Transfer matrix renormalization group}
\todo[inline]{More about connection between 1D quantum and 2D classical}
\todo[inline]{Pictures}

Nishino \cite{nishino1995density} was the first one to apply DMRG in its modern
form to approximate the transfer matrix of a two-dimensional classical lattice system.

In the limit of an $N \times \infty$ lattice, $T_{N}^{l}$ becomes proportional
to the projector onto the eigenspace of the largest eigenvalue

\begin{equation}
  \lim_{l \to \infty} T_{N}^{l} = \lim_{l \to \infty} (\lambda_0)_{N}^{l} \ket{\lambda_0}_{N}\bra{\lambda_0}_{N} \thicksim \ket{\lambda_0}_{N}\bra{\lambda_0}_N
\end{equation}

Now, the connection to ground state DMRG is clear. Recall the full-system
density matrix of an $N$-site 1D quantum lattice system (the superblock) in the
ground state (cf. \eqref{eq:density_matrix_superblock})

\begin{equation}
  \rho_N = \ket{\Psi_0}_{N}\bra{\Psi_0}_{N}
\end{equation}


\todo[inline]{Write down equations that are analogous to reduced density matrix. Pictures}



\section{Corner transfer matrices}

The concept of corner transfer matrices for 2D lattices was first introduced by
Baxter \cite{baxter1968dimers, baxter1978variational, baxter1982exactly}.
Whereas the row-to-row transfer matrix \eqref{row_to_row_transfer_matrix}
corresponds to adding a row to the lattice, the corner transfer matrix adds
a quadrant of spins. It is defined as

\begin{equation}
  A_{\bm{\sigma}, \bm{\sigma'}} =
  \begin{cases}
    \sum \prod_{\langle i, j, k, l \rangle} W(\sigma_i, \sigma_j, \sigma_k, \sigma_l) & \text{if } \sigma_{1} = \sigma_{1}' \\
    0 & \text{if } \sigma_{1} \neq \sigma_{1}'
  \end{cases}
\end{equation}

\todo[inline]{Picture.}
Here, the product runs over groups of four spins that share the same face, and
the sum is over all spins in the interior of the quadrant.
In a symmetric and isotropic model, we have

\begin{equation}
  W(a, b, c, d) = W(b, a, d, c) = W(c, a, d, b) = W(d, c, b, a)
\end{equation}

and the partition of an $N \times N$ lattice is expressed as

\begin{equation}\label{z_n_times_n}
  Z_{N \times N} = \tr A^4
\end{equation}

In the thermodynamic limit, \eqref{z_n_times_n} is equal to \eqref{z_n_times_infty}.



\subsection{Corner transfer matrix renormalization group}

\todo[inline]{More about Baxter's variational approach.}
Nishino and Okunishi combined ideas from Baxter and White to formulate the
corner transfer matrix renormalization group \cite{nishino1996corner}.

