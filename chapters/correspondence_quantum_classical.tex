The partition of a discrete quantum mechanical system is given by
\begin{equation}\label{eq:quantum_partition_function}
  Z_{q} = \tr \exp(-\beta H_{q}) =
  \sum_{\sigma} \bra{\sigma} \exp(-\beta H_{q}) \ket{\sigma}
\end{equation}
Imagine splitting the imaginary time interval $\beta$ into $N$ smaller steps:
\begin{align}
  \beta &= N \delta \tau, \\
  \exp(-\beta H_q) &= \exp(-\delta \tau H_q)^N.
\end{align}
Recall that for any orthonormal basis, the identity can be expressed as a sum over
projectors onto the basis states
\begin{equation}
    \mathbb{1} = \sum_{\sigma} \ket{\sigma}\bra{\sigma}.
\end{equation}
If we insert $N - 1$ resolutions of indentity into
\autoref{eq:quantum_partition_function}, we obtain
\begin{equation}
  Z_q = \sum_{\sigma} \sum_{\sigma_1, \dots, \sigma_{N-1}}
  \braket{\sigma | \exp(-\beta \delta \tau) | \sigma_1} \dots
  \braket{\sigma_{N-1} | \exp(-\beta \delta \tau) | \sigma}.
\end{equation}

This is the imaginary time path integral formulation of quantum mechanics. Similar to the
real-time path integral, an evolution in the imaginary time direction is expressed as a
sum over all paths connecting the initial and final state, which are the same here, since
we are taking the trace.

We turn to the partition function of a classical system, written as a product of
its transfer matrix, as in \autoref{eq:partition_function_transfer_matrix_1d}:
\begin{equation}
  Z_{\text{cl}} = \tr T^N.
\end{equation}
There is a striking similarity between a quantum mechanical partition function in $d$
dimensions and a classical partition function in $d + 1$ dimensions.
Adding a row to the classical lattice by applying the transfer matrix corresponds to time
evolution of a quantum system:
\begin{equation}
  T \longleftrightarrow \exp(-\delta \tau H_q).
\end{equation}
The classical temperature corresponds to the coupling constants in the Hamiltonian
$H_{\text{q}}$.

Letting $\beta \to \infty$ (or equivalently $T \to 0$) amounts to taking $N \to \infty$,
while keeping $\delta \tau$ fixed. In this limit, analogously to the transfer matrix for
the classical system (cf. \autoref{eq:largest_eigenvector}), the operator $\exp(-\beta
H_{\text{q}})$ becomes a projector onto the ground state.

Thus, a quantum lattice in the ground state corresponds to a
classical lattice that is infinite in its additional dimension.

\todo[inline]{What is the role $\delta \tau$? Why should it be small?
If we want to make the correspondence $T \equiv H_q$, we need $\delta \tau \to 0$.
But what does this imply for the lattice? How does the 'scaling limit' come into play?
Also: explain the correspondence between the energy scale of the quantum system and the
correlation length of the classical system. Maybe do an example of 0D quantum to 1D
classical and 1D quantum to 2D classical (Ising model?).}
