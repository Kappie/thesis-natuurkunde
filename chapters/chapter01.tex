\section{Abstract}

The variational method is introduced in its historical context -- that of
interacting 1D quantum lattice systems. An overview of White's density
matrix renormalization group \cite{white1992density} is given. Then, the
formal connection to 2D classical lattices is made.

\section{Strongly interacting 1D quantum lattice models}

\subsection{The Hilbert space is enormous}

Consider the problem of numerically finding the ground state $\ket{\Psi_0}$ of the
$N$-site 1-dimensional transverse-field Ising-model, given by

\begin{equation} 
  H = -J \sum_{i = 1}^{N} \sigma_{i}^{z}\sigma_{i+1}^{z}
  - h \sum_{i=1}^{N} \sigma_{i}^{x}
\end{equation}

The underlying Hilbert space of the system is a tensor product of the
local Hilbert spaces $\mathcal{H}_{\text{site}}$, which are spanned by the
states $\{\ket{\uparrow}, \ket{\downarrow}\}$. Thus, a general state of the system is a unit vector in
a $2^N$-dimensional space.

\begin{equation}
  \ket{\Psi} = \sum_{i_1, i_2, \ldots \in \{\ket{\uparrow}, \ket{\downarrow}\}} c_{i_1, i_2, \ldots, i_N} \ket{\sigma_1} \otimes \ket{\sigma_2} \otimes \ldots \otimes \ket{\sigma_N}
\end{equation}

So, for a system with 1000 particles, the dimensionality of the Hilbert
space comes in at about $10^301$, some 220 orders of magnitude larger than
the number of atoms in the observable universe. As it turns out, nature is
very well described by Hamiltonians that are local -- that do not contain
interactions between an arbitrary number of bodies. And for these
Hamiltonians, only an exponentially small subset of states can be explored
in the lifetime of the universe \cite{poulin2011quantum}. That is, only
exponentially few states are physical. \todo[inline]{Expand a bit. Refer
back to problem of finding ground state.}


\section{Density matrix renormalization group}

The density matrix renormalization group (DMRG), introduced in 1992 by
White \cite{white1992density}, aims to find the best approximation of
a many-body quantum state, given that only a fixed amount of basis vectors
is kept. This amounts to finding the best truncation

\begin{equation}
  \mathcal{H}_N \rightarrow \mathcal{H}_{\text{eff}}
\end{equation}

From the full $N$-particle Hilbert space to an effective lower dimensional
one. This corresponds to renormalizing the Hamiltonian $H$. Before the
DMRG paper, several methods for achieving this truncation were proposed,
most notably Wilson's real-space renormalization group
\cite{wilson1975renormalization}.

\subsection{Real-space renormalization group}

Consider the problem of finding the ground state of a many-body
Hamiltonian $H$. A natural way of renormalizing $H$ in real-space is by
partitioning the lattice in blocks, and writing $H$ as 

\begin{equation}
  H = H_A \otimes \ldots \otimes H_A
\end{equation}

where $H_A$ is the Hamiltonian of a block. \todo[inline]{Make figures.} The blocking procedure now
entails finding an effective Hamiltonian $H_{A}'$ of the two-block
Hamiltonian $H_{AA} = H_A \otimes H_A$. In numerical renormalization
introduced by Wilson, $H_{A}'$ is formed by keeping the $m$ lowest lying
eigenstates $\ket{\epsilon_{i}}$ of $H_{AA}$.

\begin{equation}
  H_{A}' = \sum_{i = 1}^{m} \epsilon_{i} \ket{\epsilon_{i}}\bra{\epsilon_{i}}
 \end{equation}

This is equivalent to writing

\begin{equation}
  H_{A}' = O H_{AA} O^{\dagger}
\end{equation}

With $O$ an $m \times 2^L$ matrix, with rows being the $m$ lowest-lying
eigenvectors of $H_{AA}$, and $L$ the number of lattice sites of a block. At
the fixed point of this iteration procedure, $H_A$ represents the hamiltonian
of a half-infinite chain. In choosing this truncation, it is assumed that the
low-lying eigenstates of the system in the thermodynamic limit are composed of
low-lying eigenstates of smaller blocks. It turns out that this method gives
poor results for many lattice systems. Following an example put forth by White
and Noack \cite{white1992real}, we establish here an intuition why.

\subsection{Single particle in a box}

Consider the Hamiltonian

\begin{equation} 
  H = 2 \sum_i \ket{i}\bra{i} - \sum_{\langle i, j \rangle} \ket{i}\bra{j}
\end{equation}

Where the second summation is over nearest neighbors $\langle i,
j \rangle$. $H$ represents the discretized version of the
particle-in-a-box Hamiltonian, so we expect its ground state to be
approximately a standing wave with wavelength double the box size.
However, the blocking procedure just described tries to build the ground
state iteratively from ground states of smaller blocks. No matter the
amount of states kept, the final result will always incur large errors.

For this simple model, White and Noack solved the problem by diagonalizing
the Hamiltonian of a block with different boundary conditions, and
combining the lowest eigenstates of each. Additionally, they noted that
diagonalizing $p > 2$ blocks, and projecting out $p - 2$ blocks to arrive
at $H_{AA}$ also gives accurate results, and that this is
a generalization of applying multiple boundary conditions. It is
a slightly changed version of this last method that is now known as DMRG.


\subsection{Density matrix method}















