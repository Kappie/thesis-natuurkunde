This appendix discusses only what is strictly necessary to understand the tensor network notation in this thesis.
For a more comprehensive introduction, see \cite{orus2014practical}.

\section{Tensors, or multidimensional arrays}
In the field of tensor networks, a tensor is a multidimensional table with numbers -- a
convenient way to organize information. It is the generalization of a vector
\begin{equation}
  v_i =
  \begin{bmatrix}
    v_1 \\
    \vdots \\
    v_n
  \end{bmatrix},
\end{equation}
which has one index, and a matrix
\begin{equation}
  M_{i j} =
  \begin{bmatrix}
  M_{1 1} & \dots & M_{1 n} \\
  \vdots  & & \vdots \\
  M_{m 1} & \dots & M_{m n}
  \end{bmatrix},
\end{equation}
which has two.
A tensor of rank $N$ has $N$ indices:\footnote{The definition of rank in this
context is not to be confused with the rank of a matrix, which is the number of
linearly independent columns. Synonyms of tensor rank are tensor degree and
tensor order.}
\begin{equation}
  T_{i_1 \dots i_N}.
\end{equation}

A tensor of rank zero is just a scalar.

\section{Tensor contraction}

Tensor contraction is the higher-dimensional generalization of the dot product
\begin{equation}
  \bm{a} \cdot \bm{b} = \sum_i a_i b_i,
\end{equation}
where a lower-dimensional tensor (in this case, a scalar, which is a
zero-dimensional tensor) is obtained by summing over all values of a repeated
index.

Examples are matrix-vector multiplication
\begin{equation}
  (M \bm{a})_{i} = \sum_j M_{i j} a_j,
\end{equation}
and matrix-matrix multiplication
\begin{equation}
  (A B)_{i j} = \sum_k A_{i k} B_{k j},
\end{equation}
but a more elaborate tensor multiplication could look like
\begin{equation}
  w_{a b c} = \sum_{d, e, f} T_{a b c d e f} v_{d e f}.
\end{equation}

As with the dot product between vectors, matrix-vector multiplication and
matrix-matrix multiplication, a contraction between tensors is only defined if
the dimensions of the indices match.

\section{Tensor networks}

A tensor network is specified by a set of tensors, together with a set of contractions to be performed. For example:
\begin{equation}
  M_{a b} = \sum_{i, j, k} A_{a i} B_{i j} C_{j k} D_{k b},
\end{equation}
which corresponds to the matrix product $A B C D$.

\subsection{Graphical notation}
It is highly convenient to introduce a graphical notation that is common in the
tensor network community. It greatly simplifies expressions and makes certain
properties manifest.

Each tensor is represented by a shape. Open-ended lines, called legs, represent unsummed
indices. See \autoref{fig:tensors_graphical_notation}. If it clear from the context, index
labels may be omitted from the open legs.

Each contracted index is represented by a connected line. See
\autoref{fig:contracted_tensors}.

Many tensor equations, while burdensome when written out, are readily
understood in this graphical way. As an example, consider the matrix trace in
\autoref{fig:contracted_tensors}, where its cyclic property is manifest.

\begin{figure}
  \begin{tikzpicture}
	\begin{pgfonlayer}{nodelayer}
		\node [style=generic, label={$c$}] (0) at (-3, 0) {};
		\node [style=generic, label={$v_i$}] (1) at (-1, 0) {};
		\node [style=generic, label={$M_{i j}$}] (2) at (1, 0) {};
		\node [style=generic, label={$T_{i j k}$}] (3) at (3, 0) {};
		\node [style=white no border] (4) at (-2, 0) {};
		\node [style=white no border] (5) at (0, 0) {};
		\node [style=white no border] (6) at (2, 0) {};
		\node [style=white no border] (7) at (4, 0) {};
		\node [style=white no border] (8) at (3, -1) {};
	\end{pgfonlayer}
	\begin{pgfonlayer}{edgelayer}
		\draw [style=simple] (1) to node[below]{$i$} (5);
		\draw [style=simple] (2) to node[below]{$i$} (5);
		\draw [style=simple, in=180, out=0, looseness=1.00] (2) to node[below]{$j$} (6);
		\draw [style=simple, in=0, out=180, looseness=1.00] (3) to node[below]{$i$} (6);
		\draw [style=simple] (3) to node[below]{$j$} (7);
		\draw [style=simple] (3) to node[left]{$k$} (8);
	\end{pgfonlayer}
\end{tikzpicture}

  \caption{Open-ended lines, called legs, represent unsummed indices. A tensor
  with no open legs is a scalar.}
  \label{fig:tensors_graphical_notation}
\end{figure}

\begin{figure}
  \begin{tikzpicture}
	\begin{pgfonlayer}{nodelayer}
		\node [style=generic, label={$\bm{a}$}] (0) at (-3, 0) {};
		\node [style=generic, label={$\bm{b}$}] (1) at (-2, 0) {};
		\node [style=generic, label={$M$}] (2) at (0, 0) {};
		\node [style=generic, label={$\bm{v}$}] (3) at (1, 0) {};
		\node [style=white no border] (4) at (-1, 0) {};
		\node [style=generic, label={$A$}] (5) at (-3, -2) {};
		\node [style=generic, label={$B$}] (6) at (-2, -2) {};
		\node [style=white no border] (7) at (-1, -2) {};
		\node [style=white no border] (8) at (-4, -2) {};
		\node [style=generic] (9) at (0, -1) {};
		\node [style=generic] (10) at (0, -2) {};
		\node [style=generic] (11) at (1, -2) {};
		\node [style=generic] (12) at (1, -1) {};
	\end{pgfonlayer}
	\begin{pgfonlayer}{edgelayer}
		\draw [style=simple] (0) to (1);
		\draw [style=simple] (4) to (2);
		\draw [style=simple] (2) to (3);
		\draw [style=simple] (8) to (5);
		\draw [style=simple] (5) to (6);
		\draw [style=simple] (6) to (7);
		\draw [style=simple] (10) to (11);
		\draw [style=simple] (11) to (12);
		\draw [style=simple] (12) to (9);
		\draw [style=simple] (9) to (10);
	\end{pgfonlayer}
\end{tikzpicture}

  \caption{Connected legs represent contracted indices. The networks in the
  figure represent $\sum_i a_i b_i$ (dot product),
  $\sum_j M_{i j} a_j$ (matrix-vector product), $\sum_{k} A_{i k} B_{k
  j}$ (matrix-matrix product) and $\tr A B C D$, respectively.}
  \label{fig:contracted_tensors}
\end{figure}

\subsection{Reshaping tensors}
Several indices can be taken together to form a single, joint index, that runs over all
combinations of the indices that fused into it. For example, an $m \times n$ matrix can be
reshaped into an $m n$ vector.
\begin{equation}
  M_{i j} = \bm{v_a} \qquad a \in \{ 1, \dots, m n \}.
\end{equation}

Some convention has to be chosen to map the joint index $i, j$ onto the single index $a$,
for example
\begin{equation}
  a = (n - 1)i + j,
\end{equation}
which orders the indices of the matrix $M$ row by row
\begin{equation}
  1 1, 1 2, \dots, 1 n, 2 1, 2 2, \dots, m n - 1, m n.
\end{equation}

Graphically, an index contraction is represented as a bundling of the open legs of a
tensor network. See \autoref{fig:reshaped_tensor} for an example.

\begin{figure}
  \begin{tikzpicture}
	\begin{pgfonlayer}{nodelayer}
		\node [style=generic] (0) at (-2, 0) {};
		\node [style=generic] (1) at (-1, 0) {};
		\node [style=generic] (2) at (0, 0) {};
		\node [style=white no border] (3) at (-2, 1) {};
		\node [style=white no border] (4) at (-1, 1) {};
		\node [style=white no border] (5) at (0, 1) {};
		\node [style=white no border] (6) at (0, -1) {};
		\node [style=white no border] (7) at (-1, -1) {};
		\node [style=white no border] (8) at (-2, -1) {};
		\node [style=white no border] (9) at (1, 0) {$=$};
		\node [style=generic2] (10) at (2, 0) {};
		\node [style=white no border] (11) at (2, 1) {};
		\node [style=white no border] (12) at (2, -1) {};
	\end{pgfonlayer}
	\begin{pgfonlayer}{edgelayer}
		\draw [style=simple] (0) to (1);
		\draw [style=simple] (1) to (2);
		\draw [style=simple] (2) to node[right]{$c$} (5);
		\draw [style=simple] (1) to node[right]{$b$} (4);
		\draw [style=simple] (0) to node[right]{$a$} (3);
		\draw [style=simple] (0) to node[right]{$d$} (8);
		\draw [style=simple, in=90, out=-90, looseness=1.00] (1) to node[right]{$e$} (7);
		\draw [style=simple] (2) to node[right]{$f$} (6);
		\draw [style=simple, ultra thick] (10) to node[right]{$i$} (11);
		\draw [style=simple, ultra thick] (10) to node[right]{$j$} (12);
	\end{pgfonlayer}

		\draw [style=simple, decoration={brace, amplitude=0.5em}, decorate] (3.center) to (5.center);
    \draw [style=simple, decoration={brace, amplitude=0.5em}, decorate, rotate=90] (6.center) to
    (8.center);
\end{tikzpicture}

  \caption{Reshaping a tensor $T_{a b c d e f}$ to $T_{i j}$}
  \label{fig:reshaped_tensor}
\end{figure}

\subsection{Computational complexity of contraction}

A contraction between two tensors
\begin{equation}
  C_{a_1 \dots a_m b_1 \dots b_n} = \sum_{z} A_{a_1 \dots a_m z} B_{z b_1 \dots b_n}
\end{equation}
takes
\begin{equation}
  \mathcal{O} \left( |z| \prod_{i} |a_i| \prod_{i} |b_i| \right)
\end{equation}
operations, where $|i|$ is the dimension of index $i$.
This is a straightforward generalisation of the fact that multiplication of an $n \times m$-matrix and an $m \times
p$-matrix takes $\mathcal{O}(n m p)$ operations.

This implies that the complexity of contracting an entire network of tensors depends vitally on the order in which the
indices are summed over (in which order the \enquote{legs are closed}).
For a simple example of this, see section 4.2.1 of \cite{schollwock2011density}.

Determining the optimal order is highly non-trivial (in fact, known to be NP-hard) and also depends on available
computer memory \cite{pfeifer2014faster}.
