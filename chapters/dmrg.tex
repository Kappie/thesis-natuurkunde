
\chapterprecishere{The density matrix renormalization group, proposed in 1992 by White
\cite{white1992density}, is introduced in its historical context. To highlight the ideas
that led to this method, we explain the real-space renormalization group, proposed by
Wilson \cite{wilson1975renormalization} in 1975. We then explain how the shortcomings of
Wilson's method led to the density matrix renormalization group.}

\section{Introduction}
Consider the problem of numerically finding the ground state $\ket{\Psi_0}$ of an
$N$-site one-dimensional spin-$\frac{1}{2}$ system.

The underlying Hilbert space of the system is a tensor product of the
local Hilbert spaces $\mathcal{H}_{\text{site}}$, which are spanned by the
states $\{\ket{\uparrow}, \ket{\downarrow}\}$. Thus, a general state of the system is a unit vector in
a $2^N$-dimensional space
\begin{equation}
  \ket{\Psi} = \sum_{\sigma_1, \sigma_2, \ldots \in \{\ket{\uparrow}, \ket{\downarrow}\}}
  c_{\sigma_1, \sigma_2, \ldots, \sigma_N} \ket{\sigma_1} \otimes \ket{\sigma_2} \otimes
  \ldots \otimes \ket{\sigma_N}.
\end{equation}

For a system with 1000 particles, the dimensionality of the Hilbert
space comes in at about $10^{301}$, some 220 orders of magnitude larger than the number of
atoms in the observable universe. How can we possibly hope to approximate states in this
space?

As it turns out, nature is very well described by Hamiltonians that are local -- that do
not contain interactions between an arbitrary number of bodies. And for these
Hamiltonians, only an exponentially small subset of states can be explored in the lifetime
of the universe \cite{poulin2011quantum}. That is, only exponentially few states are
physical. The low-energy states, especially, have special properties that allow them to be
very well approximated by a polynomial number of parameters. This explains the existence
of algorithms, of which the density matrix normalization group is the most widely
celebrated one, that can approximate certain quantum systems to machine precision.

\todo[inline]{Refer to where this will be made more precise.}

\section{Density matrix renormalization group}

The density matrix renormalization group (DMRG), introduced in 1992 by
White \cite{white1992density}, aims to find the best approximation of
a many-body quantum state, given that only a fixed amount of basis vectors
is kept. This amounts to finding the best truncation

\begin{equation}
  \mathcal{H}_N \rightarrow \mathcal{H}_{\text{eff}}
\end{equation}
from the full $N$-particle Hilbert space to an effective lower dimensional
one. This corresponds to renormalizing the Hamiltonian $H$.

Before DMRG, several methods for achieving this truncation were proposed, most notably
Wilson's real-space renormalization group \cite{wilson1975renormalization}. We will
discuss this method first, and highlight its shortcomings, which eventually led to the
invention of the density-matrix renormalization group method by White.

\subsection{Real-space renormalization group}

Consider again the problem of finding the ground state of a many-body
Hamiltonian $H$. A natural way of renormalizing $H$ in real-space is by
partitioning the lattice in blocks, and writing $H$ as

\begin{equation}
  H = H_A \otimes \ldots \otimes H_A
\end{equation}
where $H_A$ is the Hamiltonian of a block. \todo[inline]{Make figures.} The real-space
renormalization procedure now entails finding an effective Hamiltonian $H_{A}'$ of the
two-block Hamiltonian $H_{AA} = H_A \otimes H_A$. In the method introduced
by Wilson, $H_{A}'$ is formed by keeping the $m$ lowest lying eigenstates
$\ket{\epsilon_{i}}$ of $H_{AA}$:
\begin{equation}
  H_{A}' = \sum_{i = 1}^{m} \epsilon_{i} \ket{\epsilon_{i}}\bra{\epsilon_{i}}.
 \end{equation}
This is equivalent to writing
\begin{equation}
  H_{A}' = O H_{AA} O^{\dagger},
\end{equation}
with $O$ an $m \times 2^L$ matrix, with rows being the $m$ lowest-lying
eigenvectors of $H_{AA}$, and $L$ the number of lattice sites of a block. At
the fixed point of this iteration procedure, $H_A$ represents the
Hamiltonian of an infinite chain.
In choosing this truncation, it is assumed that the
low-lying eigenstates of the system in the thermodynamic limit are
composed of low-lying eigenstates of smaller blocks.

It turns out that this method gives poor results for many lattice systems. Following an
example put forth by White and Noack \cite{white1992real}, we establish an intuition
why.

\subsection{Single particle in a box}

Consider the Hamiltonian
\begin{equation}
  H = 2 \sum_i \ket{i}\bra{i} - \sum_{\langle i, j \rangle} \ket{i}\bra{j},
\end{equation}
where the second summation is over nearest neighbors $\langle i,
j \rangle$. $H$ represents the discretized version of the
particle-in-a-box Hamiltonian, so we expect its ground state to be
approximately a standing wave with wavelength double the box size.
However, the blocking procedure just described tries to build the ground
state iteratively from ground states of smaller blocks. No matter the
amount of states kept, the final result will always incur large errors.

For this simple model, White and Noack solved the problem by diagonalizing
the Hamiltonian of a block with different boundary conditions, and
combining the lowest eigenstates of each.

Additionally, they noted that
diagonalizing $p > 2$ blocks, and projecting out $p - 2$ blocks to arrive
at $H_{AA}$ also gives accurate results, and that this is a generalization
of applying multiple boundary conditions. \todo[inline]{Figure.}

In the limit $p \to \infty$
this method becomes exact, since we then find exactly the correct
contribution of $H_{AA}$ to the final ground state. It is a slightly
changed version of this last method that is now known as DMRG.

\subsection{Density matrix method}

The fundamental idea of the density matrix renormalization group method
rests on the fact that if we know the state of the final lattice, we can find the $m$
most important states for $H_{AA}$ by diagonalizing the reduced density
matrix $\rho_{AA}$ of the two blocks.

To see this, suppose, for simplicity, that the entire lattice is in a pure
state\footnote{For a proof for a mixed state, see \cite{noack1999workshop}} $\ket{\Psi} = \sum c_{b, e} \ket{b} \ket{e}$, with $b = 1, \ldots, l$ the
states of $H_{AA}$ and $e = 1, \ldots, N_{\text{env}}$ the environment states. The
reduced density matrix is given by
\begin{equation}\label{eq:density_matrix_superblock}
  \rho_{AA} = \sum_{e} \ket{\Psi} \bra{\Psi} = \sum_{b, b'} c_{b, e} c_{b', e} \ket{b} \bra{b'}
\end{equation}
We now wish to find a set of orthonormal states $\ket{\lambda} \in \mathcal{H}_{AA}$,
$\lambda = 1, \ldots, m$ with $m < l$, such that the quadratic norm
\begin{equation}\label{eq:quadratic_norm}
  \Vert \ket{\Psi} - \ket{\widetilde{\Psi}} \Vert = 1 - 2 \sum_{\lambda, b, e} a_{\lambda, e} c_{b, e} u_{\lambda, b} + \sum_{\lambda, e} a_{\lambda, e}^2
\end{equation}
is minimized. Here,
\begin{equation}
  \ket{\widetilde{\Psi}} = \sum_{\lambda = 1}^{m} \sum_{e = 1}^{N_{\text{env}}} a_{\lambda, e} \ket{\lambda} \ket{e}
\end{equation}
is the representation of $\ket{\Psi}$ given the constraint that we can only use
$m$ states from $\mathcal{H}_{AA}$. The $u_{\lambda, b}$ are given by
\begin{equation}
  \lambda = \sum_{b} u_{\lambda, b} \ket{b}.
\end{equation}

We need to minimize \eqref{eq:quadratic_norm} with respect to $a_{\lambda, e}$
and $u_{\lambda, b}$. Setting the derivative with respect to $a_{\lambda, e}$ equal to 0 yields
\begin{equation}
  -2 \sum_{\lambda, b, e} c_{b, e} u_{\lambda, b} + 2 \sum_{\lambda, e} a_{\lambda, e} = 0
\end{equation}
So we see that $a_{\lambda, e} = \sum_{b} c_{b, e} u_{\lambda, b}$, and we are left to minimize
\begin{equation}
  1 - \sum_{\lambda, b, b'} u_{\lambda, b} (\rho_{AA})_{b, b'} u_{\lambda, b'}
\end{equation}
with respect to $u_{\lambda, b}$. But this is equal to
\begin{equation}
  1 - \sum_{\lambda = 1}^{m} \bra{\lambda} \rho_{AA} \ket{\lambda}
\end{equation}
and because the eigenvalues of $\rho_{AA}$ represent probabilities and are thus
non-negative, this is clearly minimal when $\ket{\lambda}$ are the $m$
eigenvectors of $\rho_{AA}$ corresponding to the largest eigenvalues. This minimal value is
\begin{equation}\label{eq:truncation_error}
  1 - \sum_{\lambda = 1}^{m} w_{\lambda}
\end{equation}
with $w_{\lambda}$ the eigenvalues of the reduced density matrix.

\autoref{eq:truncation_error} is called the truncation error or residual
probability, and quantifies the incurred error when taking a number $m < l$ states to
represent $\mathcal{H}_{AA}$.

We have proven that the optimal (in the sense that $\Vert \ket{\Psi}
- \ket{\widetilde{\Psi}} \Vert$ is minimized\footnote{There are several other
arguments for why these states are optimal, for example, they minimize the
error in expectation values $\langle A \rangle$ of operators. For an overview,
see \cite{schollwock2005density}.}) states to keep for a subsystem are the
states given by the reduced density matrix, obtained by tracing out the entire
lattice in the ground state (or some other target state).

The problem, of
course, is that we do not know the state of the entire lattice, since that is
exactly what we're trying to approximate.

Instead then, we should try to calculate the reduced density matrix of the
system embedded in \textit{some} larger environment, as closely as possible
resembling the one in which it should be embedded.  The combination of the
system block and this environment block is usually called \textit{superblock}.

Analogous to how White and Noack solved the particle in a box problem, we could
calculate the ground state of $p > 2$ blocks,
and trace out all but 2, doubling our block size each iteration. In
practice, this doesn't work well for interacting Hamiltonians, since this
would involve finding the largest eigenvalue of a $N_{\text{block}}^p
\times N_{\text{block}}^p$ matrix (compare this with the particle in a box
Hamiltonian, which only grows linearly in the amount of lattice sites).

The widely adopted algorithm proposed by White \cite{white1993density} for
finding the ground state of a system in the thermodynamic limit proceeds
as follows.

\subsection{Infinite-system method}

\todo[inline]{Add figures.} \todo[inline]{Mention boundary conditions somewere}
Instead of using an exponential blocking procedure (doubling or tripling the
amount of effective sites in a block at each iteration), the infinite-system
method in the DMRG formulation adds a single site before truncating the Hilbert
space to have at most $m$ basis states.

\begin{enumerate}
  \item \label{step1} Consider a block $A$ of size $l$, with $l$ small. Suppose, for
  simplicity, that the number of basis states of the block is already
  $m$. States of this block can be written as
  \begin{equation}
    \ket{\Psi_A} = \sum_{b = 1}^{m} c_{i} \ket{b}.
  \end{equation}

  The Hamiltonian is written as (similarly for other operators):
  \begin{equation}
    \hat{H}_{A} = \sum_{b, b'}^{m} H_{b b'} \ket{b} \bra{b}.
  \end{equation}

  \item Construct an enlarged block with one additional site, denoted by $A
  \cdot$. States are now written
  \begin{equation}
    \ket{\Psi_{A\cdot}} = \sum_{b, \sigma} c_{b, \sigma} \ket{b} \otimes \ket{\sigma}.
  \end{equation}
  Here, $\sigma$ runs over the $d$ local basis states of $\mathcal{H}_{\text{site}}$.

  \item Construct a superblock, consisting of the enlarged system block $A
  \cdot$ and a reflected environment block $\cdot A$, together denoted by $A \cdot
  \cdot A$. Find the ground state $\ket{\Psi_0}$ of $A \cdot \cdot A$, for example
  with the Lanczos method \cite{lehoucq1996deflation}.

  \item Obtain the reduced density matrix of the enlarged block by tracing out
  the environment, and write it in diagonal form.
  \begin{equation}
  \begin{aligned}
    \rho_{A \cdot} & = \sum_{e, \sigma} (\bra{\sigma} \otimes \bra{e}) \ket{\Psi_0}
    \bra{\Psi_0} (\ket{\sigma} \otimes \ket{e}), \\
    & = \sum_{i = 1}^{d m} w_{i} \ket{\lambda_i} \bra{\lambda_i}.
  \end{aligned}
  \end{equation}

  Here, we have chosen $w_0 >= w_1 \ldots >= w_{d m}$. In this basis, the Hamiltonian is
  written as
  \begin{equation}
    \hat{H}_{A \cdot} = \sum_{i, j}^{d m} H_{ij} \ket{\lambda_i}\bra{\lambda_j}.
  \end{equation}

  \item Truncate the Hilbert space by keeping only the $m$ eigenstates of
  $\rho_{A \cdot}$ with largest eigenvalues. Operators truncate as follows:
  \begin{align}
    \widetilde{\rho}_{A \cdot} & = \sum_{i = 1}^{m} w_i \ket{\lambda_i}\bra{\lambda_j}, \\
    \widetilde{H}_{A \cdot} & = \sum_{i, j}^{m} H_{ij} \ket{\lambda_i}\bra{\lambda_j}.
  \end{align}

  \item Set $H_{A} \leftarrow \widetilde{H}_{A \cdot}$ and return to \ref{step1}.

\end{enumerate}

\todo[inline]{Expand. Present or link to some results. Finite-system algorithm. Maybe in
other chapter: rephrase in MPS, validity of approximation: primer on entropy and
eigenvalue spectrum of density matrix.}

This methods finds ground state energies with astounding accuracy, and has been
the reference point in all 1D quantum lattice simulation since its invention.
